\chapter[Temporalised Query Entailment in~\texorpdfstring{\SHQ}{SHQ}]{%
    Temporalised Query Entailment in the Description~Logic~\texorpdfstring{\SHQ}{SHQ}}\label{ch:tcqs}

Ontology-based data access (OBDA)~\cite{DEF+-DS99,PCD+-JoDS08} generalises query
answering in databases: firstly, the data are not assumed to be complete,
i.e.~we do not make the closed-world assumption, and secondly, the
interpretation of the predicates occurring in the queries is constrained by
background knowledge encoded in a knowledge base.

In this chapter, we investigate a \emph{temporalised} version of OBDA and its
corresponding decision problem: temporalised query entailment.  We show how
temporalised query entailment can be decided in the description logic~\SHQ, and
provide complexity results.
%
Most of the results contained in this chapter have already been published
in~\cite{BaBL-CADE13,BaBL-LTCS-13-01}.
%in~\cite{tcq-journal,BaBL-CADE13,BaBL-LTCS-13-01}.

This chapter is organised as follows.  In Section~\ref{sec:temporal-queries}, we
formally introduce the temporal query language that we investigate in this
chapter and discuss similar approaches to temporalising OBDA\@.  After that, in
Section~\ref{sec:complexity-query-entailment}, we show complexity results for
temporalised query entailment in our temporal query language.  Finally, in
Section~\ref{sec:tcqs-summary}, we give a brief summary of the results that we
have obtained in this chapter.


\section{The Temporal Query Language}\label{sec:temporal-queries}

Unless stated otherwise, we assume throughout this chapter that all ABoxes are
\emph{simple} (see Definition~\ref{def:syntax-abox}).  Note that, however, every
complex ABox can be rewritten to a simple ABox using a set of GCIs.  More
precisely, one can rewrite that concept assertion $C(a)$ to $A(a)$ where
$A\in\NC$ is a fresh concept name, and add the GCIs $A\sqsubseteq C$ and
$C\sqsubseteq A$ to the TBox.  However, this restriction is useful to separate
the influence of the ABox and the TBox on the complexity of reasoning problems.

Our temporal query language is a combination of conjunctive
queries~\cite{AbHV-95} and propositional LTL~\cite{Pnu-FOCS77}.  This language
is very similar to the temporalised description logics
\ALC-LTL~\cite{BaGL-ToCL12} and \SHOQ-LTL introduced in
Section~\ref{sec:syntax-semantics-shoq-ltl}.  The main difference is that we
allow conjunctive queries to occur in place of description logic axioms.
%
Thus, the results we obtained build on existing results about \ALC-LTL and also
\SHOQ-LTL\@.  Since our temporal query language generalises \ALC-LTL, some of
our hardness results for complexity follow easily from the results
in~\cite{BaGL-ToCL12}.  Moreover, we will use the results obtained in
Section~\ref{sec:consistency-boolean-shoqcap-kb} about the consistency problem
for Boolean \SHOQcap-knowledge bases.  For more information about temporalising
description logics, see Section~\ref{sec:temporalised-dls}.

However, most work on temporalised description logics focuses on the
satisfiability problem in such logics rather than query answering.  In the
following, we describe relevant related work in temporalising OBDA\@.  The
approaches from the literature have mainly been developed for light-weight
languages of the \DLLite family~\cite{CDL+-RW09}.

For instance, in~\cite{AKL+-TIME07}, various light-weight DLs are extended by
allowing the temporal operators to interfere with the DL-component.  Extending
the work of~\cite{AKL+-TIME07}, in~\cite{AKW+-IJCAI13} a temporal extension of
\DLLite is presented, which allows the temporal operators $\Diamondm$ and
$\Diamond$ on the left-hand side of GCIs and role-inclusion axioms.  In this
logic, \emph{first-order rewritability} of conjunctive queries w.r.t.\
\DLLite-knowledge bases is preserved from the atemporal case, i.e.~answering a
query over a knowledge base can be reduced to answering a rewritten query (in a
different language) over a database induced by the knowledge base.  This means
that techniques from temporal relational databases can be used to answer
temporal queries that can refer to specific points in time.

An approach to temporal query answering in \DLLite that is more similar to the
one considered in this chapter is presented
in~\cite{BoLT-FroCoS13,BoLT-DL13,BoLT-LTCS-13-05}.  There, conjunctive queries
are used as atoms in negation-free temporal formulas.  This allows for reuse of
results about atemporal first-order rewritability in \DLLite.  The paper also
presents an algorithm to answer such temporal queries over temporal relational
databases, which generalises an algorithm from~\cite{Chom-ToDS95,ChTo-05}.  The
main advantage of this algorithm is that it achieves a so-called
\emph{bounded-history encoding}, i.e.~the amount of space needed to answer a
temporal query does not depend on the length of the observed history.  Thus, it
is enough to keep track of the relevant data and storing it in the database
instead of storing all information from the past.

A similar approach is pursued in~\cite{GuKl-RR12}.  There, the authors propose a
generic framework to combine a generic DL-query component with a linear temporal
dimension.  To simplify the decision procedures, both components are decoupled
via an autoepistemic modal operator.  This allows to use atemporal
query-answering algorithms as a black-box inside a temporal satisfiability
algorithm.

In~\cite{Mot-JWS12}, temporal query answering over temporalised
RDF-triples~\cite{GuHV-ESWC05} using an extension of the query language SPARQL
is considered.

Furthermore, in~\cite{AFW+-JELIA02}, the very expressive temporalised
description logic \DLRUS is introduced, which is an extension of \DLR that
allows temporal operators to occur within concepts and roles.  Moreover, the
query-containment problem of non-recursive Datalog queries under constraints
defined in \DLRUS is investigated.  It turns out that this problem is in general
undecidable, but becomes decidable in the fragment \DLRUSminus, where no
temporal operators are allowed to occur within roles.  The query-containment
problem is then in \TwoExpTime, whereas other reasoning problems such as the
satisfiability problem and the subsumption problem in \DLRUSminus are
\ExpSpace-complete.

Whilst in principle our temporal query language can be parametrised with any
description logic, we focus in this chapter on the description logics between
\ALC and~\SHQ.  The relative expressivity of these DLs is depicted in
Figure~\ref{fig:alc-shq}.

\begin{figure}[t]
    \centering
    \begin{tikzpicture}[%
            semithick,
            auto,
            on grid,
            node distance=8ex and 8em,
            element/.style={rectangle,draw=gray,minimum height=2ex,minimum width=5em}]

        \node[element]                 (shq)    {\SHQ};
        \node[element,below=of shq]    (sq)     {\SQ};
        \node[element,left=of sq]      (sh)     {\SH};
        \node[element,right=of sq]     (alchq)  {\ALCHQ};
        \node[element,below=of sq]     (alch)   {\ALCH};
        \node[element,left=of alch]    (s)      {\Smc};
        \node[element,right=of alch]   (alcq)   {\ALCQ};
        \node[element,below=of alch]   (alc)    {\ALC};

        \draw[gray] (shq)--(sh);
        \draw[gray] (shq)--(sq);
        \draw[gray] (shq)--(alchq);
        \draw[gray] (sh)--(s);
        \draw[gray] (sh)--(alch);
        \draw[gray] (sq)--(s);
        \draw[gray] (sq)--(alcq);
        \draw[gray] (alchq)--(alch);
        \draw[gray] (alchq)--(alcq);
        \draw[gray] (s)--(alc);
        \draw[gray] (alch)--(alc);
        \draw[gray] (alcq)--(alc);
    \end{tikzpicture}
    \caption{The relative expressivity of the DLs between \ALC and~\SHQ}
    \label{fig:alc-shq}
\end{figure}

To summarise, the work described
in~\cite{AKW+-IJCAI13,BoLT-FroCoS13,AFW+-JELIA02,GuKl-RR12} is most closely
related to our approach.  Nonetheless, this related work differs from our
approach in several ways:
\begin{enumerate}
    \item We consider the expressive description logic \SHQ instead of
        inexpressive description logics such as members of the \DLLite
        family~\cite{AKW+-IJCAI13,BoLT-FroCoS13}.
    \item We consider a temporal query language instead of temporalising the
        ontology language~\cite{AKW+-IJCAI13,AFW+-JELIA02}.
    \item In contrast to~\cite{GuKl-RR12}, we consider also the case of rigid
        concept and role names.
        In~\cite{BoLT-FroCoS13,BoLT-DL13,BoLT-LTCS-13-05,AKW+-IJCAI13}, rigid
        names are also used, but in the context of inexpressive DLs.
\end{enumerate}

Since we deal with the expressive description logic~\SHQ, we take up on the
results about the complexity of atemporal conjunctive-query entailment in
expressive description logics~\cite{OrCE-AAAI06,Lut-IJCAR08,GHL+-JAIR08}.  For
the proofs of our results, it is, however, not sufficient to only apply these
results, but we need to adapt the proof methods that were developed in these
papers to show these results.  To be more precise, we adapt, for instance, the
constructions involving forest models and equivalence relations over individual
names from~\cite{GHL+-JAIR08}, and we use the results about spoilers in \SHQcap
from~\cite{Lut-IJCAR08}.  We make this connection explicit in later sections of
this chapter.

As the temporal component of our query language is propositional LTL, we also
use well-known results from that area of research.  As such, we adapt the
automata construction for propositional LTL satisfiability
from~\cite{WoVS-FOCS83,VaWo-IC94}, which is mentioned in
Section~\ref{sec:ltl-automata}.

For our temporal query language, we investigate both the \emph{combined
complexity} and the \emph{data complexity} of the temporalised query-entailment
problem in three different settings as summarised in
Table~\ref{tab:tcq-results}.  These results hold for all description logics
between \ALC and~\SHQ.  In fact, we show that the hardness results already hold
for \ALC, and we prove the complexity upper bounds for the more expressive
description logic~\SHQ.
%
\begin{table}[t]
    \centering
    \caption{The complexity of temporalised query entailment for all DLs between
        \ALC and~\SHQ in three different settings}
    \label{tab:tcq-results}
    \begin{tabular*}{\textwidth}{l@{\extracolsep{\fill}}ll}
        \toprule
            &\emph{Data complexity}
            &\emph{Combined complexity}\\
        \midrule
        \emph{Setting~(i)}
            &\coNP-complete
            &\ExpTime-complete\\
            &(Corollary~\ref{cor:lower-bounds-data-complexity} and Theorem~\ref{thm:upper-bounds-no-rigid-names})
            &(Theorems~\ref{thm:lower-bounds-combined-complexity} and~\ref{thm:upper-bounds-no-rigid-names})\\
        \midrule
        \emph{Setting~(ii)}
            &\coNP-complete
            &\coNExpTime-complete\\
            &(Corollary~\ref{cor:lower-bounds-data-complexity} and Theorem~\ref{thm:upper-bound-rigid-concepts-data-compl})
            &(Theorems~\ref{thm:lower-bounds-combined-complexity} and~\ref{thm:combined-complexity-rigid-concept-names})\\
        \midrule
        \emph{Setting~(iii)}
            &\coNP-hard\,/\,in \ExpTime
            &\TwoExpTime-complete\\
            &(Corollary~\ref{cor:lower-bounds-data-complexity} and Theorem~\ref{thm:upper-bounds-rigid-names})
            &(Theorems~\ref{thm:lower-bounds-combined-complexity} and~\ref{thm:upper-bounds-rigid-names})\\
        \bottomrule
    \end{tabular*}
    \\[1ex]
    \caption*{Settings: (i)~neither concept names nor role names are allowed to
        be rigid; (ii)~only concept names may be rigid; and (iii)~both concept
        names and role names may be rigid.}
\end{table}

Note that in~\cite{BaBL-LTCS-13-01,BaBL-CADE13}, the complexity results were
shown only for~\ALC.  Even though our complexity results are the same for \ALC
and~\SHQ, and in principle the approaches used below to prove
the upper bounds for~\SHQ are similar to the ones employed
in~\cite{BaBL-LTCS-13-01,BaBL-CADE13}, the proof details are considerably more
complex for~\SHQ.
%
For the combined complexity, the complexity results listed in
Table~\ref{tab:tcq-results} are actually identical to the ones for
\ALC-LTL~\cite{BaGL-ToCL12}, though the upper bounds are considerably harder to
show.  The data complexity results in Settings~(i) and~(ii) coincide with the
ones for atemporal query entailment, which is \coNP-complete w.r.t.\ data
complexity.  For Setting~(iii), we can show that the temporalised query
entailment problem is in \ExpTime w.r.t.\ data complexity (in contrast to
\TwoExpTime-completeness w.r.t.\ combined complexity), but we do not have a
matching lower bound.

From now on, we consider an arbitrary (but fixed) DL between \ALC and \SHQ.
Before we formally define our temporal query language in Section~\ref{sec:tcqs},
we first introduce \emph{conjunctive queries} and related notions in
Section~\ref{sec:cqs}, and \emph{temporal knowledge bases} in
Section~\ref{sec:tkbs}.


\subsection{Conjunctive Queries}\label{sec:cqs}

Our query language is based on conjunctive queries~\cite{AbHV-95}, which is a
subset of first-order queries that are well-investigated in database theory.
Basically, conjunctive queries correspond to select-project-join queries in
relational algebra, and to select-from-where queries in SQL\@.

\begin{definition}[Syntax of conjunctive queries]\label{def:syntax-cqs}
    Let \NV be a set of variables. A \emph{conjunctive query (CQ)} is
    of the form $\phi=\exists y_1,\dots,y_m.\psi$, where $y_1,\dots,y_m\in\NV$,
    and $\psi$ is a (possibly empty) finite conjunction of \emph{atoms} of the
    form
    \begin{itemize}
        \item $A(z)$ with $A\in\NC$ and $z\in\NV\cup\NI$ \emph{(concept atom)};
            or
        \item $r(z_1,z_2)$ with $r\in\NR$ and $z_1,z_2\in\NV\cup\NI$ \emph{(role
            atom)}.
    \end{itemize}
    The empty conjunction is denoted by \true.

    A \emph{union of conjunctive queries} (UCQ) is of the form
    $\phi_1\lor\dots\lor\phi_n$ with $n\ge 1$, where $\phi_1,\dots,\phi_n$ are
    CQs.
\end{definition}

\noindent
We denote the set of individual names occurring in a (U)CQ~$\phi$ by
$\Ind(\phi)$, the set of variables occurring in $\phi$ by $\Var(\phi)$, the set
of free variables occurring in $\phi$ by $\FVar(\phi)$, and the set of atoms
occurring in $\phi$ by $\At(\phi)$.
%
We call $\phi$ \emph{Boolean} if $\FVar(\phi)=\emptyset$.
%
Moreover, we denote the set of individual names occurring in a knowledge
base~\Kmc by $\Ind(\Kmc)$.

Given a (U)CQ~$\phi$ and a knowledge base~\Kmc, a basic reasoning task is
finding so-called \emph{certain answers} to~$\phi$ w.r.t.~\Kmc,
i.e.~instantiations of the free variables in~$\phi$ with individual names from
$\Ind(\Kmc)$ such that the resulting formula is satisfied in every model
of~\Kmc.  Thus, answering (U)CQs w.r.t.\ knowledge bases generalises the
entailment of ABox-axioms, i.e.~deciding whether $\Kmc\models\alpha$ holds for a
given knowledge base~\Kmc and a given ABox-axiom~$\alpha$.

We now define the semantics of Boolean (U)CQs, using the notion of
homomorphisms~\cite{ChMe-STOC77}.  This is then extended to answering arbitrary
UCQs.

\begin{definition}[Homomorphism, entailment, certain answer]
    Let $\Imc=(\Delta,\cdot^\Imc)$ be an interpretation and $\phi$ be a Boolean
    CQ\@. A mapping $\pi\colon\Var(\phi)\cup\Ind(\phi)\to\Delta$ is a
    \emph{homomorphism} of $\phi$ into \Imc if
    \begin{itemize}
        \item $\pi(a)=a^\Imc$ for every $a\in\Ind(\phi)$;
        \item $\pi(z)\in A^\Imc$ for every concept atom $A(z)\in\At(\phi)$;
            and
        \item $(\pi(z_1),\pi(z_2))\in r^\Imc$ for every role atom
            $r(z_1,z_2)\in\At(\phi)$.
    \end{itemize}
    %
    We say that \Imc is a \emph{model} of~$\phi$ (written $\Imc\models\phi$) if
    there is such a homomorphism.  Moreover, \Imc is a \emph{model} of a Boolean
    UCQ $\phi_1\lor\dots\lor\phi_n$ if it is a model of $\phi_i$ for some~$i$,
    $1\le i\le n$.

    A Boolean UCQ~$\phi$ is \emph{entailed} by a knowledge base~\Kmc (written
    $\Kmc\models\phi$) if every model of~\Kmc is also a model of~$\phi$.

    Given a (not necessarily Boolean) UCQ $\phi$, we call a mapping
    $\amf\colon\FVar(\phi)\to\Ind(\Kmc)$ a \emph{certain answer} to~$\phi$
    w.r.t.~\Kmc if $\Kmc\models\amf(\phi)$, where $\amf(\phi)$ is the Boolean
    UCQ obtained from~$\phi$ by replacing the free variables according to~\amf.
\end{definition}

\noindent
For a UCQ $\phi$ and a knowledge base \Kmc, one can compute all certain answers
by enumerating all candidate mappings $\amf\colon\FVar(\phi)\to\Ind(\Kmc)$ and
then solving the entailment problem $\Kmc\models\amf(\phi)$ for each $\amf$.
Since there are $\lvert\Ind(\Kmc)\rvert^{\lvert\FVar(\phi)\rvert}$ such
mappings, we have to solve exponentially many such entailment problems.

To analyse the complexity of deciding the entailment $\Kmc\models\amf(\phi)$, it
obviously suffices to consider the case where the UCQ is Boolean.  As discussed
in Section~\ref{sec:intro-tcqs}, usually one considers two kinds of complexity
measures: the combined complexity and data complexity.
%
For the \emph{combined complexity}, all parts of the input, i.e.~the UCQ~$\phi$
and the knowledge base $\Kmc=(\Amc,\Tmc,\Rmc)$, are taken into
account.  For the \emph{data complexity}, however, $\phi$, \Tmc, and~\Rmc are
assumed to be of constant size, and the complexity is measured only w.r.t.\ the
data, i.e.~the ABox~\Amc.
%
For this analysis, we assume in the following that the query does not introduce
new names, i.e.~it contains only concept and role names that also occur in the
TBox or the RBox.
%
This is without loss of generality since we can always introduce trivial axioms
like $A\sqsubseteq A$ or $r\sqsubseteq r$ into the TBox and RBox without
affecting data complexity or combined complexity.

Regarding data complexity, the entailment problem for \emph{concept assertions}
and \ALC-know\-ledge bases is already \coNP-hard~\cite{Sch-JIIS93,DLN+-JLC94},
and a matching upper bound has been established for the entailment problem for
UCQs and \SHQ-knowledge bases~\cite{GHL+-JAIR08}.

The entailment problem for concept assertions and \ALC-knowledge bases is
\ExpTime-hard w.r.t.\ combined complexity~\cite{DLhandbook-07}, and a matching
upper bound is known for the entailment problem for UCQs and \ALCHQ-knowledge
bases~\cite{Lut-IJCAR08}.  For the description logic~\Wmc, the problem is
already \coNExpTime-hard, while it becomes \TwoExpTime-hard for the description
logic \SH~\cite{ELO+-IJCAI09}.
%
In this chapter, we focus on a variant of the UCQ-entailment problem that is
\ExpTime-complete even for \SHQ-knowledge bases, namely, we restrict to
\emph{simple} queries, which are not allowed to use non-simple role names.  Note
that this is only a restriction in extensions of~\Wmc.

Before we are ready to consider the temporalised query-entailment problem, we
formally introduce temporal knowledge bases.


\subsection{Temporal Knowledge Bases}\label{sec:tkbs}

We extend the notion of knowledge bases and models into the temporal setting.
The setting is that there is a global TBox and a global RBox that define the
terminology, and several ABoxes that contain information about the state of the
world at the time points we have observed so far.

\begin{definition}[Syntax of temporal knowledge bases]\label{def:syntax-tkb}
    A \emph{temporal knowledge base (temporal KB)}
    $\Kmc=((\Amc_i)_{0\le i\le n},\Tmc,\Rmc)$ consists of a non-empty
    finite sequence of ABoxes $\Amc_i$, $0\le i\le n$, of \emph{length}
    $n+1>0$, a TBox~\Tmc, and an RBox~\Rmc.
\end{definition}

\noindent
As for atemporal knowledge bases, we denote by $\Ind(\Kmc)$ the set of all
individual names occurring in a temporal KB~\Kmc.
%
The semantics of temporal KBs is based on DL-LTL-structures, which were introduced in
Definition~\ref{def:semantics-shoq-ltl}.

\begin{definition}[Semantics of temporal knowledge bases]\label{def:semantics-tkb}
    We call the DL-LTL-structure $\Imf=(\Imc_i)_{i\ge 0}$ a \emph{model} of the
    temporal KB $\Kmc=((\Amc_i)_{0\le i\le n},\Tmc,\Rmc)$ (written
    $\Imf\models\Kmc$) if
    \begin{itemize}
        \item $\Imc_i\models\Amc_i$ for every $i$, $0\le i\le n$; and
        \item $\Imc_i\models\Tmc$ and $\Imc_i\models\Rmc$ for every $i\ge 0$.
    \end{itemize}
\end{definition}

\noindent
Recall that according to Definition~\ref{def:dl-ltl-structure}, we make the
constant-domain assumption and the rigid-individual assumption.

Now we are ready to formally introduce our temporal query language.


\subsection{Temporal Conjunctive Queries}\label{sec:tcqs}

We combine the notions of conjunctive queries and propositional LTL-formulas
into a new formalism, which we call \emph{temporal conjunctive queries}.

\begin{definition}[Syntax of temporal conjunctive queries]
    The set of \emph{temporal conjunctive queries} (TCQs) is the smallest set
    such that
    \begin{itemize}
        \item every conjunctive query is a TCQ; and
        \item if $\phi_1$ and $\phi_2$ are TCQs, then so are: $\lnot\phi_1$
            (negation), $\phi_1\land\phi_2$ (conjunction), $\Next\phi_1$ (next),
            $\Previous\phi_1$ (previous), $\phi_1\Until\phi_2$ (until), and
            $\phi_1\Since\phi_2$ (since).
    \end{itemize}
\end{definition}

\noindent
As for UCQs, we denote the set of individual names occurring in a TCQ~$\phi$ by
$\Ind(\phi)$, and the set of free variables occurring in~$\phi$ by
$\FVar(\phi)$.  Moreover, a \emph{Boolean TCQ} is a TCQ without free variables.

As usual in temporal logics, we use again
\begin{itemize}
    \item $\phi_1\lor\phi_2$ (disjunction) as an abbreviation for
        $\lnot(\lnot\phi_1\land\lnot\phi_2)$;
    \item $\phi_1\to\phi_2$ (implication) as an abbreviation for
        $\lnot\phi_1\lor\phi_2$;
    \item \false as an abbreviation for $\lnot\true$;\footnote{%
            Recall that \true denotes the empty conjunction, which is a CQ and
            thus also a TCQ (see Definition~\ref{def:syntax-cqs}).}
    \item $\Diamond\phi$ (diamond, which should be read as \enquote{eventually}
        or \enquote{some time in the future}) as an abbreviation for
        $\true\Until\phi$;
    \item $\Box\phi$ (box, which should be read as \enquote{always} or
        \enquote{always in the future}) as an abbreviation for
        $\lnot\Diamond\lnot\phi$;
    \item $\Diamondm\phi$ (which should be read as \enquote{once}
        or \enquote{some time in the past}) as an abbreviation for
        $\true\Since\phi$; and
    \item $\Boxm\phi$ (which should be read as \enquote{historically} or
        \enquote{always in the past}) as an abbreviation for
        $\lnot\Diamondm\lnot\phi$.
\end{itemize}

\noindent
We denote the set of conjunctive queries occurring in a TCQ~$\phi$ by
$\CQ(\phi)$.  As before, we first define the semantics for Boolean TCQs, which
is a straightforward extension of the semantics of CQs and propositional
LTL-formulas (see Definition~\ref{def:semantics-ltl}), similar to the semantics
of \SHOQ-LTL from Definition~\ref{def:semantics-shoq-ltl}.
%
Also, the notion of certain answers can then be defined exactly as in the
atemporal case.

\begin{definition}[Semantics of TCQs]\label{def:semantics-tcqs}
    For a Boolean TCQ~$\phi$, a DL-LTL-structure $\Imf=(\Imc_i)_{i\ge 0}$, and
    a time point $i\ge 0$, \emph{validity of~$\phi$ in~\Imf at time~$i$}
    (written $\Imf,i\models\phi$) is defined inductively as follows:
    \[\begin{array}{lcl}
        \Imf,i\models\exists y_1,\dots,y_m.\psi
            &\text{iff}
            &\Imc_i\models\exists y_1,\dots,y_m.\psi\\
        \Imf,i\models\lnot\phi_1
            &\text{iff}
            &\Imf,i\not\models\phi_1,\ \text{i.e.~not}\ \Imf,i\models\phi_1\\
        \Imf,i\models\phi_1\land\phi_2
            &\text{iff}
            &\Imf,i\models\phi_1\ \text{and}\ \Imf,i\models\phi_2\\
        \Imf,i\models\Next\phi_1
            &\text{iff}
            &\Imf,i+1\models\phi_1\\
        \Imf,i\models\Previous\phi_1
            &\text{iff}
            &i>0\ \text{and}\ \Imf,i-1\models\phi_1\\
        \Imf,i\models\phi_1\Until\phi_2
            &\text{iff}
            &\text{there is some $k\geq i$ such that $\Imf,k\models\phi_2$, and}\\
            & &\text{$\Imf,j\models\phi_1$ for every $j$, $i\le j<k$}\\
        \Imf,i\models\phi_1\Since\phi_2
            &\text{iff}
            &\text{there is some $k$, $0\le k\le i$, such that $\Imf,k\models\phi_2$, and}\\
            & &\text{$\Imf,j\models\phi_1$ for every $j$, $k<j\le i$}
    \end{array}\]
    %
    Given a temporal KB $\Kmc=((\Amc_i)_{0\le i\le n},\Tmc,\Rmc)$,
    we say that \Imf is a \emph{model of~$\phi$ w.r.t.~\Kmc} if
    $\Imf\models\Kmc$ and $\Imf,n\models\phi$.  We call $\phi$ \emph{satisfiable
    w.r.t.~\Kmc} if it has a model w.r.t.~\Kmc, and it is \emph{entailed
    by~\Kmc} (written $\Kmc\models\phi$) if every model~\Imf of~\Kmc satisfies
    $\Imf,n\models\phi$.

    Let~\Lmc be a description logic.  The \emph{TCQ-satisfiability problem
    in~\Lmc} is to decide, given a TCQ~$\phi$ and a temporal \Lmc-KB~\Kmc,
    whether $\phi$ is satisfiable w.r.t.~\Kmc.
    %
    Moreover, the \emph{temporalised query-entailment problem in~\Lmc} is the
    problem of deciding, given a TCQ~$\phi$ and a temporal \Lmc-KB~\Kmc, whether
    $\phi$ is entailed by~\Kmc.

    Given a (not necessarily Boolean) TCQ~$\phi$, we call a mapping
    $\amf\colon\FVar(\phi)\to\Ind(\Kmc)$ a \emph{certain answer} to~$\phi$
    w.r.t.~\Kmc if $\Kmc\models\amf(\phi)$, where $\amf(\phi)$ is the Boolean
    TCQ obtained from~$\phi$ by replacing the free variables according to~\amf.
\end{definition}

\noindent
Note that in this definition of a model, the point of reference is not the first
time point~$0$, as in propositional LTL and \SHOQ-LTL, but rather the last time
point~$n$ of a given temporal knowledge base.  Intuitively, this can be seen as
the current time point, at which we have information (e.g.~sensor data) about
the past, but not yet about the future.

As in the atemporal case, one can compute all certain answers to a TCQ~$\phi$
w.r.t.\ a temporal KB~\Kmc by enumerating the (exponentially many) mappings
$\amf\colon\FVar(\phi)\to\Ind(\Kmc)$ and then solving the entailment problem
$\Kmc\models\amf(\phi)$ for each~\amf.  We therefore focus on deciding the
entailment problem for the case where $\phi$ is Boolean.
%
It turns out to be easier to analyse the complexity of deciding the temporalised
query \emph{non-entailment} problem $\Kmc\not\models\phi$.  This problem has the
same complexity as the TCQ-satisfiability problem of~$\phi$ w.r.t.~\Kmc.  In
fact, $\Kmc\not\models\phi$ iff $\lnot\phi$ has a model w.r.t.~\Kmc, and
conversely $\phi$ has a model w.r.t.~\Kmc iff $\Kmc\not\models\lnot\phi$.

Note that, for the data complexity, we have to measure the complexity in the
size of the sequence of ABoxes in the temporal knowledge base, instead of just a
single ABox.
%
In the following, we assume without loss of generality that the query contains
only concept and role names that also occur in the global TBox or the global
RBox.

Obviously, the temporalised query-entailment problem includes as a special case
the entailment of CQs by atemporal knowledge bases, which can be seen as
temporal knowledge bases with a sequence of ABoxes of length~$1$, i.e.~having
$n=0$.  Although models of such temporal knowledge bases are formally infinite
sequences of interpretations (DL-LTL-structures), all but the first
interpretation are irrelevant for the semantics of~CQs.

On the temporal side, the TCQ-satisfiability problem generalises the
satisfiability problem in \ALC-LTL (and \SHQ-LTL) since assertions can be seen
as simple instances of Boolean CQs.
%
Although \ALC-LTL-formulas may additionally contain GCIs, they can equivalently
be expressed by negated CQs (see the proof of
Theorem~\ref{thm:lower-bounds-combined-complexity} for details).
%
On the other hand, TCQs are more expressive than \SHQ-LTL-formulas since CQs
such as $\exists y.r(y,y)$, which says that there is a loop in the model without
naming the individual which has the loop, can clearly not even be expressed in
\ALC.

An assumption on TCQs that was made in~\cite{BaBL-CADE13} is that
all Boolean CQs we encounter are \emph{connected} in the sense that the
variables and individual names are related by roles, as defined e.g.\
in~\cite{Tes-PhD01,RuGl-JAIR10}.

\begin{definition}[Connected Boolean CQs]
    We call a Boolean CQ~$\phi$ \emph{connected} if for all
    $x,y\in\Var(\phi)\cup\Ind(\phi)$, there exists a sequence
    $x_1,\dots,x_n\in\Var(\phi)\cup\Ind(\phi)$ such that $x_1=x$, $x_n=y$, and
    for every $i$, $1\le i<n$, there is an $r\in\NR$ such that either
    $r(x_i,x_{i+1})\in\At(\phi)$ or $r(x_{i+1},x_i)\in\At(\phi)$.

    A collection of Boolean CQs $\phi_1,\dots,\phi_n$ is a \emph{partition}
    of~$\phi$ if
    \begin{itemize}
        \item $\At(\phi)=\At(\phi_1)\cup\dots\cup\At(\phi_n)$;
        \item the sets $\Var(\phi_i)\cup\Ind(\phi_i)$, $1\le i\le n$, are
            pairwise disjoint; and
        \item each $\phi_i$, $1\le i\le n$, is connected.
    \end{itemize}
\end{definition}

\noindent
Similar to~\cite{Tes-PhD01,RuGl-JAIR10}, in~\cite{BaBL-CADE13}, it is assumed
without loss of generality that Boolean TCQs contain only connected CQs.
%
Indeed, if a Boolean TCQ~$\phi$ contains a CQ~$\psi$ that is not connected, we
can replace $\psi$ by the conjunction $\psi_1\land\dots\land\psi_n$, where
$\psi_1,\dots,\psi_n$ is a partition of~$\psi$.  This conjunction is of linear
size in the size of~$\psi$, and the resulting TCQ has exactly the same models
as~$\phi$ since every homomorphism of~$\psi$ into an interpretation~\Imc can be
uniquely represented as a collection of homomorphisms of $\psi_1,\dots,\psi_n$
into~\Imc.
%
Thus, in~\cite{BaBL-CADE13} it was always assumed without loss of generality
that Boolean TCQs contain only connected CQs.  Even though this assumption is
without loss of generality, it turns out that we can show the results in this
chapter also without making this assumption.

Before we investigate the complexity of the temporalised query-entailment
problem, we recall all the assumptions that we have made so far:
\begin{itemize}
    \item Every at-least restriction contains only simple roles, since otherwise
        even the problem of deciding whether a knowledge base is consistent
        would be undecidable~\cite{HoST-IGPL00}.
    \item Every role atom in a query contains only simple roles.  We make this
        restriction since then the combined complexity of atemporal query
        entailment is \ExpTime-complete in all description logics between \ALC
        and \SHQ.
        %
        This enables us to state our complexity results for all these logics at
        the same time.
        %
        Without this restriction, the combined complexity would increase
        whenever transitivity axioms are allowed~\cite{ELO+-IJCAI09}.
    \item The queries contain only concept and role names that also occur
        in the TBox or the RBox.  This restriction is without loss of
        generality.
\end{itemize}

In the next section, we will investigate the complexity of temporalised query
entailment in our temporal query language.  We show how to obtain the results of
Table~\ref{tab:tcq-results}.


\section{The Complexity of Temporalised Query Entailment}\label{sec:complexity-query-entailment}

In this section, we analyse the complexity of the temporalised query-entailment
problem in DLs between \ALC and~\SHQ.
%
As mentioned before, all our complexity results hold for any DL between \ALC
and~\SHQ, i.e.~we show the lower bounds for \ALC and the upper bounds for~\SHQ.

We first take a look at an atemporal special case of the TCQ-satisfiability
problem, which will prove to be useful for analysing the temporalised
query-entailment problem for arbitrary TCQs.  A \emph{CQ-literal} is either a
Boolean CQ or a negated Boolean CQ\@.  Note that a \emph{conjunction of
CQ-literals}~$\phi$ is a special case of a Boolean TCQ\@.  Since $\phi$ does not
contain any temporal operators, for the deciding satisfiability, it suffices to
consider a single interpretation instead of a DL-LTL-structure
$\Imf=(\Imc_i)_{i\ge 0}$.  Extending the notation for UCQs, we often write
$\Imc_i\models\phi$ instead of $\Imf,i\models\phi$ in this case.  Moreover, it
is sufficient to consider temporal KBs with only one ABox, which can be viewed
as \enquote{normal} knowledge bases.
%
The following theorem states the complexity of deciding satisfiability in this
special case.

\begin{theorem}\label{thm:complexity-conjunction-cq-literals}
    Let \Lmc be a DL between \ALC and \SHQ.  Deciding whether a conjunction of
    CQ-literals is satisfiable w.r.t.\ an \Lmc-knowledge base is
    \begin{itemize}
        \item \ExpTime-complete w.r.t.\ combined complexity, and
        \item \NP-complete w.r.t.\ data complexity.
    \end{itemize}
\end{theorem}

\begin{proof}
    The problem of deciding the entailment of concept assertions w.r.t.\
    \ALC-knowledge bases is \ExpTime-hard w.r.t.\ combined
    complexity~\cite{DLhandbook-07} and \coNP-hard w.r.t.\ data
    complexity~\cite{CDL+-KR06,Sch-JIIS93,DLN+-JLC94}.  Note that this
    entailment problem is a special case of the complement of our problem.

    Let now $\Kmc=(\Amc,\Tmc,\Rmc)$ be an \SHQ-knowledge base, and
    let $\zeta$ be a conjunction of CQ-literals.
    %
    To check whether there is an interpretation~\Imc with $\Imc\models\Kmc$ and
    $\Imc\models\zeta$, we reduce this problem to a query non-entailment problem
    of known complexity.  Let
    \[\zeta=\chi_1\land\dots\land\chi_\ell\land\lnot\rho_1\land\dots\land\lnot\rho_m\]
    for Boolean CQs $\chi_1,\dots,\chi_\ell$, $\rho_1,\dots,\rho_m$.  First, we
    instantiate the non-negated CQs $\chi_1,\dots,\chi_\ell$ by omitting the
    existential quantifiers and replacing the variables with fresh individual
    names.  The set $\Amc'$ of all resulting atoms can thus be viewed as an
    additional ABox that restricts the interpretation~\Imc.

    However, we also have to ensure that the UNA is respected for the newly
    introduced individual names.  To achieve this, we employ a trick
    from~\cite{GHL+-JAIR08}, which consists in guessing an equivalence
    relation~$\approx$ on $\Ind(\Amc\cup\Amc')$ (i.e.~the set of individual
    names occurring in~\Amc or~$\Amc'$) that specifies which individual names
    are allowed to be mapped to the same domain element, with the additional
    restriction that each equivalence class can contain at most one element from
    $\Ind(\Amc)$.  For such a relation~$\approx$, we fix a representative for
    each equivalence class such that every class that contains an
    $a\in\Ind(\Amc)$ has $a$ as its representative.
    %
    We denote by $\Amc_\approx$ the ABox resulting from $\Amc'$ by replacing
    each new individual name by the representative of its equivalence class.
    %
    Note that there are exponentially many such equivalence relations, each of
    which is of size polynomial in the size of~$\zeta$.

    We now show that the existence of an interpretation~\Imc with
    $\Imc\models\Kmc$ and $\Imc\models\zeta$ is equivalent to the existence of
    an equivalence relation~$\approx$ as above and an interpretation~$\Imc'$
    with $\Imc'\models(\Amc\cup\Amc_\approx,\Tmc,\Rmc)$ and
    $\Imc'\models\lnot\rho_1\land\dots\land\lnot\rho_m$.

    For the \enquote{if} direction, assume that $\approx$ is an equivalence
    relation on the individual names and $\Imc'$ is a model of~\Amc, \Tmc, \Rmc,
    $\Amc_\approx$, and $\lnot\rho_1\land\dots\land\lnot\rho_m$.  By mapping
    each variable occurring in $\chi_1\land\dots\land\chi_\ell$ to the
    interpretation of the representative of the equivalence class of the
    corresponding fresh individual name, we obtain homomorphisms from~$\chi_i$
    into~$\Imc'$ for each~$i$, $1\le i\le n$.  This shows that $\Imc'$ is also a
    model of~$\zeta$.

    For the \enquote{only if} direction, assume that $\Imc\models\Kmc$ and
    $\Imc\models\zeta$.  Thus, there are homomorphisms from~$\chi_i$ into~\Imc
    for every $i$, $1\le i\le n$.  We define any pair of individual names
    occurring in $\Amc\cup\Amc'$ equivalent w.r.t.~$\approx$ iff they are mapped
    to the same domain element by their respective homomorphisms or~\Imc.  The
    extension of~\Imc that maps each representative of its equivalence class to
    exactly this domain element is obviously a model of~$\Amc_\approx$.  It
    still satisfies \Amc, \Tmc, \Rmc, and
    $\lnot\rho_1\land\dots\land\lnot\rho_m$ since they do not contain the new
    individual names, and thus it is of the required form.

    The above problem is thus equivalent to finding an equivalence
    relation~$\approx$ and an interpretation~\Imc with
    $\Imc\models(\Amc\cup\Amc_\approx,\Tmc,\Rmc)$ and
    $\Imc\not\models\rho$ where $\rho:=\rho_1\lor\dots\lor\rho_m$ is the Boolean
    UCQ that results from negating the conjunction of all negated CQs
    in~$\zeta$.  This is the same as asking whether
    $(\Amc\cup\Amc_\approx,\Tmc,\Rmc)$ does not entail~$\rho$.

    For the combined complexity, we can enumerate all equivalence
    relations~$\approx$ in exponential time, and check the above non-entailment
    for the polynomial-size \SHQ-knowledge base and UCQ resulting from each
    relation~$\approx$, which can be done in \ExpTime~\cite{Lut-IJCAR08}.
    %
    For the data complexity, we can guess a relation~$\approx$ in
    non-deterministic polynomial time, and check the non-entailment in
    \NP~\cite{OrCE-AAAI06}.
    %
    Hence, we obtain the desired complexity results for the satisfiability
    problem of a conjunction of CQ-literals.
\end{proof}

\noindent
In the remainder of this section, we  present several constructions, most
of which use the above theorem, to derive the complexity results shown in
Table~\ref{tab:tcq-results} for temporalised query entailment.
%
As mentioned several times, the results depend on which symbols are allowed to
be rigid.  It is well-known that one can simulate rigid concept names by rigid
role names~\cite{BaGL-ToCL12}, which is why there are only three cases to
consider.

As argued above, we show the lower bounds for the DL \ALC, and the upper bounds
for the DL \SHQ.
%
Hence, in Section~\ref{sec:tcq-lower-bounds}, we consider the lower bounds for
temporalised query entailment in \ALC, whereas in
Section~\ref{sec:tcq-upper-bounds}, we consider the upper bounds for
temporalised query entailment in \SHQ.  However, the upper bounds for the most
complex case of rigid concept names are treated separately in
Sections~\ref{sec:tcq-data-complexity-rigid-concepts}
and~\ref{sec:tcq-combined-complexity-rigid-concepts}.


\subsection{Lower Bounds for Temporalised Query Entailment in~\texorpdfstring{\ALC}{ALC}}\label{sec:tcq-lower-bounds}

In this section, we investigate the lower bounds for temporalised query
entailment of Table~\ref{tab:tcq-results}.

For the combined complexity, we obtain the lower bounds by a simple reduction of
the satisfiability problem of~\ALC-LTL~\cite{BaGL-ToCL12}.

\begin{theorem}\label{thm:lower-bounds-combined-complexity}
    With respect to combined complexity, the temporalised query-entailment
    problem in \ALC is
    \begin{itemize}
        \item \ExpTime-hard if $\NRC=\NRR=\emptyset$;
        \item \coNExpTime-hard if $\NRC\ne\emptyset$ and $\NRR=\emptyset$; and
        \item \TwoExpTime-hard if $\NRC\ne\emptyset$ and $\NRR\ne\emptyset$.
    \end{itemize}
\end{theorem}

\begin{proof}
    As shown in~\cite{BaGL-ToCL12}, the satisfiability problem of \ALC-LTL is
    \ExpTime-complete if $\NRC=\NRR=\emptyset$, \NExpTime-complete if
    $\NRC\ne\emptyset$ and $\NRR=\emptyset$, and \TwoExpTime-complete if
    $\NRC\ne\emptyset$ and $\NRR\ne\emptyset$.

    Let now $\phi$ be an \ALC-LTL-formula, $C_1\sqsubseteq D_1$,~\dots,
    $C_p\sqsubseteq D_p$ be all GCIs occurring in~$\phi$, and $E_1(a_1)$,~\dots,
    $E_m(a_m)$ be all concept assertions occurring in~$\phi$, where
    $E_1,\dots,E_m$ are arbitrary concepts.
    %
    We define $\psi$ to be the Boolean TCQ obtained from~$\phi$ by replacing
    each $C_i\sqsubseteq D_i$ by $\lnot(\exists x.A_i(x))$ and each $E_j$ by
    $B_j$, where $A_i,B_j$ are concept names that do not occur in~$\phi$, for
    every~$i$, $1\le i\le p$, and every~$j$, $1\le j\le m$.
    %
    Moreover, we define
    \[\Tmc:=\{A_i\equiv C_i\sqcap\lnot D_i\mid 1\leq i\leq p\}\cup%
        \{B_j\equiv E_j\mid 1\leq j\leq m\}.\]
    %
    Then $\phi$ is satisfiable iff
    $(\emptyset,\Tmc,\emptyset)\not\models\lnot\psi$.
    We have thus reduced the satisfiability problem in \ALC-LTL to the
    temporalised query \emph{non-entailment} problem in \ALC, which yields the
    claimed lower bounds.
\end{proof}

\noindent
For the data complexity, we obtain the lower bounds directly from
Theorem~\ref{thm:complexity-conjunction-cq-literals}.

\begin{corollary}\label{cor:lower-bounds-data-complexity}
    With respect to data complexity, the temporalised query-entailment problem
    in~\ALC is \coNP-hard.
\end{corollary}

\begin{proof}
    Theorem~\ref{thm:complexity-conjunction-cq-literals} states that deciding
    whether a conjunction of CQ-literals~$\zeta$ is satisfiable w.r.t.\ an
    atemporal \ALC-knowledge base~\Kmc is \NP-complete w.r.t.\ data complexity.
    %
    Since $\zeta$ is a special TCQ and rigid names are irrelevant in the
    atemporal case, we obtain \coNP-hardness w.r.t.\ data complexity for the
    temporalised query-entailment problem in all the settings listed in
    Table~\ref{tab:tcq-results}.
\end{proof}

\noindent
Theorem~\ref{thm:lower-bounds-combined-complexity} and
Corollary~\ref{cor:lower-bounds-data-complexity} yield the lower bounds for
temporalised query entailment as shown in Table~\ref{tab:tcq-results}.

In the following sections, we present the ideas for the upper bounds w.r.t.\
combined complexity and data complexity.  For the former, we can match all lower
bounds that we have from Theorem~\ref{thm:lower-bounds-combined-complexity}.
For the latter, however, we cannot match the lower bound of \coNP in the
case where both concept names and role names may be rigid.
%
While our constructions need to deal with CQs and the additional expressivity
of \SHQ in an appropriate way, the basic ideas are similar to those presented
for \SHOQ-LTL in Chapter~\ref{ch:shoq-ltl}.  However, there are several
differences to the constructions of Chapter~\ref{ch:shoq-ltl}.  Firstly, we have
to deal with conjunctive queries instead of axioms, and secondly, we do not
allow nominals in this chapter.  Thirdly, in the semantics of TCQs (see
Definition~\ref{def:semantics-tcqs}), the point of reference is the last time
point~$n$ and a temporal knowledge base has to be taken into account.  Hence,
although similar, the constructions in the subsequent sections differ from the
ones in Chapter~\ref{ch:shoq-ltl}.


\subsection{Upper Bounds for Temporalised Query Entailment in~\texorpdfstring{\SHQ}{SHQ}}\label{sec:tcq-upper-bounds}

Similar to what was done for \ALC-LTL in Lemma~4.3 in~\cite{BaGL-ToCL12} and
also for \SHOQ-LTL in Lemma~\ref{lem:r-sat-t-sat}, we reduce the
\emph{TCQ-satisfiability problem} in \SHQ to two separate satisfiability
problems.

In the following, let $\Kmc=((\Amc_i)_{0\le i\le n},\Tmc,\Rmc)$ be a temporal
\SHQ-KB, and let $\phi$ be a Boolean TCQ, for which we want to decide whether
$\phi$ has a model w.r.t.~\Kmc.

We again consider the propositional abstraction of~$\phi$.  Its definition is
very similar to propositional abstraction of a \SHOQ-LTL-formula (see
Definition~\ref{def:prop-abs}).

\begin{definition}[Propositional abstraction]
    Let $\phi$ be a TCQ, and let $\Pmc_\phi$ be a finite set of propositional
    variables such that there is a bijection $\psf\colon\CQ(\phi)\to\Pmc_\phi$.
    \begin{enumerate}
        \item The propositional LTL-formula~$\phi^\psf$ is obtained from~$\phi$
            by replacing every occurrence of a CQ~$\psi$ in~$\phi$ by its
            \psf-image $\psf(\psi)$.  We call $\phi^\psf$ the
            \emph{propositional abstraction of~$\phi$ w.r.t.~\psf}.
        \item Given a DL-LTL-structure $\Imf=(\Imc_i)_{i\ge 0}$, its
            \emph{propositional abstraction w.r.t.~\psf} is the propositional
            LTL-structure $\Imf^\psf=(w_i)_{i\ge 0}$ with
            \[w_i:=\bigl\{\psf(\psi)\mid\psi\in\CQ(\phi)\ \text{and} \
                \Imc_i\models\alpha\bigr\}\]
            for every $i\ge 0$.
    \end{enumerate}
\end{definition}

\noindent
In the following, we assume that $\psf\colon\CQ(\phi)\to\Pmc_\phi$ is a
bijection.%
\footnote{As for \SHOQ-LTL-formulas, it is obvious that such a set $\Pmc_\phi$
    and such a bijection~\psf exists for every TCQ\@.}
%
Again for simplicity, for a sub-TCQ~$\psi$ of~$\phi$, we denote by~$\psi^\psf$
the propositional abstraction of~$\psi$ w.r.t.\ the restriction of~\psf
to~$\CQ(\psi)$.
%
The propositional abstraction~$\phi^\psf$ of~$\phi$ w.r.t.~\psf is a
propositional LTL-formula that allows us to analyse the temporal structure
of~$\phi$ separately from the CQ-component.  The following lemma is very similar
to Lemma~\ref{lem:prop-abs}, and its proof is analogous.

\begin{lemma}\label{lem:tcq-prop-abs}
    Let \Imf be a DL-LTL-structure with $\Imf\models\Kmc$.  Then, \Imf is a
    model of~$\phi$ w.r.t.~\Kmc iff $\Imf^\psf$ is a model of~$\phi^\psf$.
\end{lemma}

\begin{proof}
    Let $\Imf=(\Imc_i)_{i\ge 0}$ be a DL-LTL-structure with $\Imf\models\Kmc$,
    and $\Imf^\psf=(w_i)_{i\ge 0}$ its propositional abstraction w.r.t.~\psf.
    Moreover, let $i\ge 0$.  We prove this lemma by showing that
    $\Imf,i\models\phi$ iff $\Imf^\psf,i\models\phi^\psf$ by induction of the
    structure of~$\phi$.

    For the base case, let $\phi$ be an CQ\@.  Then, we have
    $\Imf,i\models\phi$ \emph{iff} $\Imc_i\models\phi$ \emph{iff}
    $\psf(\phi)\in w_i$ \emph{iff} $w_i\models\phi^\psf$ \emph{iff}
    $\Imf^\psf,i\models\phi^\psf$.

    For the induction steps, we obtain the claim by using the definition of the
    semantics and the induction hypothesis.  This can be done exactly as we did
    in the proof of Lemma~\ref{lem:prop-abs}.
\end{proof}

\noindent
Again, the \enquote{only if} direction of this lemma yields that satisfiability
of~$\phi$ w.r.t.~\Kmc implies satisfiability of~$\phi^\psf$.  Note that, however, the
\enquote{if} direction does not yield the converse of this implication as
already argued for Lemma~\ref{lem:prop-abs}.

We again consider
a set $\Wmc\subseteq 2^{\Pmc_\phi}$, which intuitively
specifies the worlds that are allowed to occur in an LTL-structure
satisfying~$\phi^\psf$.  To express this restriction, we define the
propositional LTL-formula
\[\phi^\psf_\Wmc:=\phi^\psf\land%
    \Boxm\Box\Biggl(\bigvee_{X\in\Wmc}\Biggl(\bigwedge_{p\in X}p\land%
    \bigwedge_{p\in\Pmc_\phi\setminus X}\lnot p\Biggr)\Biggr).\]
%
Note that a propositional LTL-formula $\Boxm\Box\psi$ is satisfied iff $\psi$
holds at every point in time.%
\footnote{Note also that the propositional LTL-formula~$\phi^\psf_\Wmc$ for a
    \SHOQ-LTL-formula~$\phi$ as defined in Section~\ref{sec:complexity-shoq-ltl}
    does not use the $\Boxm$-operator.  We need this here due to the definition of
    the semantics where the point of reference is time point~$n$ rather
    than~$0$.}
%
The next lemma formalises the immediate connection between~$\phi$
and~$\phi^\psf_\Wmc$.

\begin{lemma}\label{lem:tcq-prop-abs-wmc}
    If $\phi$ is satisfiable w.r.t.~\Kmc, then there is a set
    $\Wmc\subseteq 2^{\Pmc_\phi}$ and a propositional LTL-structure~\Wmf such
    that $\phi^\psf_\Wmc$ is valid in~\Wmf at time~$n$.
\end{lemma}

\begin{proof}
    Let \Imf be a DL-LTL-structure that is a model of~$\phi$ w.r.t.~\Kmc, and
    let $\Imf^\psf=(w_i)_{i\ge 0}$ be its propositional abstraction w.r.t.~\psf.
    We consider the finite set $\Wmc:=\{w_i\mid i\ge 0\}$ induced by~\Imf.
    %
    Using Lemma~\ref{lem:tcq-prop-abs}, it is easy to verify that the fact that
    $\Imf\models\Kmc$ and $\phi$ is valid in~\Imf at time~$n$ implies that
    $\phi^\psf_\Wmc$ is valid in~$\Imf^\psf$ at time~$n$.
\end{proof}

\noindent
As argued above, guessing a set~\Wmc and then checking whether there is a
propositional LTL-structure~\Wmf such that the induced propositional
LTL-formula~$\phi^\psf_\Wmc$ is valid in~\Wmf at time~$n$ is not sufficient for
checking whether~$\phi$ has a model w.r.t.~\Kmc.  We must also check whether
\Wmc can indeed be induced by some DL-LTL-structure that is a model of~\Kmc.
For that, we extend the notion of r-satisfiability from
Definition~\ref{def:r-sat}.

\begin{definition}[R-satisfiability]\label{def:tcq-r-sat}
    Let $\Wmc=\{X_1,\dots,X_k\}\subseteq 2^{\Pmc_\phi}$, and let $\iota$ be a
    mapping from $\{0,\dots,n\}$ into $\{1,\dots,k\}$.  We call \Wmc
    \emph{r-satisfiable w.r.t.~$\iota$ and~\Kmc} if there exist interpretations
    $\Jmc_1=(\Delta,\cdot^{\Jmc_1})$,~\dots, $\Jmc_k=(\Delta,\cdot^{\Jmc_k})$,
    and $\Imc_0=(\Delta,\cdot^{\Imc_0})$,~\dots,
    $\Imc_n=(\Delta,\cdot^{\Imc_n})$ such that
    \begin{itemize}
        \item $a^{\Jmc_i}=a^{\Jmc_j}=a^{\Imc_\ell}$ holds for every $a\in\NI$
            and all $i,j,\ell$, $1\le i<j\le k$ and $0\le\ell\le n$;
        \item $A^{\Jmc_i}=A^{\Jmc_j}=A^{\Imc_\ell}$ holds for every $A\in\NRC$
            and all $i,j,\ell$, $1\le i<j\le k$ and $0\le\ell\le n$;
        \item $r^{\Jmc_i}=r^{\Jmc_j}=r^{\Imc_\ell}$ holds for every $r\in\NRR$
            and all $i,j,\ell$, $1\le i<j\le k$ and $0\le\ell\le n$; and
        \item every $\Jmc_i$, $1\le i\le k$, and every $\Imc_j$, $0\le j\le n$,
            is a model of~\Tmc and~\Rmc;
        \item every $\Jmc_i$, $1\le i\le k$, is a model of the conjunction of
            CQ-literals
            \[\zeta_{X_i}:=\bigwedge_{p\in X_i}\psf^{-1}(p)\land%
                \bigwedge_{p\in\Pmc_\phi\setminus X_i}\lnot\psf^{-1}(p);\ \text{and}\]
        \item every $\Imc_i$, $0\le i\le n$, is a model of~$\Amc_i$ and
            $\zeta_{X_{\iota(i)}}$.
    \end{itemize}
\end{definition}

\noindent
The intuition underlying this definition is the following.  The existence of the
interpretations~$\Jmc_i$, $1\le i\le k$, ensures that the conjunction
$\zeta_{X_i}$ of the CQ-literals induced by~$X_i$ is consistent.  In fact, a
set~\Wmc containing a set~$X_i$ for which this does not hold cannot be induced
by a DL-LTL-structure.  The interpretations~$\Imc_i$, $0\le i\le n$, constitute
the first $n+1$ interpretations in such a DL-LTL-structure.  In addition to
inducing a set~$X_{\iota(i)}\in\Wmc$ and thus satisfying the corresponding
conjunction~$\zeta_{X_{\iota(i)}}$, the interpretation~$\Imc_i$ must also
satisfy the ABox~$\Amc_i$.  Moreover, we ensure that the interpretations share
the same domain, respect rigid names, and satisfy the TBox~\Tmc and the
RBox~\Rmc.
%
Note that we can use Theorem~\ref{thm:complexity-conjunction-cq-literals} to
check whether interpretations satisfying the last three conditions of
Definition~\ref{def:tcq-r-sat} exist.  As we will see below,
the difficulty lies in ensuring that the interpretations share the same domain
and respect rigid names.

Satisfaction of the temporal structure of~$\phi$ by a DL-LTL-structure built
this way is ensured by testing $\phi^\psf_\Wmc$ for satisfiability w.r.t.\ a
side condition that ensures that the first $n$ worlds are those chosen
by~$\iota$.  For that, we extend the notion of t-satisfiability from
Definition~\ref{def:t-sat}.

\begin{definition}[T-satisfiability]\label{def:tcq-t-sat}
    Let $\Wmc=\{X_1,\dots,X_k\}\subseteq 2^{\Pmc_\phi}$, and let $\iota$ be a
    mapping from $\{0,\dots,n\}$ into $\{1,\dots,k\}$.  We call the
    propositional LTL-formula~$\phi^\psf$ \emph{t-satisfiable w.r.t.~\Wmc
    and~$\iota$} if there exists a propositional LTL-structure
    $\Wmf=(w_i)_{i\ge 0}$ such that
    \begin{itemize}
        \item $\Wmf,n\models\phi^\psf_\Wmc$, and
        \item $w_i=X_{\iota(i)}$ for every $i$, $0\le i\le n$.
    \end{itemize}
\end{definition}

\noindent
The next lemma shows that these two satisfiability problems, namely,
t-satisfiability and r-satisfiability, can be combined to decidable the
TCQ-satisfiability problem in \SHQ.  The proof of the lemma is very similar to
the proofs of Lemmas~\ref{lem:ltl-structure-r-sat} and~\ref{lem:r-sat-t-sat}.

\begin{lemma}\label{lem:tcq-r-sat-t-sat}
    The TCQ~$\phi$ is satisfiable w.r.t.\ the temporal knowledge base~\Kmc iff
    there is a set $\Wmc=\{X_1,\dots,X_k\}\subseteq 2^{\Pmc_\phi}$ and a mapping
    $\iota\colon\{0,\dots,n\}\to\{1,\dots,k\}$ such that
    \begin{itemize}
        \item \Wmc is r-satisfiable w.r.t.~$\iota$ and~\Kmc, and
        \item $\phi^\psf$ is t-satisfiable w.r.t.~\Wmc and~$\iota$.
    \end{itemize}
\end{lemma}

\begin{proof}
    For the \enquote{only if} direction, assume that there is a DL-LTL-structure
    $\Imf=(\Imc_i)_{i\ge 0}$ that is a model of~$\phi$ w.r.t.~\Kmc,
    i.e.~$\Imf\models\Kmc$ and $\Imf,n\models\phi$.
    %
    Let $\Imf^\psf=(w_i)_{i\ge 0}$ be the propositional abstraction of~\Imf
    w.r.t.~\psf.
    %
    Recall that we have already seen in the proof of
    Lemma~\ref{lem:tcq-prop-abs-wmc} that \Imf induces a finite set
    $\Wmc:=\{w_i\mid i\ge 0\}=\{X_1,\dots,X_k\}\subseteq 2^{\Pmc_\phi}$ such
    that $\phi^\psf_\Wmc$ is valid in~$\Imf^\psf$ at time~$n$.
    %
    Moreover, we have that for every $i\ge 0$, there is an index
    $\nu_i\in\{1,\dots,k\}$ such that $\Imc_i$ induces the set~$X_{\nu_i}$,
    i.e.\
    \[X_{\nu_i}=\bigl\{\psf(\psi)\mid\psi\in\CQ(\phi)\ \text{and}\
        \Imc_i\models\psi\bigr\},\]
    and, conversely, for every $\nu\in\{1,\dots,k\}$, there is an index $i\ge 0$
    such that $\nu=\nu_i$.
    %
    We define the mapping $\iota$ as follows: $\iota(i)=\nu_i$ for every~$i$,
    $0\le i\le n$.
    %
    By definition of~$\iota$, $X_{\nu_i}$ and~$\Imf^\psf$, we also have
    $w_i=X_{\iota(i)}$ for every~$i$, $0\le i\le n$.
    %
    Thus, $\phi^\psf$ is t-satisfiable w.r.t.~\Wmc and~$\iota$.
    %
    For every $i$, $1\le i\le k$, the interpretation~$\Jmc_i$ is obtained as
    follows.  Let $\ell_1,\dots,\ell_k$ be such that $\nu_{\ell_1}=1$,~\dots,
    $\nu_{\ell_k}=k$.  Now, if we set $\Jmc_i:=\Imc_{\ell_i}$, then we clearly
    have that $\Jmc_i$ is a model of~$\zeta_{X_i}$.  It is now easy to see that
    the interpretations $\Jmc_1,\dots,\Jmc_k$, and $\Imc_0,\dots,\Imc_n$ satisfy
    the conditions for r-satisfiability of~\Wmc w.r.t.~$\iota$ and~\Kmc.

    For the \enquote{if} direction, assume that there is a set
    $\Wmc=\{X_1,\dots,X_k\}\subseteq 2^{\Pmc_\phi}$ and a mapping
    $\iota\colon\{0,\dots,n\}\to\{1,\dots,k\}$ such that \Wmc is r-satisfiable
    w.r.t.~$\iota$ and~\Kmc and $\phi^\psf$ is t-satisfiable w.r.t.~\Wmc
    and~$\iota$.  Hence, there is a propositional LTL-structure
    $\Wmf=(w_i)_{i\ge 0}$ such that $\phi^\psf_\Wmc$ is valid in~\Wmf at
    time~$n$ and $w_i=X_{\iota(i)}$ for every~$i$, $0\le i\le n$, and there are
    interpretations $\Jmc_1,\dots,\Jmc_k$, and $\Imc_0,\dots,\Imc_n$ such that
    the conditions in Definition~\ref{def:tcq-r-sat} are satisfied.

    By the definition of $\phi^\psf_\Wmc$, we have that for
    every world~$w_i$, there is exactly one index $\nu_i\in\{1,\dots,k\}$ such
    that $w_i$ satisfies
    \[\bigwedge_{p\in X_{\nu_i}}p\land\bigwedge_{p\in\Pmc_\phi\setminus X_{\nu_i}}\lnot p.\]
    Since every~$w_i$, $0\le i\le n$, satisfies exactly the propositional
    variables of~$X_{\iota(i)}$, we have $\iota(i)=\nu_i$.
    %
    We can now define a DL-LTL-structure $\Imf:=(\Imc_i)_{i\ge 0}$ as follows.
    We set $\Imc_i:=\Jmc_{\nu_i}$ for $i>n$.  By Definition~\ref{def:tcq-r-sat},
    each~$\Imc_i$ is a model of $\zeta_{X_{\nu_i}}$, i.e.~it satisfies exactly
    the CQs specified by the propositional variables in~$X_{\nu_i}$.  This
    yields since $\Wmf,n\models\phi^\psf_\Wmc$, that $\Imf,n\models\phi$.  It
    also follows directly from Definition~\ref{def:tcq-r-sat} that
    $\Imf\models\Kmc$.  Hence, we have that $\phi$ is satisfiable w.r.t.~\Kmc.
\end{proof}

\noindent
To obtain a decision procedure for the TCQ-satisfiability problem in \SHQ, we
have to non-deterministically guess or construct the set~\Wmc and the
mapping~$\iota$, and then check the two conditions of
Lemma~\ref{lem:tcq-r-sat-t-sat}.
%
Depending on which symbols are allowed to be rigid, we use different
constructions to achieve that.
%
First, we focus on deciding t-satisfiability w.r.t.\ a given set~\Wmc and a
given mapping~$\iota$.


\subsubsection{Deciding T-satisfiability}

From now on, let $\Wmc=\{X_1,\dots,X_k\}\subseteq 2^{\Pmc_\phi}$, and let
$\iota\colon\{0,\dots,n\}\to\{1,\dots,k\}$ be a mapping that specifies a
set~$X_{\iota(i)}$ for each of the ABoxes~$\Amc_i$, $0\le i\le n$.
%
We proceed similar to the proof of Lemma~\ref{lem:t-sat}, where we have shown
that we can decide t-satisfiability of the propositional abstraction of a
\SHOQ-LTL-formula w.r.t.\ a set of worlds~\Wmc in time exponential in the size
of the propositional abstraction and linear in the size of~\Wmc.  For that, we
constructed a Büchi-automaton for the propositional abstraction and removed all
transitions that are labelled with a letter that is not contained in~\Wmc.
However, the difference here is that we have to take care of the second
condition of t-satisfiability (see Definition~\ref{def:tcq-t-sat}), and that we
have to ensure that $\phi^\psf$ is satisfied as time point~$n$ rather than~$0$.

We check this by using the same idea that we employed in
Section~\ref{sec:aut-for-ltl} when we constructed a Büchi-automaton that ensures
that a propositional LTL-formula is satisfied at a given time point.
%
We attach a counter from $\{0,\dots,n+1\}$ to the states of the
Büchi-automaton.  Transitions where the counter is $i<n+1$ check if the current
world corresponds to $X_{\iota(i)}$ and increase the counter by~$1$.

In the following, let $\Nmc=(Q,\Sigma_{\Pmc_\phi},\Delta,Q_0,F)$ be a
Büchi-automaton such that for every $\omega$-word
$w=w_0w_1w_2\dotso\in\Sigma_{\Pmc_\phi}^\omega$, we have that
$w\in L_\omega(\Nmc)$ iff $\phi^\psf$ is valid in the propositional
LTL-structure $\Wmf=(w_i)_{i\ge 0}$ at time~$n$.  We have shown in
Section~\ref{sec:aut-for-ltl} how such a Büchi-automaton can be constructed.
Moreover, we have shown in Lemma~\ref{lem:ba-phi-n-exp} that we can construct
such a Büchi-automaton in time exponential in the size of~$\phi^\psf$ and
polynomial in~$n$.

We define the Büchi-automaton $\Nmc'=(Q',\Sigma_{\Pmc_\phi},\Delta',Q_0',F')$ as
follows:
\begin{itemize}
    \item $Q':=Q\times\{0,\dots,n+1\}$;
    \item $((q,\ell),\ \sigma,\ (q',\ell'))\ \in\ \Delta'$ iff
        \begin{itemize}
            \item $(q,\sigma,q')\in\Delta$,
            \item $\sigma\in\Wmc$,
            \item $\ell\le n$ implies $\sigma=X_{\iota(\ell)}$, and
            \item $\ell'=\begin{cases}
                    \ell+1 &\text{if $\ell\le n$, and}\\
                    \ell   &\text{otherwise;}
                \end{cases}$
        \end{itemize}
    \item $Q_0':=Q_0\times\{0\}$; and
    \item $F':=F\times\{n+1\}$.
\end{itemize}
%
The next lemma shows that this Büchi-automaton is correct.

\begin{lemma}
    For every $\omega$-word $w=w_0w_1w_2\dotso\in\Sigma_{\Pmc_\phi}^\omega$, we
    have $w\in L_\omega(\Nmc')$ iff $\phi^\psf_\Wmc$ is valid in the
    propositional LTL-structure $\Wmf=(w_i)_{i\ge 0}$ at time~$n$, and
    $w_i=X_{\iota(i)}$ for every~$i$, $0\le i\le n$.
\end{lemma}

\begin{proof}
    For the \enquote{only if} direction, assume that $\phi^\psf_\Wmc$ is valid
    in the propositional LTL-structure $\Wmf=(w_i)_{i\ge 0}$ at time~$n$, and
    that we have $w_i=X_{\iota(i)}$ for every~$i$, $0\le i\le n$.
    %
    Obviously, we have also $\Wmf,n\models\phi^\psf$ since $\phi^\psf$ is a
    conjunct of~$\phi^\psf_\Wmc$.  This yields that
    $w=w_0w_1w_2\dotso\in L_\omega(\Nmc)$.  Thus, there is a run $S_0S_1S_2\dots$
    of~\Nmc on~$w$.  Then,
    \[(S_0,0)(S_1,1)\dots(S_n,n)(S_{n+1},n+1)(S_{n+2},n+1)\dots\]
    is a accepting run of~$\Nmc'$ on~$w$ due to the following reasons:
    \begin{itemize}
        \item Obviously, we have $(S_i,\ell)\in Q'$ for every $i\ge 0$ and
            every~$\ell$, $0\le k\le n+1$.
        \item We have for every~$i$, $0\le i\le n$, that
            \[((S_i,i),\ w_i,\ (S_{i+1},i+1))\ \in\ \Delta',\]
            and for every $i\ge n+1$ that
            \[((S_i,n+1),\ w_i,\ (S_{i+1},n+1))\ \in\ \Delta'\]
            since:
            \begin{itemize}
                \item $(S_i,w_i,S_{i+1})\in\Delta$ by construction;
                \item $w_i\in\Wmc$ since \Wmf is a model of~$\phi^\psf_\Wmc$;
                \item $i\le n$ implies $w_i=X_{\iota(i)}$ by construction; and
                \item the condition for incrementing the second component of a
                    state (until $n+1$ is reached) is obviously also satisfied.
            \end{itemize}
        \item Since $S_0\in Q_0$, we have $(S_0,0)\in Q_0'$.
        \item Since $S_0S_1S_2\dots$ is an accepting run of~\Nmc on~w, there are
            infinitely many $j\ge 0$ such that $S_j\in F$.  The definition
            of~$F'$ yields now that the above run is accepting.
    \end{itemize}

    For the \enquote{if} direction, assume that
    $w=w_0w_1w_2\dotso\in L_\omega(\Nmc')$, i.e.~there is an accepting run
    \[(S_0,0)(S_1,1)\dots(S_n,n)(S_{n+1},n+1)(S_{n+2},n+1)\dots\]
    of~$\Nmc'$ on~$w$.

    By the definition of~$\Delta'$, we have $w_i=X_{\iota(i)}$ for every~$i$,
    $0\le i\le n$.
    %
    To show that $\phi^\psf_\Wmc$ is valid in $\Wmf:=(w_i)_{i\ge 0}$ at time~$n$
    observe that we have $w_i\in\Wmc$ for every $i\ge 0$ again by the definition
    of~$\Delta'$.  Thus, the conjunct
    \[\Boxm\Box\Biggl(\bigvee_{X\in\Wmc}\Biggl(\bigwedge_{p\in X}p\land%
        \bigwedge_{p\in\Pmc_\phi\setminus X}\lnot p\Biggr)\Biggr)\]
    of~$\phi^\psf_\Wmc$ is clearly satisfied by~\Wmf (at any time point).

    Moreover, we have that $S_0S_1S_2\dots$ is an accepting run of~\Nmc on~$w$
    by the definition of~$Q_0'$, $\Delta'$, and~$F'$.  Thus, $\phi^\psf$ is
    valid in~\Wmf at time~$n$.
    %
    Hence, we obtain that $\phi^\psf_\Wmc$ is valid in~\Wmf at time~$n$.
\end{proof}

\noindent
This lemma implies that $L_\omega(\Nmc')\ne\emptyset$ iff $\phi^\psf$ is
t-satisfiable w.r.t.~\Wmc and~$\iota$.  We can thus decide the latter problem by
checking~$\Nmc'$ for emptiness, which yields the following complexity result.

\begin{lemma}\label{lem:tcq-t-sat}
    Deciding whether $\phi^\psf$ is t-satisfiable w.r.t.~\Wmc and~$\iota$ can be
    done in time exponential in the size of~$\phi^\psf$, linear in the size
    of~\Wmc, and polynomial in~$n$.
\end{lemma}

\begin{proof}
    As mentioned above the Büchi-automaton~\Nmc can be constructed in time
    exponential in the size of~$\phi^\psf$ and polynomial in~$n$ (see
    Lemma~\ref{lem:ba-phi-n-exp}).
    %
    Note that the Büchi-automaton~$\Nmc'$ can be constructed in time linear
    in the size of~\Nmc, the size of~\Wmc, and~$n$, and thus the
    size of~$\Nmc'$ is linear in the size of~\Nmc and~$n$, and thus
    exponential in the size of~$\phi^\psf$ and polynomial in~$n$.
    %
    Since the emptiness problem for Büchi-automata can be solved in polynomial
    time~\cite{VaWo-IC94}, this yields that t-satisfiability of~$\phi^\psf$
    w.r.t.~\Wmc and~$\iota$ can be decided in time exponential in the size
    of~$\phi^\psf$, linear in the size of~\Wmc, and polynomial in~$n$.
\end{proof}

\noindent
However, due to Lemma~\ref{lem:tcq-r-sat-t-sat}, the complexity of the
TCQ-satisfiability problem also depend on the complexity of deciding whether
\Wmc is r-satisfiable w.r.t.~$\iota$ and~\Kmc.  This depends again on the fact
whether there are concept or role names that are allowed to be rigid.

In the following sections, we establish some results as to this complexity in
the cases without rigid names, and with rigid concept and role names.
%
The case without rigid role names, but with
rigid concept names, is considered in
Section~\ref{sec:tcq-data-complexity-rigid-concepts} for data
complexity and in Section~\ref{sec:tcq-combined-complexity-rigid-concepts} for
combined complexity.


\subsubsection{The Case without Rigid Names}

In this section, we consider the case where neither concept names nor role names
are allowed to be rigid, i.e.~$\NRC=\NRR=\emptyset$.  We establish the following
complexity results.

\begin{theorem}\label{thm:upper-bounds-no-rigid-names}
    If $\NRC=\NRR=\emptyset$, the temporalised query-entailment problem in \SHQ
    is
    \begin{itemize}
        \item in \ExpTime w.r.t.\ combined complexity and
        \item in \coNP w.r.t.\ data complexity.
    \end{itemize}
\end{theorem}

\begin{proof}
    Let $\phi$ be a Boolean TCQ, and let
    $\Kmc=((\Amc_i)_{0\le i\le n},\Tmc,\Rmc)$ be a temporal \SHQ-knowledge base.
    As argued above, the temporalised query \emph{non-entailment} problem has
    the same complexity as the TCQ-satisfiability problem.
    %
    We can decide whether $\phi$ is satisfiable w.r.t.~\Kmc using
    Lemma~\ref{lem:tcq-r-sat-t-sat}.  For that, let
    $\psf\colon\CQ(\phi)\to\Pmc_\phi$ be a bijection.

    For combined complexity, we proceed as follows.  We define
    \[\Wmc:=\{X\in 2^{\Pmc_\phi}\mid%
        \text{$\zeta_X$ is satisfiable w.r.t.~$(\emptyset,\Tmc,\Rmc)$}\},\]
    where $\zeta_X$ is defined as in Definition~\ref{def:tcq-r-sat}.
    %
    Note that~$\Wmc=\{X_1,\dots,X_k\}$ can be constructed in time exponential in
    the size of~$\phi$, \Tmc, and~\Rmc.  Indeed, there are exponentially many
    sets $X\in 2^{\Pmc_\phi}$, but each $\zeta_X$ can be constructed in time
    polynomial in the size of~$\phi$, and is thus of size polynomial in the size
    of~$\phi$.  By Theorem~\ref{thm:complexity-conjunction-cq-literals}, the
    problem of checking whether the conjunction of CQ-literals~$\zeta_X$ is
    satisfiable w.r.t.~$(\emptyset,\Tmc,\Rmc)$ is \ExpTime-complete.  Thus, we
    obtain the set~\Wmc after exponentially many \ExpTime-tests, i.e.~in time
    exponential in the size of~$\phi$, \Tmc, and~\Rmc.
    %
    Moreover, we enumerate all possible mappings
    $\iota\colon\{0,\dots,n\}\to\{1,\dots,k\}$ in time exponential in the size
    of~$\phi$ and~\Kmc.  For each such~$\iota$ and every~$i$, $0\le i\le n$, we
    check whether the conjunction of CQ-literals~$\zeta_{X_{\iota(i)}}$ is
    satisfiable w.r.t.~$(\Amc_i,\Tmc,\Rmc)$ in time exponential in the size
    of~$\phi$ and~\Kmc (using again
    Theorem~\ref{thm:complexity-conjunction-cq-literals}).
    %
    After that, we check for ever mapping~$\iota$ that passes this test, whether
    $\phi^\psf$ is t-satisfiable w.r.t.~\Wmc and~$\iota$, which, by
    Lemma~\ref{lem:tcq-t-sat}, can be done in time exponential in the size
    of~$\phi^\psf$ (and thus in time exponential in the size of~$\phi$), linear
    in the size of~\Wmc, and polynomial in~$n$.

    We now show that for every mapping~$\iota$ that passes the above tests, we
    have that \Wmc is r-satisfiable w.r.t.~$\iota$ and~\Kmc.
    %
    Since every~$\zeta_{X_i}$, $1\le i\le k$ is satisfiable
    w.r.t.~$(\emptyset,\Tmc,\Rmc)$, there are models $\Jmc_1,\dots,\Jmc_k$ such
    that every~$\Jmc_i$, $1\le i\le k$, is a model of~$\zeta_{X_i}$
    w.r.t.~$(\emptyset,\Tmc,\Rmc)$.  Moreover, since
    every~$\zeta_{X_{\iota(j)}}$, $0\le j\le n$, is satisfiable
    w.r.t.~$(\Amc_j,\Tmc,\Rmc)$, there are models
    $\Imc_0,\dots,\Imc_n$ such that every~$\Imc_i$, $0\le j\le n$, is a model
    of~$\zeta_{X_{\iota(j)}}$ w.r.t.~$(\emptyset,\Tmc,\Rmc)$.  We can assume
    w.l.o.g.\ that all of these models have the same domain since we can assume
    w.l.o.g.\ that their domains are countably infinite due to to the
    Löwenheim-Skolem theorem~\cite{Loe-MA15,Sko-VS20}.  Furthermore, we can
    assume w.l.o.g.\ that all individual names are interpreted by the same
    domain elements in all models.  Since $\NRC=\NRR=\emptyset$, this yields
    that \Wmc is r-satisfiable w.r.t.~$\iota$ and~\Kmc.

    Thus, we have by Lemma~\ref{lem:tcq-r-sat-t-sat} that if such a
    mapping~$\iota$ exists, then $\phi$ is satisfiable w.r.t.~\Kmc.
    %
    Conversely, again by Lemma~\ref{lem:tcq-r-sat-t-sat}, we have that if $\phi$ is
    satisfiable w.r.t.~\Kmc, then there is a set
    $\Wmc'=\{X_1',\dots,X_{k'}'\}\subseteq 2^{\Pmc_\phi}$ and a
    mapping~$\iota'\colon\{0,\dots,n\}\to\{1,\dots,k'\}$ such that $\Wmc'$ is
    r-satisfiable w.r.t.~$\iota'$ and~\Kmc, and $\phi^\psf$ is t-satisfiable
    w.r.t.~$\Wmc'$ and~$\iota'$.  The definition of~\Wmc above yields that
    $\Wmc'\subseteq\Wmc$, and thus $k'\le k$.  We define the mapping
    $\iota\colon\{0,\dots,n\}\to\{1,\dots,k\}$ such that
    $X_{\iota(i)}=X'_{\iota'(i)}$ for every $i$, $0\le i\le n$.  Hence, we have
    that \Wmc is r-satisfiable w.r.t.~$\iota$ and~\Kmc.  Moreover, it is easy to
    see that the t-satisfiability of~$\phi^\psf$ w.r.t.~$\Wmc'$ and~$\iota'$
    implies that $\phi^\psf$ is t-satisfiable w.r.t.~\Wmc and~$\iota$.

    Hence, we can check whether $\phi$ is satisfiable w.r.t.~\Kmc using the
    above decision procedure, which shows that the TCQ-satisfiability problem
    in~\SHQ is in \ExpTime w.r.t.\ combined complexity.  Since \ExpTime is
    closed under complement, we obtain that the temporalised query-entailment
    problem is in \ExpTime w.r.t.\ combined complexity.

    For data complexity, we non-deterministically guess a set
    $\Wmc=\{X_1,\dots,X_k\}\subseteq 2^{\Pmc_\phi}$ and a mapping
    $\iota\colon\{0,\dots,n\}\to\{1,\dots,k\}$.  Note that since $\Pmc_\phi$
    does not depend on the ABoxes in~\Kmc, we have that \Wmc is of constant size
    w.r.t.\ the ABoxes and~$\iota$ is of size linear in~$n$.  Thus, we can
    perform these guesses in time polynomial in the size of the ABoxes.
    %
    Moreover, for checking whether \Wmc is r-satisfiable w.r.t.~$\iota$
    and~\Kmc, all conditions of Definition~\ref{def:tcq-r-sat} can be checked in
    non-deterministic polynomial time w.r.t.\ data complexity using
    Theorem~\ref{thm:complexity-conjunction-cq-literals}.
    %
    By Lemma~\ref{lem:tcq-t-sat}, deciding whether $\phi^\psf$ is t-satisfiable
    w.r.t.~\Wmc and~$\iota$ can be done in time polynomial in~$n$ w.r.t.\ data
    complexity.
    %
    Then, by Lemma~\ref{lem:tcq-r-sat-t-sat}, we obtain that the
    TCQ-satisfiability problem in~\SHQ is in \NP w.r.t.\ data complexity.  Thus,
    we obtain that the temporalised query-entailment problem in \SHQ is in \coNP
    w.r.t.\ data complexity.
\end{proof}

\noindent
Together with Theorem~\ref{thm:lower-bounds-combined-complexity} and
Corollary~\ref{cor:lower-bounds-data-complexity}, we obtain that the
temporalised query-entailment problem in \SHQ is \ExpTime-complete w.r.t.\
combined complexity and \coNP-complete w.r.t.\ data complexity if neither
concept nor role names are allowed to be rigid.


\subsubsection{The Case of Rigid Concept and Role Names}

In this section, we consider the case where both concept and role names may be
rigid, i.e.~$\NRC\ne\emptyset$ and $\NRR\ne\emptyset$.

Let us assume in the following that a set
$\Wmc=\{X_1,\dots,X_k\}\subseteq 2^{\Pmc_\phi}$, and a mapping
$\iota\colon\{0,\dots,n\}\to\{1,\dots,k\}$ is given.
%
Note that if concept and role names may be rigid, the satisfiability checks
employed in the previous section (see the proof of
Theorem~\ref{thm:upper-bounds-no-rigid-names}) for deciding whether \Wmc is
r-satisfiable w.r.t.~$\iota$ and~\Kmc are no longer independent from each other.
To make sure that the models respect the rigid names, we use a renaming
technique similar to the one we used in Section~\ref{sec:sat-rigid}, which was
adopted from~\cite{BaGL-ToCL12}.  The difference here is that we have to
introduce more copies of the flexible symbols.

For every~$i$, $1\le i\le k+n+1$, every \emph{flexible} concept name~$A$
occurring in~\Tmc, and every \emph{flexible} role name~$r$ occurring in~\Tmc
or~\Rmc, we introduce copies $A^{(i)}$ and~$r^{(i)}$.  We call $A^{(i)}$ the
$i$-th copy of~$A$, and similarly $r^{(i)}$ the $i$-th copy of~$r$.  The
conjunctive query~$\alpha^{(i)}$ (the axiom~$\beta^{(i)}$) is obtained from a
conjunctive query~$\alpha$ (an axiom~$\beta$) by replacing every occurrence of a
flexible name by its $i$-th copy.  Similarly, for $1\le\ell\le k$, the
conjunction of CQ-literals~$\zeta_{X_\ell}^{(i)}$ is obtained
from~$\zeta_{X_\ell}$ (see Definition~\ref{def:tcq-r-sat}) by replacing each
CQ~$\alpha$ occurring in $\zeta_{X_\ell}$ by~$\alpha^{(i)}$.
%
Finally, we define
\begin{align*}
    \zeta_{\Wmc,\iota}
    &:=\bigwedge_{1\le i\le k}\zeta_{X_i}^{(i)}\land%
        \bigwedge_{0\le i\le n}\Biggl(\zeta_{X_{\iota(i)}}^{(k+i+1)}\land
        \bigwedge_{\alpha\in\Amc_i}\alpha^{(k+i+1)}\Biggr),\\
    \Tmc_{\Wmc,\iota}
    &:=\bigl\{\beta^{(i)}\mid\beta\in\Tmc\ \text{and}\ 1\le i\le k+n+1\bigr\},\ \text{and}\\
    \Rmc_{\Wmc,\iota}
    &:=\bigl\{\gamma^{(i)}\mid\gamma\in\Rmc\ \text{and}\ 1\le i\le k+n+1\bigr\}.
\end{align*}
%
Note that in the definition of $\zeta_{\Wmc,\iota}$ it is essential that the
ABoxes do not contain complex concepts, otherwise they could not be viewed as
sets of conjunctive queries, and hence $\zeta_{\Wmc,\iota}$ would not be a
conjunction of CQ-literals.

\begin{lemma}\label{lem:zeta-wmc-iota}
    The set~\Wmc is r-satisfiable w.r.t.~$\iota$ and~\Kmc iff the conjunction of
    CQ-literals~$\zeta_{\Wmc,\iota}$ is satisfiable w.r.t.\ the knowledge base
    $(\Tmc_{\Wmc,\iota},\Rmc_{\Wmc,\iota})$.
\end{lemma}

\begin{proof}
    For the \enquote{only if} direction, let
    $\Jmc_1=(\Delta,\cdot^{\Jmc_1})$,~\dots, $\Jmc_k=(\Delta,\cdot^{\Jmc_k})$,
    and $\Imc_0=(\Delta,\cdot^{\Imc_0})$,~\dots,
    $\Imc_n=(\Delta,\cdot^{\Imc_n})$ be the interpretations required by
    Definition~\ref{def:tcq-r-sat} for the r-satisfiability of~\Wmc
    w.r.t.~$\iota$ and~\Kmc.  We construct the interpretation
    $\Jmc=(\Delta,\cdot^\Jmc)$ as follows:
    \begin{itemize}
        \item every individual name and every rigid name is interpreted as
            in~$\Jmc_1$;
        \item the $i$-th copy, $1\le i\le k$, of each flexible name is
            interpreted like the original name in~$\Jmc_i$; and
        \item the $i$-th copy, $k+1\le i\le k+n+1$, of each flexible name is
            interpreted like the original name in~$\Imc_{i-k-1}$.
    \end{itemize}
    %
    It is easy to verify that \Jmc is a model of~$\zeta_{\Wmc,\iota}$ and
    $(\Tmc_{\Wmc,\iota},\Rmc_{\Wmc,\iota})$.

    For the \enquote{if} direction, let~\Jmc be a model of~$\zeta_{\Wmc,\iota}$
    w.r.t.\ $(\Tmc_{\Wmc,\iota},\Rmc_{\Wmc,\iota})$.  We obtain the
    interpretations $\Jmc_1,\dots,\Jmc_k$, and $\Imc_0,\dots,\Imc_n$ by the
    inverse construction to the one above:
    \begin{itemize}
        \item the domain of all these interpretations is the domain of~\Jmc;
        \item every individual name and every rigid name is interpreted by these
            interpretations as in~\Jmc;
        \item every flexible name is interpreted in~$\Jmc_i$, $1\le i\le k$, as
            its $i$-th copy is interpreted in~\Jmc; and
        \item every flexible name is interpreted in $\Imc_i$, $0\le i\le n$, as
            its $(k+i+1)$-st copy is interpreted in~\Jmc.
    \end{itemize}
    %
    Again, it is easy to verify that these interpretations satisfy the
    conditions for r-satisfiability of~\Wmc w.r.t.~$\iota$ and~\Kmc.
\end{proof}

\noindent
Using this lemma, we can prove the following complexity results.

\begin{theorem}\label{thm:upper-bounds-rigid-names}
    If $\NRC\ne\emptyset$ and $\NRR\ne\emptyset$, the temporalised
    query-entailment problem in \SHQ is
    \begin{itemize}
        \item in \TwoExpTime w.r.t.\ combined complexity and
        \item in \ExpTime w.r.t.\ data complexity.
    \end{itemize}
\end{theorem}

\begin{proof}
    Let $\phi$ be a Boolean TCQ, and let
    $\Kmc=((\Amc_i)_{0\le i\le n},\Tmc,\Rmc)$ be a temporal \SHQ-knowledge base.
    %
    We again consider the TCQ-satisfiability problem, which has the same
    complexity as the temporalised query non-entailment problem.  We decide
    whether $\phi$ is satisfiable w.r.t.~\Kmc using
    Lemma~\ref{lem:tcq-r-sat-t-sat}.  For that, let
    $\psf\colon\CQ(\phi)\to\Pmc_\phi$ be a bijection.
    %
    We first enumerate all sets $\Wmc=\{X_1,\dots,X_k\}\subseteq 2^{\Pmc_\phi}$
    and all mappings $\iota\colon\{0,\dots,n\}\to\{1,\dots,k\}$, which can be
    done in time doubly exponential in the size of~$\phi$ and exponential
    in~$n$.

    For every such pair~$(\Wmc,\iota)$, we check t-satisfiability of
    $\phi^\psf_\Wmc$ w.r.t.~\Wmc and~$\iota$ in time exponential in the size
    of~$\phi^\psf$ (and thus in time exponential in the size of~$\phi$), linear
    in the size of~\Wmc, and polynomial in~$n$ (by Lemma~\ref{lem:tcq-t-sat}),
    and check \Wmc for r-satisfiability w.r.t.~$\iota$ and~\Kmc.  By
    Lemma~\ref{lem:tcq-r-sat-t-sat}, $\phi$ has a model w.r.t.~\Kmc iff at least
    one pair passes both tests.

    For the r-satisfiability check, we use Lemma~\ref{lem:zeta-wmc-iota}.  We
    construct the conjunction of CQ-literals~$\zeta_{\Wmc,\iota}$ and the
    knowledge base $(\Tmc_{\Wmc,\iota},\Rmc_{\Wmc,\iota})$, which can be done in
    time exponential in the size of~$\phi$, \Tmc, and~\Rmc, and in time linear
    in the size of $\Amc_1,\dots,\Amc_n$.  Moreover, the size
    of~$\zeta_{\Wmc,\iota}$ and $(\Tmc_{\Wmc,\iota},\Rmc_{\Wmc,\iota})$ is at
    most exponential in the size of~$\phi$, \Tmc, and~\Rmc, and linear in the
    size of $\Amc_1,\dots,\Amc_n$.

    By Theorem~\ref{thm:complexity-conjunction-cq-literals} we can check whether
    $\zeta_{\Wmc,\iota}$ is satisfiable w.r.t.\
    $(\Tmc_{\Wmc,\iota},\Rmc_{\Wmc,\iota})$ in time \emph{doubly exponential} in
    the size of~$\phi$, \Tmc, and~\Rmc, and \emph{exponential} in the size of
    $\Amc_1,\dots,\Amc_n$.
    %
    Using this decision procedure, we can check whether $\phi$ is satisfiable
    w.r.t.~\Kmc.  Thus, the TCQ-satisfiability problem in \SHQ is in \TwoExpTime
    w.r.t.\ combined complexity and in \ExpTime w.r.t.\ data complexity.  Since
    both \TwoExpTime and \ExpTime are closed under complement, we obtain that
    the temporalised query-entailment problem is in \TwoExpTime w.r.t.\ combined
    complexity and in \ExpTime w.r.t.\ data complexity.
\end{proof}

\noindent
Together with Theorem~\ref{thm:lower-bounds-combined-complexity} and
Corollary~\ref{cor:lower-bounds-data-complexity}, we obtain that the
temporalised query-entailment problem in \SHQ is \TwoExpTime-complete w.r.t.\
combined complexity, and \coNP-hard and in \ExpTime w.r.t.\ data complexity if
both concept and role names may be rigid.

Unfortunately, the above approach does not allow us to match the lower bound
for data complexity, and thus leaves a gap in the data complexity results.  As
seen in the proof of Theorem~\ref{thm:upper-bounds-rigid-names}, the issue is
that the size of~$\zeta_{\Wmc,\iota}$ depends
on~$n$.  More precisely, recall that constructing $\zeta_{\Wmc,\iota}$ involves
copying the type~$\zeta_{X_{\iota(i)}}$ assigned to the ABox~$\Amc_i$ for
every~$i$, $1\le i\le n$.  Thus, we introduce linearly many \emph{negated} CQs
in~$\zeta_{\Wmc,\iota}$, and
Theorem~\ref{thm:complexity-conjunction-cq-literals} yields only an upper bound
of \ExpTime for the satisfiability problem.  Note that linearly many non-negated
CQs in $\zeta_{\Wmc,\iota}$ are not problematic, as they can be instantiated and
viewed as part of the ABox, as detailed in the proof of
Theorem~\ref{thm:complexity-conjunction-cq-literals}.
%
However, we can match the lower bound of \coNP for the data complexity in the
following special cases.

\begin{lemma}\label{lem:special-cases}
    If $\NRC\ne\emptyset$ and $\NRR\ne\emptyset$, the temporalised
    query-entailment problem in \SHQ is in \coNP w.r.t.\ data complexity if any
    of the following conditions apply:
    \begin{enumerate}
        \item The number~$n$ of the input ABoxes is bounded by a constant.
        \item The set of individual names allowed to occur in the input ABoxes
            is fixed.
    \end{enumerate}
\end{lemma}

\begin{proof}
    As done in the proof of Theorem~\ref{thm:upper-bounds-no-rigid-names}, we
    decide the TCQ-satisfiability problem as follows.  We first
    non-deterministically guess a set
    $\Wmc=\{X_1,\dots,X_k\}\subseteq 2^{\Pmc_\phi}$ and a mapping
    $\iota\colon\{0,\dots,n\}\to\{1,\dots,k\}$ in time polynomial in the size of
    the input ABoxes.
    %
    By Lemma~\ref{lem:tcq-t-sat}, deciding whether $\phi^\psf$ is t-satisfiable
    w.r.t.~\Wmc and~$\iota$ can be done in time polynomial in~$n$ w.r.t.\ data
    complexity.  Thus, due to Lemma~\ref{lem:tcq-r-sat-t-sat}, is suffices to
    show that in the above mentioned special cases r-satisfiability of~\Wmc
    w.r.t.~$\iota$ and~\Kmc can be checked in non-deterministic polynomial time
    w.r.t.\ data complexity.  For that, we use again
    Lemma~\ref{lem:zeta-wmc-iota}.
    \begin{enumerate}
        \item If $n$ is bounded by a constant, then the number of negated CQs
            in $\zeta_{\Wmc,\iota}$ is constant, and thus
            Theorem~\ref{thm:complexity-conjunction-cq-literals} yields the
            desired upper bound of \NP for the TCQ-satisfiability problem.
        \item If the set of individual names is fixed, then the number of
            possible assertions involving concept names opccurring in the TBox
            is constant.  Note that the concept names occurring only in the
            ABoxes do not affect the entailment of the TCQ, as they can only
            occur in \emph{positive} assertions, and can thus always be
            satisfied by appropriately interpreting the new names.
            %
            This allows us to restrict the formula $\zeta_{\Wmc,\iota}$ to
            contain at most one copy of~$\zeta_{X_{\iota(i)}}$ for each distinct
            combination of~$\zeta_{X_{\iota(i)}}$ and~$\Amc_i$ (ignoring
            assertions about names that do not occur in the TBox).  Clearly, the
            satisfiability of each combination of an ABox with such a
            conjunction of CQ-literals need to be checked only once.  Since
            there are only constantly many such combinations, the modified TCQ
            $\zeta_{\Wmc,\iota}'$ again contains only constantly many negated
            CQs.  As in the previous case,
            Theorem~\ref{thm:complexity-conjunction-cq-literals} yields again
            that the TCQ-satisfiability problem is in \NP.
    \end{enumerate}
    %
    Since the TCQ-satisfiability problem has the same complexity as the
    temporalised query non-entailment problem, we obtain the desired complexity
    results.
\end{proof}

\noindent
It is still open, however, where any of these conditions is necessary.


\subsection{Data Complexity for the Case of Rigid Concept Names}\label{sec:tcq-data-complexity-rigid-concepts}

In this section, we consider the case where only concept names are allowed to be
rigid, i.e.~$\NRC\ne\emptyset$ and $\NRR=\emptyset$.  We show that, in this
case, the temporalised query-entailment problem is in \coNP w.r.t.\ data
complexity.

For that, we again first assume that a set
$\Wmc=\{X_1,\dots,X_k\}\subseteq 2^{\Pmc_\phi}$ and a mapping
$\iota\colon\{0,\dots,n\}\to\{1,\dots,k\}$ are given.
%
We first show how to decide r-satisfiability of~\Wmc w.r.t.~$\iota$ and~\Kmc in
non-deterministic polynomial time w.r.t.\ data complexity.

Similar to what we did in the previous sections, we construct conjunctions of
CQ-literals which we check for satisfiability.  The approach is a mixture of
those of employed for the case without rigid names and for the case with rigid
concept and role names.  More precisely, we combine several satisfiability
checks required for r-satisfiability, but we do not go as far as compiling all
of them into just one conjunction as done to obtain
Lemma~\ref{lem:zeta-wmc-iota}.  For that we use the ideas of
Section~\ref{sec:sat-rigid-concepts} and of the proof of Lemma~6.3
in~\cite{BaGL-ToCL12}.

We consider the conjunctions of CQ-literals $\xi_i\land\zeta_\Wmc$,
$0\le i\le n$, and the knowledge base $(\Tmc_\Wmc,\Rmc_\Wmc)$, where
\begin{align*}
    \xi_i
    &:=\bigwedge_{\alpha\in\Amc_i}\alpha^{(\iota(i))},
    &\zeta_\Wmc
    &:=\bigwedge_{1\le i\le k}\zeta_{X_i}^{(i)},\\
    \Tmc_\Wmc
    &:=\bigl\{\beta^{(i)}\mid\beta\in\Tmc\ \text{and}\ 1\le i\le k\bigr\},
    &\Rmc_\Wmc
    &:=\bigl\{\gamma^{(i)}\mid\gamma\in\Rmc\ \text{and}\ 1\le i\le k\bigr\}.
\end{align*}

However, for r-satisfiability we have to make sure that rigid consequences of
the form $A(a)$ for a rigid concept name $A\in\NRC$ and an individual name
$a\in\NI$ are shared between all of these conjunctions $\xi_i\land\zeta_\Wmc$.
%
It suffices to do this for the set $\RCon(\Tmc)$ of rigid concept names
occurring in~\Tmc since those occurring only in ABox-assertions cannot affect
the entailment of the TCQ~$\phi$.

Let $\Dmc=(\RCon(\Tmc),\Ymc)$ with $\Ymc\subseteq 2^{\RCon(\Tmc)}$ be arbitrary,
and let $\tau$ be a mapping from $\Ind(\phi)\cup\Ind(\Kmc)$ to~\Ymc.  Recall
that as in Definition~\ref{def:respects-dmc}, the idea is that \Dmc fixes the
combinations of rigid concept names that are allowed to occur in the models of
$\xi_i\land\zeta_\Wmc$, $0\le i\le n$.  The mapping~$\tau$ assigns to each
individual name occurring in~$\phi$ or~\Kmc one such combination.
%
To express this formally, we extend the TBox by the axioms in
\[\Tmc_\tau:=\bigl\{A_{\tau(a)}\equiv C_{\RCon(\Tmc),\tau(a)}\mid%
    a\in\Ind(\phi)\cup\Ind(\Kmc)\bigr\},\]
where $A_{\tau(a)}$ are fresh rigid concept names and, for every
$Y\subseteq\RCon(\Tmc)$, the concept~$C_{\RCon(\Tmc),Y}$ is defined as
in Definition~\ref{def:respects-dmc}.
%
Correspondingly, we extend the conjunctions $\xi_i\land\zeta_\Wmc$ by
\[\xi_\tau:=\bigwedge_{a\in\Ind(\phi)\cup\Ind(\Kmc)}A_{\tau(a)}\]
in order to fix the behaviour of the rigid concept names on the named
individuals.

The next lemma states how these notions can be used to characterise
r-satisfiability of~\Wmc w.r.t.~$\iota$ and~\Kmc.  Its proof is very similar to
the proof of Lemma~\ref{lem:dmc-tau}.

\begin{lemma}\label{lem:tcq-dmc-tau}
    If $\NRC\ne\emptyset$ and $\NRR=\emptyset$, then \Wmc is r-satisfiable
    w.r.t.~$\iota$ and~\Kmc iff there exist a pair $\Dmc=(\RCon(\Tmc),\Ymc)$
    with $\Ymc\subseteq 2^{\RCon(\Tmc)}$ and a mapping
    $\tau\colon\Ind(\phi)\cup\Ind(\Kmc)\to\Ymc$ such that for every~$i$,
    $0\le i\le n$, the conjunction of CQ-literals
    $\xi_i\land\zeta_\Wmc\land\xi_\tau$ has a model w.r.t.\
    $(\Tmc_\Wmc\cup\Tmc_\tau,\Rmc_\Wmc)$ that respects~\Dmc.
\end{lemma}

\begin{proof}
    For the \enquote{if} direction, assume that $\Imc_i$, $0\le i\le n$, are the
    models of $\xi_i\land\zeta_\Wmc\land\xi_\tau$ w.r.t.\
    $(\Tmc_\Wmc\cup\Tmc_\tau,\Rmc_\Wmc)$, respectively, that respect~\Dmc; see
    Definition~\ref{def:respects-dmc}.
    %
    Similar to the proof of Lemma~\ref{lem:dmc-tau} (and Lemma~6.3
    in~\cite{BaGL-ToCL12}), we can assume w.l.o.g.\ that their domains
    $\Delta_i$ are countably infinite and for each $Y\in\Ymc$ there are
    countably infinitely many elements $d\in(C_{\RCon(\Tmc),Y})^{\Imc_i}$, where
    $C_{\RCon(\Tmc),Y}$ is defined in Definition~\ref{def:respects-dmc}.  This
    is a consequence of the Löwenheim-Skolem theorem~\cite{Loe-MA15,Sko-VS20}
    and the fact that the countably infinite disjoint union of~$\Imc_i$ with
    itself is again a model of $\xi_i\land\zeta_\Wmc\land\xi_\tau$ w.r.t.\
    $(\Tmc_\Wmc\cup\Tmc_\tau,\Rmc_\Wmc)$.  The latter follows from the
    observation that for every CQ~$\psi$, there is a homomorphism of~$\psi$
    into~$\Imc_i$ iff there is a homomorphism of~$\psi$ into the disjoint union
    of~$\Imc_i$ with itself.
    %
    One direction is trivial, while whenever there is a homomorphism of~$\psi$
    into the disjoint union of~$\Imc_i$ with itself, we can construct a
    homomorphism of~$\psi$ into $\Imc_i$ by replacing the elements in the image
    of this homomorphism by the corresponding elements of $\Delta_i$.  It is
    easy to see that the resulting homomorphism still satisfies all atoms of the
    CQ~$\psi$.

    Consequently, we can partition the domains~$\Delta_i$ into the countably
    infinite sets
    \[\Delta_i(Y):=\bigl\{d\in\Delta_i\mid d\in(C_{\RCon(\Tmc),Y})^{\Imc_i}\bigr\}\]
    for $Y\in\Ymc$.
    %
    By the assumptions above and the fact that every~$\Imc_i$
    satisfies~$\xi_\tau$ and~$\Tmc_\tau$, there are bijections
    $\pi_i\colon\Delta_0\to\Delta_i$, $1\le i\le n$, such that
    \begin{itemize}
        \item $\pi_i(\Delta_0(Y))=\Delta_i(Y)$ for every $Y\in\Ymc$, and
        \item $\pi_i(a^{\Imc_0})=a^{\Imc_i}$ for every
            $a\in\Ind(\phi)\cup\Ind(\Kmc)$.
    \end{itemize}
    %
    Thus, we can assume in the following that the models~$\Imc_i$,
    $0\le i\le n$, actually share the same domain~$\Delta$ and interpret the
    concept names in $\RCon(\Tmc)$ and the individual names occurring in~$\phi$
    or~\Kmc in the same way.
    %
    We can now construct the interpretations required by
    Definition~\ref{def:tcq-r-sat} by appropriately relating the flexible names
    and their copies.  For every $j$, $1\le j\le k$, we define
    $\Jmc_j=(\Delta,\cdot^{\Jmc_j})$ by interpreting
    the concept names in $\RCon(\Tmc)$ and the individual names occurring
    in~$\phi$ or~\Kmc as in~$\Imc_0$, and the flexible names as their $j$-th
    copies in~$\Imc_0$.  Since $\Imc_0$ is a model of~$\zeta_\Wmc$ and
    $(\Tmc_\Wmc,\Rmc_\Wmc)$, we have that $\Jmc_j$ is a model of~$\zeta_{X_j}$,
    \Tmc, and~\Rmc.

    Similarly, for every~$i$, $0\le i\le n$, we define
    $\Imc_i'=(\Delta,\cdot^{\Imc_i'})$ by interpreting the concept names in
    $\RCon(\Tmc)$ and the individual names occurring in~$\phi$ or~\Kmc as
    in~$\Imc_i$, and the flexible names as their $\iota(i)$-th copies
    in~$\Imc_i$.  Since $\Imc_i$ is a model of~$\xi_i$, $\zeta_\Wmc$, and
    $(\Tmc_\Wmc,\Rmc_\Wmc)$, we obtain that $\Imc_i$ is a model of~$\Amc_i$,
    $\zeta_{X_{\iota(i)}}$, \Tmc, and~\Rmc.
    %
    All these models share the same domain and interpret the rigid concept names
    in $\RCon(\Tmc)$ and the individual names occurring in~$\phi$ or~\Kmc in the
    same way.  Note that the interpretation of the names that occur neither~\Kmc
    nor in in~$\phi$ is irrelevant and can be fixed arbitrarily, as long as the
    UNA is satisfied.

    Thus, it remains to consider those rigid concept names~$A$ occurring
    in~$(\Amc_i)_{0\le i\le n}$, but not in~\Tmc.  Since they are not
    constrained by the TBox, it suffices to interpret them in such a way that
    they satisfy all ABox-assertions.  But since these assertions can only occur
    positively in the ABoxes, the set
    $\{a^{\Imc_0}\mid A(a)\in\Amc_i,\ 0\le i\le n\}$ fulfils this restriction.
    %
    Thus, the conditions required for r-satisfiability of~\Wmc w.r.t.~$\iota$
    and~\Kmc by Definition~\ref{def:tcq-r-sat} are satisfied.

    For the \enquote{only if} direction, assume that
    $\Jmc_j=(\Delta,\cdot^{\Jmc_j})$, $1\le j\le k$, and
    $\Imc_i=(\Delta,\cdot^{\Imc_i})$, $0\le i\le n$, are the interpretations
    required for r-satisfiability of~\Wmc w.r.t.~$\iota$ and~\Kmc by
    Definition~\ref{def:tcq-r-sat}.
    %
    It is easy to see that for every~$i$, $0\le i\le n$, one can combine the
    interpretations $\Imc_i$, $\Jmc_1$,~\dots, $\Jmc_k$ to obtain a
    model~$\Imc_i'$ of $\xi_i\land\zeta_\Wmc$ w.r.t.\ $(\Tmc_\Wmc,\Rmc_\Wmc)$ by
    interpreting the $\iota(i)$-th copy of a flexible name as the original name
    in~$\Imc_i$, and the $j$-th copy of a flexible name as the original name
    in~$\Jmc_j$ for each $j$, $1\le j\le k$, with $j\ne\iota(i)$.  Obviously,
    the interpretations $\Imc_0',\dots,\Imc_n'$ share the same domain, interpret
    individual names in the same way, and respect rigid concept names.  Thus,
    for every $Y\subseteq\RCon(\Tmc)$, we have that
    $(C_{\RCon(\phi),Y})^{\Imc_0'}=(C_{\RCon(\phi),Y})^{\Imc_i'}$ for every~$i$,
    $1\le i\le n$.
    %
    We define $\Dmc:=(\RCon(\Tmc),\Ymc)$ with
    \[\Ymc:=\bigl\{Y\subseteq\RCon(\phi)\mid%
        \text{there is some $d\in\Delta$ with $d\in(C_{\RCon(\Tmc),Y})^{\Imc_0'}$}\bigr\}.\]
    %
    By construction of~\Dmc, we obtain that the interpretations~$\Imc_i'$,
    $0\le i\le n$, respect~\Dmc.
    %
    Moreover, for every $a\in\Ind(\phi)\cup\Ind(\Kmc)$, we define
    $\tau(a):=Y\subseteq\RCon(\Tmc)$ iff $a\in(C_{\RCon(\Tmc),Y})^{\Imc_0'}$,
    which ensures that the interpretations~$\Imc_i'$ can be extended to models
    of~$\xi_\tau$ and~$\Tmc_\tau$ by appropriately interpreting the new concept
    names~$A_{\tau(a)}$.
    %
    Hence, we obtain models of $\xi_i\land\zeta_\Wmc\land\xi_\tau$ w.r.t.\
    $(\Tmc_\Wmc\cup\Tmc_\tau,\Rmc_\Wmc)$ that respect~\Dmc as required.
\end{proof}

\noindent
Using this lemma, we can prove our complexity result.

\begin{theorem}\label{thm:upper-bound-rigid-concepts-data-compl}
    If $\NRC\ne\emptyset$ and $\NRR=\emptyset$, the temporalised
    query-entailment problem in \SHQ is in \coNP w.r.t.\ data complexity.
\end{theorem}

\begin{proof}
    Let $\phi$ be a Boolean TCQ, and let
    $\Kmc=((\Amc_i)_{0\le i\le n},\Tmc,\Rmc)$ be a temporal \SHQ-knowledge base.
    %
    We again consider the TCQ-satisfiability problem, which has the same
    complexity as the temporalised query non-entailment problem.  We decide
    whether $\phi$ is satisfiable w.r.t.~\Kmc using
    Lemma~\ref{lem:tcq-r-sat-t-sat}.  For that, let
    $\psf\colon\CQ(\phi)\to\Pmc_\phi$ be a bijection.
    %
    We first non-deterministically guess a set
    $\Wmc=\{X_1,\dots,X_k\}\subseteq 2^{\Pmc_\phi}$ and a mapping
    $\iota\colon\{0,\dots,n\}\to\{1,\dots,k\}$ in time polynomial in the size of
    the input ABoxes.
    %
    By Lemma~\ref{lem:tcq-t-sat}, deciding whether $\phi^\psf$ is t-satisfiable
    w.r.t.~\Wmc and~$\iota$ can be done in time polynomial in~$n$ w.r.t.\ data
    complexity.

    Thus, due to Lemma~\ref{lem:tcq-r-sat-t-sat}, is suffices to show that
    r-satisfiability of~\Wmc w.r.t.~$\iota$ and~\Kmc can be checked in
    non-deterministic polynomial time w.r.t.\ data complexity.  For that, we use
    Lemma~\ref{lem:tcq-dmc-tau}.
    %
    We non-deterministically guess a set $\Ymc\subseteq 2^{\RCon(\Tmc)}$ and a
    mapping $\tau\colon\Ind(\phi)\cup\Ind(\Kmc)\to\Ymc$, which can be done in
    time polynomial in the size of the input ABoxes.  Indeed, \Ymc only depends
    on~\Tmc, and $\tau$ is of size linear in the size of the input ABoxes.
    %
    We define $\Dmc:=(\RCon(\Tmc),\Ymc)$.  Next, we construct for every~$i$,
    $0\le i\le n$, the conjunction of CQ-literals
    $\xi_i\land\zeta_\Wmc\land\xi_\tau$ and the knowledge base
    $(\Tmc_\Wmc\cup\Tmc_\tau,\Rmc_\Wmc)$.  Note that $\xi_i$ and~$\xi_\tau$
    are of size polynomial in the size of the input ABoxes, and that the sizes
    of~$\zeta_\Wmc$, $\Tmc_\Wmc$, $\Tmc_\tau$, and~$\Rmc_\Wmc$ do not depend on
    the input ABoxes.  Furthermore, only $\zeta_\Wmc$ may contain negated CQs,
    and thus their size does not depend on the size of the input ABoxes.
    %
    It remains to show that we can check the existence of a model of
    $\xi_i\land\zeta_\Wmc\land\xi_\tau$ w.r.t.\
    $(\Tmc_\Wmc\cup\Tmc_\tau,\Rmc_\Wmc)$ that respects~\Dmc in non-deterministic
    polynomial time w.r.t.\ data complexity.
    %
    For that, observe that the restriction imposed by~\Dmc can equivalently be
    expressed as the conjunction of CQ-literals
    \[\zeta_\Dmc:=\bigl(\lnot\exists x.A_\Dmc(x)\bigr)\land\bigwedge_{Y\in\Ymc}\exists x.A_Y(x),\]
    where $A_\Dmc$ and $A_Y$ are fresh concept names that are restricted by
    adding the following axioms to the TBox:
    $A_Y\equiv C_{\RCon(\Tmc),Y}$ and
    $C_{\RCon(\Tmc),Y}\sqsubseteq A_Y$ for every $Y\in\Ymc$, and
    $A_\Dmc\equiv\bigsqcap_{Y\in\Dmc}\lnot A_Y$.\footnote{%
        We did not add all the axioms $A_Y\equiv C_{\RCon(\Tmc),Y}$ earlier
        since we reuse Lemma~\ref{lem:tcq-dmc-tau} in the following section
        about combined complexity, and these additional axioms cause an
        exponential blow-up in the size of the TBox.}
    We denote by~$\Tmc_\Wmc'$ the resulting extension of the
    TBox~$\Tmc_\Wmc\cup\Tmc_\tau$.  Thus, it is enough to check whether
    $\xi_i\land\zeta_\Wmc\land\xi_\tau\land\zeta_\Dmc$ has a model w.r.t.\
    $(\Tmc_\Wmc',\Rmc_\Wmc)$.
    %
    Note that neither $\zeta_\Dmc$ nor~$\Tmc_\Wmc'$ depend on the input ABoxes.
    Hence, one can see from the proof of
    Theorem~\ref{thm:complexity-conjunction-cq-literals} that this
    satisfiability problem can be decided in non-deterministic polynomial time
    w.r.t.\ data complexity.  Thus, we obtain that the TCQ-satisfiability
    problem in \SHQ is in \NP w.r.t.\ data complexity, which shows that the
    temporalised query-entailment problem in \SHQ is in \coNP w.r.t.\ data
    complexity.
\end{proof}

\noindent
Together with Corollary~\ref{cor:lower-bounds-data-complexity}, this yields that
the temporalised query-entailment problem in \SHQ is \coNP-complete w.r.t.\ data
complexity if only concept names are allowed to be rigid.


\subsection{Combined Complexity for the Case of Rigid Concept Names}\label{sec:tcq-combined-complexity-rigid-concepts}

In this section, we again consider the case where only concept names are allowed
to be rigid, i.e.~$\NRC\ne\emptyset$ and $\NRR=\emptyset$.  However, we consider
the combined complexity of the temporalised query-entailment problem.
%
Unfortunately, the approach used in the previous section does not yield a
combined complexity of \coNExpTime.  The reason is that the conjunctions of
CQ-literals $\zeta_\Wmc$ and $\zeta_\Dmc$ are of size exponential in the size
of~$\phi$, and thus Theorem~\ref{thm:complexity-conjunction-cq-literals} only
yields an upper bound of \TwoExpTime.
%
Therefore, we describe a different approach with a combined complexity of
\coNExpTime.

As a first step, we rewrite the Boolean TCQ~$\phi$ into a Boolean TCQ~$\psi$ of
size polynomial in the size of~$\phi$ and the temporal KB~\Kmc such that
answering $\phi$ at time~$n$ w.r.t.~\Kmc is equivalent to answering $\psi$ at
time~$0$ w.r.t.\ a temporal KB containing only a trivial sequence of ABoxes.
This is done by compiling the ABoxes into the query and postponing the
query~$\phi$ using the $\Next$-operator.

\begin{lemma}\label{lem:initial-validity}
    Let $\Kmc=((\Amc_i)_{0\le i\le n},\Tmc,\Rmc)$ be a temporal KB
    and $\phi$ be a Boolean TCQ\@.
    %
    Then there is a Boolean TCQ~$\psi$ of size polynomial in the size of~$\phi$
    and~\Kmc such that $\Kmc\models\phi$ iff $(\emptyset,\Tmc,\Rmc)\models\psi$.
\end{lemma}

\begin{proof}
    We define the Boolean TCQ
    \[\psi:=(\gamma_0\land\Next\gamma_1\land\dots\land\Next^n\gamma_n)\to\Next^n\phi,\]
    where $\gamma_i:=\bigwedge\Amc_i$ for $i$, $0\le i\le n$, and $\Next^j$
    abbreviates $j$ nested $\Next$-operators.  Obviously, the size of~$\psi$ is
    polynomial in the size of~$\phi$ and~\Kmc.
    %
    It remains to prove that $\Kmc\models\phi$ iff
    $\Kmc':=(\emptyset,\Tmc,\Rmc)\models\psi$.  We have:
    \begin{itemize}
        \item[]
            $\Kmc\models\phi$
        \item[\emph{iff}]
            $((\Amc_i)_{0\le i\le n},\Tmc,\Rmc)\models\phi$
        \item[\emph{iff}]
            $\Imf,n\models\phi$ for every~\Imf with
            $\Imf\models((\Amc_i)_{0\le i\le n},\Tmc,\Rmc)$
        \item[\emph{iff}]
            $\Imf,n\models\phi$ for every~\Imf with
            $\Imf\models(\emptyset,\Tmc,\Rmc)$ and
            $\Imf,0\models\gamma_0$;
            $\Imf,1\models\gamma_1$;~\dots;
            $\Imf,n\models\gamma_n$
        \item[\emph{iff}]
            $\Imf,0\models\Next^n\phi$ for every~\Imf with
            $\Imf\models\Kmc'$ and
            $\Imf,0\models\gamma_0$;
            $\Imf,0\models\Next\gamma_1$;~\dots;
            $\Imf,0\models\Next^n\gamma_n$
        \item[\emph{iff}]
            $\Imf,0\models\psi$ for every~\Imf with $\Imf\models\Kmc'$
        \item[\emph{iff}]
            $\Kmc'\models\psi$.
            %\qedhere
    \end{itemize}
\end{proof}

\noindent
To obtain our complexity result, we can thus focus on deciding whether a Boolean
TCQ~$\phi$ has a model w.r.t.\ a temporal KB $\Kmc=(\emptyset,\Tmc,\Rmc)$
containing only one empty ABox, i.e.~we have $n=0$.
%
Note that this compilation approach does not yield a low \emph{data
complexity} for the TCQ-satisfiability problem since, after encoding the ABoxes
into~$\phi$, the size of the conjunction of CQ-literals~$\zeta_\Wmc$ is
exponential in the size of the input ABoxes.  Moreover, then
Lemma~\ref{lem:tcq-t-sat} yields that the t-satisfiability check is exponential
in the size of the input ABoxes.

Assume from now on that a set
$\Wmc=\{X_1,\dots,X_k\}\subseteq 2^{\Pmc_\phi}$ and a mapping
$\iota\colon\{0\}\to\{1,\dots,k\}$ are given.
%
We first show how to decide r-satisfiability of~\Wmc w.r.t.~$\iota$ and~\Kmc in
non-deterministic exponential time w.r.t.\ combined complexity.

For that, we use the idea of Lemma~\ref{lem:tcq-dmc-tau}.  Since $\xi_0=\true$,
according to this lemma, it suffices to non-deterministically guess a pair~\Dmc
and a mapping~$\tau$ such that $\zeta_\Wmc\land\xi_\tau$ has a model w.r.t.\
$(\Tmc_\Wmc\cup\Tmc_\tau,\Rmc_\Wmc)$ that respects~\Dmc.
%
Instead of constructing the conjunction of CQ-literals~$\zeta_\Dmc$, and then
applying Theorem~\ref{thm:complexity-conjunction-cq-literals}
directly to this problem, which would yield a complexity of \TwoExpTime, we
split the problem into separate sub-problems for each $\zeta_{X_i}$.

\begin{lemma}\label{lem:tcq-dmc-tau-combined}
    If $\NRC\ne\emptyset$ and $\NRR=\emptyset$, then \Wmc is r-satisfiable
    w.r.t.~$\iota$ and $\Kmc=(\emptyset,\Tmc,\Rmc)$ iff there exist a pair
    $\Dmc=(\RCon(\Tmc),\Ymc)$ with $\Ymc\subseteq 2^{\RCon(\Tmc)}$ and a mapping
    $\tau\colon\Ind(\phi)\to\Ymc$ such that for every~$i$, $1\le i\le k$, the
    conjunction of CQ-literals $\zeta_{X_i}\land\xi_\tau$ has a model
    w.r.t.\ $(\Tmc\cup\Tmc_\tau,\Rmc)$ that respects~\Dmc.
\end{lemma}

\begin{proof}
    For the \enquote{if} direction, assume that $\Imc_i$, $1\le i\le k$, are the
    models of $\zeta_{X_i}\land\xi_\tau$ w.r.t.\ $(\Tmc\cup\Tmc_\tau,\Rmc)$ that
    respect~\Dmc.  As in the proof of Lemma~\ref{lem:tcq-dmc-tau}, we can ensure
    that they share the same domain and interpret the rigid concept names in
    $\RCon(\Tmc)$ and the individual names in $\Ind(\phi)$ in the same way.
    %
    Similar to before, we construct a model~\Jmc of $\zeta_\Wmc\land\xi_\tau$
    and $(\Tmc_\Wmc\cup\Tmc_\tau,\Rmc_\Wmc)$ over the shared domain of
    $\Imc_1,\dots,\Imc_k$ as follows: interpret the $i$-th copy of a flexible
    name as the original name in~$\Imc_i$, and every rigid name as in~$\Imc_1$.
    Since the interpretations of the names in $\RCon(\Tmc)$ are not changed,
    \Jmc also respects~\Dmc.  By Lemma~\ref{lem:tcq-dmc-tau}, we obtain that
    \Wmc is r-satisfiable w.r.t.~$\iota$ and~\Kmc.

    For the \enquote{only if} direction, assume that \Wmc is r-satisfiable
    w.r.t.~$\iota$ and~\Kmc.  Lemma~\ref{lem:tcq-dmc-tau} yields that there
    exist a pair $\Dmc=(\RCon(\Tmc),\Ymc)$ with $\Ymc\subseteq 2^{\RCon(\Tmc)}$
    and a mapping $\tau\colon\Ind(\phi)\to\Ymc$ such that
    $\zeta_\Wmc\land\xi_\tau$ has a model~\Jmc w.r.t.\
    $(\Tmc_\Wmc\cup\Tmc_\tau,\Rmc_\Wmc)$ that respects~\Dmc.
    %
    As before, for every~$i$, $1\le i\le k$, we obtain a model~$\Imc_i$ of
    $\zeta_{X_i}\land\xi_\tau$ and $(\Tmc\cup\Tmc_\tau,\Rmc)$ over the domain
    of~\Jmc by interpreting the rigid names as in~\Jmc and the flexible names as
    their $i$-th copies in~\Jmc.  Again, these models still respect~\Dmc.
\end{proof}

\noindent
To obtain our complexity result, we show how to decide whether the conjunction
of CQ-literals~$\zeta_{X_i}\land\xi_\tau$ has a model w.r.t.\
$(\Tmc\cup\Tmc_\tau,\Rmc)$ that respects~\Dmc in time exponential in the size
of~$\phi$ and~\Kmc.
%
Similar to the proof of Theorem~\ref{thm:complexity-conjunction-cq-literals}, we
can reduce this problem to a \emph{non-entailment problem for a union of Boolean
CQs}: there is a model of~$\zeta_{X_i}\land\xi_\tau$ w.r.t.\
$(\Tmc\cup\Tmc_\tau,\Rmc)$ that respects~\Dmc iff there is a model of
$\Kmc_i:=(\Amc_i,\Tmc\cup\Tmc_\tau,\Rmc)$ that respects~\Dmc and is not a model
of~$\rho_i$ (written $\Kmc_i\not\models\rho_i$ w.r.t.~\Dmc), where $\Amc_i$ is
an ABox obtained by instantiating the non-negated CQs
of~$\zeta_{X_i}\land\xi_\tau$ with fresh individual names and $\rho_i$ is a
union of CQs constructed from the negated CQs of~$\zeta_{X_i}\land\xi_\tau$.
%
Since all $\Kmc_i$ and~$\rho_i$ are of size polynomial in the size of~$\phi$
and~\Kmc, it thus suffices to show that we can decide the query non-entailment
$\Kmc_i\not\models\rho_i$ w.r.t.~\Dmc in time exponential in the size
of~$\Kmc_i$ and~$\rho_i$.

It is known that $\Kmc_i\not\models\rho_i$ iff there is a \emph{forest model}~\Imc
of~$\Kmc_i$ such that $\Imc\not\models\rho_i$~\cite{GHL+-JAIR08,Lut-IJCAR08}.
%
We define here forest models for the more general case of \emph{Boolean
\SHQcap-knowledge bases} since this will be needed later.

As introduced in Chapter~\ref{ch:shoq-ltl}, the description logic \SHQcap extends
\SHQ with \emph{role conjunctions}.  Recall that role conjunctions are of the
form $r_1\sqcap\dots\sqcap r_\ell$, $\ell\ge 1$, where $r_1,\dots,r_\ell$ are
\emph{simple} role names.
%
Such role conjunctions are allowed to occur in existential restrictions instead
of a single role, but not in at-least restrictions or role axioms.
%
An interpretation~\Imc is extended to a role conjunction as follows:
$(r_1\sqcap\dots\sqcap r_\ell)^\Imc:=r_1^\Imc\cap\dots\cap r_\ell^\Imc$.

In the following, we denote by $\Ind(\Psi)$ the set of individuals occurring in
the Boolean knowledge base $\Bmc=(\Psi,\Rmc)$.
%
We are now ready to define forest models over Boolean knowledge bases
(for a similar definition, see~\cite{GHL+-JAIR08}).

\begin{definition}[Forest model]\label{def:forest-model}
    A \emph{tree} is a non-empty prefix-closed subset of $\Nbb^*$, where
    $\Nbb^*$ denotes the set of all finite words over the non-negative integers.

    Let $\Imc=(\Delta^\Imc,\cdot^\Imc)$ be an interpretation, and let
    $\Bmc=(\Psi,\Rmc)$ be a Boolean knowledge base.  We say that
    \Imc is a \emph{forest base for~\Bmc} if
    \begin{itemize}
        \item $\Delta^\Imc\subseteq\Ind(\Psi)\times\Nbb^*$ such that for every
            $a\in\Ind(\Psi)$, the set $\{u\mid(a,u)\in\Delta^\Imc\}$ is a tree;
        \item if $((a,u),(b,v))\in r^\Imc$, then either $u=v=\varepsilon$, or
            $a=b$ and $v=u\cdot c$ for some $c\in\Nbb$, where $\cdot$ denotes
            concatenation; and
        \item for every $a\in\Ind(\Psi)$, we have $a^\Imc=(a,\varepsilon)$.
    \end{itemize}
    %
    We call a model $\Jmc=(\Delta^\Jmc,\cdot^\Jmc)$ of~\Bmc a \emph{forest model
    of~\Bmc{}} if there is a forest base $\Imc=(\Delta^\Imc,\cdot^\Imc)$
    for~\Bmc such that $\Delta^\Imc=\Delta^\Jmc$, for each $A\in\NC$, we have
    $A^\Imc=A^\Jmc$, for each $a\in\NI$, we have $a^\Imc=a^\Jmc$, and for each
    $r\in\NR$, we have
    \[r^\Jmc=r^\Imc\cup\bigcup_{\Rmc\models s\sqsubseteq r,\
        \Rmc\models\trans(s)}(s^\Imc)^+,\]
    where $\cdot^+$ denotes the transitive closure.
\end{definition}

\noindent
As an example of a forest model, consider Figure~\ref{fig:forest-model}, where a
graphical representation of a forest model is given.  It depicts the individual
names $a$, $b$, and $c$, which represent the roots $(a,\varepsilon)$,
$(b,\varepsilon)$, and $(c,\varepsilon)$ of three trees.  Moreover, $s$ is a
simple role name, and $r$ is a transitive role name.  The solid arrows denote
the role connections that are present in the corresponding forest base, and the
dashed arrows denote role connection that are introduced due to transitivity.

\begin{figure}[t]
    \centering
    \begin{tikzpicture}[>=stealth',semithick,bend angle=20,shorten <=4pt,shorten >=4pt]
        \draw[draw=none,fill=gray!30]           (0,0)--(-2.1,-3)--(2.1,-3)--cycle;
        \draw[draw=none,fill=gray!30]           (5,0)--(2.9,-3)--(7.1,-3)--cycle;
        \draw[draw=none,fill=gray!30]           (10,0)--(7.9,-3)--(12.1,-3)--cycle;
        \draw[gray,shorten <=0pt,shorten >=0pt] (0,0)       edge (-2.1,-3);
        \draw[gray,shorten <=0pt,shorten >=0pt] (0,0)       edge (2.1,-3);
        \draw[gray,shorten <=0pt,shorten >=0pt] (5,0)       edge (2.9,-3);
        \draw[gray,shorten <=0pt,shorten >=0pt] (5,0)       edge (7.1,-3);
        \draw[gray,shorten <=0pt,shorten >=0pt] (10,0)      edge (7.9,-3);
        \draw[gray,shorten <=0pt,shorten >=0pt] (10,0)      edge (12.1,-3);
        \draw[fill]                             (0,0)       circle (2pt);
        \draw[fill]                             (5,0)       circle (2pt);
        \draw[fill]                             (10,0)      circle (2pt);
        \draw                                   (0,0)       edge [->] node [yshift=5pt] {$r$} (5,0);
        \draw                                   (5,0)       edge [->,bend left] node [yshift=5pt] {$s$} (10,0);
        \draw                                   (10,0)      edge [->,bend left] node [yshift=-5pt] {$s$} (5,0);
        \draw[fill]                             (4.7,-1.5)  circle (2pt);
        \draw                                   (5,0)       edge [->] node [xshift=-4pt] {$r$} (4.7,-1.5);
        \draw[fill]                             (5.25,-2.5) circle (2pt);
        \draw                                   (4.7,-1.5)  edge [->] node [xshift=-5pt] {$r$} (5.25,-2.5);
        \draw[dashed]                           (0,0)       edge [->] node [yshift=5pt] {$r$} (4.7,-1.5);
        \draw[dashed]                           (0,0)       edge [->] node [yshift=-6pt] {$r$} (5.25,-2.5);
        \draw[dashed]                           (5,0)       edge [->] node [xshift=5pt] {$r$} (5.25,-2.5);
        \draw                                   (-.3,.28)   node [right] {$a$};
        \draw                                   (4.7,.28)   node [right] {$b$};
        \draw                                   (9.85,.28)  node [right] {$c$};
    \end{tikzpicture}
    \caption{An example of a forest model}\label{fig:forest-model}
\end{figure}

In the following, we call a model $\Jmc=(\Delta^\Jmc,\cdot^\Jmc)$ a forest model
of a knowledge base $\Kmc=(\Amc,\Tmc,\Rmc)$ if \Jmc is a forest
model of the induced Boolean knowledge base
$(\bigwedge\Amc\land\bigwedge\Tmc,\Rmc)$.

We show now that the restriction to forest models when checking for the
consistency of a Boolean \SHQcap-knowledge base w.r.t.\ a pair
$\Dmc=(\Umc,\Ymc)$ is without loss of generality.
%
First, note that $\Bmc=(\Psi,\Rmc)$ has a model that respects~\Dmc
iff $(\Psi\land A(a),\Rmc)$ has a model that respects~\Dmc, where
$a$ is a fresh individual name and $A$ is a fresh concept name.  In the
following, we thus assume without loss of generality that $\Psi$ contains
at least one individual name.  This is necessary to ensure that there is a
non-empty forest base for~\Bmc.
%
The construction in the proof of the following lemma is very similar to the one
in~\cite{GHL+-JAIR08}, but we extend the previous result to \emph{Boolean}
knowledge bases, and take a pair~\Dmc into account.

\begin{lemma}\label{lem:forest-model}
    Let \Bmc be a Boolean \SHQcap-knowledge base, and let
    $\Dmc=(\Umc,\Ymc)$ be a pair such that \Umc is a set of concept
    names occurring in~\Bmc and $\Ymc\subseteq 2^\Umc$.  Then \Bmc is consistent
    w.r.t.~\Dmc iff it has a forest model that respects~\Dmc.
\end{lemma}

\begin{proof}
    The \enquote{if} direction is trivial.  For the \enquote{only if} direction,
    assume that $\Imc=(\Delta^\Imc,\cdot^\Imc)$ is a model of
    $\Bmc=(\Phi,\Rmc)$ that respects~\Dmc.
    %
    Moreover, we assume that $\Delta^\Imc$ is countable, which is without loss
    of generality due to the downward Löwenheim-Skolem
    theorem~\cite{Loe-MA15,Sko-VS20}.  We can thus assume that
    $\Delta^\Imc\subseteq\Nbb$.
    %
    We define now a forest base $\Jmc=(\Delta^\Jmc,\cdot^\Jmc)$ for~\Bmc with
    domain
    \begin{align*}
        \Delta^\Jmc:=\bigl\{(a,d_1\dots d_m)\mid{}
            &\text{$a\in\Ind(\Psi)$, $m\ge 0$, $d_1,\dots,d_m\in\Delta^\Imc$, and}\\
            &\text{there is no $b\in\Ind(\Psi)$ with $d_1=b^\Imc$}\bigr\}
    \end{align*}
    as follows:
    \begin{itemize}
        \item $a^\Jmc:=(a,\varepsilon)$ for every $a\in\Ind(\Psi)$;
        \item $b^\Jmc$ for each $b\in\NI\setminus\Ind(\Psi)$ can be fixed
            arbitrarily, as long as the UNA is satisfied;
        \item $A^\Jmc:=\{(a,\varepsilon)\mid a^\Imc\in A^\Imc\}\cup%
            \{(a,d_1\dots d_m)\mid d_m\in A^\Imc\}$; and
        \item $\displaystyle\begin{aligned}[t]
                r^\Jmc:={}
                &\{((a,\varepsilon),(b,\varepsilon))\mid(a^\Imc,b^\Imc)\in r^\Imc\}\cup{}\\
                &\{((a,\varepsilon),(a,d))\mid(a^\Imc,d)\in r^\Imc\}\cup{}\\
                &\{((a,d_1\dots d_m),(a,d_1\dots d_md_{m+1}))\mid\text{$m>0$, $(d_m,d_{m+1})\in r^\Imc$}\}.
            \end{aligned}$
    \end{itemize}
    %
    Obviously, \Jmc satisfies the conditions of a forest base for~\Bmc.  We
    construct now a forest model $\Jmc'=(\Delta^{\Jmc'},\cdot^{\Jmc'})$
    for~\Bmc.  For that, we define $\Delta^{\Jmc'}:=\Delta^\Jmc$, for each
    $A\in\NC$, $A^{\Jmc'}:=A^\Jmc$ for each $a\in\NI$, $a^{\Jmc'}:=a^\Jmc$, and
    for each $r\in\NR$:
    \[r^{\Jmc'}:=r^\Jmc\cup\bigcup_{\Rmc\models s\sqsubseteq r,\ \Rmc\models\trans(s)}(s^\Jmc)^+.\]
    %
    To prove that $\Jmc'$ is indeed a forest model for~\Bmc, we first show the
    following claim by structural induction.

    \begin{claim}\label{claim:forest-model-concepts}
        For every $(a,d_1\dots d_m)\in\Delta^{\Jmc'}$ and every concept $C$, we
        have $(a,d_1\dots d_m)\in C^{\Jmc'}$ iff either $m=0$ and $a^\Imc\in
        C^\Imc$, or $d_m\in C^\Imc$.
    \end{claim}

    \noindent
    For the base case, where $C$ is a concept name, the claim is directly
    implied by the definition of~$\Jmc'$.

    For the case where $C$ is of the form $\lnot D$, we have:
    \begin{itemize}
        \item[]
            $(a,d_1\dots d_m)\in(\lnot D)^{\Jmc'}$
        \item[\emph{iff}]
            $(a,d_1\dots d_m)\notin D^{\Jmc'}$
        \item[\emph{iff}]
            either $m=0$ and $a^\Imc\notin D^\Imc$, or $d_m\notin D^\Imc$
        \item[\emph{iff}]
            either $m=0$ and $a^\Imc\in(\lnot D)^\Imc$, or $d_m\in(\lnot D)^\Imc$.
    \end{itemize}

    \noindent
    For the case where $C$ is of the form $D\sqcap E$, we have:
    \begin{itemize}
        \item[]
            $(a,d_1\dots d_m)\in(D\sqcap E)^{\Jmc'}$
        \item[\emph{iff}]
            $(a,d_1\dots d_m)\in D^{\Jmc'}$ and $(a,d_1\dots d_m)\in E^{\Jmc'}$
        \item[\emph{iff}]
            either $m=0$ and $a^\Imc\in D^\Imc$ and $a^\Imc\in E^\Imc$, or
            $d_m\in D^\Imc$ and $d_m\in E^\Imc$
        \item[\emph{iff}]
            either $m=0$ and $a^\Imc\in(D\sqcap E)^\Imc$, or
            $d_m\in(D\sqcap E)^\Imc$.
    \end{itemize}

    \noindent
    For the case where $C$ is of the form $\exists(r_1\sqcap\dots\sqcap
    r_\ell).D$ with $\ell>1$, we have that
    $r_1,\dots,r_\ell$ are simple role names, and thus
    $r_1^{\Jmc'}\cap\dots\cap r_\ell^{\Jmc'}=r_1^\Jmc\cap\dots\cap r_\ell^\Jmc$.
    This yields:
    \begin{itemize}
        \item[]
            $(a,d_1\dots d_m)\in(\exists(r_1\sqcap\dots\sqcap
            r_\ell).D)^{\Jmc'}$
        \item[\emph{iff}]
            either $m=0$ and
            \begin{itemize}
              \item there is a $(b,\varepsilon)\in D^{\Jmc'}$ such that
                  $((a,\varepsilon),(b,\varepsilon))\in r_1^\Jmc\cap\dots\cap
                  r_\ell^\Jmc$, or
              \item there is a $(a,d)\in D^{\Jmc'}$ such that
                  $((a,\varepsilon),(a,d))\in r_1^\Jmc\cap\dots\cap
                  r_\ell^\Jmc$;
            \end{itemize}
            or there is a $(a,d_1\dots d_m d_{m+1})\in D^{\Jmc'}$ such that
            $((a,d_1\dots d_m),(a,d_1\dots d_m d_{m+1}))$ is in
            $r_1^\Jmc\cap\dots\cap r_\ell^\Jmc$
        \item[\emph{iff}]
            either $m=0$ and there is a $d\in D^\Imc$ such that $(a^\Imc,d)\in
            r_1^\Imc\cap\dots\cap r_\ell^\Imc$, or there is a $d\in D^\Imc$ such
            that $(d_m,d)\in r_1^\Imc\cap\dots\cap r_\ell^\Imc$
        \item[\emph{iff}]
            either $m=0$ and $a^\Imc\in(\exists(r_1\sqcap\dots\sqcap
            r_\ell).D)^\Imc$, or $d_m\in(\exists(r_1\sqcap\dots\sqcap
            r_\ell).D)^\Imc$.
    \end{itemize}

    \noindent
    For the case where $C$ is of the form $\exists r.D$, we have
    \begin{itemize}
        \item[]
            $(a,d_1\dots d_m)\in(\exists r.D)^{\Jmc'}$
        \item[\emph{iff}]
            there is an $x\in D^{\Jmc'}$ with either $((a,d_1\dots d_m),x)\in
            r^\Jmc$ or there is an $s\in\NR$ with $\Rmc\models s\sqsubseteq r$,
            $\Rmc\models\trans(s)$, and $((a,d_1\dots d_m),x)\in (s^\Jmc)^+$
        \item[\emph{iff}]
            either $m=0$ and
            \begin{itemize}
                \item there is a $(b,\varepsilon)\in D^{\Jmc'}$ with
                    $((a,\varepsilon),(b,\varepsilon))\in r^\Jmc$,
                \item there is a $(a,d)\in D^{\Jmc'}$ with
                    $((a,\varepsilon),(a,d))\in r^\Jmc$, or
                \item there is an $s\in\NR$ with $\Imc\models s\sqsubseteq r$
                    and $\Imc\models\trans(s)$, and a sequence
                    $(a_0,\varepsilon)$, $(a_1,\varepsilon)$, \dots,
                    $(a_n,\varepsilon)$, $(a_n,e_1)$, \dots, $(a_n,e_1\dots e_k)$
                    of elements of~$\Delta^{\Jmc'}$ such that $a_0=a$,
                    $(a_n,e_1\dots e_k)\in D^{\Jmc'}$, and each two consecutive
                    elements of this sequence are connected via $s^\Jmc$;
            \end{itemize}
            or there is a sequence $(a,d_1\dots d_m)$, $(a,d_1\dots d_{m+1})$,
            \dots, $(a,d_1\dots d_{m+n})$ of elements of $\Delta^{\Jmc'}$ such
            that $n\ge 1$, $(a,d_1\dots d_{m+n})\in D^{\Jmc'}$, and each two
            consecutive elements of this sequence are connected via $s^\Jmc$,
            where $s$ is a role name such that either $n=1$ and $s=r$, or
            $\Imc\models s\sqsubseteq r$ and $\Imc\models\trans(s)$
        \item[\emph{iff}]
            either $m=0$ and
            \begin{itemize}
                \item there is a $d\in D^\Imc$ such that $(a^\Imc,d)\in r^\Imc$,
                    or
                \item there is an $s\in\NR$ with $\Imc\models s\sqsubseteq r$
                    and $\Imc\models\trans(s)$, and an $e_k\in\Delta^\Imc$ such
                    that $(a^\Imc,e_k)\in s^\Imc\subseteq r^\Imc$ and $e_k\in
                    D^\Imc$;
            \end{itemize}
            or there is a $d\in D^\Imc$ such that $(d_m,d)\in s^\Imc\subseteq
            r^\Imc$, where $s$ is a role name such that either $s=r$, or
            $\Imc\models s\sqsubseteq r$ and $\Imc\models\trans(s)$
        \item[\emph{iff}]
            either $m=0$ and $a^\Imc\in(\exists r.D)^\Imc$, or $d_m\in(\exists
            r.D)^\Imc$.
    \end{itemize}

    \noindent
    For the case where $C$ is of the form $\atLeast{n}{r}{D}$ for a simple role
    name $r$, we again have $r^{\Jmc'}=r^\Jmc$, and thus
    \begin{itemize}
        \item[]
            $(a,d_1\dots d_m)\in(\atLeast{n}{r}{D})^{\Jmc'}$
        \item[\emph{iff}]
            there is a subset $X\subseteq D^{\Jmc'}$ with $\lvert X\rvert=n$
            such that $((a,\varepsilon),x)\in r^\Jmc$ for every $x\in X$, and
            either
            \begin{itemize}
                \item $m=0$ and every $x\in X$ is either of the form
                    $(b,\varepsilon)$ or $(a,d)$, or
                \item every $x\in X$ is of the form $(a,d_1\dots d_m d_{m+1})$
            \end{itemize}
        \item[\emph{iff}]
            there is a subset $Y\subseteq D^\Imc$ with $\lvert Y\rvert=n$ such
            that $m=0$ and $(a^\Imc,y)\in r^\Imc$ for every $y\in Y$, or
            $(d_m,y)\in r^\Imc$ for every $y\in Y$
        \item[\emph{iff}]
            either $m=0$ and $a^\Imc\in(\atLeast{n}{r}{D})^\Imc$, or
            $d_m\in(\atLeast{n}{r}{D})^\Imc$.
    \end{itemize}
    %
    The second equivalence holds since each $r^\Imc$-successor of a named
    domain element $a^\Imc\in\Delta^\Imc$ is represented by \emph{exactly one}
    $r^{\Jmc'}$-successor of $(a,\varepsilon)\in\Delta^{\Jmc'}$, which holds due
    the fact that $\Delta^{\Jmc'}$ does not contain domain elements of the form
    $(a,b^\Imc)$ for $b\in\Ind(\Psi)$.
    %
    This finishes the proof of Claim~\ref{claim:forest-model-concepts}.

    It remains only to be shown that $\Jmc'$ is indeed a model of~\Bmc that
    respects~\Dmc.  For this, we prove first the following claim by structural
    induction.

    \begin{claim}\label{claim:forest-model-fcl}
        For every $\psi\in\FCl(\Psi)$, we have $\Jmc'\models\psi$ iff
        $\Imc\models\psi$.
    \end{claim}

    \noindent
    For the first base case, assume that $\psi$ is of the form $A(a)$ for some
    $A\in\NC$ and $a\in\NI$.  We have $a^\Imc\in A^\Imc$ iff
    $a^{\Jmc'}=a^\Jmc=(a,\varepsilon)\in A^\Jmc=A^{\Jmc'}$ by definition.

    For the second base case, assume that $\psi$ is of the form $r(a,b)$ for
    some $r\in\NR$ and $a,b\in\NI$.  If $\Imc\models r(a,b)$, then
    $(a^\Imc,b^\Imc)\in r^\Imc$, and thus
    \[(a^{\Jmc'},b^{\Jmc'})=(a^\Jmc,b^\Jmc)=((a,\varepsilon),(b,\varepsilon))\in
        r^\Jmc\subseteq r^{\Jmc'}.\]
    Conversely, if $((a,\varepsilon),(b,\varepsilon))\in r^{\Jmc'}$, then there
    is an $s\in\NR$ and a sequence $(a_0,\varepsilon)$,~\dots,
    $(a_n,\varepsilon)$ of elements of $\Delta^{\Jmc'}$ with $n\ge 1$ such that
    $a_0=a$, $a_n=b$, each two consecutive elements of this sequence are
    connected via $s^\Jmc$, and either $n=1$ and $s=r$, or $\Rmc\models
    s\sqsubseteq r$ and $\Rmc\models\trans(s)$.  By the definition of~$s^\Jmc$,
    the properties of~$s$, and since $\Imc\models\Rmc$, we can infer that
    $(a^\Imc,b^\Imc)\in r^\Imc$, and thus $\Imc\models r(a,b)$.

    For the third base case, assume that $\psi$ is of the form $C\sqsubseteq D$.
    For the \enquote{if} direction, assume that $\Imc\models C\sqsubseteq D$ and
    thus $C^\Imc\subseteq D^\Imc$.  Suppose that there is a $(a,d_1\dots d_m)\in
    C^{\Jmc'}$ with $(a,d_1\dots d_m)\notin D^{\Jmc'}$.  By
    Claim~\ref{claim:forest-model-concepts}, either $m=0$ and we have $a^\Imc\in
    C^\Imc$ and $a^\Imc\notin D^\Imc$, or $d_m\in C^\Imc$ and $d_m\notin
    D^\Imc$, which contradicts our assumption that $C^\Imc\subseteq D^\Imc$.

    For the \enquote{only if} direction, assume that $C^{\Jmc'}\subseteq
    D^{\Jmc'}$.  Suppose that there is a $d\in C^\Imc$ with $d\notin D^\Imc$.
    By the definition of $\Delta^{\Jmc'}$, we have $(a,d'd)\in\Delta^{\Jmc'}$
    for any $a\in\Ind(\Psi)$ and $d'\in\Delta^\Imc$ such that there is no
    $b\in\Ind(\Psi)$ with $d'=b^\Imc$.  By
    Claim~\ref{claim:forest-model-concepts}, we obtain $(a,d'd)\in C^{\Jmc'}$
    and $(a,d'd)\notin D^{\Jmc'}$, which again yields a contradiction.

    For the induction step, assume first that $\psi$ is of the form
    $\lnot\psi'$.  We have that $\Jmc'\models\lnot\psi'$ iff
    $\Jmc'\not\models\psi'$ iff $\Imc\not\models\psi'$ iff
    $\Imc\models\lnot\psi'$.
    %
    Assume now that $\psi$ is of the form $\psi_1\land\psi_2$.  We have that
    $\Jmc'\models\psi_1\land\psi_2$ iff $\Jmc'\models\psi_1$ and
    $\Jmc'\models\psi_2$ iff $\Imc\models\psi_1$ and $\Imc\models\psi_2$ iff
    $\Imc\models\psi_1\land\psi_2$.
    %
    This finishes the proof of Claim~\ref{claim:forest-model-fcl}.

    Since $\Psi\in\FCl(\Psi)$, this shows that $\Jmc'$ is indeed a model
    of~$\Psi$.
    %
    We show that $\Jmc'$ is also a model of~\Rmc in the following claim.

    \begin{claim}\label{claim:forest-model-rbox}
        For every $\alpha\in\Rmc$, we have $\Jmc'\models\alpha$.
    \end{claim}

    \noindent
    Assume first that $\alpha$ is of the form $r\sqsubseteq s$.  Since
    $\Imc\models\Rmc$, we have $\Imc\models r\sqsubseteq s$ and thus
    $r^\Imc\subseteq s^\Imc$.  We first show that $r^\Jmc\subseteq s^\Jmc$.  For
    this, take $(x,y)\in r^\Jmc$.  There are three cases to consider:
    \begin{itemize}
        \item If $x=(a,\varepsilon)$ and $y=(b,\varepsilon)$ with
            $a,b\in\Ind(\Psi)$, we have $(a^\Imc,b^\Imc)\in r^\Imc$ and thus
            $(a^\Imc,b^\Imc)\in s^\Imc$.  Hence, the definition of $s^\Jmc$
            yields that $(x,y)\in s^\Jmc$.
        \item If $x=(a,\varepsilon)$ and $y=(a,d)$ with
            $a\in\Ind(\Psi)$ and $d\in\Delta^\Imc$, we have $(a^\Imc,d)\in
            r^\Imc$ and thus $(a^\Imc,d)\in s^\Imc$.  Again, the definition of
            $s^\Jmc$ yields that $(x,y)\in s^\Jmc$.
        \item If we have $x=(a,d_1\dots d_m)$ and $y=(a,d_1\dots d_m d_{m+1})$
            with $a\in\Ind(\Psi)$, $m>0$, and $d_1,\dots,d_{m+1}\in\Delta^\Imc$,
            we have also $(d_m,d_{m+1})\in r^\Imc$ and thus
            $(d_m,d_{m+1})\in s^\Imc$.  Again, the definition of $s^\Jmc$ yields
            that $(x,y)\in s^\Jmc$.
    \end{itemize}
    %
    To show that $r^{\Jmc'}\subseteq s^{\Jmc'}$, take $(x,y)\in r^{\Jmc'}$.  If
    $(x,y)\in r^\Jmc$, we have $(x,y)\in s^\Jmc$ and thus $(x,y)\in s^{\Jmc'}$.
    Otherwise, we have that $(x,y)\in(t^\Jmc)^+$ with $\Rmc\models t\sqsubseteq
    r$ and $\Rmc\models\trans(t)$.  Since $r\sqsubseteq s\in\Rmc$, we have
    obviously $\Rmc\models r\sqsubseteq s$.  It is easy to see that this implies
    $\Rmc\models t\sqsubseteq s$.  Then the definition of $s^{\Jmc'}$ yields
    that $(t^\Jmc)^+\subseteq s^{\Jmc'}$.  Hence $(x,y)\in s^{\Jmc'}$.

    Assume now that $\psi$ is of the form $\trans(r)$.  Since $\Imc\models\Rmc$,
    we have $\Imc\models\trans(r)$ and thus $r^\Imc\circ r^\Imc\subseteq
    r^\Imc$.  By the same arguments as above, we have for every $t\in\NR$ with
    $t^\Imc\subseteq r^\Imc$, that $t^\Jmc\subseteq r^\Jmc$, and thus
    $(t^\Jmc)^+\subseteq(r^\Jmc)^+$ since the transitive closure is monotonic.
    Since $r^\Imc\subseteq r^\Imc$, we have also $\Imc\models r\sqsubseteq r$.
    The definition of $r^{\Jmc'}$ yields now that $r^{\Jmc'}=(r^\Jmc)^+$, and
    hence $\Jmc'$ is a model of $\trans(r)$.
    %
    This finishes the proof of Claim~\ref{claim:forest-model-rbox}.

    Claim~\ref{claim:forest-model-fcl}, the fact that $\Psi\in\FCl(\Psi)$, and
    Claim~\ref{claim:forest-model-rbox} yield that $\Jmc'$ is indeed a model
    of $\Bmc=(\Psi,\Rmc)$.
    %
    Finally, we show that $\Jmc'$ respects $\Dmc=(\Umc,\Ymc)$.
    %
    Since \Imc respects~\Dmc, we have
    \[\Ymc=\bigl\{Y\subseteq\Umc\mid%
        \text{there is a $d\in\Delta^\Imc$ with $d\in(C_{\Umc,Y})^\Imc$}\bigr\},\]
    where $C_{\Umc,Y}$ is defined as in Definition~\ref{def:respects-dmc}.
    %
    We now set $\Dmc':=(\Umc,\Ymc')$ where
    \[\Ymc':=\bigl\{Y\subseteq\Umc\mid%
        \text{there is an $x\in\Delta^{\Jmc'}$ with $x\in(C_{\Umc,Y})^{\Jmc'}$}\bigr\},\]
    and show that $\Dmc=\Dmc'$.
    %
    Since $\Jmc'$ respects~$\Dmc'$, this implies that $\Jmc'$ respects~\Dmc.

    For the direction~($\subseteq$), assume that $Y\in\Ymc$, and thus there is a
    $d\in(C_{\Umc,Y})^\Imc$.  By Claim~\ref{claim:forest-model-concepts} and the
    definition of~$\Delta^{\Jmc'}$, there is a $(a,d'd)\in(C_{\Umc,Y})^{\Jmc'}$,
    and hence $Y\in\Ymc'$.
    %
    Conversely, for the direction~($\supseteq$), assume that $Y\in\Ymc'$,
    i.e.~there is a $(a,d_1\dots d_m)\in(C_{\Umc,Y})^{\Jmc'}$.  By
    Claim~\ref{claim:forest-model-concepts} and the definition
    of~$\Delta^{\Jmc'}$, there is a $d\in(C_{\Umc,Y})^\Imc$, where for $m=0$, we
    can set $d:=a^\Imc$, and for $m>0$, we can take $d:=d_m$.  Hence, we have
    that $Y\in\Ymc$.
\end{proof}

\noindent
Moreover, we can extend the aforementioned result about the non-entailment
problem for UCQs from~\cite{GHL+-JAIR08,Lut-IJCAR08} to our setting.  In the
following, we assume that the UCQ~$\rho$ contains only individual names that
also occur in the ABox (or Boolean axiom formula).  This is without loss of
generality, because for any individual name~$a$ not occurring in the ABox (or
Boolean axiom formula), we can simply add the assertion $A(a)$ to the ABox,
where $A$ is a fresh concept name.

\begin{lemma}\label{lem:forest-model-non-entailment}
    Let $\Kmc:=(\Amc,\Tmc,\Rmc)$ be a knowledge base, let $\rho$ be
    a union of Boolean CQs, and let $\Dmc=(\Umc,\Ymc)$ be a pair
    such that \Umc is a set of concept names and $\Ymc\subseteq 2^\Umc$.
    %
    Then, we have $\Kmc\not\models\rho$ w.r.t.~\Dmc iff there is a forest
    model~\Jmc of~\Kmc that respects \Dmc with $\Jmc\not\models\rho$.
\end{lemma}

\begin{proof}
    The \enquote{if} direction is trivial.  For the \enquote{only if} direction,
    assume that there is a model $\Imc=(\Delta^\Imc,\cdot^\Imc)$ of~\Kmc that
    respects~\Dmc such that $\Imc\not\models\rho$.  As shown in the proof of
    Lemma~\ref{lem:forest-model}, the model~\Imc can be transformed into a
    forest model $\Jmc'=(\Delta^{\Jmc'},\cdot^{\Jmc'})$ of~\Kmc that
    respects~\Dmc.
    %
    Assume that \Jmc and $\Jmc'$ are the forest base and the forest model,
    respectively, obtained from~\Imc as in the proof of
    Lemma~\ref{lem:forest-model}.

    It is left to be shown that $\Jmc'\not\models\rho$.  Assume to the contrary
    that $\Jmc'\models\rho$.  Then, there is a Boolean CQ~$\rho_i$ in the
    UCQ~$\rho$ such that there is a homomorphism~$\pi$ from~$\rho_i$
    into~$\Jmc'$.  We define a homomorphism~$\pi'$ from~$\rho_i$ into~\Imc as
    follows:
    \begin{itemize}
        \item $\pi'(a):=a^\Imc$ for every individual name~$a\in\Ind(\Amc)$; and
        \item for every $v\in\Var(\rho_i)$, we define
            \[\pi'(v):=\begin{cases}
                    a^\Imc
                    &\text{if $\pi(v)=(a,\varepsilon)$ with $a\in\Ind(\Amc)$, and}\\
                    d_m
                    &\text{if $\pi(v)=(a,d_1\dots d_m)$ with $m>0$.}
                \end{cases}\]
    \end{itemize}
    %
    We now show that $\pi'$ is indeed a homomorphism from~$\rho_i$ into~\Imc.

    Consider first a concept atom $A(a)\in\At(\rho_i)$ with $a\in\Ind(\Amc)$.
    Since $\pi$ is a homomorphism from~$\rho_i$ into~$\Jmc'$, we have
    $\pi(a)=a^{\Jmc'}=(a,\varepsilon)\in A^{\Jmc'}$.  By
    Claim~\ref{claim:forest-model-concepts}, we obtain
    $a^\Imc\in A^\Imc$, and thus $\pi'(a)\in A^\Imc$.

    Similarly, for a concept atom $A(v)\in\At(\rho_i)$ with $v\in\Var(\rho_i)$,
    we have $\pi(v)\in A^{\Jmc'}$, and thus $\pi'(v)\in A^\Imc$ again by
    Claim~\ref{claim:forest-model-concepts}.

    For a role atom $r(a,b)\in\At(\rho_i)$, we can show
    $(\pi'(a),\pi'(b))=(a^\Imc,b^\Imc)\in r^\Imc$ using the same arguments as in
    the proof of Claim~\ref{claim:forest-model-fcl}: Since $\pi$ is a
    homomorphism from~$\rho_i$ into~$\Jmc'$, we have
    $(\pi(a),\pi(b))=(a^{\Jmc'},b^{\Jmc'})=((a,\varepsilon),(b,\varepsilon))\in r^{\Jmc'}$,
    and thus that there is an $s\in\NR$ and a sequence
    $(a_0,\varepsilon)$,~\dots, $(a_n,\varepsilon)$ of elements of
    $\Delta^{\Jmc'}$ with $n\ge 1$ such that $a_0=a$, $a_n=b$, each two
    consecutive elements of this sequence are connected via~$s^\Jmc$, and either
    $n=1$ and $s=r$, or $\Rmc\models s\sqsubseteq r$ and $\Rmc\models\trans(s)$.
    By the definition of~$s^\Jmc$, the properties of~$s$, and since
    $\Imc\models\Rmc$, we can infer that $(a^\Imc,b^\Imc)\in r^\Imc$.

    If there is a role atom of the form $r(a,v)\in\At(\rho_i)$ with
    $a\in\Ind(\Amc)$ and $v\in\Var(\rho_i)$, we have
    $(\pi(a),\pi(v))=(a^{\Jmc'},\pi(v))=((a,\varepsilon),\pi(v))\in r^{\Jmc'}$.
    If $\pi(v)=(b,\varepsilon)$ with $b\in\Ind(\Amc)$, we can argue as in the
    previous case that $(a^\Imc,b^\Imc)=(\pi'(a),\pi'(b))\in r^\Imc$.  Otherwise
    there are again two cases to consider.
    %
    First, if we have $((a,\varepsilon),\pi(v))\in r^\Jmc$, then we have also
    $(a^\Imc,\pi'(v))=(\pi'(a),\pi'(v))\in r^\Imc$ by the definitions of~\Jmc
    and~$\pi'$.  Otherwise, there must be a role name $s\in\NR$ such that
    $\Rmc\models s\sqsubseteq r$, $\Rmc\models\trans(s)$, and
    $((a,\varepsilon),\pi(v))\in(s^\Jmc)^+$.  This yields that there is a
    sequence $(a_0,\varepsilon)$, $(a_1,\varepsilon)$,~\dots,
    $(a_n,\varepsilon)$, $(a_n,e_1)$,~\dots, $(a_n,e_1\dots e_k)$ of elements
    of~$\Delta^{\Jmc'}$ with $n\ge 1$ and $k\ge 1$ such that $a_0=a$,
    $\pi(v)=(a_n,e_1\dots e_k)$, and each two consecutive elements of this
    sequence are connected via~$s^\Jmc$.  By the definitions of~$s^\Jmc$
    and~$\pi'$, we obtain $(a^\Imc,\pi'(v))\in s^\Imc$, and thus since
    $s^\Imc\subseteq r^\Imc$, also $(\pi'(a),\pi'(v))\in r^\Imc$.

    For any role atom $r(v,a)\in\At(\rho_i)$ with $v\in\Var(\rho_i)$ and
    $a\in\Ind(\Amc)$, we have that
    $(\pi(v),\pi(a))=(\pi(v),a^{\Jmc'})=(\pi(v),(a,\varepsilon))\in r^{\Jmc'}$.
    By the definition of $r^{\Jmc'}$, this implies that there is a sequence
    $(a_0,\varepsilon)$,~\dots, $(a_n,\varepsilon)$ of elements
    of~$\Delta^{\Jmc'}$ with $n\ge 1$ such that $a_n=a$,
    $\pi(v)=(a_0,\varepsilon)$, and each two consecutive elements of this
    sequence are connected via~$s^\Jmc$, where $s$ is a role name such that
    either $n=1$ and $s=r$, or $\Rmc\models s\sqsubseteq r$ and
    $\Rmc\models\trans(s)$.  By the definitions of~$s^\Jmc$ and~$\pi'$, the
    properties of~$s$, and since $\Imc\models\Rmc$, this yields that
    $(\pi'(v),\pi'(a))=(\pi'(v),a^\Imc)=(a_0^\Imc,a_n^\Imc)\in r^\Imc$.

    Finally, consider a role atom $r(v,v')\in\At(\rho_i)$ with
    $v,v'\in\Var(\rho_i)$.  Again, since $\pi$ is a homomorphism from~$\rho_i$
    into~$\Jmc'$, we have $(\pi(v),\pi(v'))\in r^{\Jmc'}$.  If
    $\pi(v)=(a,\varepsilon)$ for some $a\in\Ind(\Amc)$, we can show as in
    second-last case that $(\pi'(v),\pi'(v'))=(a^\Imc,\pi'(v'))\in r^\Imc$.
    Otherwise, we have $\pi(v)=(a,d_1\dots d_m)$ with $m>0$ and that there is a
    sequence $(a,d_1\dots d_m)$, $(a,d_1\dots d_{m+1})$,~\dots,
    $(a,d_1\dots d_{m+n})$ of elements of the domain~$\Delta^{\Jmc'}$ such that
    we have that $n\ge 1$, $\pi(v')=(a,d_1\dots d_{m+n})$, and each two
    consecutive elements of this sequence are connected via~$s^\Jmc$, where $s$
    is a role name such that either $n=1$ and $s=r$, or
    $\Rmc\models s\sqsubseteq r$ and $\Rmc\models\trans(s)$.  Thus, we obtain
    $(\pi'(v),\pi'(v'))=(d_m,d_{m+n})\in s^\Imc\subseteq r^\Imc$ by similar
    arguments as before.

    Hence, $\pi'$ is a homomorphism from~$\rho_i$ into~\Imc, and thus
    $\Imc\models\rho_i$.  But this yields that $\Imc\models\rho$, which
    contradicts our assumption that $\Imc\not\models\rho$.
\end{proof}

\noindent
Recall that we want to decide the existence of such a forest model in time
exponential in the size of $\Kmc=(\Amc,\Tmc,\Rmc)$, and~$\rho$.
%
To achieve this, we further reduce this decision problem following an idea
from~\cite{Lut-IJCAR08}.  There, the notion of a spoiler is introduced.
According to~\cite{Lut-IJCAR08}, a \emph{spoiler for~\Kmc and~$\rho$} is an
\SHQcap-knowledge base $\Kmc'=(\Amc',\Tmc',\emptyset)$ that states
properties that must be satisfied such that $\rho$ is not entailed by~\Kmc.
Note that the ABox~$\Amc'$ of such a spoiler may also contain \emph{negated}
assertions.  Furthermore, a spoiler may contain role conjunctions.  Thus,
according to our notation, $\Kmc'$ is a Boolean \SHQcap-knowledge base,
i.e.~$\Kmc'=(\bigwedge\Amc'\land\bigwedge\Tmc',\emptyset)$.  The following
proposition is shown in~\cite{Lut-IJCAR08}.

\begin{proposition}
    Let $\Kmc:=(\Amc,\Tmc,\Rmc)$ be a \SHQ-knowledge base, and let
    $\rho$ be a union of Boolean CQs.
    %
    Then, we have $\Kmc\not\models\rho$ iff there is a spoiler
    $\Kmc'=(\bigwedge\Amc'\land\bigwedge\Tmc',\emptyset)$ for~\Kmc
    and~$\rho$ such that
    $(\bigwedge\Amc\land\bigwedge\Amc'\land\bigwedge\Tmc\land\bigwedge\Tmc',\Rmc)$
    is consistent.
\end{proposition}

\noindent
In addition, it is shown in~\cite{Lut-IJCAR08} that all spoilers for~\Kmc
and~$\rho$ can be computed in time exponential in the size of~\Kmc and~$\rho$,
and that each spoiler is of polynomial size.
%
In the proof of these results, one only has to deal with forest models, which
furthermore do not need to be modified.
%
More formally, for any forest model~\Imc of $(\Amc,\Tmc,\Rmc)$ that
does not satisfy~$\rho$ there is a spoiler
$(\bigwedge\Amc'\land\bigwedge\Tmc',\emptyset)$ that also has \Imc
as a model and, conversely, every forest model of the knowledge base
$(\Amc,\Tmc,\Rmc)$ that also satisfies a spoiler
$(\bigwedge\Amc'\land\bigwedge\Tmc',\emptyset)$ does not
satisfy~$\rho$ (see the proof of Lemma~3 in~\cite{Lut-DL08}).
%
This yields the following more general result that also takes into account the
pair~\Dmc.

\begin{proposition}\label{prop:spoilers}
    Let $\Kmc:=(\Amc,\Tmc,\Rmc)$ be a \SHQ-knowledge base, let
    $\rho$ be a union of Boolean CQs, and let $\Dmc=(\Umc,\Ymc)$ be a pair
    such that \Umc is a set of concept names and $\Ymc\subseteq 2^\Umc$.
    %
    Then, we have $\Kmc\not\models\rho$ w.r.t.~\Dmc iff there is a spoiler
    $\Kmc'=(\bigwedge\Amc'\land\bigwedge\Tmc',\emptyset)$ for~\Kmc
    and~$\rho$ such that there is a model of
    $(\bigwedge\Amc\land\bigwedge\Amc'\land\bigwedge\Tmc\land\bigwedge\Tmc',\Rmc)$
    that respects~\Dmc.
\end{proposition}

\noindent
It remains to show that the existence of such a model can be checked in time
exponential in the size of
$(\bigwedge\Amc\land\bigwedge\Amc'\land\bigwedge\Tmc\land\bigwedge\Tmc',\Rmc)$.
%
But this follows directly from Theorem~\ref{thm:cons-bmc-dmc}, where it is shown
that consistency of Boolean \SHOQcap-knowledge base~\Bmc w.r.t.\ a pair~\Dmc can be
decided in time exponential in the size of~\Bmc.
%
Thus, we obtain the following theorem.

\begin{theorem}\label{thm:non-entailment-exp}
    Let $\Kmc:=(\Amc,\Tmc,\Rmc)$ be a \SHQ-knowledge base, let
    $\rho$ be a union of Boolean CQs, and let $\Dmc=(\Umc,\Ymc)$ be a pair
    such that \Umc is a set of concept names and $\Ymc\subseteq 2^\Umc$.
    %
    Then, we can decide whether $\Kmc\not\models\rho$ w.r.t.~\Dmc in time
    exponential in the size of~\Kmc and~$\rho$.
\end{theorem}

\noindent
Combining this with the reductions in Lemmas~\ref{lem:initial-validity}
and~\ref{lem:tcq-dmc-tau-combined}, we obtain the desired complexity result.

\begin{theorem}\label{thm:combined-complexity-rigid-concept-names}
    If $\NRC\ne\emptyset$ and $\NRR=\emptyset$, the
    temporalised query-entailment problem in \SHQ is in \coNExpTime w.r.t.\
    combined complexity.
\end{theorem}

\begin{proof}
    Let $\phi$ be a Boolean TCQ, and let
    $\Kmc=((\Amc_i)_{0\le i\le n},\Tmc,\Rmc)$ be a temporal \SHQ-knowledge base.
    %
    By Lemma~\ref{lem:initial-validity}, we can construct a Boolean TCQ~$\psi$
    polynomial in the size of~$\phi$ and~\Kmc such that $\Kmc\models\phi$ iff
    $\Kmc':=(\emptyset,\Tmc,\Rmc)\models\psi$.
    %
    We again consider the TCQ-satisfiability problem, which has the same
    complexity as the temporalised query non-entailment problem.  We decide
    whether $\psi$ is satisfiable w.r.t.~$\Kmc'$ using
    Lemma~\ref{lem:tcq-r-sat-t-sat}.  For that, let
    $\psf\colon\CQ(\psi)\to\Pmc_\psi$ be a bijection.
    %
    We first non-deterministically guess a set
    $\Wmc=\{X_1,\dots,X_k\}\subseteq 2^{\Pmc_\psi}$ and a mapping
    $\iota\colon\{0\}\to\{1,\dots,k\}$ in time exponential in the size of~$\psi$
    and~$\Kmc'$.
    %
    By Lemma~\ref{lem:tcq-t-sat}, deciding whether $\psi^\psf$ is t-satisfiable
    w.r.t.~\Wmc and~$\iota$ can be done in time exponential in the size
    of~$\psi^\psf$ (and thus also in time exponential in the size of~$\psi$),
    linear in the size of~\Wmc, and polynomial in~$n$.

    Thus, due to Lemma~\ref{lem:tcq-r-sat-t-sat}, is suffices to show that
    r-satisfiability of~\Wmc w.r.t.~$\iota$ and~$\Kmc'$ can be checked in
    non-deterministic exponential time w.r.t.\ combined complexity.  For that,
    we use Lemma~\ref{lem:tcq-dmc-tau-combined}.
    %
    We non-deterministically guess a set $\Ymc\subseteq 2^{\RCon(\Tmc)}$ and a
    mapping $\tau\colon\Ind(\psi)\to\Ymc$, which can be done in
    time exponential in the size of~$\psi$ and~$\Kmc'$ since \Ymc is of size
    exponential in~\Tmc and~$\tau$ is of size polynomial in the size of~$\psi$
    and~\Tmc.
    %
    We define $\Dmc:=(\RCon(\Tmc),\Ymc)$.  Next, we construct for every~$i$,
    $1\le i\le k$, the conjunction of CQ-literals
    $\zeta_{X_i}\land\xi_\tau$, the knowledge base~$\Kmc_i$ and the Boolean
    union of CQs~$\rho_i$.
    %
    Note that the size of each $\zeta_{X_i}\land\xi_\tau$, $\Kmc_i$,
    and~$\rho_i$ is polynomial in the size of~$\psi$ and~$\Kmc'$ and the
    number~$k$ of these conjunctions is exponential in the size of~$\psi$.
    %
    Thus, it remains to show that we can decide every query non-entailment
    $\Kmc_i\not\models\rho_i$ w.r.t.~\Dmc in time exponential in the size
    of~$\Kmc_i$ and~$\rho_i$, and thus in time exponential in the size of~$\psi$
    and~$\Kmc'$, which we obtain by Theorem~\ref{thm:non-entailment-exp}.

    Hence, we can check whether~$\phi$ is satisfiable w.r.t.~\Kmc using the
    above decision procedure, which shows that the TCQ-satisfiability problem in
    \SHQ is in \NExpTime w.r.t.\ combined complexity.  Thus, we obtain that the
    temporalised query-entailment problem is in \coNExpTime w.r.t.\ combined
    complexity.
\end{proof}

\noindent
Together with Theorem~\ref{thm:lower-bounds-combined-complexity}, we obtain that
the temporalised query-entailment problem in \SHQ is \coNExpTime-complete
w.r.t.\ combined complexity if only concept names are allowed to be rigid.


\section{Summary}\label{sec:tcqs-summary}

In this chapter, we have shown all the complexity results that are summarised in
Table~\ref{tab:tcq-results} for the proposed temporal query language.  More
precisely, we considered both the combined complexity and the data complexity of
temporalised query entailment for all description logics between \ALC and \SHQ
in the settings where (i)~neither concept names nor role names are allowed to be
rigid, (ii)~only concept names may be rigid, and (iii)~both concept names and
role names may be rigid.  It turned out that in Setting~(i), the temporalised
query-entailment problem is as hard as entailment of conjunctive queries w.r.t.\
atemporal \ALC- and \SHQ-knowledge bases, namely \coNP-complete w.r.t.\ data
complexity and \ExpTime-complete w.r.t.\ combined complexity.  However, if we
allow rigid concept names (but no rigid role names), the picture changes.
Whilst the data complexity remains the same as in the atemporal case, the
combined complexity of the temporalised query-entailment problem increases to
\coNExpTime, i.e.~the temporalised query non-entailment problem is as hard as
the satisfiability problem in the temporalised description logic \ALC-LTL\@.  If
we further allow rigid role names, the combined complexity of the temporalised
query (non-)entailment problem again increases in accordance with the complexity
of the satisfiability problem in \ALC-LTL\@.  In fact, all three problems are
\TwoExpTime-complete.  For the data complexity, it is still open whether adding
rigid role names results in an increase of the complexity.  We have shown an
upper bound of \ExpTime---which is one exponential better than the combined
complexity---, but the only lower bound we have is the trivial one of \coNP.

Further work will include trying to close this gap.  Moreover, it would be
interesting to find out what effect the addition of inverse roles has on the
complexity of query entailment in the temporal case.  Given the results for
\ALCI and \SHIQ in the atemporal case, where the query entailment problem is
\TwoExpTime-complete w.r.t.\ combined complexity~\cite{Lut-IJCAR08} and
\coNP-complete w.r.t.\ data complexity~\cite{OrCE-AAAI06}, there is the
possibility that also in the temporal case, the query entailment problem remains
\coNP-complete w.r.t.\ data complexity and \TwoExpTime-complete w.r.t.\ combined
complexity for all three settings considered in this chapter.  But showing this
will require considerable extensions of the proof techniques employed until now
since the presence of inverse roles creates additional problems.
%
We have also left open the complexity of the temporalised query entailment
problem for the case where non-simple roles are allowed to occur in the queries.
This problem is, however, already \TwoExpTime-hard w.r.t.\ combined complexity
for the description logic~\SH~\cite{ELO+-IJCAI09} in the atemporal case.
