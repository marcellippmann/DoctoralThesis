\chapter{The Temporalised Description Logic \texorpdfstring{\SHOQ-LTL}{SHOQ-LTL}}\label{ch:shoq-ltl}

Temporalised description logics extend description logics with temporal
modalities.  As discussed in Section~\ref{sec:temporalised-dls}, there are
many different approaches to temporalising description logics.  In this thesis,
we follow the approach that was taken by introducing
\ALC-LTL~\cite{BaGL-ToCL12}, i.e.~a combination of \ALC with propositional LTL
where \ALC-axioms replace propositional variables, and temporal operators are
only allowed to occur in front of \ALC-axioms rather than inside of concepts.

The temporalised description logic \SHOQ-LTL, which we examine in this chapter,
generalises \ALC-LTL\@.  In fact, several constructions in the present chapter
are adaptations of those for \ALC-LTL, in particular the ones used to show
Lemmas~4.3 and~6.4 in~\cite{BaGL-ToCL12}.
%
Some of the results of this chapter have already been published
in~\cite{BaLi-LTCS-14-01}.
%in~\cite{tcq-journal,BaLi-LTCS-14-01}.

This chapter is organised as follows.  In
Section~\ref{sec:syntax-semantics-shoq-ltl}, we formally introduce the
temporalised description logic \SHOQ-LTL\@.  Then, in
Section~\ref{sec:complexity-shoq-ltl}, we show complexity results for the
satisfiability problem in \SHOQ-LTL\@.  Finally, we provide a brief summary of
the obtained results in Section~\ref{sec:shoq-ltl-summary}.


\section{Syntax and Semantics of \texorpdfstring{\SHOQ-LTL}{SHOQ-LTL}}\label{sec:syntax-semantics-shoq-ltl}

The temporalised description logic \SHOQ-LTL combines the description logic
\SHOQ with the temporal logic LTL\@.  Its syntax is very similar to the one of
propositional LTL (see Definition~\ref{def:syntax-ltl}), but in \SHOQ-LTL
propositional variables are replaced by \SHOQ-axioms.

\begin{definition}[Syntax of \SHOQ-LTL]
    Let \Rmc be an RBox.  The set of \emph{\SHOQ-LTL-formulas w.r.t.~\Rmc} is the smallest
    set such that
    \begin{itemize}
        \item every ABox-axiom and every TBox-axiom in which at-least
            restrictions contain only simple roles w.r.t.~\Rmc is a
            \SHOQ-LTL-formula w.r.t.~\Rmc; and
        \item if $\phi_1$ and $\phi_2$ are \SHOQ-LTL-formulas w.r.t.~\Rmc, then
            so are: $\lnot\phi_1$ (negation), $\phi_1\land\phi_2$ (conjunction),
            $\Next\phi_1$ (next), $\Previous\phi_1$ (previous),
            $\phi_1\Until\phi_2$ (until), and $\phi_1\Since\phi_2$ (since).
    \end{itemize}
\end{definition}

\noindent
We denote the set of axioms occurring in a \SHOQ-LTL-formula~$\phi$ by
$\Ax(\phi)$.  Clearly, the cardinality of~$\Ax(\phi)$ is bounded by the size
of~$\phi$.  Similar to propositional LTL, we use
\begin{itemize}
    \item $\phi_1\lor\phi_2$ (disjunction) as an abbreviation for
        $\lnot(\lnot\phi_1\land\lnot\phi_2)$;
    \item $\phi_1\to\phi_2$ (implication) as an abbreviation for
        $\lnot\phi_1\lor\phi_2$;
    \item \true as an abbreviation for an arbitrary but fixed
        tautology such as $A(a)\lor\lnot A(a)$ with $A\in\NC$ and $a\in\NI$;
    \item \false as an abbreviation for $\lnot\true$;
    \item $\Diamond\phi$ (diamond, which should be read as \enquote{eventually}
        or \enquote{some time in the future}) as an abbreviation for
        $\true\Until\phi$;
    \item $\Box\phi$ (box, which should be read as \enquote{always} or
        \enquote{always in the future}) as an abbreviation for
        $\lnot\Diamond\lnot\phi$;
    \item $\Diamondm\phi$ (which should be read as \enquote{once}
        or \enquote{some time in the past}) as an abbreviation for
        $\true\Since\phi$; and
    \item $\Boxm\phi$ (which should be read as \enquote{historically} or
        \enquote{always in the past}) as an abbreviation for
        $\lnot\Diamondm\lnot\phi$.
\end{itemize}

The semantics of \SHOQ-LTL is based on DL-LTL-structures, which are infinite
sequences of interpretations over the same non-empty domain~$\Delta$
(\emph{constant-domain assumption}).
%
As discussed in Section~\ref{sec:temporalised-dls}, for some concept and role
names it is also not desirable that their interpretation changes over time.
%
Thus, we assume in the following that a subset of the set of concept and role
names can be designated as being rigid.  Let \NRC denote the set of \emph{rigid
concept names} and \NRR the set of \emph{rigid role names} where
$\NRC\subseteq\NC$ and $\NRR\subseteq\NR$.  All concept and role names in
$\NC\setminus\NRC$ and $\NR\setminus\NRR$ are then called \emph{flexible}.
%
Moreover, we make the rigid-individual assumption, i.e.~we assume that every
individual name stands for a unique element of the domain~$\Delta$.

\begin{definition}[DL-LTL-structure]\label{def:dl-ltl-structure}
    We call an infinite sequence $\Imf=(\Imc_i)_{i\ge 0}$ of interpretations
    $\Imc_i=(\Delta,\cdot^{\Imc_i})$ a DL-LTL-structure if
    \begin{itemize}
        \item $a^{\Imc_i}=a^{\Imc_j}$ holds for every $a\in\NI$ and all
            $i,j\ge 0$ (\emph{rigid-individual assumption});
        \item $A^{\Imc_i}=A^{\Imc_j}$ holds for every $A\in\NRC$ and all
            $i,j\ge 0$; and
        \item $r^{\Imc_i}=r^{\Imc_j}$ holds for every $r\in\NRR$ and all
            $i,j\ge 0$.
    \end{itemize}
\end{definition}

\noindent
This notion is now used to define the semantics of \SHOQ-LTL-formulas.

\begin{definition}[Semantics of \SHOQ-LTL]\label{def:semantics-shoq-ltl}
    Given a \SHOQ-LTL-formula~$\phi$ w.r.t.\ an RBox~\Rmc, a
    DL-LTL-structure $\Imf=(\Imc_i)_{i\ge 0}$, and a time point $i\ge 0$,
    \emph{validity of~$\phi$ in~\Imf at time~$i$} (written $\Imf,i\models\phi$)
    is defined inductively as follows:
    \[\begin{array}{lcl}
        \Imf,i\models\alpha
            &\text{iff}
            &\Imc_i\models\alpha\\
        \Imf,i\models\lnot\phi_1
            &\text{iff}
            &\Imf,i\not\models\phi_1,\ \text{i.e.~not}\ \Imf,i\models\phi_1\\
        \Imf,i\models\phi_1\land\phi_2
            &\text{iff}
            &\Imf,i\models\phi_1\ \text{and}\ \Imf,i\models\phi_2\\
        \Imf,i\models\Next\phi_1
            &\text{iff}
            &\Imf,i+1\models\phi_1\\
        \Imf,i\models\Previous\phi_1
            &\text{iff}
            &i>0\ \text{and}\ \Imf,i-1\models\phi_1\\
        \Imf,i\models\phi_1\Until\phi_2
            &\text{iff}
            &\text{there is some $k\geq i$ such that $\Imf,k\models\phi_2$, and}\\
            & &\text{$\Imf,j\models\phi_1$ for every $j$, $i\le j<k$}\\
        \Imf,i\models\phi_1\Since\phi_2
            &\text{iff}
            &\text{there is some $k$, $0\le k\le i$, such that $\Imf,k\models\phi_2$, and}\\
            & &\text{$\Imf,j\models\phi_1$ for every $j$, $k<j\le i$}
    \end{array}\]
    %
    If $\Imc_i\models\Rmc$ for every $i\ge 0$ (written $\Imf\models\Rmc$), and
    $\Imf,0\models\phi$, then we call \Imf a \emph{model} of~$\phi$ w.r.t.~\Rmc.
    %
    We call~$\phi$ \emph{satisfiable w.r.t.~\Rmc} if it has a model w.r.t.~\Rmc.

    The \emph{satisfiability problem in \SHOQ-LTL} is the problem of deciding,
    given a \SHOQ-LTL-formula~$\phi$ w.r.t.\ an RBox~\Rmc, whether $\phi$ is
    satisfiable w.r.t.~\Rmc.

    Moreover, two \SHOQ-LTL-formulas $\phi_1,\phi_2$ w.r.t.\ an RBox~\Rmc are
    \emph{equivalent} (written $\phi_1\equiv\phi_2$) if they have the same
    models w.r.t.~\Rmc.
\end{definition}

\noindent
In~\cite{BaGL-ToCL12}, the temporalised description logic \ALC-LTL, which is a
fragment of \SHOQ-LTL, is considered.  There, the authors show that
satisfiability in \ALC-LTL is \ExpTime-complete if no rigid names are present.
If rigid concept names are allowed, the problem becomes \NExpTime-complete, and
if additionally rigid role names are allowed, the problem becomes even
\TwoExpTime-complete.%
\footnote{Additional intermediate cases (such as the case where GCIs occurring
    in the formula are global, i.e.~required to hold at every point in time; or
    where only the temporal operator~$\Diamond$ may be used) are also considered
    in~\cite{BaGL-ToCL12}, but are not considered in this thesis.}
%
In the next section, we show that the same complexity bounds also apply to the
satisfiability problem in \SHOQ-LTL (see Table~\ref{tab:shoq-ltl-results}).%
\footnote{Thus, the complexity of the satisfiability problem in \Lmc-LTL is the
    same as of \ALC-LTL for any description logic~\Lmc between \ALC and \SHOQ.}

\begin{table}[t]
    \centering
    \caption{The complexity of the satisfiability problem in \SHOQ-LTL}
    \label{tab:shoq-ltl-results}
    \begin{tabular*}{\textwidth}{l@{\extracolsep{\fill}}ll}
        \toprule
        \emph{Setting~(i)}
            &\ExpTime-complete
            &(Theorems~\ref{thm:lower-bounds-shoq-ltl} and~\ref{thm:upper-bound-shoq-ltl-no-rigid})\\
        \midrule
        \emph{Setting~(ii)}
            &\NExpTime-complete
            &(Theorems~\ref{thm:lower-bounds-shoq-ltl} and~\ref{thm:upper-bound-shoq-ltl-rigid-concepts})\\
        \midrule
        \emph{Setting~(iii)}
            &\TwoExpTime-complete
            &(Theorems~\ref{thm:lower-bounds-shoq-ltl} and~\ref{thm:upper-bound-shoq-ltl-rigid-roles})\\
        \bottomrule
    \end{tabular*}
    \\[1ex]
    \caption*{Settings: (i)~neither concept names nor role names are allowed to
        be rigid; (ii)~only concept names may be rigid; and (iii)~both concept
        names and role names may be rigid.}
\end{table}

The results obtained in the next sections are used in later chapters of this
thesis.  For instance, the ideas ideas underlying the decision procedures to
show the complexity upper bounds for the cases without rigid names and with
rigid concept and role names can also be used to obtain automata-based decision
procedures.  These constructions of $\omega$-automata are then used in
Chapter~\ref{ch:monitor}.
%
Moreover, the proof ideas and in particular the results of
Section~\ref{sec:consistency-boolean-shoqcap-kb} are used in
Chapter~\ref{ch:tcqs}.


\section{The Complexity of Satisfiability in \texorpdfstring{\SHOQ-LTL}{SHOQ-LTL}}\label{sec:complexity-shoq-ltl}

In this section, we examine the complexity of the satisfiability problem in
\SHOQ-LTL\@.  We consider three different settings: (i)~neither concept names
nor role names are allowed to be rigid, i.e.~$\NRC=\NRR=\emptyset$, (ii)~only
concept names may be rigid, i.e.~$\NRC\ne\emptyset$ and $\NRR=\emptyset$, and
(iii)~both concept and role names may be rigid, i.e.~$\NRC\ne\emptyset$ and
$\NRR\ne\emptyset$.  It is well-known that one can simulate rigid concept names
by rigid role names~\cite{BaGL-ToCL12}, which is why there are only three cases
to consider.  The results of this section are summarised in
Table~\ref{tab:shoq-ltl-results}.

Since \ALC-LTL is a fragment of \SHOQ-LTL, we immediately obtain the following
lower bounds for the satisfiability problem.

\begin{theorem}\label{thm:lower-bounds-shoq-ltl}
    The satisfiability problem in \SHOQ-LTL is
    \begin{enumerate}
        \item \ExpTime-hard if $\NRC=\NRR=\emptyset$;
        \item \NExpTime-hard if $\NRC\ne\emptyset$ and $\NRR=\emptyset$; and
        \item \TwoExpTime-hard if $\NRC\ne\emptyset$ and $\NRR\ne\emptyset$.
    \end{enumerate}
\end{theorem}

\begin{proof}
    We reduce the satisfiability problem in \ALC-LTL\@.  This problem is
    \ExpTime-complete if $\NRC=\NRR=\emptyset$, \NExpTime-complete if
    $\NRC\ne\emptyset$ and $\NRR=\emptyset$, and \TwoExpTime-complete if
    $\NRC\ne\emptyset$ and $\NRR\ne\emptyset$ as shown in~\cite{BaGL-ToCL12}.

    Let now $\phi$ be an \ALC-LTL-formula.  Obviously, $\phi$ is a
    \SHOQ-LTL-formula w.r.t.\ the empty RBox.  Thus, we have that the
    \ALC-LTL-formula~$\phi$ is satisfiable iff the \SHOQ-LTL-formula~$\phi$
    is satisfiable w.r.t.~$\emptyset$.
\end{proof}

\noindent
To obtain the corresponding upper bounds, we reduce the satisfiability problem
in \SHOQ-LTL to two separate satisfiability problems.  For that, we use the idea
of Lemma~4.3 in~\cite{BaGL-ToCL12}, where this was done for \ALC-LTL\@.

In the following, let \Rmc be an RBox, and let $\phi$ be a \SHOQ-LTL-formula
w.r.t.~\Rmc.
%
For the first satisfiability problem is called \emph{t-satisfiability}, because
it takes care of the \emph{temporal} structure of~$\phi$.  For this, we consider
the propositional abstraction.  The propositional abstraction of~$\phi$ is
constructed by replacing each axiom occurring in~$\phi$
with a propositional variable such that there is a 1--1 relationship between the
axioms $\alpha_1,\dots,\alpha_m$ occurring in~$\phi$ and the propositional
variables $p_1,\dots,p_m$ occurring in its abstraction.

\begin{definition}[Propositional abstraction]\label{def:prop-abs}
    Let \Rmc be an RBox, and let $\phi$ be a \SHOQ-LTL-formula w.r.t.~\Rmc.
    Furthermore, let $\Pmc_\phi$ be a finite set of propositional variables such
    that there is a bijection $\psf\colon\Ax(\phi)\to\Pmc_\phi$.
    \begin{enumerate}
        \item The propositional LTL-formula~$\phi^\psf$ is obtained from~$\phi$
            by replacing every occurrence of an axiom~$\alpha$ in~$\phi$ by its
            \psf-image $\psf(\alpha)$.  We call $\phi^\psf$ the
            \emph{propositional abstraction of~$\phi$ w.r.t.~\psf}.
        \item Given a DL-LTL-structure $\Imf=(\Imc_i)_{i\ge 0}$, its
            \emph{propositional abstraction w.r.t.~\psf} is the propositional
            LTL-structure $\Imf^\psf=(w_i)_{i\ge 0}$ with
            \[w_i:=\bigl\{\psf(\alpha)\mid\alpha\in\Ax(\phi)\ \text{and} \
                \Imc_i\models\alpha\bigr\}\]
            for every $i\ge 0$.
    \end{enumerate}
\end{definition}

\noindent
In the following, we assume that $\psf\colon\Ax(\phi)\to\Pmc_\phi$ is a
bijection.%
\footnote{Obviously, for every \SHOQ-LTL-formula~$\phi$ w.r.t.\ an RBox~\Rmc,
    there is a finite set~$\Pmc_\phi$ of propositional variables such that a
    bijection $\psf\colon\Ax(\phi)\to\Pmc_\phi$ exists.}
%
For simplicity, for a subformula~$\psi$ of~$\phi$, we denote by~$\psi^\psf$ the
propositional abstraction of~$\psi$ w.r.t.\ the restriction of~\psf to
$\Ax(\psi)$.
%
We now give an example of a propositional abstraction.

\begin{example}\label{ex:prop-abs}
    Let
    $\phi_\text{ex}:=\Next\bigl(A(a)\bigr)\land\bigl((A\sqsubseteq B)\Until((\lnot B)(a))\bigr)$
    be a \SHOQ-LTL-formula w.r.t.\ the empty RBox, and let
    $\psf\colon\Ax(\phi_\text{ex})\to\{p_1,p_2,p_3\}$ be the bijection that maps
    $A(a)$ to~$p_1$, $A\sqsubseteq B$ to~$p_2$, and $(\lnot B)(a)$ to~$p_3$.
    Then the propositional LTL-formula $\Next p_1\land(p_2\Until p_3)$ is the
    propositional abstraction of~$\phi_\text{ex}$ w.r.t.~\psf.
\end{example}

\noindent
The propositional abstraction~$\phi^\psf$ of~$\phi$ w.r.t.~\psf allows us to
analyse the temporal structure of~$\phi$ separately from the DL-component.
%
The following lemma states the relationships between~$\phi$ and its
propositional abstraction~$\phi^\psf$.

\begin{lemma}\label{lem:prop-abs}
    Let \Imf be a DL-LTL-structure with $\Imf\models\Rmc$.  Then, \Imf is a
    model of~$\phi$ w.r.t.~\Rmc iff $\Imf^\psf$ is a model of~$\phi^\psf$.
\end{lemma}

\begin{proof}
    Let $\Imf=(\Imc_i)_{i\ge 0}$ be a DL-LTL-structure with $\Imf\models\Rmc$,
    and $\Imf^\psf=(w_i)_{i\ge 0}$ its propositional abstraction w.r.t.~\psf.
    We prove this lemma by showing that $\Imf,i\models\phi$ iff
    $\Imf^\psf,i\models\phi^\psf$ for every~$i\ge 0$ by induction of the
    structure of~$\phi$.

    For the base case, let $\phi$ be an axiom.  Then, we have for every~$i\ge 0$
    that $\Imf,i\models\phi$ \emph{iff} $\Imc_i\models\phi$ \emph{iff}
    $\psf(\phi)\in w_i$ \emph{iff} $w_i\models\phi^\psf$ \emph{iff}
    $\Imf^\psf,i\models\phi^\psf$.

    If $\phi$ is of the form $\lnot\phi_1$, we have for every~$i\ge 0$ that
    $\Imf,i\models\lnot\phi_1$ \emph{iff} $\Imf,i\not\models\phi_1$ \emph{iff}
    $\Imf^\psf,i\not\models\phi_1^\psf$ \emph{iff}
    $\Imf^\psf\models(\lnot\phi_1)^\psf$.

    If $\phi$ is of the form $\phi_1\land\phi_2$, we have for every~$i\ge 0$
    that $\Imf,i\models\phi_1\land\phi_2$ \emph{iff} $\Imf,i\models\phi_1$ and
    $\Imf,i\models\phi_2$ \emph{iff} $\Imf^\psf,i\models\phi_1^\psf$ and
    $\Imf^\psf,i\models\phi_2^\psf$ \emph{iff}
    $\Imf^\psf,i\models(\phi_1\land\phi_2)^\psf$.

    If $\phi$ is of the form $\Next\phi_1$, we have for every~$i\ge 0$ that
    $\Imf,i\models\Next\phi_1$ \emph{iff} $\Imf,i+1\models\phi_1$ \emph{iff}
    $\Imf^\psf,i+1\models\phi_1^\psf$ \emph{iff}
    $\Imf^\psf,i\models(\Next\phi_1)^\psf$.

    If $\phi$ is of the form $\Previous\phi_1$, we have for every~$i\ge 0$ that
    $\Imf,i\models\Previous\phi_1$ \emph{iff} $i>0$ and $\Imf,i-1\models\phi_1$
    \emph{iff} $i>0$ and
    $\Imf^\psf,i-1\models\phi_1^\psf$ \emph{iff}
    $\Imf^\psf,i\models(\Previous\phi_1)^\psf$.

    If $\phi$ is of the form $\phi_1\Until\phi_2$, we have for every~$i\ge 0$
    that $\Imf,i\models\phi_1\Until\phi_2$ \emph{iff} there is some $k\ge i$
    such that $\Imf,k\models\phi_2$, and $\Imf,j\models\phi_1$ for every~$j$,
    $i\le j<k$ \emph{iff} there is some $k\ge i$ such that
    $\Imf^\psf,k\models\phi_2^\psf$, and $\Imf^\psf,j\models\phi_1^\psf$
    for every~$j$, $i\le j<k$ \emph{iff}
    $\Imf^\psf,i\models(\phi_1\Until\phi_2)^\psf$.

    Finally, if $\phi$ is of the form $\phi_1\Since\phi_2$, we have for
    every~$i\ge 0$ that $\Imf,i\models\phi_1\Since\phi_2$ \emph{iff} there is
    some~$k$, $0\le k\le i$, such that $\Imf,k\models\phi_2$, and
    $\Imf,j\models\phi_1$ for every~$j$, $k<j\le i$ \emph{iff} there is some
    $k$, $0\le k\le i$, such that $\Imf^\psf,k\models\phi_2^\psf$, and
    $\Imf^\psf,j\models\phi_1^\psf$ for every~$j$, $k<j\le i$ \emph{iff}
    $\Imf^\psf,i\models(\phi_1\Since\phi_2)^\psf$.
\end{proof}

\noindent
The \enquote{only if} direction of this lemma yields that satisfiability
of~$\phi$ w.r.t.~\Rmc implies satisfiability of~$\phi^\psf$.  However, the
\enquote{if} direction does not yield the converse of this implication.  In
fact, the propositional LTL-formula~$\phi^\psf$ may turn out to be satisfiable
even though the original \SHOQ-LTL-formula~$\phi$ is not.  The reason is that
there may exist propositional LTL-structures that are not propositional
abstractions of DL-LTL-structures.

\begin{example}
    Take the \SHOQ-LTL-formula~$\phi_\text{ex}$ and the bijection~\psf of
    Example~\ref{ex:prop-abs}, and consider the propositional LTL-structure
    $\Wmf=(w_i)_{i\ge 0}$ with $w_i=\{p_1,p_2,p_3\}$ for every $i\ge 0$.
    Obviously, we have $\Wmf,0\models\phi_\text{ex}^\psf$, but there is no
    DL-LTL-structure~\Imf such that $\Imf^\psf=\Wmf$.  In fact, every world of
    this DL-LTL-structure would need to satisfy the three axioms in
    $\Ax(\phi_\text{ex})$ simultaneously, which is clearly not possible.
\end{example}

\noindent
To address this problem, we need some more notions.  We consider a set
$\Wmc\subseteq 2^{\Pmc_\phi}$, which intuitively specifies the worlds that are
allowed to occur in a propositional LTL-structure satisfying $\phi^\psf$.  To
express this restriction, we the define the propositional LTL-formula
\[\phi^\psf_\Wmc:=\phi^\psf\land%
    \Box\Biggl(\bigvee_{X\in\Wmc}\Biggl(\bigwedge_{p\in X}p\land%
    \bigwedge_{p\in\Pmc_\phi\setminus X}\lnot p\Biggr)\Biggr).\]

\noindent
The second satisfiability problem is called \emph{r-satisfiability}, because it
is used to determine whether it is possible to satisfy the \emph{rigidity}
constraints for the names in~\NRC and~\NRR.  Thus, it can be used to check
whether the set~\Wmc can indeed be induced by a DL-LTL-structure that is a model
of~$\phi$ w.r.t.~\Rmc.

\begin{definition}[R-satisfiability]\label{def:r-sat}
    Let $\Wmc=\{X_1,\dots,X_k\}\subseteq 2^{\Pmc_\phi}$.  We call \Wmc
    \emph{r-satisfiable w.r.t.~\Rmc} if there exist interpretations
    $\Imc_1=(\Delta,\cdot^{\Imc_1})$,~\dots, $\Imc_k=(\Delta,\cdot^{\Imc_k})$
    such that
    \begin{itemize}
        \item $a^{\Imc_i}=a^{\Imc_j}$ holds for every $a\in\NI$ and all $i,j$,
            $1\le i<j\le k$;
        \item $A^{\Imc_i}=A^{\Imc_j}$ holds for every $A\in\NRC$ and all $i,j$,
            $1\le i<j\le k$;
        \item $r^{\Imc_i}=r^{\Imc_j}$ holds for every $r\in\NRR$ and all $i,j$,
            $1\le i<j\le k$; and
        \item every $\Imc_i$, $1\le i\le k$, is a model of the Boolean knowledge
            base
            \[\Bmc_{X_i}:=\Biggl(\bigwedge_{p\in X_i}\psf^{-1}(p)\land%
                \bigwedge_{p\in\Pmc_\phi\setminus X_i}\lnot\psf^{-1}(p),\quad\Rmc\Biggr).\]
    \end{itemize}
\end{definition}

\noindent
Note that any subset of a set~\Wmc that is r-satisfiable w.r.t.~\Rmc is again
r-satisfiable w.r.t.~\Rmc.  In particular, the empty set is always r-satisfiable
w.r.t.~\Rmc.

The intuition underlying the definition of r-satisfiability is the following.
The existence of the interpretations~$\Imc_i$, $1\le i\le k$, ensures that the
Boolean knowledge base induced by~$X_i$ and~\Rmc is consistent.  In fact, a
set~\Wmc containing a set~$X_i$ for which this does not hold cannot be induced
by a DL-LTL-structure.  Moreover, we ensure that the interpretations share the
same domain and respect rigid names.  As we will see later, for deciding whether
a set~\Wmc is r-satisfiable w.r.t.~\Rmc, the difficulty lies in ensuring that
the interpretations share the same domain and respect rigid names.

Satisfaction of the temporal structure of~$\phi$ by a DL-LTL-structure built
this way is ensured by testing $\phi^\psf_\Wmc$ for satisfiability.  This is
captured in the notion of t-satisfiability.

\begin{definition}[T-satisfiability]\label{def:t-sat}
    Let $\Wmc\subseteq 2^{\Pmc_\phi}$.  We call the propositional LTL-formula
    $\phi^\psf$ \emph{t-satisfiable w.r.t.~\Wmc} if there exists a propositional
    LTL-structure $\Wmf=(w_i)_{i\ge 0}$ such that $\Wmf,0\models\phi^\psf_\Wmc$.
\end{definition}

\noindent
The next two lemmas show that these two satisfiability problems, namely,
t-satisfiability and r-satisfiability, can be combined to decidable the
satisfiability problem in \SHOQ-LTL\@.  The statements of the lemmas were also
proved in~\cite{BaGL-ToCL12} for \ALC-LTL, but again the same arguments can also
be used to prove them for \SHOQ-LTL\@.  Yet, we repeat these arguments for the
sake of completeness.

\begin{lemma}\label{lem:ltl-structure-r-sat}
    For every propositional LTL-structure $\Wmf=(w_i)_{i\ge 0}$ with
    $w_i\subseteq\Pmc_\phi$ for every $i\ge 0$, the following two statements are
    equivalent:
    \begin{enumerate}
        \item There is a model~\Imf of~$\phi$ w.r.t.~\Rmc with $\Imf^\psf=\Wmf$.
        \item \Wmf is a model of~$\phi^\psf$ and the set
            $\Wmc:=\{w_i\mid i\ge 0\}$ is r-satisfiable w.r.t.~\Rmc.
    \end{enumerate}
\end{lemma}

\begin{proof}
    For the direction \enquote{$1\implies 2$}, assume that there is a
    DL-LTL-structure $\Imf=(\Imc_i)_{i\ge 0}$ that is a model of~$\phi$
    w.r.t.~\Rmc with $\Imf^\psf=\Wmf$.
    %
    Since $\Imf\models\Rmc$, we have by Lemma~\ref{lem:prop-abs} that \Wmf is a
    model of~$\phi^\psf$.
    %
    Since $w_i\subseteq\Pmc_\phi$ for every~$i\ge 0$, we have that
    $\Wmc=\{w_i\mid i\ge 0\}\subseteq 2^{\Pmc_\phi}$ is finite.  Let
    $\Wmc=\{X_1,\dots,X_k\}$.
    %
    We have that for every $i\ge 0$, there is an index $\nu_i\in\{1,\dots,k\}$
    such that $\Imc_i$ induces the set~$X_{\nu_i}$, i.e.\
    \[X_{\nu_i}=\bigl\{\psf(\alpha)\mid\alpha\in\Ax(\phi)\ \text{and}\
        \Imc_i\models\alpha\bigr\},\]
    and, conversely, for every $\nu\in\{1,\dots,k\}$, there is an index $i\ge 0$
    such that $\nu=\nu_i$.
    %
    For every~$i$, $1\le i\le k$, the interpretation~$\Jmc_i$ is obtained as
    follows.  Let $\ell_1,\dots,\ell_k$ be such that $\nu_{\ell_1}=1$,~\dots,
    $\nu_{\ell_k}=k$.  Now, if we set $\Jmc_i:=\Imc_{\ell_i}$, then we clearly
    have $\Jmc_i\models\Bmc_{X_i}$.  It is now easy to see that the
    interpretations $\Jmc_1,\dots,\Jmc_k$ satisfy the conditions for
    r-satisfiability of~\Wmc w.r.t.~\Rmc.

    For the direction \enquote{$2\implies 1$}, assume that \Wmf is a model
    of~$\phi^\psf$ and the set $\Wmc=\{w_i\mid i\ge 0\}$ is r-satisfiable
    w.r.t.~\Rmc.
    %
    Since $w_i\subseteq\Pmc_\phi$ for every~$i\ge 0$, we have that
    $\Wmc=\{w_i\mid i\ge 0\}\subseteq 2^{\Pmc_\phi}$ is finite.  Let
    $\Wmc=\{X_1,\dots,X_k\}$.
    %
    Since \Wmc is r-satisfiable w.r.t.~\Rmc, there are interpretations
    $\Jmc_1,\dots,\Jmc_k$ such that the conditions in Definition~\ref{def:r-sat}
    are satisfied.
    %
    Moreover, we have that for every world~$w_i$, there is exactly one index
    $\nu_i\in\{1,\dots,k\}$ such that $w_i$ satisfies
    \[\bigwedge_{p\in X_{\nu_i}}p\land\bigwedge_{p\in\Pmc_\phi\setminus X_{\nu_i}}\lnot p.\]
    We can now define a DL-LTL-structure $\Imf:=(\Imc_i)_{i\ge 0}$ as follows.
    We set $\Imc_i:=\Jmc_{\nu_i}$ for $i\ge 0$.  By construction, we have
    $\Imf^\psf=\Wmf$.  By Definition~\ref{def:r-sat},
    each~$\Imc_i$ is a model of $\Bmc_{X_{\nu_i}}$, i.e.~it is a model of~\Rmc and
    satisfies exactly the axioms specified by the propositional variables
    in~$X_{\nu_i}$.  This yields that $\Imf\models\Rmc$, and since
    $\Wmf,0\models\phi^\psf$, we have by Lemma~\ref{lem:prop-abs} that
    $\Imf,0\models\phi$.  Thus, \Imf is a model of~$\phi$ w.r.t.~\Rmc.
\end{proof}

\noindent
The following lemma is an immediate consequence of the previous lemma.

\begin{lemma}\label{lem:r-sat-t-sat}
    The \SHOQ-LTL-formula~$\phi$ is satisfiable w.r.t.\ the RBox~\Rmc iff there
    is a set $\Wmc=\bigl\{X_1,\dots,X_k\bigr\}\subseteq 2^{\Pmc_\phi}$ such that
    \begin{itemize}
        \item \Wmc is r-satisfiable w.r.t.~\Rmc, and
        \item $\phi^\psf$ is t-satisfiable w.r.t.~\Wmc.
    \end{itemize}
\end{lemma}

\begin{proof}
    For the \enquote{only if} direction, assume that there is a DL-LTL-structure
    $\Imf=(\Imc_i)_{i\ge 0}$ that is a model of~$\phi$ w.r.t.~\Rmc.
    %
    Let $\Imf^\psf=(w_i)_{i\ge 0}$, and let $\Wmc=\{w_i\mid i\ge 0\}$.
    %
    By Lemma~\ref{lem:ltl-structure-r-sat}, we have that $\Imf^\psf$ is a model
    of~$\phi^\psf$ and that \Wmc is r-satisfiable w.r.t.~\Rmc.
    %
    By construction of~\Wmc, we have also that $\Imf^\psf$ is a model
    of~$\phi^\psf_\Wmc$.  Hence, $\phi^\psf$ is t-satisfiable w.r.t.~\Wmc.

    For the \enquote{if} direction, assume that there is a set
    $\Wmc=\{X_1,\dots,X_k\}\subseteq 2^{\Pmc_\phi}$ such that \Wmc is
    r-satisfiable w.r.t.~\Rmc and $\phi^\psf$ is t-satisfiable w.r.t.~\Wmc.
    Hence, there is a propositional LTL-structure $\Wmf=(w_i)_{i\ge 0}$ such
    that \Wmf is a model of~$\phi^\psf_\Wmc$.
    %
    Hence \Wmf is a model of~$\phi^\psf$.  We define
    $\Wmc':=\{w_i\mid i\ge 0\}$.  Since \Wmf is a model of~$\phi^\psf_\Wmc$, we
    have that $\Wmc'\subseteq\Wmc$.  Since \Wmc is r-satisfiable w.r.t.~\Rmc,
    this yields that $\Wmc'$ is r-satisfiable w.r.t.~\Rmc.  Then,
    Lemma~\ref{lem:ltl-structure-r-sat} yields that there is a model \Imf
    of~$\phi$ w.r.t.~\Rmc with $\Imf^\psf=\Wmf$, i.e.~$\phi$ is satisfiable
    w.r.t.~\Rmc.
\end{proof}

\noindent
To obtain a decision procedure for the satisfiability problem in \SHOQ-LTL, we
have to non-deterministically guess or construct the set~\Wmc, and then check
the two conditions of Lemma~\ref{lem:r-sat-t-sat}.
%
Depending on which symbols are allowed to be rigid, we use different
constructions to achieve that.

First, we focus on deciding t-satisfiability w.r.t.\ a given set~\Wmc.
%
From now on, let $\Wmc\subseteq 2^{\Pmc_\phi}$.  Obviously, the size
of~$\phi^\psf_\Wmc$ may be exponential in the size of~$\phi$.  Since we can
decide satisfiability of a propositional LTL-formula in
\PSpace~\cite{SiCl-JACM85,LiPZ-CLP85}, this yields an \ExpSpace-decision
procedure for deciding the satisfiability of~$\phi^\psf_\Wmc$.  However, using a
trick from~\cite{BaGL-ToCL12}, we can reduce the complexity to \ExpTime.

\begin{lemma}\label{lem:t-sat}
    Deciding whether $\phi^\psf$ is t-satisfiable w.r.t.~\Wmc can be done in
    time exponential in the size of~$\phi^\psf$ and linear in the size of~\Wmc.
\end{lemma}

\begin{proof}
    We first construct a Büchi-automaton for~$\phi^\psf$, which can be done in
    time exponential in the size of~$\phi^\psf$ as discussed in
    Section~\ref{sec:aut-for-ltl}.  Let
    $\Nmc=(Q,\Sigma_{\Pmc_\phi},\Delta,Q_0,F)$ be a Büchi-automaton
    for~$\phi^\psf$.  We obtain the Büchi-automaton
    $\Nmc'=(Q,\Sigma_{\Pmc_\phi},\Delta',Q_0,F)$ by removing all transitions
    that are labelled with a letter $\sigma\in\Sigma_{\Pmc_\phi}\setminus\Wmc$,
    i.e.~we define
    \[\Delta':=\bigl\{(q,\sigma,q')\in\Delta\mid\sigma\in\Wmc\bigr\}.\]
    %
    It is easy to verify that $\Nmc'$ is a Büchi-automaton for~$\phi^\psf_\Wmc$.

    Note that the Büchi-automaton~$\Nmc'$ can be constructed in time polynomial
    in the size of~\Nmc and linear in the size of~\Wmc, and thus the size
    of~$\Nmc'$ is exponential in the size of~$\phi^\psf$.  Since the emptiness
    problem for Büchi-automata can be solved in polynomial
    time~\cite{VaWo-IC94}, this yields that t-satisfiability of~$\phi^\psf$
    w.r.t.~\Wmc can be decided in time exponential in the size of~$\phi^\psf$
    and linear in the size of~\Wmc.
\end{proof}

\noindent
Due to Lemma~\ref{lem:r-sat-t-sat}, the complexity of the satisfiability problem
in \SHOQ-LTL also depends on the complexity of the deciding whether \Wmc is
r-satisfiable w.r.t.~\Rmc.  However, this depends on the fact whether there are
concept or role names that are allowed to be rigid.

In Section~\ref{sec:sat-no-rigid}, we consider the case without rigid names, and
in Section~\ref{sec:sat-rigid}, we consider the most general case with rigid
concept and role names.  Finally, we consider the case with rigid concept names
in Section~\ref{sec:sat-rigid-concepts}.  A result that is needed in
Section~\ref{sec:sat-rigid-concepts} is proved in a separate section, namely, in
Section~\ref{sec:consistency-boolean-shoqcap-kb}.


\subsection{Satisfiability in \texorpdfstring{\SHOQ-LTL}{SHOQ-LTL} for the Case without Rigid Names}\label{sec:sat-no-rigid}

In this section, we consider the case where neither concept names nor role names
are allowed to be rigid, i.e.~$\NRC=\NRR=\emptyset$.
%
We establish the following complexity result.

\begin{theorem}\label{thm:upper-bound-shoq-ltl-no-rigid}
    The satisfiability problem in \SHOQ-LTL is in \ExpTime if
    $\NRC=\NRR=\emptyset$.
\end{theorem}

\begin{proof}
    Let \Rmc be an RBox, and let $\phi$ be a \SHOQ-LTL-formula w.r.t.~\Rmc.  We
    can decide satisfiability of~$\phi$ w.r.t.~\Rmc using
    Lemma~\ref{lem:r-sat-t-sat}.
    %
    For that, let $\psf\colon\Ax(\phi)\to\Pmc_\phi$ be a bijection, and define
    \[\Wmc:=\{X\in 2^{\Pmc_\phi}\mid\Bmc_X\ \text{is consistent}\},\]
    where $\Bmc_X$ is defined as in Definition~\ref{def:r-sat}.  We first show
    that $\Wmc=\{X_1,\dots,X_k\}$ is r-satisfiable w.r.t.~\Rmc.
    %
    Since every~$\Bmc_{X_i}$, $1\le i\le k$ is consistent, there are models
    $\Imc_1,\dots,\Imc_k$ such that every~$\Imc_i$, $1\le i\le k$, is a model
    of~$\Bmc_{X_i}$.  We can assume w.l.o.g.\ that all of these models have the
    same domain since we can assume w.l.o.g.\ that their domains are countably
    infinite due to to the Löwenheim-Skolem theorem~\cite{Loe-MA15,Sko-VS20}.
    Furthermore, we can assume w.l.o.g.\ that all individual names are
    interpreted by the same domain elements in all models.  Since
    $\NRC=\NRR=\emptyset$, this yields that \Wmc is r-satisfiable w.r.t.~\Rmc.

    Thus, we have by Lemma~\ref{lem:r-sat-t-sat} that if $\phi^\psf$ is
    t-satisfiable w.r.t.~\Wmc, then $\phi$ is satisfiable w.r.t.~\Rmc.
    %
    Conversely, again by Lemma~\ref{lem:r-sat-t-sat}, we have that if $\phi$ is
    satisfiable w.r.t.~\Rmc, then there is a set $\Wmc'\subseteq 2^{\Pmc_\phi}$
    such that $\Wmc'$ is r-satisfiable w.r.t.~\Rmc and $\phi^\psf$ is
    t-satisfiable w.r.t.~$\Wmc'$.  The definition of~\Wmc yields that \Wmc is
    the maximal subset of $2^{\Pmc_\phi}$ that is r-satisfiable w.r.t.~\Rmc.
    Thus, we have that $\Wmc'\subseteq\Wmc$.  It is easy to see that the
    t-satisfiability of~$\phi^\psf$ w.r.t.~$\Wmc'$ implies that $\phi^\psf$ is
    t-satisfiable w.r.t.~\Wmc.
    %
    Hence, we have that $\phi$ is satisfiable w.r.t.~\Rmc iff $\phi^\psf$ is
    t-satisfiable w.r.t.~\Wmc.

    Note that \Wmc can be constructed in time exponential in the size of~$\phi$
    and~\Rmc.  Indeed, there are exponentially many $X\in 2^{\Pmc_\phi}$, but
    each~$\Bmc_X$ can be constructed in time polynomial in the size of~$\phi$
    and~\Rmc, and is thus of size polynomial in the size of~$\phi$ and~\Rmc.  We
    show in Corollary~\ref{cor:cons-boolean-shoq-kb} (see
    Section~\ref{sec:consistency-boolean-shoqcap-kb}) that consistency of a
    Boolean \SHOQ-knowledge base~\Bmc can be decided in time exponential in the
    size of~\Bmc.  Thus, overall, deciding for every~$\Bmc_X$ whether it is
    consistent can be done in time exponential in the size of~$\phi$ and~\Rmc.
    %
    Due to Lemma~\ref{lem:t-sat}, deciding whether $\phi^\psf$ is t-satisfiable
    w.r.t.~\Wmc can be done in time exponential in the size of $\phi^\psf$ (and
    thus in time exponential in the size of~$\phi$) and linear in the size
    of~\Wmc.  Thus, we can decide whether $\phi$ is satisfiable w.r.t.~\Rmc in
    time exponential in the size of~$\phi$ and~\Rmc.
\end{proof}

\noindent
Together with Theorem~\ref{thm:lower-bounds-shoq-ltl}, we obtain that the
satisfiability problem in \SHOQ-LTL is \ExpTime-complete if neither concept nor
role names are allowed to be rigid.


\subsection{Satisfiability in \texorpdfstring{\SHOQ-LTL}{SHOQ-LTL} for the Case of Rigid Concept Names and Role Names}\label{sec:sat-rigid}

In this section, we consider the case where both concept and role names may be
rigid, i.e.~$\NRC\ne\emptyset$ and $\NRR\ne\emptyset$.

Let us assume that a set $\Wmc=\{X_1,\dots,X_k\}\subseteq 2^{\Pmc_\phi}$ is
given.  Note that deciding whether \Wmc is r-satisfiable w.r.t.~\Rmc cannot be
done by simply checking for each $X\in\Wmc$ whether the Boolean knowledge
base~$\Bmc_X$ is consistent as we did in Section~\ref{sec:sat-no-rigid} for the
case without rigid names.  In fact, these consistency checks are not independent
any longer since one has to ensure that rigid concept and role names are
interpreted in the same way.  To achieve this, we adopt the renaming technique
from~\cite{BaGL-ToCL12} that works by introducing copies of the flexible
symbols.

For every $i$, $1\le i\le k$, every \emph{flexible} concept name~$A$ occurring
in~$\phi$, and every \emph{flexible} role name~$r$ occurring in~$\phi$ or~\Rmc,
we introduce copies~$A^{(i)}$ and~$r^{(i)}$.  We call $A^{(i)}$ the $i$-th
copy of~$A$, and similarly $r^{(i)}$ the $i$-th copy of~$r$.  The
axiom~$\alpha^{(i)}$ is obtained from the axiom~$\alpha$ by replacing every
occurrence of a flexible name by its $i$-th copy.  Similarly, the Boolean axiom
formula~$\Psi^{(i)}$ (RBox~$\Rmc^{(i)}$) is obtained from the Boolean axiom
formula~$\Psi$ (RBox~\Rmc) by replacing each axiom~$\alpha$ occurring in~$\Psi$
(\Rmc) by $\alpha^{(i)}$.

Moreover, let $\Bmc_{X_i}=(\Psi_{X_i},\Rmc)$, $1\le i\le k$, denote the Boolean
knowledge bases defined in Definition~\ref{def:r-sat}.  We define
\[\Bmc_\Wmc:=\Biggl(\bigwedge_{1\le i\le k}\Psi_{X_i}^{(i)},\quad%
    \bigcup_{1\le i\le k}\Rmc^{(i)}\Biggr).\]
%
The next lemma states that consistency of~$\Bmc_\Wmc$ is enough for ensuring
r-satisfiability of~\Wmc w.r.t.~\Rmc.

\begin{lemma}\label{lem:bmc-wmc}
    The set~\Wmc is r-satisfiable w.r.t.~\Rmc iff the Boolean knowledge
    base~$\Bmc_\Wmc$ is consistent.
\end{lemma}

\begin{proof}
    For the \enquote{only if} direction, let
    $\Imc_1=(\Delta,\cdot^{\Imc_1})$,~\dots, $\Imc_k=(\Delta,\cdot^{\Imc_k})$ be
    the interpretations required by Definition~\ref{def:r-sat} for the
    r-satisfiability of~\Wmc w.r.t.~\Rmc.  We construct the interpretation
    $\Jmc=(\Delta,\cdot^\Jmc)$ as follows:
    \begin{itemize}
        \item every individual name and every rigid name is interpreted as
            in~$\Imc_1$; and
        \item the $i$-th copy, $1\le i\le k$, of each flexible name is
            interpreted like the original name in~$\Imc_i$.
    \end{itemize}
    %
    It is easy to verify that~\Jmc is a model of~$\Bmc_\Wmc$.

    For the \enquote{if} direction, let \Jmc be a model of~$\Bmc_\Wmc$.
    We obtain the interpretations $\Imc_1,\dots,\Imc_k$ by the inverse
    construction to the one above:
    \begin{itemize}
        \item the domain of all these interpretations is the domain of~\Jmc;
        \item every individual name and every rigid name is interpreted by these
            interpretations as in~\Jmc; and
        \item every flexible name is interpreted in $\Imc_i$, $1\le i\le k$, as
            its $i$-th copy is interpreted in~\Jmc.
    \end{itemize}
    %
    It is again easy to verify that the interpretations $\Imc_1,\dots,\Imc_k$
    satisfy the conditions for r-satisfiability of~\Wmc w.r.t.~\Rmc.
\end{proof}

\noindent
Using this lemma, we can prove our complexity result.

\begin{theorem}\label{thm:upper-bound-shoq-ltl-rigid-roles}
    The satisfiability problem in \SHOQ-LTL is in \TwoExpTime if
    $\NRC\ne\emptyset$ and $\NRR\ne\emptyset$.
\end{theorem}

\begin{proof}
    Let \Rmc be an RBox, let $\phi$ be a \SHOQ-LTL-formula w.r.t.~\Rmc, and let
    $\psf\colon\Ax(\phi)\to\Pmc_\phi$ be a bijection.  We use again
    Lemma~\ref{lem:r-sat-t-sat} for deciding satisfiability of~$\phi$
    w.r.t.~\Rmc.  We first enumerate all sets $\Wmc\subseteq 2^{\Pmc_\phi}$,
    which can be done in time doubly exponential in~$\phi$ and~\Rmc.  For each
    of these sets~\Wmc, we check t-satisfiability of~$\phi^\psf$ w.r.t.~\Wmc and
    r-satisfiability of~\Wmc w.r.t.~\Rmc.  By Lemma~\ref{lem:t-sat}, the
    t-satisfiability check can be done in time exponential in the size
    of~$\phi^\psf$ (and thus in time exponential in the size of~$\phi$) and
    linear in the size of~\Wmc.

    For the r-satisfiability check, we use Lemma~\ref{lem:bmc-wmc}.  We
    construct the Boolean knowledge base~$\Bmc_\Wmc$, which can be done in time
    exponential in the size of~$\phi$ and~\Rmc.  Also, $\Bmc_\Wmc$ is of size at
    most exponential in the size of~$\phi$ and~\Rmc.  Consistency of~$\Bmc_\Wmc$
    can be checked in time exponential in the size of~$\Bmc_\Wmc$ by
    Corollary~\ref{cor:cons-boolean-shoq-kb}, which we prove in
    Section~\ref{sec:consistency-boolean-shoqcap-kb}.  Thus, checking whether
    \Wmc is r-satisfiable w.r.t.~\Rmc can be done in time doubly exponential in
    the size of~$\phi$ and~\Rmc.

    Thus, overall, we can decide whether $\phi$ is satisfiable w.r.t.~\Rmc in
    time doubly exponential in the size of~$\phi$ and~\Rmc.
\end{proof}

\noindent
Together with Theorem~\ref{thm:lower-bounds-shoq-ltl}, we obtain that the
satisfiability problem in \SHOQ-LTL is \TwoExpTime-complete if both concept and
role names are allowed to be rigid.


\subsection{Satisfiability in \texorpdfstring{\SHOQ-LTL}{SHOQ-LTL} for the Case of Rigid Concept Names}\label{sec:sat-rigid-concepts}

In this section, we consider the case where only concept names may be rigid,
i.e.~$\NRC\ne\emptyset$ and $\NRR=\emptyset$.

Let us again assume that a set $\Wmc=\{X_1,\dots,X_k\}\subseteq 2^{\Pmc_\phi}$
is given.  By Lemma~\ref{lem:bmc-wmc}, for checking whether \Wmc is
r-satisfiable w.r.t.~\Rmc, it is enough to construct the Boolean knowledge
base~$\Bmc_\Wmc$ and to check it for consistency.  As we have seen in the proof
of Theorem~\ref{thm:upper-bound-shoq-ltl-rigid-roles}, this yields a \TwoExpTime
decision procedure.  However, we can reduce the complexity to \NExpTime by using
the ideas of the proof of Lemma~6.3 in~\cite{BaGL-ToCL12}, where the same
complexity result is shown for \ALC-LTL\@.  The principal idea is to fix the
combinations of rigid concept names that are allowed to occur in the models of
the Boolean knowledge bases~$\Bmc_{X_i}$, $1\le i\le k$.  For that, we need some
more notation.

\begin{definition}[Interpretation respects~\Dmc]\label{def:respects-dmc}
    Let $\Imc=(\Delta^\Imc,\cdot^\Imc)$ be an interpretation, and let
    $\Dmc=(\Umc,\Ymc)$ be a pair such that \Umc is a set of concept
    names and $\Ymc\subseteq 2^\Umc$.

    We say that \Imc \emph{respects~\Dmc} if
    \[\Ymc=\bigl\{Y\subseteq\Umc\mid%
        \text{there is some $d\in\Delta^\Imc$ with $d\in(C_{\Umc,Y})^\Imc$}\bigr\},\]
    where
    \[C_{\Umc,Y}:=\bigsqcap_{A\in Y}A\quad\sqcap\quad\bigsqcap_{A\in\Umc\setminus Y}\lnot A.\]
\end{definition}

\noindent
Intuitively, this definitions states that every combination of concept names
in~\Ymc is realised by a domain element of~\Imc, and conversely, every such
combination that is realised by a domain element of~\Imc must occur in~\Ymc.

Let $\RCon(\phi)$ denote the set of rigid concept names occurring in~$\phi$, and
let $\Ind(\phi)$ denote the set of individual names occurring in~$\phi$.
Furthermore, let $\Dmc=(\RCon(\phi),\Ymc)$ with $\Ymc\subseteq 2^{\RCon(\phi)}$
be arbitrary, and let $\tau$ be a mapping from $\Ind(\phi)$ to~\Ymc.  The idea
is that \Dmc fixes the combinations of rigid concept names that are allowed to
occur in the models of~$\Bmc_{X_i}$, $1\le i\le k$.  The mapping~$\tau$ assigns
to each individual name occurring in~$\phi$ one such combination.
%
We define $\Psi_\tau$ to be the following Boolean axiom formula:
\[\Psi_\tau:=\bigwedge_{a\in\Ind(\phi)}C_{\RCon(\phi),\tau(a)}(a).\]
%
The next lemma states how these notions can be used to characterise
r-satisfiability of~\Wmc w.r.t.~\Rmc.

\begin{lemma}\label{lem:dmc-tau}
    If $\NRC\ne\emptyset$ and $\NRR=\emptyset$, then \Wmc is r-satisfiable
    w.r.t.~\Rmc iff there exist a pair $\Dmc=(\RCon(\phi),\Ymc)$ with
    $\Ymc\subseteq 2^{\RCon(\phi)}$ and a mapping $\tau\colon\Ind(\phi)\to\Ymc$
    such that for every $i$, $1\le i\le k$, the Boolean knowledge base
    $(\Psi_{X_i}\land\Psi_\tau,\Rmc)$ has a model that respects~\Dmc.
\end{lemma}

\begin{proof}
    For the \enquote{if} direction, assume that $\Imc_i$, $1\le i\le k$, are the
    models of $(\Psi_{X_i}\land\Psi_\tau,\Rmc)$, respectively, that
    respect~\Dmc.
    %
    Similar to the proof of Lemma~6.3 in~\cite{BaGL-ToCL12}, we can assume
    w.l.o.g.\ that their domains $\Delta_i$ are countably infinite and for each
    $Y\in\Ymc$ there are countably infinitely many elements
    $d\in(C_{\RCon(\phi),Y})^{\Imc_i}$.  This is a consequence of the
    Löwenheim-Skolem theorem~\cite{Loe-MA15,Sko-VS20} and the fact that the
    countably infinite disjoint union of~$\Imc_i$ with itself is again a model
    of $(\Psi_{X_i}\land\Psi_\tau,\Rmc)$.

    Consequently, we can partition the domains~$\Delta_i$ into the countably
    infinite sets
    \[\Delta_i(Y):=\bigl\{d\in\Delta_i\mid d\in(C_{\RCon(\phi),Y})^{\Imc_i}\bigr\}\]
    for $Y\in\Ymc$.
    %
    By the assumptions above and the fact that every~$\Imc_i$
    satisfies~$\Psi_\tau$, there are bijections
    $\pi_i\colon\Delta_1\to\Delta_i$, $2\le i\le k$, such that
    \begin{itemize}
        \item $\pi_i(\Delta_1(Y))=\Delta_i(Y)$ for every $Y\in\Ymc$, and
        \item $\pi_i(a^{\Imc_1})=a^{\Imc_i}$ for every $a\in\Ind(\phi)$.
    \end{itemize}
    %
    Thus, we can assume in the following that the models~$\Imc_i$,
    $1\le i\le k$, actually share the same domain and interpret the concept
    names in $\RCon(\phi)$ and the individual names occurring in~$\phi$ in the
    same way.  Note that the interpretation of the names in
    $\NRC\setminus\RCon(\phi)$ and $\NI\setminus\Ind(\phi)$ is irrelevant and
    can be fixed arbitrarily, as long as the UNA is satisfied.

    Since for every $i$, $1\le i\le k$, we have that $\Imc_i$ is a model of
    $(\Psi_{X_i}\land\Psi_\tau,\Rmc)$, we have also that $\Imc_i$ is a model of
    $\Bmc_{X_i}=(\Psi_{X_i},\Rmc)$.  Thus, the conditions required for the
    r-satisfiability of~\Wmc w.r.t.~\Rmc by Definition~\ref{def:r-sat} are
    satisfied.

    For the \enquote{only if} direction, assume that
    $\Imc_i=(\Delta,\cdot^{\Imc_i})$, $1\le i\le k$, are the interpretations
    required for r-satisfiability of~\Wmc w.r.t.~\Rmc by
    Definition~\ref{def:r-sat}.
    %
    Since these interpretations share the same domain, interpret the rigid
    concept names in the same way, it follows that for every
    $Y\subseteq\RCon(\phi)$, we have that
    $(C_{\RCon(\phi),Y})^{\Imc_1}=(C_{\RCon(\phi),Y})^{\Imc_i}$ for every~$i$,
    $2\le i\le k$.
    %
    We define $\Dmc:=(\RCon(\phi),\Ymc)$ with
    \[\Ymc:=\bigl\{Y\subseteq\RCon(\phi)\mid%
        \text{there is some $d\in\Delta$ with $d\in(C_{\RCon(\phi),Y})^{\Imc_1}$}\bigr\}.\]
    %
    Moreover, for every $a\in\Ind(\phi)$, we define
    $\tau(a):=Y\subseteq\RCon(\phi)$ iff $a\in(C_{\RCon(\phi),Y})^{\Imc_1}$.
    %
    Since the interpretations $\Imc_1,\dots,\Imc_k$ interpret the individual
    names in the same way, and for every $i$, $1\le i\le k$, the
    interpretation~$\Imc_i$ is a model of~$\Bmc_{X_i}=(\Psi_{X_i},\Rmc)$, we
    have thus that $\Imc_i$ is also a model of
    $(\Psi_{X_i}\land\Psi_\tau,\Rmc)$.
    %
    Moreover, every~$\Imc_i$, $1\le i\le k$, respects~\Dmc by construction
    of~\Dmc.
\end{proof}

\noindent
Using this lemma, we can prove our complexity result.

\begin{theorem}\label{thm:upper-bound-shoq-ltl-rigid-concepts}
    The satisfiability problem in \SHOQ-LTL is in \NExpTime if
    $\NRC\ne\emptyset$ and $\NRR=\emptyset$.
\end{theorem}

\begin{proof}
    Let \Rmc be an RBox, let $\phi$ be a \SHOQ-LTL-formula w.r.t.~\Rmc, and let
    $\psf\colon\Ax(\phi)\to\Pmc_\phi$ be a bijection.  Again, we use
    Lemma~\ref{lem:r-sat-t-sat} for deciding whether $\phi$ is satisfiable
    w.r.t.~\Rmc.  We first non-deterministically guess a set
    $\Wmc=\{X_1,\dots,X_k\}\subseteq 2^{\Pmc_\phi}$, which is of size at most
    exponential in the size of~$\phi$ and~\Rmc.
    %
    Next, we check whether $\phi^\psf$ is t-satisfiable w.r.t.~\Wmf, which can
    be done in time exponential in the size of~$\phi^\psf$ (and thus in time
    exponential in the size of~$\phi$) and linear in the size of~\Wmf by
    Lemma~\ref{lem:t-sat}.

    For the r-consistency check, we use Lemma~\ref{lem:dmc-tau}.  For that, we
    non-deterministically guess a set $\Ymc\subseteq 2^{\RCon(\phi)}$ and a
    mapping $\tau\colon\Ind(\phi)\to\Ymc$, which also can be done in time
    exponential in the size of~$\phi$ and~\Rmc.
    %
    We define $\Dmc:=(\RCon(\phi),\Ymc)$.  For every $i$, $1\le i\le k$, we
    construct the Boolean knowledge base $(\Psi_{X_i}\land\Psi_\tau,\Rmc)$,
    which is of size polynomial in the size of~$\phi$ and~\Rmc, and can be
    constructed in time exponential in the size of~$\phi$ and~\Rmc (due to the
    mapping~$\tau$).  Thus, it is only left to show that we can check whether
    the Boolean knowledge base $(\Psi_{X_i}\land\Psi_\tau,\Rmc)$ has a model
    that respects~\Dmc in time exponential in the size of the Boolean knowledge
    base, and thus exponential in the size of~$\phi$ and~\Rmc.  This follows
    immediately from Theorem~\ref{thm:cons-bmc-dmc}, which we show in
    Section~\ref{sec:consistency-boolean-shoqcap-kb}.

    Thus, overall, we can decide whether $\phi$ is satisfiable w.r.t.~\Rmc in
    \NExpTime.
\end{proof}

\noindent
Together with Theorem~\ref{thm:lower-bounds-shoq-ltl}, we obtain now that the
satisfiability problem in \SHOQ-LTL is \NExpTime-complete if only concept names
are allowed to be rigid.


\subsection{Consistency of Boolean \texorpdfstring{\SHOQcap}{SHOQ\^{}cap}-knowledge bases}\label{sec:consistency-boolean-shoqcap-kb}

In this section, we prove the result that is needed to finish the proof of
Theorem~\ref{thm:upper-bound-shoq-ltl-rigid-concepts}, namely, that the
consistency of a Boolean \SHOQ-knowledge base w.r.t.\ a pair~\Dmc can be checked
in time exponential in the size of this Boolean knowledge base, where we call a
Boolean KB \emph{consistent w.r.t.\ a pair~\Dmc} if it has a model that
respects~\Dmc.  Moreover, we derive the corollary that checking the consistency
of a Boolean \SHOQ-knowledge base (without~\Dmc) can also be done in time
exponential in the size of this Boolean knowledge base.  This result then
finishes the proofs of Theorems~\ref{thm:upper-bound-shoq-ltl-no-rigid}
and~\ref{thm:upper-bound-shoq-ltl-rigid-roles}.

In Chapter~\ref{ch:tcqs}, we deal with Boolean \SHQcap-knowledge bases.  The
description logic \SHQcap extends \SHQ with \emph{role conjunctions} of the form
$r_1\sqcap\dots\sqcap r_\ell$, $\ell\ge 1$, where $r_1,\dots,r_\ell$ are
\emph{simple} role names.  Such role conjunctions are allowed to occur in
existential restrictions instead of a single role, but not in at-least
restrictions or role assertions.  An interpretation~\Imc is extended to role
conjunction as follows:
$(r_1\sqcap\dots\sqcap r_\ell)^\Imc:=r_1^\Imc\cap\dots\cap r_\ell^\Imc$.
Therefore, in this section, we consider Boolean \SHOQcap-knowledge bases rather
than Boolean \SHOQ-knowledge bases.

In the following, let $\Bmc=(\Psi,\Rmc)$ be a Boolean \SHOQcap-knowledge base,
and let $\Dmc=(\Umc,\Ymc)$ be a pair such that \Umc is a set of concept names
occurring in~\Bmc and $\Ymc\subseteq 2^\Umc$.
%
We assume here that all GCIs occurring in~$\Psi$ are of the form
$\top\sqsubseteq C$; this is without loss of generality since any GCI
$C\sqsubseteq D$ is equivalent to $\top\sqsubseteq\lnot(C\sqcap\lnot D)$.

We show that consistency of~\Bmc w.r.t.~\Dmc can be decided in time exponential
in the size of~\Bmc.  This complexity result is tight since
already for \enquote{classical} \SHQcap-knowledge bases (without nominals), the
consistency problem (without~\Dmc) is
\ExpTime-complete~\cite{Tob-PhD01,Lut-IJCAR08}.
%
The complexity of this problem even remains in \ExpTime when simple role
conjunctions are allowed to occur in at-least restrictions and non-simple roles
are allowed in role conjunctions in existential restrictions~\cite{GlKa-LPAR08}.
%
However, if we move to \SHOQcap, i.e.~we consider \enquote{classical}
\SHOQcap-knowledge bases where simple role conjunctions are allowed to occur in
at-least restrictions and non-simple roles are allowed in role conjunctions in
existential restrictions, the best known upper bound of the consistency problem
(without~\Dmc) is \TwoExpTime~\cite{GlHS-KR08,Gli-PhD07}.

The proof of our result is an adaptation of the proof of Lemma~6.4
in~\cite{BaGL-ToCL12}, which is again an adaptation of the proof of Theorem~2.27
in~\cite{GKW+-03}, which shows that consistency of Boolean \ALC-knowledge bases
can be decided in exponential time.
%
An earlier version of this proof for \ALCcap can be found
in~\cite{BaBL-LTCS-13-01,BaBL-CADE13}.
%
There, for role conjunctions, additional concept names are introduced that
function as so-called \emph{pebbles} that mark elements that have specific role
predecessors, an idea borrowed from~\cite{Dan-SCT84,DeMa-IC00,Mas-IJCAI01}.
%
%TODO
%Moreover, in~\cite{tcq-journal}, the result was shown for Boolean
%\SHQcap-knowledge bases.  However, this proof uses the forest-model property of
%\SHQcap (see Definition~\ref{def:forest-model} for a definition of forest
%models), which does not hold for \SHOQcap.  Moreover, the proof gets simpler if
%nominals are allowed.
%
%Similar to the proof in~\cite{tcq-journal}, 
Here, we employ systems of equations over
non-negative integers to deal with role conjunctions, transitivity axioms,
role-inclusion axioms, and at-least restrictions simultaneously.

For the subsequent construction, we extend the notion of a \emph{quasimodel}
from~\cite{BaGL-ToCL12}, which is an abstract description of a model.
Quasimodels characterise domain elements by the concepts they satisfy.
%
We start by introducing several auxiliary notions that we need in the
construction.

We define $\Con(\Psi)$ to be the set of all concepts occurring in~$\Psi$,
$\Ind(\Psi)$ to be the set of all individual names occurring in~$\Psi$, and
$\Rol(\Bmc)$ to the set of all role names occurring in~\Bmc.  Then, $\CCl(\Bmc)$
is defined to be the closure under negation of the set
\begin{align*}
    \Con(\Psi)
    &\cup\bigl\{\exists r.C\mid\exists s.C\in\Con(\Psi),\Rmc\models r\sqsubseteq s,\ \text{and}\ \Rmc\models\trans(r)\bigr\}\\
    &\cup\bigl\{\{a\}\mid a\in\Ind(\Psi)\bigr\}\\
    &\cup\bigl\{\exists r.\{a\}\mid r\in\Rol(\Bmc)\ \text{and}\ a\in\Ind(\Psi)\bigr\}.
\end{align*}
%
The reason why we consider these additional sets is that they are needed to
properly deal with transitive roles and nominals (see
Definition~\ref{def:concept-type}).
%
Similarly, we define $\FCl(\Psi)$ to be the closure under negation of the set of
all subformulas of~$\Psi$.

In the following, we identify $\lnot\lnot\psi$ with $\psi$ for every
concept~$\psi$.  Similarly, we identify $\lnot\lnot\Psi$ with $\Psi$ for every
Boolean axiom formula~$\Psi$.  Thus, all sets introduced above are of size
polynomial in the size of~\Bmc, and can also be constructed in time polynomial
in the size of~\Bmc.

\begin{definition}[Concept type]\label{def:concept-type}
    A \emph{concept type} for~\Bmc is a set $\cbb\subseteq\CCl(\Bmc)$ such that:
    \begin{itemize}
        \item for every $C\sqcap D\in\CCl(\Bmc)$, we have $C\sqcap D\in\cbb$ iff
            $\{C,D\}\subseteq\cbb$;
        \item for every $\lnot C\in\CCl(\Bmc)$, we have $\lnot C\in\cbb$ iff
            $C\notin\cbb$;
        \item for every $\{a\}\in\CCl(\Bmc)$, we have $\{a\}\in\cbb$
            implies $\{b\}\notin\cbb$ for every $\{b\}\in\CCl(\Bmc)$ with
            $\{b\}\ne\{a\}$; and
        \item for every $\exists r.\{a\}\in\CCl(\Bmc)$, we have that if $\exists
            r.\{a\}\in\cbb$ and $\Rmc\models r\sqsubseteq s$, then $\exists
            s.\{a\}\in\cbb$.
    \end{itemize}
    %
    Given two concept types $\cbb,\dbb$ for~\Bmc and a role name
    $r\in\Rol(\Bmc)$, we say that \cbb and \dbb are \emph{$r$-compatible
    w.r.t.~\Rmc} (written $\rcomp{\cbb}{\dbb}{r}{\Rmc}$)
    if the following conditions
    are satisfied:
    \begin{itemize}
        \item for every $\lnot(\exists r.D)\in\cbb$, we have $\lnot D\in\dbb$;
            and
        \item for every $s\in\Rol(\Bmc)$ with $\Rmc\models r\sqsubseteq s$,
            $\Rmc\models\trans(r)$, and $\lnot(\exists s.D)\in\cbb$, we have
            $\lnot(\exists r.D)\in\dbb$.
    \end{itemize}
\end{definition}

\noindent
Obviously, the number of concept types for~\Bmc is exponential in the size
of~\Bmc.
%
Intuitively, the $r$-compatibility of two concept types $\cbb,\dbb$ w.r.t.~\Rmc
indicates that it is possible to connect them with an $r$-edge without violating
the value restrictions in~\cbb.  These conditions are very similar to the
tableau rules $(\forall)$ and $(\forall_+)$ that deal with value restrictions in
the presence of role-inclusion axioms and transitivity axioms (see
e.g.~\cite{HoST-IGPL00}).

\begin{definition}[Role type]\label{def:role-type}
    A \emph{role type} for~\Bmc is a set $\rbb\subseteq\Rol(\Bmc)$ such that:
    \begin{itemize}
        \item if $\Rmc\models s\sqsubseteq r$, then $s\in\rbb$ implies
            $r\in\rbb$.
    \end{itemize}
    We denote the set of all role types for~\Bmc by $\Rmf(\Bmc)$.

    Given two concept types $\cbb,\dbb$ for~\Bmc and a role type
    $\rbb\in\Rmf(\Bmc)$, we say that \cbb and \dbb are \emph{\rbb-compatible
    w.r.t.~\Rmc} (written $\rcomp{\cbb}{\dbb}{\rbb}{\Rmc}$) iff
    $\rcomp{\cbb}{\dbb}{r}{\Rmc}$ for every $r\in\rbb$.
\end{definition}

\noindent
Again, the number of role types for~\Bmc is exponential in the size of~\Bmc.
%
Finally, a quasimodel also has to determine which of the axioms in~$\Psi$ it
satisfies.

\begin{definition}[Formula type]\label{def:formula-type}
    A \emph{formula type} for~\Bmc is a set $\fbb\subseteq\FCl(\Psi)$ such that:
    \begin{itemize}
        \item $\Psi\in\fbb$;
        \item for every $\lnot\psi\in\FCl(\Psi)$, we have $\lnot\psi\in\fbb$ iff
            $\psi\notin\fbb$; and
        \item for every $\Psi_1\land\Psi_2\in\FCl(\Psi)$, we have
            $\Psi_1\land\Psi_2\in\fbb$ iff $\{\Psi_1,\Psi_2\}\subseteq\fbb$.
    \end{itemize}
\end{definition}

\noindent
The number of formula types for~\Bmc is exponential in the size of~\Bmc.
%
Using these definitions, we can now define the notion of a model candidate, and
later refine this notion to characterise quasimodels.

\begin{definition}[Model candidate]\label{def:model-candidate}
    A \emph{model candidate} for~\Bmc is a triple $\Mmc=(\Wmc,\iota,\fbb)$ such
    that:
    \begin{itemize}
        \item \Wmc is a set of concept types for~\Bmc such that for any
            $\cbb,\dbb\in\Wmc$ with $\cbb\ne\dbb$, we have
            $\cbb\cap\dbb\cap\bigl\{\{a\}\mid a\in\Ind(\Psi)\bigr\}=\emptyset$;
        \item $\iota\colon\Ind(\Psi)\to\Wmc$ is a function such that
            $\{a\}\in\iota(a)$ for every $a\in\Ind(\Psi)$; and
        \item \fbb is a formula type for~\Bmc.
    \end{itemize}
\end{definition}

\noindent
Intuitively, the set~\Wmc determines the behaviour of the domain elements, while
the function~$\iota$ fixes the interpretation of the named domain elements, and
the formula type~\fbb ensures that~\Bmc is satisfied.
%
In the following, we denote by~\Wmcu the set $\Wmc\setminus\iota(\Ind(\Psi))$,
i.e.~the set of all those concept types for~\Bmc that do not contain a nominal
concept $\{a\}$ with $a\in\Ind(\Psi)$.  Those types represent the unnamed domain
elements of the model candidate.
%
To define quasimodels, we add to the definition of a model candidate several
conditions that ensure that it can indeed be transformed into a model of~\Bmc.

To satisfy the (negated) at-least restrictions in the concept types of a model
candidate $\Mmc=(\Wmc,\iota,\fbb)$, we introduce for each $\cbb\in\Wmc$, a
system of equations $E_{\Mmc,\cbb}$ with variables ranging over the non-negative
integers.
%
In $E_{\Mmc,\cbb}$, we use variables of the form $x_{\cbb,\rbb,\dbb}$, which
determine for a domain element of type~\cbb, the number of \emph{unnamed} role
successors of type~\rbb (called \rbb-successors) of concept type~\dbb, where we
require that $\rcomp{\cbb}{\dbb}{\rbb}{\Rmc}$ and $\dbb\in\Wmcu$, i.e.~\cbb
and~\dbb are \rbb-compatible w.r.t.~\Rmc and \dbb does not represent a named
individual.

Given $\cbb\in\Wmc$, $C\in\CCl(\Bmc)$, and $\rbb\in\Rmf(\Bmc)$, we can now count
the number of \emph{unnamed} $\rbb$-successors of~\cbb that satisfy~$C$ using
the following expression:
\[\Xi_{\Mmc,\cbb,\rbb,C}:=
    \sum_{C\in\dbb\in\Wmcu,\ \rcomp{\cbb}{\dbb}{\rbb}{\Rmc}}
    x_{\cbb,\rbb,\dbb}.\]
%
To count the \emph{named} $\rbb$-successors of~\cbb that satisfy~$C$, we define
the following constant:
\[\Gamma_{\Mmc,\cbb,\rbb,C}:=
    \bigl\lvert\bigl\{b\in\Ind(\Psi)\mid\text{$C\in\iota(b)$, and $\exists
        r.\{b\}\in\cbb$ iff $r\in\rbb$}\bigr\}\bigr\rvert.\]
%
To ensure that the at-least restrictions in~\cbb are satisfied, we add the
following equation to $E_{\Mmc,\cbb}$ for each $\atLeast{n}{r}{C}\in\cbb$:
\begin{equation}\label{eq:atleast}\tag{E1}
    -y_{\cbb,\atLeast{n}{r}{C}}+
    \sum_{r\in\rbb\in\Rmf(\Bmc)}(\Xi_{\Mmc,\cbb,\rbb,C}+\Gamma_{\Mmc,\cbb,\rbb,C})=n,
\end{equation}
where $y_{\cbb,\atLeast{n}{r}{C}}$ is a slack variable that is used to obtain an
equation instead of an inequation.
%
Similarly, for each $\lnot(\atLeast{n}{r}{C})\in\cbb$, we add
\begin{equation}\label{eq:atmost}\tag{E2}
    y_{\cbb,\lnot(\atLeast{n}{r}{C})}+
    \sum_{r\in\rbb\in\Rmf(\Bmc)}(\Xi_{\Mmc,\cbb,\rbb,C}+\Gamma_{\Mmc,\cbb,\rbb,C})=n-1.
\end{equation}
%
For each existential restriction $D=\exists(r_1\sqcap\dots\sqcap r_\ell).C\in\cbb$,
we add the following equation to $E_{\Mmc,\cbb}$:
\begin{equation}\label{eq:exists}\tag{E3}
    -y_{\cbb,D}+
    \sum_{\{r_1,\dots,r_\ell\}\subseteq\rbb\in\Rmf(\Bmc)}(\Xi_{\Mmc,\cbb,\rbb,C}+\Gamma_{\Mmc,\cbb,\rbb,C})=1.
\end{equation}
%
Finally, for each $\lnot(\exists(r_1\sqcap\dots\sqcap r_\ell).C)\in\cbb$, we
add the equation
\begin{equation}\label{eq:forall}\tag{E4}
    \sum_{\{r_1,\dots,r_\ell\}\subseteq\rbb\in\Rmf(\Bmc)}(\Xi_{\Mmc,\cbb,\rbb,C}+\Gamma_{\Mmc,\cbb,\rbb,C})=0,
\end{equation}
where no slack variable is needed since the sum cannot be smaller than $0$.

This finishes the description of the system of equations $E_{\Mmc,\cbb}$.  Note
that this system contains \emph{exponentially} many variables in the size
of~\Bmc, but only \emph{polynomially} many equations, and thus it can be solved
in exponential time, even if the numbers are given in binary
encoding~\cite{Pap-JACM81} (for details, see the proof of
Theorem~\ref{thm:cons-bmc-dmc}).

Now we are ready to introduce the notion of a quasimodel.

\begin{definition}[Quasimodel]\label{def:quasimodel}
    The model candidate $\Mmc=(\Wmc,\iota,\fbb)$ for~\Bmc is a \emph{quasimodel}
    for~\Bmc if it satisfies the following properties:
    \begin{enumerate}[label=(\alph*)]
        \item\label{enum:qm-nonempty}
            \Wmc is not empty;
        \item\label{enum:qm-concept}
            for every $A(a)\in\FCl(\Psi)$, we have $A(a)\in\fbb$ iff
            $A\in\iota(a)$;
        \item\label{enum:qm-role}
            for every $r(a,b)\in\FCl(\Psi)$, we have $r(a,b)\in\fbb$ iff
            $\exists r.\{b\}\in\iota(a)$;
        \item\label{enum:qm-gci}
            for every $\top\sqsubseteq C\in\fbb$ and every $\cbb\in\Wmc$, we
            have $C\in\cbb$;
        \item\label{enum:qm-gci-neg}
            for every $\lnot(\top\sqsubseteq C)\in\fbb$, there is a
            $\cbb\in\Wmc$ such that $C\notin\cbb$;
        \item\label{enum:qm-rcomp}
            for every $\cbb\in\Wmc$, if $\exists r.\{a\}\in\cbb$, then
            $\rcomp{\cbb}{\iota(a)}{r}{\Rmc}$; and
        \item\label{enum:qm-equations}
            for every $\cbb\in\Wmc$, the system of equations $E_{\Mmc,\cbb}$ has
            a solution over the non-negative integers.
    \end{enumerate}
    %
    The quasimodel $\Mmc=(\Wmc,\iota,\fbb)$ for~\Bmc
    \emph{respects~$\Dmc=(\Umc,\Ymc)$} if it additionally satisfies:
    \begin{enumerate}[label=(\alph*),resume]
        \item\label{enum:qm-dmc1}
            for every $\cbb\in\Wmc$, there is a set $Y\in\Ymc$ such that
            $Y=\cbb\cap\Umc$; and
        \item\label{enum:qm-dmc2}
            for every $Y\in\Ymc$, there is a concept type $\cbb\in\Wmc$ such
            that $Y=\cbb\cap\Umc$.
    \end{enumerate}
\end{definition}

\noindent
We can show to in order to check consistency of~\Bmc w.r.t.~\Dmc it suffices to
search for a quasimodel for~\Bmc that respects~\Dmc.

\begin{lemma}\label{lem:quasimodels}
    Let \Bmc be a Boolean \SHOQcap-knowledge base, and let
    $\Dmc=(\Umc,\Ymc)$ be a pair such that \Umc is a set of concept
    names occurring in~\Bmc and $\Ymc\subseteq 2^\Umc$.  Then, \Bmc is
    consistent w.r.t.~\Dmc iff there is a quasimodel for~\Bmc that
    respects~\Dmc.
\end{lemma}

\begin{proof}
    For the \enquote{if} direction, suppose that $\Mmc=(\Wmc,\iota,\fbb)$ is a
    quasimodel for $\Bmc=(\Psi,\Rmc)$ that respects~\Dmc.
    %
    Then by Condition~\ref{enum:qm-equations}, for each $\cbb\in\Wmc$, the
    system of equations $E_{\Mmc,\cbb}$ has a solution $\nu_{\cbb}$ that maps
    the variables in $E_{\Mmc,\cbb}$ to non-negative integers.  Let $z_\Mmc$ be
    the greatest non-negative integer that occurs in any of these solutions, and
    let \Zmf denote the set $\{1,\dots,z_\Mmc\}$.

    We define the interpretation $\Jmc=(\Delta^\Jmc,\cdot^\Jmc)$ as follows:
    \begin{itemize}
        \item $\Delta^\Jmc:=\Anon\cup\Ind(\Psi)$, where
            $\Anon:=\Wmcu\times\Zmf\times\Rmf(\Bmc)$;
        \item $a^\Jmc:=a$ for every $a\in\Ind(\Psi)$;\footnote{%
                For now, we ignore the individual names in
                $\NI\setminus\Ind(\Psi)$ since they are irrelevant for the
                consistency of~\Bmc.  After constructing the model~\Imc below,
                one can ensure that it respects the UNA by constructing the
                countably infinite disjoint union of~\Imc with itself to allow
                for different interpretations of each of these individual names.}
        \item $A^\Jmc:=\{(\cbb,i,\rbb)\in\Anon\mid A\in\cbb\}\cup\{a\in\Ind(\Psi)\mid A\in\iota(a)\}$
            for every $A\in\NC$; and
        \item for every $r\in\NR$, $(\cbb,i,\rbb),(\dbb,j,\sbb)\in\Anon$, and
            $a,b\in\Ind(\Psi)$, we define:
            \[\begin{array}{lcl}
                (a,b)\in r^\Jmc
                    &\text{iff}
                    &\text{$\exists r.\{b\}\in\iota(a)$;}\\
                \bigl((\cbb,i,\rbb),b\bigr)\in r^\Jmc
                    &\text{iff}
                    &\text{$\exists r.\{b\}\in\cbb$;}\\
                (a,(\dbb,j,\sbb))\in r^\Jmc
                    &\text{iff}
                    &\text{$r\in\sbb$, $\rcomp{\iota(a)}{\dbb}{\sbb}{\Rmc}$, and
                        $\nu_{\iota(a)}(x_{\iota(a),\sbb,\dbb})\ge j$;}\\
                \bigl((\cbb,i,\rbb),(\dbb,j,\sbb)\bigr)\in r^\Jmc
                    &\text{iff}
                    &\text{$r\in\sbb$, $\rcomp{\cbb}{\dbb}{\sbb}{\Rmc}$, and
                        $\nu_{\cbb}(x_{\cbb,\sbb,\dbb})\ge j$.}
            \end{array}\]
    \end{itemize}
    %
    Note that $\Delta^\Jmc\ne\emptyset$ since even if $\Ind(\Psi)=\emptyset$, we
    have that $\Wmcu\ne\emptyset$ by Condition~\ref{enum:qm-nonempty}.

    Now we construct a model $\Imc=(\Delta^\Imc,\cdot^\Imc)$ of~\Bmc by defining
    $\Delta^\Imc:=\Delta^\Jmc$, for each $A\in\NC$, $A^\Imc:=A^\Jmc$, for each
    $a\in\Ind(\Psi)$, $a^\Imc:=a^\Jmc$, and for each $r\in\NR$,
    \[r^\Imc:=r^\Jmc\cup\bigcup_{\Rmc\models s\sqsubseteq r,\ \Rmc\models\trans(s)}(s^\Jmc)^+,\]
    where $\cdot^+$ denotes the transitive closure.

    We denote by $\kappa\colon\Delta^\Imc\to\Wmc$ the following function:
    \[\kappa(d):=\begin{cases}
            \cbb        &\text{if $d=(\cbb,i,\rbb)\in\Anon$, and}\\
            \iota(b)    &\text{if $d=b\in\Ind(\Psi)$.}
        \end{cases}\]
    %
    The following claim can be proved by a careful case distinction.

    \begin{claim}\label{claim:qm-kappa}
        Let $d\in\Delta^\Imc$.  If $\lnot(\exists r.D)\in\kappa(d)$, and there
        is an $s\in\Rol(\Bmc)$ and an $e\in\Delta^\Imc$ with $\Rmc\models
        s\sqsubseteq r$ and $(d,e)\in s^\Jmc$, then we have:
        \begin{itemize}
            \item $\lnot D\in\kappa(e)$, and
            \item if $\Rmc\models\trans(s)$, then $\lnot(\exists
                s.D)\in\kappa(e)$.
        \end{itemize}
    \end{claim}

    \noindent
    Assume first that $e=b\in\Ind(\Psi)$.  Then the definition of~\Jmc yields
    that $\exists s.\{b\}\in\kappa(d)$.  Since $\kappa(d)$ is a concept type, we
    by Definition~\ref{def:concept-type} that $\exists r.\{b\}\in\kappa(d)$.  By
    Condition~\ref{enum:qm-rcomp}, this implies
    $\rcomp{\kappa(d)}{\iota(b)}{r}{\Rmc}$, and thus
    $\rcomp{\kappa(d)}{\kappa(e)}{r}{\Rmc}$.  By
    Definition~\ref{def:concept-type}, we obtain $\lnot D\in\kappa(e)$.
    Moreover, by Condition~\ref{enum:qm-rcomp}, we have
    $\rcomp{\kappa(d)}{\iota(b)}{s}{\Rmc}$, and thus
    $\rcomp{\kappa(d)}{\kappa(e)}{s}{\Rmc}$.  Hence, if $\Rmc\models\trans(s)$,
    we have by Definition~\ref{def:concept-type} also that $\lnot(\exists
    s.D)\in\kappa(e)$.

    Assume now that $e=(\dbb,j,\sbb)\in\Anon$.  The definition of~\Jmc yields
    that $s\in\sbb$, $\rcomp{\kappa(d)}{\dbb}{\sbb}{\Rmc}$, and
    $\nu_{\kappa(d)}(x_{\kappa(d),\sbb,\dbb})\ge j$.  Since \sbb is a role type,
    and $\Rmc\models s\sqsubseteq r$, we have $r\in\sbb$.  Thus, we have by
    Definition~\ref{def:role-type} that $\rcomp{\kappa(d)}{\dbb}{r}{\Rmc}$.  By
    Definition~\ref{def:concept-type}, we obtain $\lnot D\in\dbb$, and thus
    $\lnot D\in\kappa(e)$.  Moreover, since $s\in\sbb$, we have by
    Definition~\ref{def:role-type} that $\rcomp{\kappa(d)}{\dbb}{s}{\Rmc}$.
    Hence, if $\Rmc\models\trans(s)$, we have by
    Definition~\ref{def:concept-type} also that $\lnot(\exists s.D)\in\dbb$, and
    thus $\lnot(\exists s.D)\in\kappa(e)$.
    %
    This finishes the proof of Claim~\ref{claim:qm-kappa}.

    Using Claim~\ref{claim:qm-kappa}, we now prove the following claim by
    structural induction.
    \begin{claim}\label{claim:qm-concepts}
        For every concept $C\in\CCl(\Bmc)$, we have
        $C^\Imc=\{d\in\Delta^\Imc\mid C\in\kappa(d)\}$.
    \end{claim}

    \noindent
    For the base case, $C$ being a concept name, the definition of~\Imc nd the
    definition of~$\kappa$ immediately imply the claim.

    For the case that $C$ is of the form $\lnot D$, we have by the semantics of
    \SHOQcap, the induction hypothesis, the definition of~\Imc, the definition
    of~$\kappa$, and the definition of concept types the following for every
    $d\in\Delta^\Imc$:
    \[d\in(\lnot D)^\Imc\text{ \emph{iff} } d\notin D^\Imc\text{ \emph{iff} }
        D\notin\kappa(d)\text{ \emph{iff} }\lnot D\in\kappa(d).\]

    \noindent
    For the case that $C$ is of the form $D\sqcap E$, we have by similar
    arguments the following for every $d\in\Delta^\Imc$:
    \[d\in(D\sqcap E)^\Imc\text{ \emph{iff} }d\in D^\Imc\text{ and }d\in E^\Imc
        \text{ \emph{iff} }D\in\kappa(d)\text{ and }E\in\kappa(d)\text{ \emph{iff} }
        D\sqcap E\in\kappa(d).\]

    \noindent
    For the case that $C$ is of the form $\exists(r_1\sqcap\dots\sqcap
    r_\ell).D$, we have by similar arguments the following:
    \begin{multline*}
        (\exists(r_1\sqcap\dots\sqcap r_\ell).D)^\Imc\\
        \begin{aligned}
            &=\{d\in\Delta^\Imc\mid\text{there is an $e\in\Delta^\Imc$ with
                $(d,e)\in r_1^\Imc\cap\dots\cap r_\ell^\Imc$ and $e\in D^\Imc$}\}\\
            &=\{d\in\Delta^\Imc\mid\text{there is an $e\in\Delta^\Imc$ with
                $(d,e)\in r_1^\Imc\cap\dots\cap r_\ell^\Imc$ and $D\in\kappa(e)$}\}\\
            &\overset{\ast}{=}\{d\in\Delta^\Imc\mid\exists(r_1\sqcap\dots\sqcap r_\ell).D\in\kappa(d)\}
        \end{aligned}
    \end{multline*}
    %
    The starred equality $\overset{\ast}{=}$ holds due to the following
    arguments.
    %
    Assume, for the direction~($\supseteq$), that $d\in\Delta^\Imc$ and
    $\exists(r_1\sqcap\dots\sqcap r_\ell).D\in\kappa(d)$.  Since
    $\nu_{\kappa(d)}$ solves~\eqref{eq:exists}, there is
    an $\rbb\in\Rmc(\Bmc)$ such that
    $\{r_1,\dots,r_\ell\}\subseteq\rbb$ and
    \begin{itemize}
        \item either there is a $\dbb\in\Wmcu$ with $D\in\dbb$,
            $\rcomp{\kappa(d)}{\dbb}{\rbb}{\Rmc}$, and
            $\nu_{\kappa(d)}(x_{\kappa(d),\rbb,\dbb})\ge 1$; or
        \item there is a $b\in\Ind(\Psi)$ such that $D\in\iota(b)$ and
            $\bigl\{\exists r_1.\{b\},\dots,\exists
            r_\ell.\{b\}\bigr\}\subseteq\kappa(d)$.
    \end{itemize}
    The definition of $r_1^\Jmc,\dots,r_\ell^\Jmc$ yields in the first case that
    \[(d,(\dbb,1,\rbb))\in r_1^\Jmc\cap\dots\cap
        r_\ell^\Jmc\subseteq r_1^\Imc\cap\dots\cap r_\ell^\Imc,\]
    and in the second case that
    \[(d,b)\in r_1^\Jmc\cap\dots\cap
        r_\ell^\Jmc\subseteq r_1^\Imc\cap\dots\cap r_\ell^\Imc.\]
    %
    Since we have also $D\in\kappa\bigl((\dbb,1,\rbb)\bigr)$ and
    $D\in\kappa(b)$, this finishes this direction.

    For the other direction~($\subseteq$), take any $d\in\Delta^\Imc$ such that
    there is an $e\in\Delta^\Imc$ with the property that
    $(d,e)\in r_1^\Imc\cap\dots\cap r_\ell^\Imc$ and $D\in\kappa(e)$.  We show
    that $C=\exists(r_1\sqcap\dots\sqcap r_\ell).D\in\kappa(d)$.  Assume to the
    contrary that $C\notin\kappa(d)$, and thus $\lnot C\in\kappa(d)$.
    \begin{itemize}
        \item For the case $\ell>1$, we have that $r_1,\dots,r_\ell$ are
            \emph{simple} role names, and thus that $(d,e)\in
            r_1^\Jmc\cap\dots\cap r_\ell^\Jmc$.  For the case that
            $e=b\in\Ind(\Psi)$, we have by the definition of~\Jmc that
            $\bigl\{\exists r_1.\{b\},\dots,\exists
            r_\ell.\{b\}\bigr\}\in\kappa(d)$.  Take the set
            $\rbb:=\bigl\{r\in\Rol(\Bmc)\mid\exists r.\{b\}\in\kappa(d)\bigr\}$.
            Since $\kappa(d)$ is a concept type, we have by
            Definition~\ref{def:concept-type} that \rbb is a role type that
            contains $r_1,\dots,r_\ell$.  Since $D\in\kappa(e)=\iota(b)$, we
            have that $\Gamma_{\Mmc,\kappa(d),\rbb,D}\ge 1$, which contradicts
            the assumption that~\eqref{eq:forall} has a solution.
            %
            For the case that $e=(\dbb,j,\sbb)\in\Anon$, we have by the
            definition of~\Jmc that $\{r_1,\dots,r_\ell\}\subseteq\sbb$,
            $\rcomp{\kappa(d)}{\dbb}{\sbb}{\Rmc}$, and
            $\nu_{\kappa(d)}(x_{\kappa(d),\sbb,\dbb})\ge j\ge 1$.  Since
            $\nu_{\kappa(d)}$ is a solution of~\eqref{eq:forall}, we must have
            $\nu_{\kappa(d)}(x_{\kappa(d),\sbb,\dbb})=0$, which is again a
            contradiction.
        \item For the case $\ell=1$, we have by the definition of $r_1^\Imc$
            that $(d,e)\in r_1^\Jmc$ or $(d,e)\in (s^\Jmc)^+$ for some $s\in\NR$
            with $\Rmc\models s\sqsubseteq r_1$ and $\Rmc\models\trans(s)$.  The
            first case can be handled as in the case of $\ell>1$.  In the second
            case, there is a sequence $d_0,\dots,d_n$ in $\Delta^\Imc$ such
            that $n\ge 1$, $d_0=d$, $d_n=e$, and for every $k$, $0\le k\le n-1$,
            we have that $(d_k,d_{k+1})\in s^\Jmc$.
            \begin{itemize}
                \item If $n=1$, then we have $(d,e)\in s^\Jmc$.  Thus, we have
                    by Claim~\ref{claim:qm-kappa} that $\lnot D\in\kappa(e)$,
                    which is a contradiction.
                \item If $n>1$, we have by Claim~\ref{claim:qm-kappa} that
                    $\lnot(\exists s.D)\in\kappa(d_1)$.  Since $\Rmc\models
                    s\sqsubseteq s$, using Claim~\ref{claim:qm-kappa} again, we
                    obtain that $\lnot(\exists s.D)\in\kappa(d_{n-1})$.  By
                    Claim~\ref{claim:qm-kappa}, we have $\lnot
                    D\in\kappa(d_n)=\kappa(e)$, which is again a contradiction.
            \end{itemize}
    \end{itemize}

    \noindent
    Finally, consider the case that $C$ is of the form $\atLeast{n}{r}{D}$.
    Recall that $r$ must be \emph{simple}, and thus $r^\Imc=r^\Jmc$.
    %
    We first count, for any element $d\in\Delta^\Imc$, the number $n_1$ of
    \emph{unnamed} $r^\Jmc$-successors that satisfy $D$.  For a fixed role type
    $\sbb\in\Rmf(\Bmc)$ and concept type $\dbb\in\Wmcu$ with $r\in\sbb$,
    $D\in\dbb$, and $\rcomp{\kappa(d)}{\dbb}{\sbb}{\Rmc}$, we have by definition
    of~\Jmc that $(d,(\dbb,j,\sbb))\in r^\Jmc$ iff
    $\nu_{\kappa(d)}(x_{\kappa(d),\sbb,\dbb})\ge j$.  Thus, the number of
    $r^\Jmc$-successors of $d$ that are of the form $(\dbb,j,\sbb)$ is exactly
    $\nu_{\kappa(d)}(x_{\kappa(d),\sbb,\dbb})$.  By induction, we obtain the
    following:
    \begin{align*}
        n_1
        &=\lvert\{(\dbb,j,\sbb)\in\Anon\mid(d,(\dbb,j,\sbb))\in r^\Jmc,\ (\dbb,j,\sbb)\in D^\Imc\}\rvert\\[1ex]
        &=\lvert\{(\dbb,j,\sbb)\in\Anon\mid(d,(\dbb,j,\sbb))\in r^\Jmc,\ D\in\dbb\}\rvert\\[1ex]
        &=\sum_{\substack{r\in\sbb\in\Rmf(\Bmc)\\[.5ex] D\in\dbb\in\Wmcu,\ \rcomp{\kappa(d)}{\dbb}{\sbb}{\Rmc}}}
            \lvert\{j\in\Zmf\mid(d,(\dbb,j,\sbb))\in r^\Jmc\}\rvert\\[1ex]
        &=\sum_{\substack{r\in\sbb\in\Rmf(\Bmc)\\[.5ex] D\in\dbb\in\Wmcu,\ \rcomp{\kappa(d)}{\dbb}{\sbb}{\Rmc}}}
            \nu_{\kappa(d)}(x_{\kappa(d),\sbb,\dbb})\
        =\sum_{r\in\sbb\in\Rmf(\Bmc)}\nu_{\kappa(d)}(\Xi_{\Mmc,\kappa(d),\sbb,D}).
    \end{align*}
    %
    Similarly, we count the number $n_2$ of \emph{named} $r^\Jmc$-successors of
    $d\in\Delta^\Imc$ that satisfy~$D$.  Take again the set
    $\rbb:=\bigl\{r\in\Rol(\Bmc)\mid\exists r.\{b\}\in\kappa(d)\bigr\}$.  Since
    $\kappa(d)$ is a concept type, we have by Definition~\ref{def:concept-type}
    that \rbb is a role type.  By the definition of~\Jmc, this yields for every
    $b\in\Ind(\Psi)$ that $(d,b)\in r^\Jmc$ iff $\exists r.\{b\}\in\kappa(d)$
    iff $r\in\rbb$.  Thus, by induction, we obtain the following:
    \begin{align*}
        n_2
        &=\lvert\{b\in\Ind(\Psi)\mid(d,b)\in r^\Jmc,\ b\in D^\Imc\}\rvert\\[1ex]
        &=\lvert\{b\in\Ind(\Psi)\mid(d,b)\in r^\Jmc,\ D\in\iota(b)\}\rvert\\[1ex]
        &=\sum_{r\in\sbb\in\Rmf(\Bmc)}
            \lvert\{b\in\Ind(\Psi)\mid\text{$D\in\iota(b)$, and $(d,b)\in s^\Jmc$ iff $s\in\sbb$}\}\rvert\\[1ex]
        &=\sum_{r\in\sbb\in\Rmf(\Bmc)}
            \bigl\lvert\bigl\{b\in\Ind(\Psi)\mid\text{$D\in\iota(b)$, and $\exists
            s.\{b\}\in\kappa(d)$ iff $s\in\sbb$}\bigr\}\bigr\rvert\\[1ex]
        &=\sum_{r\in\sbb\in\Rmf(\Bmc)}
            \Gamma_{\Mmc,\kappa(d),\sbb,D}.
    \end{align*}
    %
    For every $d\in\Delta^\Imc$, we know that $\nu_{\kappa(d)}$ solves the
    equations in~\eqref{eq:atleast} and~\eqref{eq:atmost}.  Thus, we have
    $\atLeast{n}{r}{D}\in\kappa(d)$ iff the sum of $n_1$ and $n_2$ is greater or
    equal to $n$ iff $d\in(\atLeast{n}{r}{D})^\Imc$.
    %
    This finishes the proof of Claim~\ref{claim:qm-concepts}.

    To show that \Imc is indeed a model of~\Bmc, we first show the following
    claim by structural induction.

    \begin{claim}\label{claim:qm-formula-type}
        For all $\psi\in\FCl(\Psi)$, we have $\psi\in\fbb$ iff
        $\Imc\models\psi$.
    \end{claim}

    \noindent
    For the first base case, assume that $\psi$ is of the form $A(a)$ for
    $A\in\NC$ and $a\in\NI$.  We have $A(a)\in\fbb$ iff $A\in\iota(a)$ by
    Condition~\ref{enum:qm-concept}.  Thus, $A(a)\in\fbb$ iff
    $a^\Imc=a^\Jmc=a\in A^\Jmc=A^\Imc$ iff $\Imc\models A(a)$.

    For the second base case, assume that $\psi$ is of the form $r(a,b)$ for
    $a,b\in\NI$ and $r\in\NR$.  If $r(a,b)\in\fbb$, we have by
    Condition~\ref{enum:qm-role} that $\exists r.\{b\}\in\iota(a)$, and thus
    $(a,b)\in r^\Jmc$ by the definition of~$r^\Jmc$.  Since $r^\Jmc\subseteq
    r^\Imc$, $a=a^\Imc$, and $b=b^\Imc$, we obtain $(a^\Imc,b^\Imc)\in r^\Imc$,
    and thus $\Imc\models r(a,b)$.

    Conversely, if $\Imc\models r(a,b)$, we have by the definition of~$r^\Imc$
    that $(a,b)\in r^\Jmc$ or $(a,b)\in(s^\Jmc)^+$ for some $s\in\NR$ with
    $\Rmc\models s\sqsubseteq r$ and $\Rmc\models\trans(s)$.  If $(a,b)\in
    r^\Jmc$, the definition of $r^\Jmc$ implies that $\exists
    r.\{b\}\in\iota(a)$.  This yields by Condition~\ref{enum:qm-role} that
    $r(a,b)\in\fbb$.
    %
    Otherwise, there is a sequence $d_0,\dots,d_n$ in $\Delta^\Imc$ such that
    $n\ge 1$, $d_0=a$, $d_n=b$, and for every $k$, $0\le k\le n-1$, we have that
    $(d_k,d_{k+1})\in s^\Jmc$.  Assume to the contrary that $\exists
    s.\{b\}\notin\iota(a)$.  Thus, $\lnot(\exists s.\{b\})\in\iota(a)$.
    \begin{itemize}
        \item If $n=1$, then we have $(a,b)\in s^\Jmc$.  Thus, we have by
            Claim~\ref{claim:qm-kappa} that $\lnot\{b\}\in\iota(b)$, which is a
            contradiction.
        \item If $n>1$, we have by Claim~\ref{claim:qm-kappa} that
            $\lnot(\exists s.\{b\})\in\kappa(d_1)$.  Using
            Claim~\ref{claim:qm-kappa} again, we obtain that $\lnot(\exists
            s.\{b\})\in\kappa(d_{n-1})$.  By Claim~\ref{claim:qm-kappa}, we have
            $\lnot\{b\}\in\kappa(d_n)=\iota(b)$, which is again a contradiction.
    \end{itemize}
    Hence, $\exists s.\{b\}\in\iota(a)$.  Since $\iota(a)$ is a concept type,
    Definition~\ref{def:concept-type} yields that $\exists r.\{b\}\in\iota(a)$.
    Thus, by Condition~\ref{enum:qm-role}, we obtain $r(a,b)\in\fbb$.

    For the third base case, assume that $\psi$ is of the form $\top\sqsubseteq
    C$.  If $\top\sqsubseteq C\in\fbb$, then for every $\cbb\in\Wmc$, we have
    $C\in\cbb$ by Condition~\ref{enum:qm-gci}.  Since $\iota$ maps into~\Wmc, we
    have that $C\in\kappa(d)$ for every $d\in\Delta^\Imc$.  Hence,
    Claim~\ref{claim:qm-concepts} yields $C^\Imc=\Delta^\Imc$.
    %
    For the converse direction, if $\top\sqsubseteq C\notin\fbb$, then by the
    definition of a formula type, $\lnot(\top\sqsubseteq C)\in\fbb$.  Then, by
    Condition~\ref{enum:qm-gci-neg}, there is a $\cbb\in\Wmc$ such that
    $C\notin\cbb$, which implies $\lnot C\in\cbb$, because \cbb is a concept
    type.  Hence, there is a $d\in\Delta^\Imc$ such that $\lnot C\in\kappa(d)$.
    Claim~\ref{claim:qm-concepts} yields that $d\in(\lnot C)^\Imc$.  Thus, we
    have that $C^\Imc\ne\Delta^\Imc$.

    For the induction step, assume first that $\psi$ is of the
    form~$\lnot\psi_1$.  By induction, we have $\psi\in\fbb$ iff
    $\psi_1\notin\fbb$ iff $\Imc\not\models\psi_1$ iff $\Imc\models\lnot\psi_1$.
    %
    Similarly, if $\psi$ is of the form $\psi_1\land\psi_2$, then $\psi\in\fbb$
    iff $\{\psi_1,\psi_2\}\subseteq\fbb$ iff $\Imc\models\psi_1$ and
    $\Imc\models\psi_2$ iff $\Imc\models\psi_1\land\psi_2$.
    %
    This finishes the proof of Claim~\ref{claim:qm-formula-type}.

    Since \fbb is a formula type for~$\Psi$, we have $\Psi\in\fbb$, and thus
    together with Claim~\ref{claim:qm-formula-type} that $\Imc\models\Psi$.  We
    now show that \Imc is also a model of~\Rmc.

    \begin{claim}\label{claim:qm-rbox}
        For every $\alpha\in\Rmc$, we have $\Imc\models\alpha$.
    \end{claim}

    \noindent
    Assume first that $\alpha$ is of the form $r\sqsubseteq s$.  Since
    $r\sqsubseteq s\in\Rmc$, we have also $\Rmc\models r\sqsubseteq s$.  We
    first show that $r^\Jmc\subseteq s^\Jmc$.  For this, take $(d,e)\in r^\Jmc$.
    There are two cases to consider:
    \begin{itemize}
        \item If $e=b\in\Ind(\Psi)$, then the definition of~$r^\Jmc$ yields that
            $\exists r.\{b\}\in\kappa(d)$.  Since $\Rmc\models r\sqsubseteq s$,
            we have $\exists s.\{b\}\in\kappa(d)$ since $\kappa(d)$ is a concept
            type (see Definition~\ref{def:concept-type}).  The definition of
            $s^\Jmc$ yields that $(d,b)\in s^\Jmc$.
        \item If $e=(\dbb,j,\sbb)\in\Anon$, then the definition of~$r^\Jmc$
            yields that $r\in\sbb$, $\rcomp{\kappa(d)}{\dbb}{\sbb}{\Rmc}$, and
            $\nu_{\kappa(d)}(x_{\kappa(d),\sbb,\dbb})\ge j$.  Then, $\Rmc\models
            r\sqsubseteq s$ yields $s\in\sbb$ since \sbb is a role type (see
            Definition~\ref{def:role-type}).  Hence, the definition of $s^\Jmc$
            yields that $(d,(\dbb,j,\sbb))\in s^\Jmc$.
    \end{itemize}
    %
    To show that $r^\Imc\subseteq s^\Imc$, take $(d,e)\in r^\Imc$.  If $(d,e)\in
    r^\Jmc$, we have $(d,e)\in s^\Jmc$, and thus $(d,e)\in s^\Imc$.  Otherwise,
    we have $(d,e)\in(t^\Jmc)^+$ with $\Rmc\models t\sqsubseteq r$ and
    $\Rmc\models\trans(t)$.  Since $\Rmc\models r\sqsubseteq s$, we also have
    $\Rmc\models t\sqsubseteq s$.  The definition of $s^\Imc$ yields that
    $(t^\Jmc)^+\subseteq s^\Imc$, and hence $(d,e)\in s^\Imc$.

    Assume now that $\alpha$ is of the form $\trans(r)$.  Since
    $\trans(r)\in\Rmc$, we have also that $\Rmc\models\trans(r)$.  Obviously,
    also $\Rmc\models r\sqsubseteq r$ holds.  By the same arguments as above, we
    have that for each $t$ with $\Rmc\models t\sqsubseteq r$ that
    $t^\Jmc\subseteq r^\Jmc$, and thus $(t^\Jmc)^+\subseteq(r^\Jmc)^+$ since the
    transitive closure is monotonic.  This yields that $r^\Imc=(r^\Jmc)^+$, and
    thus $\Imc\models\trans(r)$.
    %
    This finishes the proof of Claim~\ref{claim:qm-rbox}.

    Thus, we have shown that \Imc is indeed a model of~\Bmc.  It only remains to
    be shown that \Imc respects~\Dmc.

    By Condition~\ref{enum:qm-dmc1}, we have for every $d\in\Delta^\Imc$ that
    there is a set $Y\in\Ymc$ such that $Y=\kappa(d)\cap\Umc$.
    Claim~\ref{claim:qm-concepts} yields that $d\in(C_{\Umc,Y})^\Imc$.
    %
    By Condition~\ref{enum:qm-dmc2}, we have for every $Y\in\Ymc$ that there is
    a $d\in\Delta^\Imc$ such that $Y=\kappa(d)\cap\Umc$.
    Claim~\ref{claim:qm-concepts} yields again that $d\in(C_{\Umc,Y})^\Imc$.
    %
    This shows that \Imc respects~\Dmc, and finishes the proof of the
    \enquote{if} direction of the lemma.

    For the \enquote{only if} direction, assume that there is a model
    $\Imc=(\Delta^\Imc,\cdot^\Imc)$ of $\Bmc=(\Psi,\Rmc)$ that
    respects~\Dmc.
    %
    We now construct a quasimodel for~\Bmc.  Let $\tau(d):=\{C\in\CCl(\Bmc)\mid
    d\in C^\Imc\}$ for $d\in\Delta^\Imc$.  We define $\Mmc=(\Wmc,\iota,\fbb)$ as
    follows:
    \begin{itemize}
        \item $\Wmc:=\{\tau(d)\mid d\in\Delta^\Imc\}$;
        \item $\iota(a):=\tau(a^\Imc)$ for every $a\in\Ind(\Psi)$; and
        \item $\fbb:=\{\psi\in\FCl(\Psi)\mid\Imc\models\psi\}$.
    \end{itemize}
    %
    We first show that \Wmc is a set of concept types for~\Bmc.  For this, we
    take any $d\in\Delta^\Imc$, and show that $\tau(d)$ is a concept type
    for~\Bmc.  The semantics of~\SHOQcap and the definition of~$\tau$ yield
    immediately that for every $C\sqcap D\in\CCl(\Bmc)$, we have $C\sqcap
    D\in\tau(d)$ iff $\{C,D\}\subseteq\tau(d)$.  Similarly, for every $\lnot
    C\in\CCl(\Bmc)$, we have $\lnot C\in\tau(d)$ iff $C\notin\tau(d)$.  Because
    of the UNA, we have also that for every $\{a\}\in\CCl(\Bmc)$,
    $\{a\}\in\tau(d)$ implies that $\{b\}\notin\tau(d)$ for every
    $\{b\}\in\CCl(\Bmc)$ with $\{b\}\ne\{a\}$.  The semantics of~\SHOQcap and
    the definition of~$\tau$ yield also that for every $\exists
    r.\{a\}\in\CCl(\Bmc)$, we have that if $\exists r.\{a\}\in\tau(d)$ and
    $\Rmc\models r\sqsubseteq s$, then $\exists s.\{a\}\in\tau(d)$.

    Obviously, \fbb is a formula type for~\Bmc, and
    we have also, by the UNA, that for any $\cbb,\dbb\in\Wmc$ with $\cbb\ne\dbb$
    that $\cbb\cap\dbb\cap\bigl\{\{a\}\mid a\in\Ind(\Psi)\bigr\}=\emptyset$.  By
    definition, $\{a\}\in\tau(a^\Imc)=\iota(a)$ for every $a\in\Ind(\Psi)$.
    Hence, \Mmc is a model candidate for~\Bmc.  We continue showing the
    following claim.

    \begin{claim}\label{claim:qm-rcomp}
        For every $d,e\in\Delta^\Imc$ and every $r\in\Rol(\Bmc)$, we have that
        $(d,e)\in r^\Imc$ implies $\rcomp{\tau(d)}{\tau(e)}{r}{\Rmc}$.
    \end{claim}

    \noindent
    To prove the claim, take any $(d,e)\in r^\Imc$.  For the first condition of
    $r$-compatibility (see Definition~\ref{def:concept-type}), take any
    $\lnot(\exists r.D)\in\tau(d)$, which implies that $d\in(\lnot\exists
    r.D)^\Imc$.  By the semantics of~\SHOQcap, we have $e\in(\lnot D)^\Imc$, and
    thus $\lnot D\in\tau(e)$.

    For the second condition of $r$-compatibility, take any $s\in\Rol(\Bmc)$
    with $\Rmc\models r\sqsubseteq s$, $\Rmc\models\trans(r)$, and
    $\lnot(\exists s.D)\in\tau(d)$.  Since \Imc is a model of~\Rmc, we have that
    $r^\Imc\subseteq s^\Imc$ and that $r^\Imc$ is transitive.  Suppose that
    $\lnot(\exists r.D)\notin\tau(e)$, and thus $\exists r.D\in\tau(e)$.  Then
    there is an $e'\in\Delta^\Imc$ with $e'\in D^\Imc$ and $(e,e')\in r^\Imc$.
    Since $r^\Imc$ is transitive, we have also $(d,e')\in r^\Imc$, and thus
    $(d,e')\in s^\Imc$, which yields a contradiction to $\lnot(\exists
    s.D)\in\tau(d)$.
    %
    This finishes the proof of Claim~\ref{claim:qm-rcomp}.

    We now use this claim to show that \Mmc is a quasimodel for~\Bmc that
    respects~\Dmc.
    %
    Condition~\ref{enum:qm-nonempty} is obviously satisfied since
    $\Delta^\Imc\ne\emptyset$ by definition.

    For Condition~\ref{enum:qm-concept}, we have for every $A(a)\in\FCl(\Psi)$
    that $A(a)\in\fbb$ iff $\Imc\models A(a)$ iff $a^\Imc\in A^\Imc$ iff
    $A\in\tau(a^\Imc)=\iota(a)$.

    For Condition~\ref{enum:qm-role}, we have for every $r(a,b)\in\FCl(\Psi)$
    that $r(a,b)\in\fbb$ iff $\Imc\models r(a,b)$ iff $(a^\Imc,b^\Imc)\in
    r^\Imc$ iff $a^\Imc\in(\exists r.\{b\})^\Imc$ iff $\exists
    r.\{b\}\in\tau(a^\Imc)=\iota(a)$.

    For Condition~\ref{enum:qm-gci}, take any $\top\sqsubseteq C\in\fbb$ and any
    $\cbb\in\Wmc$.  The definition of \fbb yields $\Imc\models\top\sqsubseteq
    C$, and thus $C^\Imc=\Delta^\Imc$.  Hence, $C\in\tau(d)$ for any
    $d\in\Delta^\Imc$, which yields by the definition of \Wmc that $C\in\cbb$.

    For Condition~\ref{enum:qm-gci-neg}, take any $\lnot(\top\sqsubseteq
    C)\in\fbf$.  By the definition of \fbb, this implies
    $\Imc\not\models\top\sqsubseteq C$.  Thus, there is a $d\in\Delta^\Imc$ with
    $d\notin C^\Imc$.  Thus, we have $C\notin\tau(d)\in\Wmc$.

    For Condition~\ref{enum:qm-rcomp}, take any $d\in\Delta^\Imc$ with $\exists
    r.\{a\}\in\tau(d)$.  Then, $d\in(\exists r.\{a\})^\Imc$, and thus
    $(d,a^\Imc)\in r^\Imc$.  Claim~\ref{claim:qm-rcomp} yields that
    $\rcomp{\tau(d)}{\tau(a^\Imc)}{r}{\Rmc}$,
    i.e.~$\rcomp{\tau(d)}{\iota(a)}{r}{\Rmc}$.

    For Condition~\ref{enum:qm-equations}, take any $d\in\Delta^\Imc$.  We
    construct a solution $\nu_{\tau(d)}$ of the system of equations
    $E_{\Mmc,\tau(d)}$.
    %
    Let $z$ denote the maximal integer that occurs in any number restriction
    in~\Bmc.
    %
    We denote by $\Deltau^\Imc$ the set $\{d\in\Delta^\Imc\mid d\ne a^\Imc\text{
    for every $a\in\NI$}\}$ of \emph{unnamed} domain elements, and by
    $\Deltan^\Imc$ the set $\Delta^\Imc\setminus\Deltau^\Imc$ of \emph{named}
    domain elements.
    %
    We first consider the variables $x_{\tau(d),\rbb,\dbb}$.  Take any
    $\rbb\in\Rmf(\Bmc)$ and any $\dbb\in\Wmcu$ such that
    $\rcomp{\tau(d)}{\dbb}{\rbb}{\Rmc}$.  Then we define
    \[\nu_{\tau(d)}(x_{\tau(d),\rbb,\dbb}):=\min\bigl\{z,\ \lvert\{e\in\Deltau^\Imc\mid%
        \text{$\tau(e)=\dbb$, and $(d,e)\in s^\Imc$ iff $s\in\rbb$}\}\rvert\bigr\}.\]
    %
    We set $\nu_{\tau(d)}(x_{\tau(d),\rbb,\dbb})$ to at most $z$ to ensure that
    this value is finite.  Note that this value counts the \emph{unnamed} role
    successors of type~$\rbb$ of concept type~$\dbb$.

    The following claim implies that the equations of the
    form~\eqref{eq:atleast} and~\eqref{eq:atmost} are satisfiable by
    appropriately defining $\nu_{\tau(d)}(y_{\tau(d),\atLeast{n}{r}{C}})$ and
    $\nu_{\tau(d)}(y_{\tau(d),\lnot(\atLeast{n}{r}{C})})$.

    \begin{claim}\label{claim:qm-eq1}
        For every $\atLeast{n}{r}{C}\in\CCl(\Bmc)$, we have
        \[\atLeast{n}{r}{C}\in\tau(d)\text{ iff }
            \sum_{r\in\rbb\in\Rmf(\Bmc)}
            (\nu_{\tau(d)}(\Xi_{\Mmc,\tau(d),\rbb,C})+\Gamma_{\Mmc,\tau(d),\rbb,C})\ge n.\]
    \end{claim}

    \noindent
    Assume first that there are $\dbb\in\Wmcu$ and $\rbb\in\Rmf(\Bmc)$ such that
    $C\in\dbb$, $r\in\rbb$, $\rcomp{\tau(d)}{\dbb}{\rbb}{\Rmc}$, and
    $\nu_{\tau(d)}(x_{\tau(d),\rbb,\dbb})=z\ge n$.  Then by definition of
    $\nu_{\tau(d)}$, there are at least $n$ unnamed domain elements
    $e\in\Deltau^\Imc$ with $C\in\dbb=\tau(e)$ and $(d,e)\in r^\Imc$, which
    implies that $d\in(\atLeast{n}{r}{C})^\Imc$, and thus
    $\atLeast{n}{r}{C}\in\tau(d)$.  Additionally,
    $\nu_{\tau(d)}(\Xi_{\Mmc,\tau(d),\rbb,C})\ge z\ge n$, which shows that
    Claim~\ref{claim:qm-eq1} holds.

    We assume in the following that for every $\dbb\in\Wmcu$ and
    $\rbb\in\Rmf(\Bmc)$ with $C\in\dbb$, $r\in\rbb$, and
    $\rcomp{\tau(d)}{\dbb}{\rbb}{\Rmc}$, we have
    $\nu_{\tau(d)}(x_{\tau(d),\rbb,\dbb})=\lvert\{e\in\Deltau^\Imc\mid\text{$\tau(e)=\dbb$,
    and $(d,e)\in s^\Imc$ iff $s\in\rbb$}\}\rvert\le z$.
    %
    It now follows that, for each $\rbb\in\Rmf(\Bmc)$, we have
    \begin{align*}
        \nu_{\tau(d)}(\Xi_{\Mmc,\tau(d),\rbb,C})
        &=\sum_{C\in\dbb\in\Wmcu,\ \rcomp{\tau(d)}{\dbb}{\rbb}{\Rmc}}
            \nu_{\tau(d)}(x_{\tau(d),\rbb,\dbb})\\[1ex]
        &=\sum_{C\in\dbb\in\Wmcu,\ \rcomp{\tau(d)}{\dbb}{\rbb}{\Rmc}}
            \lvert\{e\in\Deltau^\Imc\mid\text{$\tau(e)=\dbb$, and $(d,e)\in s^\Imc$
            iff $s\in\rbb$}\}\rvert\\[1ex]
        &=\lvert\{e\in C^\Imc\cap\Deltau^\Imc\mid\text{$(d,e)\in s^\Imc$ iff
            $s\in\rbb$}\}\rvert,
    \end{align*}
    where the third equality follows by Claim~\ref{claim:qm-rcomp}.
    %
    Thus,
    \[\sum_{r\in\rbb\in\Rmf(\Bmc)}\nu_{\tau(d)}(\Xi_{\Mmc,\tau(d),\rbb,C})
        =\lvert\{e\in C^\Imc\cap\Deltau^\Imc\mid(d,e)\in r^\Imc\}\rvert.\]

    Moreover, we have
    \begin{align*}
        \sum_{r\in\rbb\in\Rmf(\Bmc)}\Gamma_{\Mmc,\tau(d),\rbb,C}
        &=\sum_{r\in\rbb\in\Rmf(\Bmc)}
            \bigl\lvert\bigl\{b\in\Ind(\Psi)\mid\text{$C\in\iota(b)$, and $\exists
            s.\{b\}\in\tau(d)$ iff $s\in\rbb$}\bigr\}\bigr\rvert\\[1ex]
        &=\bigl\lvert\bigl\{b\in\Ind(\Psi)\mid\text{$C\in\iota(b)$ and $\exists
            r.\{b\}\in\tau(d)$}\bigr\}\bigr\rvert\\[1ex]
        &=\lvert\{b\in\Ind(\Psi)\mid\text{$b^\Imc\in C^\Imc$ and $(d,b^\Imc)\in
            r^\Imc$}\}\rvert\\[1ex]
        &=\lvert\{e\in C^\Imc\cap\Deltan^\Imc\mid(d,e)\in r^\Imc\}\rvert.
    \end{align*}

    Since $\{\Deltau^\Imc,\Deltan^\Imc\}$ is a partition of $\Delta^\Imc$, we
    have that $\atLeast{n}{r}{C}\in\tau(d)$ \emph{iff} $d\in(\atLeast{n}{r}{C})^\Imc$
    \emph{iff} $\lvert\{e\in C^\Imc\mid(d,e)\in r^\Imc\}\rvert\ge n$ \emph{iff}
    \[\sum_{r\in\rbb\in\Rmf(\Bmc)}(\nu_{\cbb}(\Xi_{\Mmc,\cbb,\rbb,C})+\Gamma_{\Mmc,\cbb,\rbb,C})\ge n\]
    by the above equations, which finishes the proof of
    Claim~\ref{claim:qm-eq1}.

    Consider now any $E=\exists(r_1\sqcap\dots\sqcap r_\ell).C\in\CCl(\Bmc)$.
    As above, if there are $\dbb\in\Wmcu$ and $\rbb\in\Rmf(\Bmc)$ such that
    $C\in\dbb$, $\{r_1,\dots,r_\ell\}\subseteq\rbb$,
    $\rcomp{\tau(d)}{\dbb}{\rbb}{\Rmc}$, and
    $\nu_{\tau(d)}(x_{\tau(d),\rbb,\dbb})=z\ge 1$, then there is at least one
    unnamed domain element $e\in\Deltau^\Imc$ with $C\in\dbb=\tau(e)$ and
    $(d,e)\in r_1^\Imc\cap r_\ell^\Imc$.  This implies that $d\in E^\Imc$, and
    thus $E\in\tau(d)$.  Also, $\nu_{\tau(d)}(\Xi_{\Mmc,\tau(d),\rbb,C})\ge z\ge
    1$, which shows that the corresponding equation of the
    form~\eqref{eq:exists} is satisfied if $\nu_{\tau(d)}(y_{\tau(d),E})$ is set
    appropriately.

    We consider now the case where for every $\dbb\in\Wmcu$ and every
    $\rbb\in\Rmf(\Bmc)$ with $C\in\dbb$, $\{r_1,\dots,r_\ell\}\subseteq\rbb$,
    and $\rcomp{\tau(d)}{\dbb}{\rbb}{\Rmc}$, we have
    $\nu_{\tau(d)}(x_{\tau(d),\rbb,\dbb})\le z$.
    %
    Then, by similar arguments as above, we have:
    \[\sum_{\{r_1,\dots,r_\ell\}\subseteq\rbb\in\Rmf(\Bmc)}\nu_{\tau(d)}(\Xi_{\Mmc,\tau(d),\rbb,C})
        =\lvert\{e\in C^\Imc\cap\Deltau^\Imc\mid(d,e)\in r_1^\Imc\cap\dots\cap r_\ell^\Imc\}\rvert,\]
    and also
    \[\sum_{\{r_1,\dots,r_\ell\}\subseteq\rbb\in\Rmf(\Bmc)}\Gamma_{\Mmc,\tau(d),\rbb,C}
        =\lvert\{e\in C^\Imc\cap\Deltan^\Imc\mid(d,e)\in r_1^\Imc\cap\dots\cap r_\ell^\Imc\}\rvert.\]

    Again, this yields that we have $E\in\tau(d)$ \emph{iff} $d\in E^\Imc$
    \emph{iff} there is at least one $e\in C^\Imc$ with
    $(d,e)\in r_1^\Imc\cap\dots\cap r_\ell^\Imc$ \emph{iff}
    \[\sum_{\{r_1,\dots,r_\ell\}\subseteq\rbb\in\Rmf(\Bmc)}
        (\nu_{\tau(d)}(\Xi_{\Mmc,\tau(d),\rbb,C})+\Gamma_{\Mmc,\tau(d),\rbb,C})\ge 1,\]
    which shows that the equations of the form~\eqref{eq:exists}
    and~\eqref{eq:forall} can be satisfied by appropriately setting
    $\nu_{\tau(d)}(y_{\tau(d),E})$.
    %
    This finishes the proof that \Mmc satisfies
    Condition~\ref{enum:qm-equations}.

    For Condition~\ref{enum:qm-dmc1}, take any $d\in\Delta^\Imc$.  Since \Imc
    respects~\Dmc, there must be a set $Y\in\Ymc$ such that
    $d\in(C_{\Umc,Y})^\Imc$.  Hence, by definition of $\tau(d)$, we have
    $Y=\tau(d)\cap\Umc$.

    For Condition~\ref{enum:qm-dmc2}, let $Y\in\Ymc$.  Since \Imc respects~\Dmc,
    there must be a $d\in(C_{\Umc,Y})^\Imc$.  Thus, by definition of $\tau(d)$,
    we have $Y=\tau(d)\cap\Umc$ with $\tau(d)\in\Wmc$.
\end{proof}

\noindent
It remains to be shown that one can check the existence of a quasimodel for~\Bmc
that respects~\Dmc in time exponential in the size of~\Bmc.  For this, consider
the following algorithm.
%
Given $\Bmc=(\Psi,\Rmc)$ and~\Dmc, it enumerates all model
candidates $(\Wmc,\iota,\fbb)$ for~\Bmc, where \Wmc is the set of \emph{all}
concept types for~\Bmc.  We denote these candidates by $\Mmc_1,\dots,\Mmc_N$.
Note that each of them is of size exponential in the size of~\Bmc.  It should be
clear that
\[N\le 2^{\lvert\CCl(\Bmc)\rvert\cdot\lvert\Ind(\Psi)\rvert}\cdot
    2^{\lvert\FCl(\Psi)\rvert},\]
and thus the enumeration of $\Mmc_1,\dots,\Mmc_N$ can be done in exponential
time since $\CCl(\Bmc)$ and $\FCl(\Psi)$ are of size polynomial in the size
of~\Bmc.

The algorithm works as follows.  First, initialise $i:=1$ and consider
$\Mmc_i=(\Wmc,\iota,\fbb)$.
\begin{enumerate}[label=\emph{Step~\arabic*.},ref=Step~\arabic*,leftmargin=4em]
    \item\label{step:assertions}
        Check whether $\Mmc_i$ satisfies Conditions~\ref{enum:qm-concept}
        and~\ref{enum:qm-role}.

        If it does, continue with \ref{step:defective}.  Otherwise, stop
        considering $\Mmc_i$ and go to \ref{step:iteration}.

    \item\label{step:defective}
        Check each concept type $\cbb\in\Wmc$.  We call a concept
        type~$\cbb\in\Wmc$ \emph{defective} if it violates
        Condition~\ref{enum:qm-gci} for some $\top\sqsubseteq C\in\fbb$, it
        violates Condition~\ref{enum:qm-rcomp}, or it violates
        Condition~\ref{enum:qm-dmc1}.

        If we find a defective concept type~\cbb, and have $\cbb\in\Wmcu$, then
        we set $\Wmc:=\Wmc\setminus\{\cbb\}$ and continue with
        \ref{step:defective}.
        %
        If we find a defective $\cbb\notin\Wmcu$,
        i.e.~$\cbb\in\iota(\Ind(\Psi))$, then stop considering $\Mmc_i$ and go
        to \ref{step:iteration}.
        %
        If we have found no defective concept types in~\Wmc, continue with
        \ref{step:equations}.

    \item\label{step:equations}
        Consider the model candidate $\Mmc'=(\Wmc',\iota,\fbb)$ obtained from
        the previous step.  For every $\cbb\in\Wmc'$, check whether
        $E_{\Mmc',\cbb}$ has a solution.

        If we find a $\cbb\in\Wmcu'$ such that $E_{\Mmc',\cbb}$ has no solution,
        then set $\Wmc':=\Wmc'\setminus\{\cbb\}$ and redo \ref{step:equations}.
        %
        If we find a $\cbb\in\iota(\Ind(\Psi))$ such that $E_{\Mmc',\cbb}$ has
        no solution, then go to \ref{step:iteration}.
        %
        If we have found no such concept type in~$\Wmc'$, continue with
        \ref{step:final}.

    \item\label{step:final}
        Check whether the model candidate $\Mmc''=(\Wmc'',\iota,\fbb)$ obtained
        from \ref{step:equations} satisfies Conditions~\ref{enum:qm-nonempty},
        \ref{enum:qm-gci-neg}, and~\ref{enum:qm-dmc2}.

        If it does, stop with output \enquote{quasimodel that respects~\Dmc found}.
        %
        Otherwise, continue with \ref{step:iteration}.

    \item\label{step:iteration}
        Set $i:=i+1$. If $i\leq N$, continue with \ref{step:assertions}.
        Otherwise, stop with output \enquote{no quasimodel that respects~\Dmc
        exists}.
\end{enumerate}
%
Using this algorithm, we are now ready to prove one of the main results of this
section.

\begin{theorem}\label{thm:cons-bmc-dmc}
    Let \Bmc be a Boolean \SHOQcap-knowledge base, and let
    $\Dmc=(\Umc,\Ymc)$ be a pair such that \Umc is a set of concept
    names occurring in~\Bmc and $\Ymc\subseteq 2^\Umc$.  Then, consistency
    of~\Bmc w.r.t.~\Dmc can be decided in time exponential in the size of~\Bmc.
\end{theorem}

\begin{proof}
    By Lemma~\ref{lem:quasimodels}, it suffices to show that the algorithm
    described above to find quasimodels for~\Bmc that respect~\Dmc is sound,
    complete, and terminates in time exponential in the size of~\Bmc.

    If the algorithm has constructed a model candidate $\Mmc=(\Wmc,\iota,\fbb)$
    that passed all tests, then \Mmc obviously satisfies
    Conditions~\ref{enum:qm-nonempty}--\ref{enum:qm-dmc2} of
    Definition~\ref{def:quasimodel}.

    Conversely, if $\Mmc=(\Wmc,\iota,\fbb)$ is a quasimodel for~\Bmc that
    respects~\Dmc, then $\iota$ and \fbb must be enumerated by the algorithm at
    some point.  Since $\iota$ and~\fbb satisfy Conditions~\ref{enum:qm-concept}
    and~\ref{enum:qm-role}, they pass the tests in \ref{step:assertions}, and we
    continue with \ref{step:defective}.  There, a model candidate
    $\Mmc'=(\Wmc',\iota,\fbb)$ is constructed.  Note that the concept types
    in~\Wmc cannot be defective because of Conditions~\ref{enum:qm-gci},
    \ref{enum:qm-rcomp}, and~\ref{enum:qm-dmc1}.  Since $\iota$ maps to concept
    types in~\Wmc, we indeed obtain $\Mmc'$ with $\Wmc\subseteq\Wmc'$, and
    continue with \ref{step:equations}.  There, a model candidate
    $\Mmc''=(\Wmc'',\iota,\fbb)$ is constructed.  Since for every $\cbb\in\Wmc$,
    $E_{\Mmc,\cbb}$ has a solution, also $E_{\Mmc'',\cbb}$ has a solution.
    Indeed, the additional variables that occur in $E_{\Mmc'',\cbb}$ but not in
    $E_{\Mmc,\cbb}$ can be set to $0$.  Since $\iota$ maps to the concept types
    in~\Wmc, we indeed obtain a model candidate $\Mmc''$ with
    $\Wmc\subseteq\Wmc''$ that satisfies Condition~\ref{enum:qm-equations}, and
    continue with \ref{step:final}.  Finally, $\Wmc''$ satisfies the
    Conditions~\ref{enum:qm-nonempty}, \ref{enum:qm-gci-neg},
    and~\ref{enum:qm-dmc2}, because $\Wmc\subseteq\Wmc''$.  This shows that the
    algorithm detects the existence of a quasimodel for~\Bmc that respects~\Dmc.

    To analyse the running time of the algorithm, observe first that
    $r$-compatibility w.r.t.~\Rmc can be checked in polynomial time since this
    only involves inclusion tests for two concept types, which are sets of
    polynomial size, and entailment tests of role axioms w.r.t.~\Rmc, which can
    be done in time polynomial in the size of~\Rmc~\cite{HoST-IGPL00}.

    As mentioned above, the number~$N$ of model candidates is at most
    exponential in the size of~\Bmc, while each model candidate $\Mmc_i$ is of
    size exponential in the size of~\Bmc.  Also the sequence of model candidates
    $\Mmc_1,\dots,\Mmc_N$ can be enumerated in time exponential in the size
    of~\Bmc.

    For each of these exponentially many model candidates, the checks in
    \ref{step:assertions} can be done in time polynomial in the size of~\Bmc,
    and the checks in \ref{step:defective} are done at most exponentially often
    since each time one of the exponentially many concept types in~\Wmc is
    removed.  Each of these checks can be done in time exponential in the size
    of~\Bmc since the following conditions are checked for at most exponentially
    many concept types~\cbb:
    \begin{itemize}
        \item for Condition~\ref{enum:qm-gci}, we check for inclusion of
            polynomially many concepts in~\cbb;
        \item for Condition~\ref{enum:qm-rcomp}, we have polynomially many
            $r$-compatibility tests; and
        \item for Condition~\ref{enum:qm-dmc1}, we enumerate all (at most
            exponentially many) elements of~\Ymc and do a simple test.
    \end{itemize}
    %
    By similar arguments as above, the checks in \ref{step:equations} are done
    at most exponentially often since each time one of the exponentially many
    concept types in~$\Wmc'$ is removed.  Each time \ref{step:equations} is
    performed, for exponentially many concept types $\cbb\in\Wmc'$, it must be
    checked whether $E_{\Mmc',\cbb}$ has a solution.  We denote the number of
    variables in $E_{\Mmc',\cbb}$ by $n$, and the number of equations in
    $E_{\Mmc',\cbb}$ by $m$.  Note that $n$ may be exponential in the size
    of~\Bmc since there are exponentially many concept types and role types.
    However, $m$ is polynomial in the size of~\Bmc since we have one equation
    per at-least and existential restriction occurring in~$\Psi$.
    In~\cite{Pap-JACM81}, it was shown that $E_{\Mmc',\cbb}$ can be solved in
    time $O(n^{2m+2}(ma)^{(m+1)(2m+1)})$, where $a$ denotes the value of the
    largest number appearing in the equations.  Thus, even if the numbers in
    the at-least restrictions occurring in~$\Psi$ are given in binary encoding,
    checking whether the system of equations $E_{\Mmc',\cbb}$ has a solution can
    be done in time exponential in the size of~\Bmc.  Overall,
    \ref{step:equations} takes only time exponential in the size of~\Bmc.

    Finally, \ref{step:final} can also be done in time exponential in the size
    of~\Bmc.  Checking Condition~\ref{enum:qm-nonempty} is trivial.  Checking
    Condition~\ref{enum:qm-gci-neg} involves enumerating polynomially many
    $\lnot(\top\sqsubseteq C)\in\fbb$ and at most exponentially many concept
    types in $\Wmc''$ to do an inclusion test.  For
    Condition~\ref{enum:qm-dmc2}, we enumerate at most exponentially many
    elements of~\Ymc and at most exponentially many concept types in $\Wmc''$ to
    do a simple test.

    Overall, the algorithm runs in time exponential in the size of~\Bmc.
\end{proof}

\noindent
The following corollary captures the special case of Boolean \SHOQ-knowledge
bases without a set~\Dmc.

\begin{corollary}\label{cor:cons-boolean-shoq-kb}
    Let \Bmc be a Boolean \SHOQ-knowledge base.  Then, consistency of~\Bmc can
    be decided in time exponential in the size of~\Bmc.
\end{corollary}

\begin{proof}
    Obviously, \Bmc is also a Boolean \SHOQcap-knowledge base as \SHOQ is a
    fragment of \SHOQcap.  We define $\Dmc:=(\Umc,\Ymc)$ with $\Umc:=\emptyset$
    and $\Ymc:=\{\emptyset\}$.  It is easy to see that \Bmc is consistent iff
    \Bmc is consistent w.r.t.~\Dmc.  Indeed, the \enquote{if} direction is
    trivial.  For the \enquote{only if} direction, assume that \Bmc is
    consistent.  Then, there is a model $\Imc=(\Delta^\Imc,\cdot^\Imc)$ of~\Bmc.
    Note that $C_{\emptyset,\emptyset}$ is equivalent to $\top$, and thus
    $(C_{\emptyset,\emptyset})^\Imc=\Delta^\Imc\ne\emptyset$.  Hence, we have
    \[\Ymc=\{\emptyset\}=\bigl\{Y\subseteq\Umc=\emptyset\mid%
        \text{there is a $d\in\Delta^\Imc$ with $d\in(C_{\Umc,Y})^\Imc$}\bigr\},\]
    which shows that \Imc respects~\Dmc.
    %
    Thus, Theorem~\ref{thm:cons-bmc-dmc} yields that consistency of~\Bmc can be
    decided in time exponential in the size of~\Bmc.
\end{proof}

\noindent
Theorem~\ref{thm:cons-bmc-dmc} and Corollary~\ref{cor:cons-boolean-shoq-kb}
yield the results that are needed in the proofs of
Theorems~\ref{thm:upper-bound-shoq-ltl-no-rigid},
\ref{thm:upper-bound-shoq-ltl-rigid-roles},
and~\ref{thm:upper-bound-shoq-ltl-rigid-concepts}.


\section{Summary}\label{sec:shoq-ltl-summary}

In this chapter, we obtained complexity results for the satisfiability problem
in the temporalised description logic \SHOQ-LTL as shown in
Table~\ref{tab:shoq-ltl-results}.  More precisely, we considered the
satisfiability problem in the settings where (i)~neither concept names nor role
names are allowed to be rigid, (ii)~only concept names may be rigid, and
(iii)~both concept names and role names may be rigid.  It turned out that in all
three settings, the satisfiability problem in \SHOQ-LTL has the same complexity
as the satisfiability problem in the less expressive temporalised description
logic \ALC-LTL\@.  Hence, for every description logic~\Lmc between \ALC and
\SHOQ, we have that the satisfiability problem in \Lmc-LTL is \ExpTime-complete
in Setting~(i), which is the same complexity as the satisfiability problem
in~\Lmc.  Moreover, the satisfiability problem in \Lmc-LTL is \NExpTime-complete
if we allow rigid concept names (but no rigid role names), i.e.~in Setting~(ii),
and \TwoExpTime-complete in Setting~(iii), where we further allow rigid role
names.
