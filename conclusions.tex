\chapter{Conclusions}\label{ch:conclusions}

In this chapter, we first provide a brief summary in
Section~\ref{sec:main-results} about what was achieved in the present
thesis.  Then, in Section~\ref{sec:future-work}, we mention some future work.


\section{Main Results}\label{sec:main-results}

In this thesis, we have shown how to verify properties of dynamical systems
whose behaviour can be partially observed.  Whereas we assumed that we do not
have a complete description of neither the system itself nor the current state
of the system, we assumed that we have access to some background knowledge that
encodes basic knowledge about the functioning of the observed system.  Moreover,
since the systems' states may have a complex internal structure, the expressive
power of the formalism for representing the observations should go beyond
propositional logic.
%
We used description logics and extensions of them as one way to address these
requirements.

After introducing the temporalised description logic \SHOQ-LTL in
Chapter~\ref{ch:shoq-ltl}, which extends propositional LTL by allowing
\SHOQ-axioms to occur in place of propositional variables, we have shown
complexity results for the satisfiability problem in that temporalised DL\@.
We have considered the problem in three different settings:
\begin{enumerate}[label=(\roman*)]
    \item\label{enum:setting-no-rigid}
        neither concept names nor role names are allowed to be rigid,
    \item\label{enum:setting-rigid-concepts}
        only concept names are allowed to be rigid, and
    \item\label{enum:setting-rigid-roles}
        both concept and role names are allowed to be rigid.
\end{enumerate}
We have shown that the complexity is the same as in the less expressive
temporalised description logic \ALC-LTL~\cite{BaGL-ToCL12}, namely
\ExpTime-complete in Setting~\ref{enum:setting-no-rigid}, \NExpTime-complete in
Setting~\ref{enum:setting-rigid-concepts}, and \TwoExpTime-complete in
Setting~\ref{enum:setting-rigid-roles}.  Table~\ref{tab:shoq-ltl-results}
mentions the respective theorems that state those results.  Moreover, we have
shown in this chapter that the consistency problem for (an extension of) Boolean
\SHOQ-knowledge bases (w.r.t.\ some side condition) can be decided in
exponential time.

Using these results, we considered in the following three different contexts.
Firstly, in Chapter~\ref{ch:monitor}, we have shown how to perform runtime
verification using \SHOQ-LTL\@.  In this chapter, we provided a construction for
monitors that runs in doubly exponential time, and produces monitors of doubly
exponential size, even in the most complex case where both rigid concept names
and rigid role names are allowed, i.e.~in
Setting~\ref{enum:setting-rigid-roles}.  For that we have shown how to construct
Büchi-automata for \SHOQ-LTL-formulas using results from
Chapter~\ref{ch:shoq-ltl}.  Moreover, we have shown that this doubly exponential
blow-up in the construction of the monitor cannot be avoided as it already
occurs for propositional LTL\@.  Finally, we have considered the related
decision problems of liveness and monitorability and have shown some complexity
results for them.  Our results are only tight in
Setting~\ref{enum:setting-rigid-roles}.  In this setting, both problems are
\TwoExpTime-complete.  For the other cases, a gap remains: both problems are
\ExpTime-hard and in \TwoExpTime in Setting~\ref{enum:setting-no-rigid}, and
\coNExpTime-hard and in \TwoExpTime in
Setting~\ref{enum:setting-rigid-concepts}.  However, the exact complexity of
these problems are not even known for propositional LTL\@.

Secondly, in Chapter~\ref{ch:tcqs}, we have considered a temporal version of
ontology-based data access.  More precisely, we proved complexity results for
query entailment in a temporal query language that extends propositional LTL by
allowing conjunctive queries in place of propositional variables.  Moreover,
background knowledge is encoded in a TBox that is formulated in a description
logic between \ALC and \SHQ.  We considered both the data complexity and the
combined complexity for this problem for the three settings above.  In
Setting~\ref{enum:setting-no-rigid}, temporalised query entailment is
\coNP-complete w.r.t.\ data complexity and \ExpTime-complete w.r.t.\ combined
complexity.  In Setting~\ref{enum:setting-rigid-concepts}, the problem is
\coNP-complete w.r.t.\ data complexity and \coNExpTime-complete w.r.t.\ combined
complexity.  Finally, in Setting~\ref{enum:setting-rigid-roles}, the problem is
\coNP-hard and in \ExpTime w.r.t.\ data complexity and \TwoExpTime-complete
w.r.t.\ combined complexity.  For showing these results, some results of
Chapter~\ref{ch:shoq-ltl} are used.  The obtained results are summarised in
Table~\ref{tab:tcq-results}, where also the respective theorems and corollaries
that state these results are listed.

Thirdly, in Chapter~\ref{ch:ramifications}, we have considered an action
formalism based on any description logic between \ALC and \ALCQIO that is
capable of treating ramifications that arise naturally if domain constraints are
encoded in general TBoxes.  For this, we have extended the DL-based action
formalism introduced in~\cite{BLM+-AAAI05} (which could deal only with acyclic
TBoxes) with causal relationships.  We have shown that important inference
problems such as the consistency problem and the projection problem are
decidable in our new formalism, and continue the work of~\cite{BaLM-ECAI10} by
generalising the verification problem.  We have derived a number of complexity
results from the obtained decision procedures.  Depending on the base DL, the
complexity results range from \PSpace-complete to \coNExpTime-hard and in
\PTimeToNExpTime for the consistency problem, and from \PSpace-complete to
\coNExpTime-complete for the projection problem.  For the verification problem,
the complexity ranges from in \ExpSpace to in \coTwoNExpTime, and it is unknown
whether these bounds are tight.  The obtained results are summarised in
Table~\ref{tab:ramification-results}.


\section{Future Work}\label{sec:future-work}

Some more technical directions for future work have been already mentioned at
the end of each chapter.  These include tightening some complexity bounds and
considering slight extensions of the approaches introduced.

We now give some more general remarks on future research.  Throughout the
thesis, we have considered extensions of propositional LTL for specifying
temporal properties.  It would be interesting to consider also different
temporal formalisms.  Especially in the area of DL-based action formalisms, it
is worthwile to examine the verification problem for extensions of the temporal
logic CTL~\cite{ClEm-LP81} and its extension CTL$^*$~\cite{EmHa-JACM86} that
encompasses propositional LTL\@.  Furthermore, it might make sense to consider
\emph{real-time extensions} of temporal logics; see
e.g.~\cite{Koy-RTS90,AlHe-IC93,AlFH-JACM96,RaSc-JALC99}.

With respect to temporal query languages, it is interesting to consider
light-weight description logics such as members of the
\DLLite-family~\cite{CDL+-AAAI05,ACK+-JAIR09,CDL+-RW09} in our context since
they allow first-order rewritability, i.e.~query answering can be reduced to
classical database reasoning.  It is interesting to see whether the approaches
considered in the present thesis work also for \DLLite.  First steps in that
direction have been done in~\cite{BoLT-FroCoS13,BoLT-DL13,BoLT-LTCS-13-05}.

Additionally, it makes sense to consider extensions of the presented approaches
that are capable of dealing with faulty sensor information.  For instance,
temporal extensions of (decidable) \emph{probabilistic description
logics}~\cite{Luk-AIJ08,LuSc-KR10} and (decidable) \emph{fuzzy description
logics}~\cite{Str-JAIR01,BoSt-FSS09,BDG+-IJUF12,BoPe-JoDS13,BoPe-IJAR14} may be
useful.
%
Moreover, one may want to deal with concrete numerical values, and thus it is
interesting to consider also temporal extensions of description logics with
\emph{concrete
domains}~\cite{BaHa-IJCAI91,Lut-AiML02,BaSa-IS03,Lut-ToCL04,LAHS-JAIR05,LuMi-JAR07}.
Concrete domains allow to use concrete values such as numbers or strings within
concepts.  Description logics with concrete domains allow furthermore to use
predicates on such concrete values. Thus, one can, for instance, compare
concrete values.

Moreover, it would be interesting to combine some of the above mentioned
extensions.  It is very challenging, however, to find a useful combination that
remains decidable.
