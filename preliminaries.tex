\chapter{Preliminaries}\label{ch:preliminaries}

In this chapter, we set a basis for the later chapters by introducing the basic
notions that we need.  Firstly, we introduce \emph{description logics (DLs)} as
the logical formalism that we use throughout the thesis.  Secondly, we give the
basic definitions of \emph{propositional linear-time temporal logic} and recall the
relationship with $\omega$-automata, i.e.~automata working on infinite words.
These notions are needed to obtain the \emph{linear-time temporalised
description logic \SHOQ-LTL} (see Chapter~\ref{ch:shoq-ltl}).

More specific notions like the basics of DL-based query answering and action
formalisms based on description logics are not covered in this chapter but
introduced in the respective later chapters.


\section{Basic Notions of Description Logics}\label{sec:dls}

As already sketched in the Chapter~\ref{ch:introduction}, description
logics~\cite{DLhandbook-07} are a successful family of logic-based knowledge
representation formalisms.  In this section, we introduce the basic notions of
DLs that are relevant for this thesis.  For a more thorough introduction to DLs,
the interested reader is referred to the \emph{Description Logic
Handbook}~\cite{DLhandbook-07}.


\subsection{Description Logic Concepts}\label{sec:dl-concepts}

As discussed in Section~\ref{sec:intro-dls}, \emph{concepts} are defined using
concept names, role names, individual names, and concept and role constructors.
Throughout the thesis, let \NC, \NR, and \NI, respectively, denote pairwise
disjoint sets of concept names, role names, and individual names.
%
We introduce now the concept and role constructors that are relevant for this
thesis, and show how they are used to define the syntax of concepts (sometimes
called \emph{concept descriptions}).

\begin{definition}[Syntax of concepts]
    A \emph{role} $r$ is either a role name, i.e.~$r\in\NR$, or it is of the
    form $s^-$ for $s\in\NR$ (inverse role).
    %
    The set of \emph{concepts} is the smallest set such that
    \begin{itemize}
        \item every concept name $A\in\NC$ is a concept; and
        \item if $C,D$ are concepts, $a\in\NI$, $r$ is a role, and $n$ is a
            non-negative integer, then the following are also concepts: $\lnot
            C$ (negation), $C\sqcap D$ (conjunction), $\{a\}$ (nominal),
            $\exists r.C$ (existential restriction), and $\atLeast{n}{r}{C}$
            (at-least restriction).
    \end{itemize}
\end{definition}

\noindent
As usual in description logics, we use
\begin{itemize}
    \item $C\sqcup D$ (disjunction) as an abbreviation for $\lnot(\lnot
        C\sqcap\lnot D)$;
    \item $C\to D$ (implication) as an abbreviation for $\lnot C\sqcup D$;
    \item $\top$ (top) as an abbreviation for $A\sqcup\lnot A$ where $A\in\NC$
        is arbitrary but fixed;
    \item $\bot$ (bottom) as an abbreviation for $\lnot\top$;
    \item $\forall r.C$ (value restriction) as an abbreviation for
        $\lnot(\exists r.\lnot C)$; and
    \item $\atMost{n}{r}{C}$ (at-most restriction) as an abbreviation for
        $\lnot(\atLeast{(n+1)}{r}{C})$.
\end{itemize}

Note that there are more concept and role constructors introduced in the
literature.  These are either beyond the scope of this thesis or introduced
where needed.

The semantics of concepts is given in a model-theoretic way using the notion of
an interpretation.

\begin{definition}[Semantics of concepts]\label{def:semantics-concepts}
    An \emph{interpretation} is a pair $\Imc=(\Delta^\Imc,\cdot^\Imc)$, where the
    \emph{domain} $\Delta^\Imc$ is a non-empty set, and the \emph{interpretation
    function} $\cdot^\Imc$ assigns to every $A\in\NC$ a set
    $A^\Imc\subseteq\Delta^\Imc$, to every $r\in\NR$ a binary relation
    $r^\Imc\subseteq\Delta^\Imc\times\Delta^\Imc$, and to every $a\in\NI$ an
    element $a^\Imc\in\Delta^\Imc$ such that the \emph{unique-name assumption
    (UNA)} holds, i.e.~for all $a,b\in\NI$ with $a\ne b$, we have
    $a^\Imc\ne b^\Imc$.  This function is extended to inverse roles and concepts
    as follows:
    \begin{itemize}
        \item $(s^-)^\Imc:=\{(e,d)\mid(d,e)\in s^\Imc\}$;
        \item $(\lnot C)^\Imc:=\Delta^\Imc\setminus C^\Imc$;
        \item $(C\sqcap D)^\Imc:=C^\Imc\cap D^\Imc$;
        \item $\{a\}^\Imc:=\{a^\Imc\}$;
        \item $(\exists r.C)^\Imc:=\{d\in\Delta^\Imc\mid\text{there exists an
            $e\in\Delta^\Imc$ with $(d,e)\in r^\Imc$ and $e\in C^\Imc$}\}$; and
        \item
            $(\atLeast{n}{r}{C})^\Imc:=\bigl\{d\in\Delta^\Imc\mid\lvert\{e\in\Delta^\Imc\mid(d,e)\in
            r^\Imc\text{ and }e\in C^\Imc\}\rvert\ge n\bigr\}$.
    \end{itemize}
    We call a concept $C$ \emph{satisfiable} if there is a interpretation~\Imc
    such that $C^\Imc\ne\emptyset$.
\end{definition}

\noindent
Note that in our definition of the semantics, we make the unique-name
assumption, which is an assumption often made in DLs.
%
We continue by giving an example of the notions introduced so far.

\begin{example}\label{ex:camel-concept}
    Let $C$ be the following concept:
    \[\{\leah\}\sqcap\lnot\Dromedary\sqcap\exists\likes.\Foliage\sqcap%
        \atMost{1}{\has}{\Hump}\sqcap\exists\isFatherOf^-.\{\hassan\}.\]
    It is not hard to see that $C$ is satisfiable.
    Figure~\ref{fig:camel-concept} depicts the graphical representation of an
    interpretation~\Imc with $C^\Imc=\{\leah^\Imc\}\ne\emptyset$.  Note
    that for this interpretation~\Imc, we have $\Dromedary^\Imc=\emptyset$.
\end{example}

\begin{figure}[t]
    \centering
    \begin{tikzpicture}[%
            ->,
            >=stealth',
            semithick,
            shorten <=2pt,
            shorten >=2pt,
            auto,
            on grid,
            node distance=10ex and 10em,
            element/.style={circle,fill,draw,inner sep=0,minimum size=4pt}]

         \node[element] (leah)    [label=below:\leah]                  {};
         \node[element] (hassan)  [right=of leah,label=below:\hassan]  {};
         \node[element] (hump)    [left=of leah,label=above:\Hump]     {};
         \node[element] (foliage) [above=of leah,label=above:\Foliage] {};

         \path[->]  (leah)   edge node [swap] {\has}        (hump)
                             edge node [swap] {\likes}      (foliage)
                    (hassan) edge node [swap] {\isFatherOf} (leah);
    \end{tikzpicture}
    \caption{An interpretation \Imc such that $C^\Imc\ne\emptyset$ for the
        concept $C$ of Example~\ref{ex:camel-concept}}
    \label{fig:camel-concept}
\end{figure}

\noindent
It is important to note that the semantics of a concept is entirely given by an
interpretation.  The concept, role, and individual names itself do not imply
anything.  For instance, the concept name \Dromedary does not necessarily denote
dromedaries.  It is just a name, and the actual interpretation has to ensure the
expected meaning.


\subsection{Knowledge Bases}

To restrict ourselves to certain kinds of interpretations, we capture the domain
knowledge in a so-called knowledge base (KB).  Each KB consists of three parts:
a \emph{TBox} (terminological box), an \emph{RBox} (role box), and an
\emph{ABox} (assertional box).  Intuitively, the RBox states knowledge about
roles, the TBox states knowledge about all domain elements, whilst the ABox
states knowledge about specific individuals.\footnote{%
    In the literature, sometimes the information from the RBox is included in
    the TBox.  However, we keep them separately here as this turns out to be
    useful later.}

\begin{definition}[Syntax of TBoxes]\label{def:syntax-tbox}
    An \emph{concept definition} is of the form $A\equiv C$ where $A\in\NC$ and
    $C$ is a concept.
    %
    A \emph{general concept inclusion (GCI)} is of the form $C\sqsubseteq D$
    where $C,D$ are concepts.
    %
    We call both concept definitions and GCIs \emph{TBox-axioms}.

    A \emph{(general) TBox} is a finite set of TBox-axioms.
    %
    An \emph{acyclic TBox} \Tmc is a finite set of concept definitions such that
    the following two conditions are satisfied:
    \begin{itemize}
        \item if $A\equiv C,\ A\equiv D\in\Tmc$, then $C=D$ (unambiguity); and
        \item there is no sequence $A_1\equiv C_1,\dots,A_n\equiv C_n\in\Tmc$
            with $n\ge 1$ such that $A_{i+1}$ occurs in $C_i$ (for $1\le i<n$)
            and $A_1$ occurs in $C_n$ (no cyclic definitions).
    \end{itemize}
    We call the concept names that occur on the left-hand side of some concept
    definition in \Tmc \emph{defined concept names} whereas we call the others
    \emph{primitive concept names}.
\end{definition}

\noindent
Intuitively, an acyclic TBox consists of \enquote{macros}, i.e.\ definitions of
shorthands for (complex) concepts.  Therefore, acyclic TBoxes are sometimes
called \emph{unfoldable} TBoxes.
%
The semantics of TBoxes can now be defined in a straightforward manner.

\begin{definition}[Semantics of TBoxes]\label{def:semantics-tbox}
    The interpretation~\Imc is a \emph{model}
    \begin{itemize}
        \item of the concept definition $A\equiv C$ (written $\Imc\models
            A\equiv C$) if $A^\Imc=C^\Imc$; and
        \item of the GCI $C\sqsubseteq D$ (written $\Imc\models C\sqsubseteq D$)
            if $C^\Imc\subseteq D^\Imc$.
    \end{itemize}
    \Imc is a model of the TBox~\Tmc (written $\Imc\models\Tmc$) if it is a
    model of each TBox-axiom in~\Tmc.
    %
    We call a TBox \emph{consistent} if it has a model.
\end{definition}

\noindent
Note that the concept definition $A\equiv C$ can be captured by two GCIs, namely
$A\sqsubseteq C$ and $C\sqsubseteq A$.  For ease of presentation, we thus often
assume in the following that general TBoxes do not contain concept definitions
or that a concept definition is an \enquote{abbreviation} for two GCIs.

\begin{definition}[Syntax of RBoxes]
    A \emph{transitivity axiom} is of the form $\trans(r)$ where $r$ is a
    role, and a \emph{role-inclusion axiom} is of the form
    $r\sqsubseteq s$ where $r,s$ are roles.
    %
    We call both transitivity axioms and role-inclusion axioms
    \emph{RBox-axioms}.
    %
    An \emph{RBox} is a finite set of RBox-axioms.
\end{definition}

\noindent
Again, the semantics is straightforward.

\begin{definition}[Semantics of RBoxes]
    The interpretation \Imc is a \emph{model}
    \begin{itemize}
        \item of the transitivity axiom $\trans(r)$ (written
            $\Imc\models\trans(r)$) if $r^\Imc\circ r^\Imc\subseteq r^\Imc$,
            i.e.\ $r^\Imc$ is transitive; and
        \item of the role-inclusion axiom $r\sqsubseteq s$ (written $\Imc\models
            r\sqsubseteq s$) if $r^\Imc\subseteq s^\Imc$.
    \end{itemize}
    \Imc is a model of the RBox \Rmc (written $\Imc\models\Rmc$) if it is
    a model of each RBox-axiom in~\Rmc.
    %
    We call \Rmc \emph{consistent} if it has a model.
\end{definition}

\noindent
Finally, we define the syntax of ABoxes as follows.

\begin{definition}[Syntax of ABoxes]\label{def:syntax-abox}
    A \emph{concept assertion} is of the form $C(a)$ where $C$ is a concept, and
    $a\in\NI$.  A \emph{role assertion} is of the form $r(a,b)$ where $r\in\NR$,
    and $a,b\in\NI$.  We call both concept assertions and role assertions
    \emph{ABox-axioms}.
    %
    An ABox-axiom is \emph{atomic} if it is either a role assertion or an atomic
    concept assertion, i.e.\ it is of the form $A(a)$ where $A\in\NC$ and
    $a\in\NI$.

    A (complex) \emph{ABox} is a finite set of ABox-axioms.
    %
    A \emph{simple ABox} is a finite set of atomic ABox-axioms.
\end{definition}

\noindent
We call ABox-axioms sometimes \emph{assertions}.  Note that every simple ABox is
also a (complex) ABox.
%
The semantics of ABoxes is defined as follows.

\begin{definition}[Semantics of ABoxes]\label{def:semantics-abox}
    The interpretation \Imc is a \emph{model}
    \begin{itemize}
        \item of the concept assertion $C(a)$ (written $\Imc\models
            C(a)$) if $a^\Imc\in C^\Imc$; and
        \item of the role assertion $r(a,b)$ (written $\Imc\models
            r(a,b)$) if $(a^\Imc,b^\Imc)\in r^\Imc$.
    \end{itemize}
    \Imc is a model of the ABox~\Amc (written $\Imc\models\Amc$) if it is a
    model of each ABox-axiom in~\Amc.
    %
    We call \Amc \emph{consistent} if it has a model.
\end{definition}

\noindent
Note that in the definition of the syntax of role assertions, we do not allow
inverse roles.  This is not a real restriction as an \enquote{assertion} of
the form $r^-(a,b)$ can be equivalently expressed by $r(b,a)$.

In the following, we often call TBox-axioms, RBox-axioms, and ABox-axioms
simply \emph{axioms}.  Now, we are ready to give the formal definition of the
syntax and the semantics of knowledge bases.

\begin{definition}[Knowledge base]\label{def:kb}
    A \emph{knowledge base} is a triple $\Kmc=(\Amc,\Tmc,\Rmc)$
    where \Amc is an ABox, \Tmc is a TBox, and \Rmc is an RBox.

    The interpretation~\Imc is a \emph{model} of~\Kmc (written
    $\Imc\models\Kmc$) if it is a model of \Amc, \Tmc, and~\Rmc.
    %
    We call \Kmc \emph{consistent} if it has a model.
    %
    We say that \Kmc \emph{entails} an axiom $\alpha$ (written
    $\Kmc\models\alpha$) if every model of~\Kmc is also a model of~$\alpha$.
\end{definition}

\noindent
If a component of a knowledge base is empty, we may also omit it, i.e.~we write
e.g.\ $(\Amc,\Tmc)$ instead of $(\Amc,\Tmc,\emptyset)$
if the RBox is empty.

We now consider an example of a knowledge base that illustrates how knowledge
bases may be used.  Note that this example serves only
didactic purposes, and its content might not reflect everybody's mindset and is
highly debatable.

\begin{example}\label{ex:camels}
    Let \Amc be the ABox that consists of the following assertions:
    \[\NiceCamel(\leah)\text{, }\quad
        \isFatherOf(\hassan,\leah)\text{, }\quad
        \isFatherOf(\yusuf,\hassan).\]
    Intuitively, the first assertion states that Leah is a nice camel.  The
    second one states that Hassan is the father of Leah, and the third one
    states that Yusuf is the father of Hassan.

    Let \Rmc be the RBox that consists of the following two axioms:
    \[\isFatherOf\sqsubseteq\isAncestorOf\text{, }\quad
        \trans(\isAncestorOf).\]
    The first axiom states that if $d$ is the father of $e$, then $d$ is also an
    ancestor of $e$.  The second axiom states that \isAncestorOf is transitive.

    Let \Tmc be the TBox that consists of the following GCIs and concept
    definitions:
    \begin{itemize}
        \item $\BactrianCamel\sqcup\Dromedary\sqsubseteq\Camel$;
        \item $\Llama\sqcup\Guanaco\sqcup\Alpaca\sqcup\Vicuna\sqsubseteq\Camel$;
        \item $\TrueCamel\equiv\BactrianCamel\sqcup\Dromedary$; and
        \item $\NiceCamel\equiv\lnot\Dromedary\sqcap\exists\likes.\Foliage$.
    \end{itemize}
    Intuitively, the GCIs state that Bactrian camels, dromedaries, llamas, etc.\
    are camels.  The first concept definition states that true camels are
    Bactrian camels or dromedaries, and the second one states that nice camels
    are no dromedaries and they like foliage.

    Figure~\ref{fig:camels-model} depicts the graphical representation of a
    model~\Imc of $\Kmc:=(\Amc,\Tmc,\Rmc)$.
    %
    Note that \Imc is also a model of the axiom $\Llama(\hassan)$, which
    is, however, not entailed by~\Kmc.
\end{example}

\begin{figure}[t]
    \centering
    \begin{tikzpicture}[%
            ->,
            >=stealth',
            semithick,
            shorten <=2pt,
            shorten >=2pt,
            auto,
            on grid,
            node distance=12ex and 14em,
            element/.style={circle,fill,draw,inner sep=0,minimum size=4pt}]

         \node[element] (leah)    [label=above left:\leah,label=below:{\shortstack{\NiceCamel,\\ \Llama, \Camel}}] {};
         \node[element] (hassan)  [right=of leah,label=above:\hassan,label=below:{\Llama, \Camel}]                 {};
         \node[element] (yusuf)   [right=of hassan,label=above right:\yusuf,label=below:{\Llama, \Camel}]          {};
         \node[element] (foliage) [above=of leah,label=above:\Foliage]                                             {};

         \path[->]  (leah)   edge              node [swap] {\likes}         (foliage)
                    (hassan) edge              node        {\isFatherOf}    (leah)
                             edge              node [swap] {\isAncestorOf,} (leah)
                    (yusuf)  edge              node        {\isFatherOf}    (hassan)
                             edge              node [swap] {\isAncestorOf,} (hassan)
                             edge [bend right] node [swap] {\isAncestorOf}  (leah);
    \end{tikzpicture}
    \caption{A model~\Imc of the KB~\Kmc of Example~\ref{ex:camels}}
    \label{fig:camels-model}
\end{figure}


\subsection{Specific Description Logics}

As mentioned above, what differs from DL to DL is which concept and role
constructors are available.  The smallest propositionally closed DL is
\ALC~\cite{ScSm-AIJ91}.  In this DL, the allowed concept constructors are
negation, conjunction, and existential restrictions, and thus also disjunctions,
universal restrictions, and the top and bottom concepts can be expressed.

If additional concept or role constructors are available, this is denoted by
concatenating a corresponding letter: \Qmc means (qualified) number
restrictions, \Imc means inverse roles, \Omc means nominals, and \Hmc means
role-inclusion axioms (role hierarchies).  For instance, the DL which is an
extension of \ALC and allows inverse roles is called \ALCI.  The extension of
\ALC with transitivity axioms is usually denoted by \Smc due to its close
relationship with the modal logic~\Sfour.  Thus, the DL that allows all the
concept and role constructors introduced above is called \SHOIQ.

Throughout this thesis, we sometimes prefix some notions with the specific DL to
make clear which DL is used to construct the concepts or axioms.  For instance,
we may write \enquote{\ALC-knowledge base} to make clear that the knowledge base
is constructed using concepts expressible in \ALC, and does not contain e.g.\
inverse roles.  If the DL under consideration is clear from the context, we omit
this prefix for ease of presentation.

Given a knowledge base $\Kmc=(\Amc,\Tmc,\Rmc)$, we say that a role
name~$r$ is \emph{transitive} (w.r.t.\ \Kmc) if $\Kmc\models\trans(r)$, and $r$
is a \emph{subrole} of a role name~$s$ (w.r.t.\ \Kmc) if $\Kmc\models
r\sqsubseteq s$.  Moreover, we call $r$ \emph{simple} (w.r.t.~\Kmc) if it has no
transitive subrole.
%
Note that entailments of the form $\Kmc\models\trans(r)$ and $\Kmc\models
r\sqsubseteq s$ only depend on the RBox~\Rmc.  Such entailments can be decided
in time polynomial in the size of \Rmc~\cite{HoST-IGPL00}.
%
As shown in~\cite{HoST-IGPL00}, already for the DL~\SHQ, the problem of deciding
whether a given knowledge base is consistent is undecidable, even if all
at-least restrictions are unqualified, i.e.\ of the form $\atLeast{n}{r}{\top}$.
One reason for that is the occurrence of non-simple role names in such
restrictions.  To regain decidability of this important inference problem, role
names occurring in at-least restrictions are therefore usually required to be
simple.  In the following, we also make this restriction to the
syntax of~\SHQ and every of its superlogics.

Under this assumption, the problem of deciding the consistency of knowledge
bases is in~\ExpTime, even if the numbers occurring in at-least restrictions are
given in binary encoding~\cite{Tob-PhD01}.  On the other hand, the problem is
\ExpTime-hard already in \ALC~\cite{Sch-IJCAI91}.  If we add nominals
(\SHOQ)~\cite{Sch-DKE94,HoSa-IJCAI01} or inverse roles
(\SHIQ)~\cite{Sch-DKE94,Tob-PhD01}, the complexity of this problem stays in
\ExpTime, but it increases to \NExpTime-complete if we include both
(\SHOIQ)~\cite{Sch-DKE94,Tob-JAIR00,Pra-JLLI05}.


\subsection{Boolean Knowledge Bases}

The notion of a knowledge base (see Definition~\ref{def:kb})
can be generalised to Boolean knowledge bases.

\begin{definition}[Boolean knowledge base]\label{def:boolean-kb}
    Let \Rmc be an RBox.  The set of \emph{Boolean axiom formulas w.r.t.~\Rmc}
    is the smallest set such that
    \begin{itemize}
        \item every ABox-axiom and every TBox-axiom in which at-least
            restrictions contain only simple roles w.r.t.~\Rmc is a Boolean
            axiom formula; and
        \item if $\Psi_1$ and $\Psi_2$ are Boolean axiom formulas, then so are
            $\lnot\Psi_1$ (negation) and $\Psi_1\land\Psi_2$ (conjunction).
    \end{itemize}
    %
    A \emph{Boolean knowledge base} is a pair $\Bmc=(\Psi,\Rmc)$,
    where \Rmc is an RBox, and $\Psi$ is a Boolean axiom formula w.r.t.~\Rmc.

    The interpretation~\Imc is a \emph{model} of~\Bmc (written
    $\Imc\models\Bmc$) iff $\Imc\models\Rmc$ and $\Imc\models\Psi$, where the
    latter is defined inductively as follows:
    \begin{itemize}
        \item $\Imc\models\lnot\Psi_1$ iff $\Imc\not\models\Psi_1$; and
        \item $\Imc\models\Psi_1\land\Psi_2$ iff $\Imc\models\Psi_1$ and
            $\Imc\models\Psi_2$.
    \end{itemize}
    %
    We call \Bmc \emph{consistent} if it has a model.  We say that \Bmc
    \emph{entails} the axiom $\alpha$ (written $\Bmc\models\alpha$) if every
    model of \Bmc is also a model of $\alpha$.
\end{definition}

\noindent
For convenience, we use the Boolean knowledge base
$(\Psi_1\lor\Psi_2,\Rmc)$ as an abbreviation for
$(\lnot(\lnot\Psi_1\land\lnot\Psi_2),\Rmc)$.

The reason why we do not allow RBox-axioms as Boolean axiom formulas is that the
notion of simple roles does not make sense w.r.t.\ a Boolean combination of
RBox-axioms.

According to this definition, knowledge bases can be seen as special kinds of
Boolean knowledge bases.  In fact, the knowledge base
$\Kmc=(\Amc,\Tmc,\Rmc)$ \emph{induces} the Boolean knowledge base
$\Bmc_\Kmc=(\Psi_\Kmc,\Rmc)$ with
$\Psi_\Kmc:=\bigwedge\Amc\land\bigwedge\Tmc$, where $\bigwedge\Amc$ denotes
$\bigwedge_{\alpha\in\Amc}\alpha$, and $\bigwedge\Tmc$ denotes
$\bigwedge_{\beta\in\Tmc}\beta$.  Thus, Boolean knowledge bases generalise
classical knowledge bases as introduced in Definition~\ref{def:kb}.

The problem of deciding the consistency of Boolean knowledge bases, however, is
not so well-investigated as for \enquote{classical} knowledge bases.  It is
known that for the description logic \ALC, this problem is
\ExpTime-complete~\cite{GKW+-03}, and we show in
Section~\ref{sec:consistency-boolean-shoqcap-kb} that it remains in \ExpTime for
an extension of the description logic \SHOQ.


\section{Propositional Linear-Time Temporal Logic and \texorpdfstring{$\omega$}{Omega}-Automata}\label{sec:ltl-automata}

In this section, we recall the definitions for the prominent temporal logic
\emph{propositional linear-time temporal logic (LTL)}~\cite{Pnu-FOCS77} that are
relevant for this thesis.  After introducing the syntax and semantics of
propositional LTL in Section~\ref{sec:ltl}, we consider its connection to
$\omega$-automata in Section~\ref{sec:aut-for-ltl}.


\subsection{Syntax and Semantics of Propositional LTL}\label{sec:ltl}

Propositional LTL extends propositional logic with modal operators that can be
used to talk about the past and the future.
%
The syntax of propositional LTL is defined as follows.

\begin{definition}[Syntax of propositional LTL]\label{def:syntax-ltl}
    Let $\Pmc=\{p_1,\dots,p_m\}$ be a finite set of \emph{propositional
    variables}.  The set of \emph{propositional LTL-formulas over~\Pmc} is the
    smallest set such that
    \begin{itemize}
        \item if $p\in\Pmc$, then $p$ is a propositional LTL-formula over~\Pmc;
            and
        \item if $\phi_1$ and~$\phi_2$ are propositional LTL-formulas over~\Pmc,
            then so are: $\lnot\phi_1$ (negation), $\phi_1\land\phi_2$
            (conjunction), $\Next\phi_1$ (next), $\Previous\phi_1$ (previous),
            $\phi_1\Until\phi_2$ (until), and $\phi_1\Since\phi_2$ (since).
    \end{itemize}
\end{definition}

\noindent
If the set of propositional variables is clear from the context or irrelevant,
we talk about propositional LTL-formulas rather than propositional LTL-formulas
over~\Pmc.

As usual in temporal logics, we use
\begin{itemize}
    \item $\phi_1\lor\phi_2$ (disjunction) as an abbreviation for
        $\lnot(\lnot\phi_1\land\lnot\phi_2)$;
    \item $\phi_1\to\phi_2$ (implication) as an abbreviation for
        $\lnot\phi_1\lor\phi_2$;
    \item \true as an abbreviation for an arbitrary but fixed propositional
        tautology such as $p\lor\lnot p$ with $p\in\Pmc$;
    \item \false as an abbreviation for $\lnot\true$;
    \item $\Diamond\phi$ (diamond, which should be read as \enquote{eventually}
        or \enquote{some time in the future}) as an abbreviation for
        $\true\Until\phi$;
    \item $\Box\phi$ (box, which should be read as \enquote{always} or
        \enquote{always in the future}) as an abbreviation for
        $\lnot\Diamond\lnot\phi$;
    \item $\Diamondm\phi$ (which should be read as \enquote{once}
        or \enquote{some time in the past}) as an abbreviation for
        $\true\Since\phi$; and
    \item $\Boxm\phi$ (which should be read as \enquote{historically} or
        \enquote{always in the past}) as an abbreviation for
        $\lnot\Diamondm\lnot\phi$.
\end{itemize}

The semantics of propositional LTL is defined using the non-negative integers as
discrete linear flow of time.  For each point in time, i.e.~non-negative
integer, the semantic structure determines which of the propositional variables
are true at this point.  This is captured in the notion of a propositional
LTL-structure.

\begin{definition}[Semantics of propositional LTL]\label{def:semantics-ltl}
    Let $\Pmc=\{p_1,\dots,p_m\}$ be a set of propositional variables.  A
    \emph{propositional LTL-structure over~\Pmc} is an infinite sequence
    $\Wmf=(w_i)_{i\ge 0}$ of sets $w_i\subseteq\Pmc$, which we call
    \emph{worlds}.

    Given a propositional LTL-formula~$\phi$, a propositional LTL-structure
    $\Wmf=(w_i)_{i\ge 0}$, and a time point $i\ge 0$, \emph{validity of~$\phi$
    in~\Wmf at time~$i$} (written $\Wmf,i\models\phi$) is defined inductively as
    follows:
    \[\begin{array}{lcl}
        \Wmf,i\models p
            &\text{iff}
            &p\in w_i\\
        \Wmf,i\models\lnot\phi_1
            &\text{iff}
            &\Wmf,i\not\models\phi_1,\ \text{i.e.~not}\ \Wmf,i\models\phi_1\\
        \Wmf,i\models\phi_1\land\phi_2
            &\text{iff}
            &\Wmf,i\models\phi_1\ \text{and}\ \Wmf,i\models\phi_2\\
        \Wmf,i\models\Next\phi_1
            &\text{iff}
            &\Wmf,i+1\models\phi_1\\
        \Wmf,i\models\Previous\phi_1
            &\text{iff}
            &i>0\ \text{and}\ \Wmf,i-1\models\phi_1\\
        \Wmf,i\models\phi_1\Until\phi_2
            &\text{iff}
            &\text{there is some $k\geq i$ such that $\Wmf,k\models\phi_2$, and}\\
            & &\text{$\Wmf,j\models\phi_1$ for every $j$, $i\le j<k$}\\
        \Wmf,i\models\phi_1\Since\phi_2
            &\text{iff}
            &\text{there is some $k$, $0\le k\le i$, such that $\Wmf,k\models\phi_2$, and}\\
            & &\text{$\Wmf,j\models\phi_1$ for every $j$, $k<j\le i$}
    \end{array}\]
    %
    If $\Wmf,0\models\phi$, then we call \Wmf a \emph{model} of~$\phi$.  We call
    the propositional LTL-formula~$\phi$ \emph{satisfiable} if it has a model.

    The \emph{satisfiability problem in propositional LTL} is the problem of
    deciding, given a propositional LTL-formula~$\phi$, whether $\phi$ is
    satisfiable.

    Two propositional LTL-formulas $\phi_1,\phi_2$ are \emph{equivalent}
    (written $\phi_1\equiv\phi_2$) if they have the same models.
\end{definition}

\noindent
Note that we defined here the so-called \emph{non-strict $\Until$} and
\emph{non-strict $\Since$}.  For the strict version $\sUntil$ of $\Until$, one
needs to replace in the definition of the semantics of $\Until$ \enquote{there
is some $k\ge i$} by \enquote{there is some $k>i$}.  It is not hard to see that,
in the presence of $\Next$, both $\Until$ and $\sUntil$ have the same expressive
power.  In fact, the formula $\phi_1\Until\phi_2$ is equivalent to
$\phi_2\lor(\phi_1\sUntil\phi_2)$, and conversely, $\phi_1\sUntil\phi_2$ is
equivalent to $\phi_1\land\Next(\phi_1\Until\phi_2)$.  Similar arguments apply
to the strict version $\sSince$ of $\Since$.

We continue by giving an example of a propositional LTL-formula.

\begin{example}\label{ex:ltl}
    Let $\phi:=\Next p_1\land(p_2\Until p_3)$ be a propositional LTL-formula.
    Consider the two propositional LTL-structures $\Wmf_1,\Wmf_2$ that are
    depicted in Figure~\ref{fig:ltl-structures} in a graphical representation.
    We have $\Wmf_1,0\models\phi$, and thus $\phi$ is satisfiable.  Moreover, we
    have $\Wmf_2,0\not\models\phi$, but $\Wmf_2,1\models\phi$.
\end{example}

\begin{figure}[t]
    \centering
    \begin{tikzpicture}[%
            ->,
            >=stealth',
            semithick,
            shorten <=2pt,
            shorten >=2pt,
            auto,
            on grid,
            node distance=8ex and 5em,
            element/.style={circle,fill,draw,inner sep=0,minimum size=4pt}]

            \node          (W1) [label=right:{$\Wmf_1$:}]                 {};
            \node[element] (a0) [right=of W1,label=below:{$\{p_2\}$}]     {};
            \node[element] (a1) [right=of a0,label=below:{$\{p_1,p_2\}$}] {};
            \node[element] (a2) [right=of a1,label=below:{$\{p_2\}$}]     {};
            \node[element] (a3) [right=of a2,label=below:{$\{p_3\}$}]     {};
            \node[element] (a4) [right=of a3,label=below:{$\emptyset$}]   {};
            \node          (d1) [right=of a4]                             {$\dots$};

            \draw (a0)--(a1);
            \draw (a1)--(a2);
            \draw (a2)--(a3);
            \draw (a3)--(a4);
            \draw (a4)--(d1);

            \node          (W2) [below=of W1,label=right:{$\Wmf_2$:}]     {};
            \node[element] (b0) [right=of W2,label=below:{$\emptyset$}]   {};
            \node[element] (b1) [right=of b0,label=below:{$\{p_1,p_2\}$}] {};
            \node[element] (b2) [right=of b1,label=below:{$\{p_1,p_2\}$}] {};
            \node[element] (b3) [right=of b2,label=below:{$\{p_3\}$}]     {};
            \node[element] (b4) [right=of b3,label=below:{$\{p_2\}$}]     {};
            \node          (d2) [right=of b4]                             {$\dots$};

            \draw (b0)--(b1);
            \draw (b1)--(b2);
            \draw (b2)--(b3);
            \draw (b3)--(b4);
            \draw (b4)--(d2);
    \end{tikzpicture}
    \caption{The propositional LTL-structures $\Wmf_1,\Wmf_2$ of
        Example~\ref{ex:ltl}}
    \label{fig:ltl-structures}
\end{figure}

\noindent
We call the temporal operators $\Next$ and $\Until$ \emph{future operators},
whereas we call $\Previous$ and $\Since$ \emph{past operators}.
%
Note that propositional LTL is normally defined using only future
operators~\cite{Pnu-FOCS77}, and the extension with past
operators~\cite{GPS+-POPL80} is usually called \emph{propositional Past-LTL}.
It is a well-known result, however, that the past operators do not add
expressive power~\cite{GPS+-POPL80}, even though some properties are easier to
express using past operators~\cite{LiPZ-CLP85}.  Indeed, using Gabbay's
separation theorem~\cite{Gab-TLS87}, one can construct for each propositional
LTL-formula $\phi$ with past operators, an equivalent propositional
LTL-formula~$\phi'$ that does not contain past operators.  However, this
construction is in general non-elementary in the size of~$\phi$, as basically
the size of the constructed formula increases by one exponential for
each~$\Until$ nested inside an~$\Since$, and vice versa.  This upper bound can
be improved, but no constructions of size less than triply exponential in the
size of~$\phi$ are known~\cite{LaMS-LICS02}.  For the lower bound, it is known
that past operators make propositional LTL exponentially more
succinct~\cite{LaMS-LICS02}, i.e.~there is a propositional LTL-formula~$\phi$
with past operators such that the size of an equivalent propositional
LTL-formula $\phi'$ without past operators is bounded by
$2^{\Omega(\lvert\phi\rvert)}$, where $\lvert\phi\rvert$ denotes the size
of~$\phi$.

Moreover, the satisfiability problem in propositional LTL is \PSpace-complete
irrespective of the use of past operators~\cite{SiCl-JACM85,Mar-AI04}.
%
In the next section, we recall the connection between propositional LTL and
$\omega$-automata.


\subsection{\texorpdfstring{$\omega$}{Omega}-Automata and Their Connection to Propositional LTL}\label{sec:aut-for-ltl}

For propositional LTL, the satisfiability problem can be decided by first
constructing an $\omega$-automaton for the given formula, and then testing this
automaton for emptiness.
%
In general, $\omega$-automata accept \emph{$\omega$-words} over an
alphabet~$\Sigma$, i.e.~infinite sequences of letters
$w=\sigma_0\sigma_1\sigma_2\dots$ with $\sigma_i\in\Sigma$ for every $i\ge 0$.
The set of all $\omega$-words over~$\Sigma$ is denoted by $\Sigma^\omega$, and a
subset~$L$ of~$\Sigma^\omega$ is called \emph{$\omega$-language}.

There are various $\omega$-automata models that can be employed for solving the
satisfiability problem for propositional LTL such as
Büchi-automata~\cite{Bue-ICLMPS62}, Muller-automata~\cite{Mul-SWCT63},
Rabin-automata~\cite{Rab-ToAMS69}, and Streett-automata~\cite{Str-IC82}.  To
keep the explanations simple, in this thesis we focus on (non-deterministic)
Büchi-automata.

\begin{definition}[Generalised Büchi-automaton]\label{def:gnba}
    A \emph{generalised Büchi-automaton~\Gmc} is a tuple
    $\Gmc=(Q,\Sigma,\Delta,Q_0,\Fmc)$ consisting of a finite set of states~$Q$,
    a finite input alphabet~$\Sigma$, a transition relation
    $\Delta\subseteq Q\times\Sigma\times Q$, a set of initial states
    $Q_0\subseteq Q$, and a set of sets of final states $\Fmc\subseteq 2^Q$.

    Given an $\omega$-word $w=\sigma_0\sigma_1\sigma_2\dotso\in\Sigma^\omega$, a
    \emph{run} of~\Gmc on~$w$ is an $\omega$-word $q_0q_1q_2\dotso\in Q^\omega$
    such that $q_0\in Q_0$ and $(q_i,\sigma_i,q_{i+1})\in\Delta$ for every
    $i\ge 0$.  This run is \emph{accepting} if for every $F\in\Fmc$, there are
    infinitely many $i\ge 0$ such that $q_i\in F$.
    %
    The \emph{language $L_\omega(\Gmc)$ accepted by~\Gmc} is defined as
    \[L_\omega(\Gmc):=\{w\in\Sigma^\omega\mid\text{there is an accepting run of \Gmc on $w$}\}.\]
    %
    The \emph{emptiness problem} for generalised Büchi-automata is the problem
    of deciding, given a generalised Büchi-automaton~\Gmc, whether
    $L_\omega(\Gmc)=\emptyset$ or not.

    Moreover, we call~\Gmc \emph{deterministic} if $\lvert Q_0\rvert=1$ and for
    every $q\in Q$ and $\sigma\in\Sigma$, there is at most one $q'\in Q$ with
    $(q,\sigma,q')\in\Delta$.
\end{definition}

\noindent
Normal Büchi-automata are a special case of a generalised Büchi-automata where
$\Fmc=\{F\}$, and are denoted by $\Nmc=(Q,\Sigma,\Delta,Q_0,F)$.
%
It is common knowledge that every generalised Büchi-automaton~\Gmc, can be
transformed into a Büchi~automaton~\Nmc such that
$L_\omega(\Gmc)=L_\omega(\Nmc)$ in time polynomial in the size
of~\Gmc~\cite{GPV+-PSTV96,BaKa-08}.

Regarding the complexity of the emptiness problem for Büchi-automata, it is
well-known that it can be solved in time polynomial in the size of the
Büchi-automaton~\cite{VaWo-IC94}.  Together with the arguments of the previous
paragraph, this yields that the emptiness problem for generalised Büchi-automata
can also be solved in time polynomial in the size of the generalised
Büchi-automaton.

Additionally, there is a well-known connection between (generalised)
Büchi-automata and propositional LTL\@.  In fact, given a propositional
LTL-formula~$\phi$ over~\Pmc, we can view any propositional LTL-structure
$\Wmf=(w_i)_{i\ge 0}$ as an $\omega$-word
$w=w_0w_1w_2\dotso\in\Sigma_\Pmc^\omega$, where the alphabet $\Sigma_\Pmc$
consists of all subsets of~\Pmc.  It is well-known that one can build a
generalised Büchi-automaton that accepts exactly the models of~$\phi$.

\begin{definition}[Büchi-automaton for propositional LTL-formula]\label{def:ba-for-ltl}
    Let $\phi$ be a propositional LTL-formula over~\Pmc, and let \Gmc be a
    generalised Büchi-automaton working on the alphabet~$\Sigma_\Pmc$.  We
    define
    \[L_\omega(\phi):=\{w_0w_1w_2\dotso\in\Sigma_\Pmc^\omega\mid%
        \Wmf=(w_i)_{i\ge 0}\ \text{is a model of $\phi$}\},\]
    and say that \Gmc is a \emph{Büchi-automaton for~$\phi$} if
    $L_\omega(\Gmc)=L_\omega(\phi)$.
\end{definition}

\noindent
If \Gmc is a Büchi-automaton for~$\phi$, then $\phi$ is satisfiable iff
$L_\omega(\Gmc)\ne\emptyset$.  Thus, by constructing a Büchi-automaton
for~$\phi$, we can reduce the satisfiability problem in propositional LTL to the
emptiness problem for Büchi-automata.
%
It is well-known that, given a propositional LTL-formula~$\phi$, one can
construct a (generalised) Büchi-automaton for~$\phi$ in time exponential in the
size of $\phi$~\cite{WoVS-FOCS83,VaWo-IC94,LiPZ-CLP85}.
%
However, for propositional LTL-formulas involving past-operators, this
construction is often not done explicitly.  We include it here, and generalise
it as follows.  Instead of constructing a Büchi-automaton for a propositional
LTL-formula~$\phi$ over~\Pmc, we define a Büchi-automaton that, given $n\ge 0$,
accepts all $\omega$-words $w_0w_1w_2\dotso\in\Sigma_\Pmc^\omega$ such that
$\phi$ is valid in $\Wmf=(w_i)_{i\ge 0}$ at time~$n$.  This generalisation will
prove to be useful in Section~\ref{sec:tcq-upper-bounds}.

To define the Büchi-automaton, we need a few more notions.  From now on, let
$\phi$ be a propositional LTL-formula over~\Pmc, and let $n\ge 0$.  As usual,
the set of \emph{subformulas} of~$\phi$ is the smallest set containing all
propositional LTL-formulas occurring in~$\phi$ (including $\phi$ itself).
%
We define $\PCl(\phi)$ to be the closure under negation of the set of
subformulas of~$\phi$.  In the following, we identify $\lnot\lnot\psi$ with
$\psi$ for every subformula~$\psi$ of~$\phi$.  Thus, the set $\PCl(\phi)$ is of
size polynomial in the size of~$\phi$.

\begin{definition}[Propositional LTL-type]
    Let $\phi$ be a propositional LTL-formula.  A \emph{propositional LTL-type}
    for~$\phi$ is a set $T\subseteq\PCl(\phi)$ such that:
    \begin{itemize}
        \item for every $\psi_1\land\psi_2\in\PCl(\phi)$, we have
            $\psi_1\land\psi_2\in T$ iff $\{\psi_1,\psi_2\}\subseteq T$; and
        \item for every $\lnot\psi\in\PCl(\phi)$, we have $\lnot\psi\in T$ iff
            $\psi\notin T$.
    \end{itemize}
\end{definition}

\noindent
Obviously, the set of all propositional LTL-types for a given propositional
LTL-formula~$\phi$ is exponential in the size of~$\phi$.

Now, we are ready to define the generalised Büchi-automaton with the above
properties by equipping the set of states with a counter from $\{0,\dots,n+1\}$.
Transitions where the counter is $i=n$ ensure that $\phi$ is satisfied.  We
construct the generalised Büchi-automaton
$\Gmc_{\phi,n}=(Q,\Sigma_\Pmc,\Delta,Q_0,\Fmc)$ as follows:
\begin{itemize}
    \item $Q:=\bigl\{(q,k)\mid%
            \text{$q$ is a propositional LTL-type for~$\phi$, and $0\le k\le n+1$}\bigr\}$;
    \item $((T,k),\ \sigma,\ (T',k'))\ \in\ \Delta$ iff
        \begin{itemize}
            \item $\sigma=T\cap\Pmc$;
            \item for every $\Next\psi\in\PCl(\phi)$, we have $\Next\psi\in
                T$ iff $\psi\in T'$;
            \item for every $\Previous\psi\in\PCl(\phi)$, we have
                $\Previous\psi\in T'$ iff $\psi\in T$;
            \item for every $\psi_1\Until\psi_2\in\PCl(\phi)$, we have
                $\psi_1\Until\psi_2\in T$ iff (i)~$\psi_2\in T$ or
                (ii)~$\psi_1\in T$ and $\psi_1\Until\psi_2\in T'$;
            \item for every $\psi_1\Since\psi_2\in\PCl(\phi)$, we have
                $\psi_1\Since\psi_2\in T'$ iff (i)~$\psi_2\in T'$ or
                (ii)~$\psi_1\in T'$ and $\psi_1\Since\psi_2\in T$;
            \item $k=n$ implies $\phi\in T$; and
            \item $k'=\begin{cases}
                    k+1 &\text{if $k\le n$, and}\\
                    k   &\text{otherwise;}
                \end{cases}$
        \end{itemize}
    \item $Q_0:=\bigl\{(T,0)\mid\text{there is no $\Previous\psi\in T$, and %
            for every $\psi_1\Since\psi_2\in T$, we have $\psi_2\in T$}\bigr\}$; and
    \item
        $\Fmc:=\bigl\{F_{\psi_1\Until\psi_2}\times\{n+1\}\mid%
            \psi_1\Until\psi_2\in\PCl(\phi)\bigr\}$, where
        \[F_{\psi_1\Until\psi_2}:=\{T\mid\text{for every}\
            \psi_1\Until\psi_2\in T,\ \text{we have}\ \psi_2\in T\}.\]
\end{itemize}
%
We show now that $\Gmc_{\phi,n}$ has indeed the above properties.

\begin{lemma}\label{lem:gba-phi-n}
    For every $\omega$-word $w=w_0w_1w_2\dotso\in\Sigma_\Pmc^\omega$, we have
    $w\in L_\omega(\Gmc_{\phi,n})$ iff $\phi$ is valid in the propositional
    LTL-structure $\Wmf=(w_i)_{i\ge 0}$ at time~$n$.
\end{lemma}

\begin{proof}
    For the \enquote{only if} direction, assume that $\phi$ is valid in the
    propositional LTL-structure $\Wmf=(w_i)_{i\ge 0}$ at time~$n$.  We define
    $S_i:=\{\psi\in\PCl(\phi)\mid\Wmf,i\models\psi\}$ for $i\ge 0$.  Then,
    \[(S_0,0)(S_1,1)\dots(S_n,n)(S_{n+1},n+1)(S_{n+2},n+1)\dots\]
    is a run of~$\Gmc_{\phi,n}$ on $w_0w_1w_2\dots$ due to the following
    reasons:
    %
    \begin{itemize}
        \item We have $(S_i,k)\in Q$ for every $i\ge 0$ and every~$k$,
            $0\le k\le n+1$, since:
            \begin{itemize}
                \item For every $\psi_1\land\psi_2\in\PCl(\phi)$, we have
                    $\Wmf,i\models\psi_1\land\psi_2$ iff $\Wmf,i\models\psi_1$
                    and $\Wmf,i\models\psi_2$.  Thus, we have $\psi_1\land\psi_2\in S_i$ iff
                    $\{\psi_1,\psi_2\}\subseteq S_i$.
                \item For every $\lnot\psi\in\PCl(\phi)$, we have either
                    $\Wmf,i\models\lnot\psi$ or $\Wmf,i\models\psi$.  Thus, we
                    have $\lnot\psi\in S_i$ iff $\psi\notin S_i$.
            \end{itemize}
        \item We have for every~$i$, $0\le i\le n$, that
            \[((S_i,i),\ w_i,\ (S_{i+1},i+1))\ \in\ \Delta,\]
            and for every $i\ge n+1$ that
            \[((S_i,n+1),\ w_i,\ (S_{i+1},n+1))\ \in\ \Delta\]
            since:
            \begin{itemize}
                \item by the definition of $S_i$, we have $w_i=S_i\cap\Pmc$;
                \item for every $\Next\psi\in\PCl(\phi)$, we have
                    $\Next\psi\in S_i$ iff $\Wmf,i\models\Next\psi$ iff
                    $\Wmf,i+1\models\psi$ iff $\psi\in S_{i+1}$;
                \item for every $\Previous\psi\in\PCl(\phi)$, we have
                    $\Previous\psi\in S_{i+1}$ iff
                    $\Wmf,i+1\models\Previous\psi$ iff $\Wmf,i\models\psi$ iff
                    $\psi\in S_i$;
                \item for every $\psi_1\Until\psi_2\in\PCl(\phi)$, we have
                    $\psi_1\Until\psi_2\in S_i$ iff
                    $\Wmf,i\models\psi_1\Until\psi_2$ iff
                    (i)~$\Wmf,i\models\psi_2$ or (ii)~$\Wmf,i\models\psi_1$ and
                    $\Wmf,i+1\models\psi_1\Until\psi_2$ iff (i)~$\psi_2\in S_i$
                    or (ii)~$\psi_1\in S_i$ and $\psi_1\Until\psi_2\in S_{i+1}$;
                \item for every $\psi_1\Since\psi_2\in\PCl(\phi)$, we have
                    $\psi_1\Since\psi_2\in S_{i+1}$ iff
                    $\Wmf,i+1\models\psi_1\Since\psi_2$ iff
                    (i)~$\Wmf,i+1\models\psi_2$ or (ii)~$\Wmf,i+1\models\psi_1$
                    and $\Wmf,i\models\psi_1\Since\psi_2$ iff
                    (i)~$\psi_2\in S_{i+1}$ or (ii)~$\psi_1\in S_{i+1}$ and
                    $\psi_1\Since\psi_2\in S_i$;
                \item for $i=n$, we have $\phi\in S_i$ since
                    $\Wmf,n\models\phi$; and
                \item the condition for incrementing the second component of a
                    state (until $n+1$ is reached) is obviously also satisfied.
            \end{itemize}
        \item We have for every $\Previous\psi\in\PCl(\phi)$ that
            $\Wmf,0\not\models\Previous\psi$, and thus $\Previous\psi\notin S_0$.
            Additionally, we have for every $\psi_1\Since\psi_2\in S_0$, since
            $\Wmf,0\models\psi_1\Since\psi_2$, also $\Wmf,0\models\psi_2$, and
            thus $\psi_2\in S_0$.  This yields that $(S_0,0)\in Q_0$.
    \end{itemize}
    %
    Moreover, the above run is accepting.  We prove this by contradiction.
    Suppose that for some $\psi_1\Until\psi_2\in\PCl(\phi)$, the set
    $\{i\ge 0\mid S_i\in F_{\psi_1\Until\psi_2}\}$ is finite.  Then there exists
    some $k\ge 0$ such that $S_\ell\notin F_{\psi_1\Until\psi_2}$ for every
    $\ell\geq k$.  This means $\psi_1\Until\psi_2\in S_\ell$ and
    $\psi_2\notin S_\ell$ for every $\ell\ge k$.  Hence,
    $\Wmf,k\models\psi_1\Until\psi_2$ and $\Wmf,\ell\not\models\psi_2$ for every
    $\ell\ge k$.  This contradicts the semantics of~$\Until$.

    For the \enquote{if} direction, assume that
    $w_0w_1w_2\dotso\in L_\omega(\Gmc_{\phi_n})$, i.e.~there is an accepting run
    \[(S_0,0)(S_1,1)\dots(S_n,n)(S_{n+1},n+1)(S_{n+2},n+1)\dots\]
    of~$\Gmc_{\phi,n}$ on~$w_0w_1w_2\dots$.
    %
    We show that $\phi$ is valid in $\Wmf:=(w_i)_{i\ge 0}$ at time~$n$.  We have
    $\phi\in S_n$ by the definition of~$\Delta$.  Thus, it is enough to show
    that for every $\psi\in\PCl(\phi)$ and every $i\ge 0$, we have $\psi\in S_i$
    iff $\Wmf,i\models\psi$.  This can be shown by induction on the structure
    of~$\psi$ using the definition of~$\Delta$.
    %
    \begin{itemize}
        \item If $\psi\in\Pmc$, we have $\psi\in S_i$ iff $\psi\in w_i$ iff
            $\Wmf,i\models\psi$.
        \item If $\psi=\lnot\chi$, we have $\lnot\chi\in S_i$ iff
            $\chi\notin S_i$ iff $\Wmf,i\not\models\chi$ iff
            $\Wmf,i\models\lnot\chi$.
        \item If $\psi=\chi_1\land\chi_2$, we have $\chi_1\land\chi_2\in S_i$
            iff $\{\chi_1,\chi_2\}\subseteq S_i$ iff $\Wmf,i\models\chi_1$ and
            $\Wmf,i\models\chi_2$ iff $\Wmf,i\models\chi_1\land\chi_2$.
        \item If $\psi=\Next\chi$, we have $\Next\chi\in S_i$ iff
            $\chi\in S_{i+1}$ iff $\Wmf,i+1\models\chi$ iff
            $\Wmf,i\models\Next\chi$.
        \item If $\psi=\Previous\chi$, we have $\Previous\chi\in S_i$ iff $i>0$
            and $\chi\in S_{i-1}$ iff $i>0$ and $\Wmf,i-1\models\chi$ iff
            $\Wmf,i\models\Previous\chi$.  The first iff holds because of the
            definition of $Q_0$.
        \item If $\psi=\chi_1\Until\chi_2$, we prove $\chi_1\Until\chi_2\in S_i$
            iff $\Wmf,i\models\chi_1\Until\chi_2$ as follows.
            %
            For the \enquote{if} direction, assume that
            $\Wmf,i\models\chi_1\Until\chi_2$. Then there is some $k\ge i$ such
            that $\Wmf,k\models\chi_2$ and $\Wmf,\ell\models\chi_1$ for every
            $\ell$, $i\le\ell<k$.  We show by induction on~$j$ that
            $\chi_1\Until\chi_2\in S_{k-j}$ for $j$, $j\le k-i$.

            For $j=0$, we have: $\Wmf,k\models\chi_2$ implies $\chi_2\in S_k$ by
            the outer induction hypothesis, and the definition of~$\Delta$
            yields $\chi_1\Until\chi_2\in S_k$.

            For $j>0$, we have: $\Wmf,k-j\models\chi_1$ implies
            $\chi_1\in S_{k-j}$ by the outer induction hypothesis.  By the inner
            induction hypothesis, we have $\chi_1\Until\chi_2\in S_{k-j+1}$.
            Thus, by the definition of~$\Delta$, it follows that
            $\chi_1\Until\chi_2\in S_{k-j}$.

            For the \enquote{only if} direction, assume that
            $\chi_1\Until\chi_2\in S_i$.  Since states of
            $F_{\chi_1\Until\chi_2}\times\{n+1\}$ occur infinitely often among
            $S_0,S_1,S_2,\dots$, there is some $k\ge i$ such that
            $S_k\in F_{\chi_1\Until\chi_2}$.  Let $k$ be the smallest index with
            that property.  Then it follows that $\chi_1\Until\chi_2\in S_\ell$
            and $\chi_2\notin S_\ell$ for every~$\ell$, $i\le\ell<k$.

            Since $\chi_1\Until\chi_2\in S_\ell$ and $\chi_2\notin S_\ell$ for
            every~$\ell$, $i\le\ell<k$, we have $\chi_1\in S_\ell$ because of
            the definition of~$\Delta$.  Thus, $\Wmf,\ell\models\chi_1$ for
            every~$\ell$, $i\le\ell<k$~($\ast$).

            Moreover, $\chi_1\Until\chi_2\in S_{k-1}$ and $\chi_2\notin S_{k-1}$
            imply $\chi_1\Until\chi_2\in S_k$ because of the definition
            of~$\Delta$.  This yields $\chi_2\in S_k$ since
            $S_k\in F_{\chi_1\Until\chi_2}$, and thus
            $\Wmf,k\models\chi_2$~($\ast\ast$).

            Finally, ($\ast$) and~($\ast\ast$) yield that
            $\Wmf,i\models\chi_1\Until\chi_2$ by the semantics of~$\Until$.
        \item If $\psi=\chi_1\Since\chi_2$, we prove $\chi_1\Since\chi_2\in S_i$
            iff $\Wmf,i\models\chi_1\Since\chi_2$ similarly.
            %
            For the \enquote{if} direction, assume that
            $\Wmf,i\models\chi_1\Since\chi_2$.  Then there is some $k$,
            $0\le k\le i$ such that $\Wmf,k\models\chi_2$ and
            $\Wmf,\ell\models\chi_1$ for every~$\ell$, $k<\ell\le i$.  We show
            by induction on~$j$ that $\chi_1\Since\chi_2\in S_{k+j}$ for $j$,
            $j\le i-k$.

            For $j=0$, we have: $\Wmf,k\models\chi_2$ implies $\chi_2\in S_k$ by
            the outer induction hypothesis, and the definition of~$\Delta$
            yields $\chi_1\Since\chi_2\in S_k$.

            For $j>0$, we have: $\Wmf,k+j\models\chi_1$ implies
            $\chi_1\in S_{k+j}$ by the outer induction hypothesis.  By the inner
            induction hypothesis, we have $\chi_1\Since\chi_2\in S_{k+j-1}$.
            Thus, by the definition of~$\Delta$, it follows that
            $\chi_1\Since\chi_2\in S_{k+j}$.

            For the \enquote{only if} direction, assume that
            $\chi_1\Since\chi_2\in S_i$.  There are two cases to consider:
            either $i=0$ or $i>0$.

            For $i=0$, we have: $\chi_1\Since\chi_2\in S_0$ implies
            $\chi_2\in S_0$ by the definition of~$Q_0$.  This yields
            $\Wmf,0\models\chi_2$, and thus $\Wmf,0\models\chi_1\Since\chi_2$.

            For $i>0$, we have again two cases: either $\chi_2\in S_i$ or
            $\chi_1\in S_i$ and $\chi_1\Since\chi_2\in S_{i-1}$.  For the case
            where $\chi_1\in S_i$, it directly follows that
            $\Wmf,i\models\chi_1\Since\chi_2$.  For the other case where
            $\chi_1\in S_i$ and $\chi_1\Since\chi_2\in S_{i-1}$, we have by the
            inner induction hypothesis that $\Wmf,i-1\models\chi_1\Since\chi_2$.
            Thus, there is a $k$, $0\le k\le i-1$, such that
            $\Wmf,k\models\chi_2$ and $\Wmf,j\models\chi_1$ for every $j$,
            $k<j\le i-1$.  Since we have by the outer induction hypothesis also
            that $\Wmf,i\models\chi_1$, it follows that there is some $k$,
            $0\le k\le i$, such that $\Wmf,k\models\chi_2$ and
            $\Wmf,j\models\chi_1$ for every $j$, $k<j\le i$.  Hence,
            $\Wmf,i\models\chi_1\Since\chi_2$.
    \end{itemize}
\end{proof}

\noindent
We immediately obtain the following corollary.

\begin{corollary}
    The generalised Büchi-automaton~$\Gmc_{\phi,0}$ is a Büchi-automaton
    for~$\phi$.
\end{corollary}

\begin{proof}
    By Lemma~\ref{lem:gba-phi-n}, we have
    \[L_\omega(\Gmc_{\phi,0})=\{w_0w_1w_2\dotso\in\Sigma_\Pmc^\omega\mid%
        \Wmf=(w_i)_{i\ge 0}\ \text{is a model of $\phi$}\}=L_\omega(\phi).\]
\end{proof}

\noindent
As already mentioned above, one can transform the
Büchi-automaton~$\Gmc_{\phi,n}$ into a \enquote{normal}
Büchi-automaton~$\Nmc_{\phi,n}$ such that
$L_\omega(\Gmc_{\phi,n})=L_\omega(\Nmc_{\phi,n})$ in time polynomial in the size
of~$\Gmc_{\phi,n}$~\cite{GPV+-PSTV96,BaKa-08}.
%
An analysis of the construction of~$\Gmc_{\phi,n}$ yields the following lemma.

\begin{lemma}\label{lem:ba-phi-n-exp}
    The Büchi-automaton~$\Nmc_{\phi,n}$ is of size exponential in the size
    of~$\phi$ and polynomial in~$n$, and can be constructed in time exponential in
    the size of~$\phi$ and polynomial in~$n$.
\end{lemma}

\begin{proof}
    Note that $\Gmc_{\phi,n}$, and thus $\Nmc_{\phi,n}$, have exponentially
    many states in the size of~$\phi$ and linearly many states in the size
    of~$n$, and each state can be represented using only space polynomial in the
    size of~$\phi$ and~$n$.
    %
    Moreover, the alphabet~$\Sigma_{\Pmc_\phi}$ is exponential in the size
    of~$\phi$.  The set of final states of~$\Gmc_{\phi,n}$ contains linearly
    many sets of size at most exponential in~$\phi$, while the size of the set
    of initial states and the transition relation is bounded polynomially in the
    size of the set of states, which is exponential in the size of~$\phi$ and
    linear in~$n$.

    Since all conditions that need to be checked to construct the components
    of~$\Gmc_{\phi,n}$ can be checked in time exponential in the size of~$\phi$
    and polynomial in~$n$, and $\Nmc_{\phi,n}$ can be constructed in time
    polynomial in the size of~$\Gmc_{\phi,n}$, we obtain the claim of the lemma.
\end{proof}

\noindent
We will use the result of this lemma later in
Section~\ref{sec:tcq-upper-bounds}.  Note that instead of constructing the
Büchi-automaton~$\Nmc_{\phi,n}$, we could also omit the counter and construct a
Büchi-automaton for~$\Next^n\phi$, e.g.~$\Nmc_{\Next^n\phi,0}$, that accepts the
same $\omega$-language.  However, the size of this Büchi-automaton is
exponential in the size of~$\Next^n\phi$, and thus also exponential in~$n$.
With the construction above, we have shown that we can do better.

Regarding the satisfiability problem in propositional LTL,
Lemma~\ref{lem:ba-phi-n-exp} yields the following.  If we first compute an
exponentially large Büchi-automaton for~$\phi$ and then apply the emptiness test
for Büchi-automata, we obtain an \ExpTime decision procedure for the
satisfiability problem.  In order to reduce the complexity to \PSpace, one has
to generate the relevant parts of the Büchi-automaton on-the-fly while
performing the emptiness check~\cite{SiCl-JACM85,LiPZ-CLP85}.

It can be shown that, in the worst case, an exponential blow-up in the
construction of the Büchi-automaton for a propositional LTL-formula cannot be
avoided (see Theorem~5.42 in~\cite{BaKa-08} for a proof).  However, there are
optimised implementations of the construction that try to keep the number of
states as small as possible (see
e.g.~\cite{GaOd-CAV01,GaOd-MFCS03,GPV+-PSTV96}).  Experiments with these
implementations show that an exponential blow-up can frequently be avoided.  For
example, the tool \emph{LTL2BA}\footnote{%
    See~\url{http://www.lsv.ens-cachan.fr/~gastin/ltl2ba/}.}
is widely used in practice to generate Büchi-automata from propositional
LTL-formulas.\footnote{%
    Unfortunately, like most other such tools, LTL2BA does \emph{not} support
    propositional LTL \emph{with past-operators}.}

\begin{example}
    Reconsider the propositional LTL-formula
    $\phi=\Next p_1\land(p_2\Until p_3)$ of Example~\ref{ex:ltl}.
    Figure~\ref{fig:ba-phi} depicts a Büchi-automaton for this formula, which
    was generated by LTL2BA\@.  Note that edges with propositional
    formulas~$\psi$ as labels are used as abbreviations for sets of edges
    labelled with those subsets of~\Pmc that represent models of~$\psi$.  For
    example, the edge with label $p_1\land p_2$ from $q_1$ to $q_3$ stands for
    two edges between these states, one with label $\{p_1,p_2\}$ and one with
    label $\{p_1,p_2,p_3\}$.

    This automaton accepts, for example, the $\omega$-word
    $\{p_2\}\{p_1,p_2\}\{p_2\}\{p_3\}\emptyset\dots$,
    which corresponds to the propositional LTL-structure $\Wmf_1$ of
    Example~\ref{ex:ltl} with $\Wmf_1,0\models\phi$.
\end{example}

\begin{figure}
    \centering
    \begin{tikzpicture}[%
            ->,
            >=stealth',
            semithick,
            initial text=,
            shorten <=2pt,
            shorten >=2pt,
            auto,
            on grid,
            node distance=8ex and 8em,
            every state/.style={minimum size=0pt,inner sep=2pt,text height=1.5ex,text depth=.25ex},
            bend angle=20]

        \node[state,initial]   (q0) {$q_0$};
        \node[state]           (q1) [above right=of q0] {$q_1$};
        \node[state,accepting] (q2) [below right=of q0] {$q_2$};
        \node[state]           (q3) [right=of q1] {$q_3$};
        \node[state,accepting] (q4) [right=of q2] {$q_4$};

        \path[->] (q0) edge node {$p_2$} (q1)
                       edge node {$p_3$} (q2)
                  (q1) edge node {$p_1\land p_2$} (q3)
                       edge node {$p_1\land p_3$} (q4)
                  (q2) edge node {$p_1$} (q4)
                  (q3) edge [loop right] node {$p_2$} (q3)
                       edge node {$p_3$} (q4)
                  (q4) edge [loop right] node {\true} (q4);
    \end{tikzpicture}
    \caption{A Büchi-automaton for the propositional LTL-formula~$\phi$ of
        Example~\ref{ex:ltl}}
    \label{fig:ba-phi}
\end{figure}

\noindent
In this chapter, we have introduced description logics.  Moreover, we have given
the basic definitions of propositional LTL and have recalled its relationship
with $\omega$-automata.
%
This sets a basis for the later chapters.
