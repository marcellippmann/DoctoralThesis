\chapter{Introduction}\label{ch:introduction}

In this day and age, it is not possible to imagine our world without complex
hardware and software systems.  Inevitably, it becomes more and more important
to verify that the systems that surround us have certain properties.  This is
indeed unavoidable for safety-critical systems such as power plants and
intensive-care units.
%
Throughout this thesis, we refer to the term \enquote{system} in a broad sense:
it may be a \enquote{man-made system} (e.g.\ a computer system) or a
\enquote{natural system} (e.g.\ a patient in an intensive-care unit).

Model Checking~\cite{ClGP-99,BaKa-08} is a prominent field of research that
addresses these issues.  However, there it is assumed that one has complete
knowledge about the functioning of the system, which is not always a reasonable
assumption.
%
In the present thesis, we consider an \emph{open-world scenario}.  Instead of
having a complete model of the system, we assume that we can only observe the
behaviour of the actual running system by \enquote{sensors}.  Such an abstract
sensor could, for instance, sense the blood pressure of a patient or the air
traffic observed by radar.  Then the observed data are preprocessed
appropriately and stored in a \emph{fact base}.  Based on the data available in
the fact base, situation-awareness tools~\cite{BBB+-TABLEAUX09,End-HF95} are
supposed to help the user to detect certain situations, e.g.\ situations that
require intervention by an expert.  Such situations could be, for instance, that
the heart-rate of a patient is rather high while the blood pressure is low, or
that a collision of two aeroplanes is about to happen.  Such critical situations
can be overcome by reacting accordingly, e.g.\ giving an appropriate medication
to the patient and informing the pilots of the aeroplanes, respectively.
%
Moreover, the information in the fact base can be used by \emph{monitors} to
verify that the system has certain properties.  Such a property could be, for
instance, that nothing \enquote{bad} will happen, i.e.~a so-called \emph{safety
property}.

It is not realistic, however, to assume that the sensors always yield a complete
description of the current state of the observed system.  Thus, it makes sense
to assume that information that is not present in the fact base is unknown
rather than to assume that this information does not hold, which we call the
\emph{open-world assumption}.
%
Moreover, very often one has \emph{some} knowledge about the functioning of the
system.  This \emph{background knowledge} can be used to draw conclusions about
the possible future behaviour of the system.

Employing description logics~\cite{DLhandbook-07} is one way to deal with these
requirements.  In this thesis, we tackle the sketched problem in three different
contexts: (i)~runtime verification using a temporal extension of a description
logic, (ii)~temporalised query entailment, and (iii)~verification in action
formalisms based on description logics.

In the remainder of this chapter, we give an abstract overview of the present
work.  In Section~\ref{sec:intro-dls}, we provide the reader with an intuitive
understanding of basic notions in description logics.  Then in
Section~\ref{sec:temporalised-dls}, we go one step further and consider temporal
extensions of description logics.  After that in
Section~\ref{sec:intro-monitor}, we give details about Context~(i), i.e.~runtime
verification using a temporalised description logic.
Section~\ref{sec:intro-tcqs} deals with Context~(ii), i.e.~temporalised query
entailment, and Section~\ref{sec:intro-actions} respectively deals with
Context~(iii), i.e.~verification in the research area of action formalisms that
are based on description logics.  Finally, Section~\ref{sec:outline} contains an
outline of the present thesis and summarises the main contributions.


\section{Description Logics}\label{sec:intro-dls}

Description logics (DLs)~\cite{DLhandbook-07} are a family of logic-based
knowledge representation formalisms.  Since they are logic-based, DLs have the
advantage of being equipped with a formal semantics, which lacks in early
knowledge representation formalisms such as Quillian's \emph{Semantic
Networks}~\cite{Qui-BS67} and Minsky's \emph{Frames}~\cite{Min-MD81}.
%
More information about the history of DLs can be found in the \emph{Description
Logic Handbook}~\cite{DLhandbook-07}.

Each DL is defined using a set of \emph{concept names}, a set of  \emph{role
names}, and a set of \emph{individual names}.  These names are used to express
knowledge in the respective application domain.  In the following, we consider a
zoo as application domain in order to be able to give simple examples that are
easy to comprehend.
%
In principle, concept names describe simple properties of elements in the
domain.  For example, the concept name \Camel can be used to describe all camels
(domain elements) in a zoo (domain).  Role names, e.g.\ \isFatherOf, describe
binary relations between domain elements.  Finally, individual names give names
to specific domain elements, e.g.\ \leah can be used as a name of a specific
camel in the zoo.
%
These sets are used by \emph{concept} and \emph{role constructors} to form
complex \emph{concepts} and \emph{roles}.  Which concept and role constructors
are available depends on the specific DL\@.

The smallest propositionally closed description logic is called
\ALC~\cite{ScSm-AIJ91}.  The name \ALC stands for \enquote{attributive language
with complements}.  There are, however, DLs that are less expressive than \ALC
such as \EL~\cite{Baa-IJCAI03,Bra-ECAI04} and extensions of
\EL~\cite{BaBL-IJCAI05,BaBL-OWLEDDC08} for which standard reasoning problems are
tractable.  With different reasoning problems in mind, other light-weight DLs
such as members of the \DLLite family~\cite{CDL+-AAAI05,ACK+-JAIR09,CDL+-RW09}
have been developed.  On the other hand, there are also DLs whose expressive
power goes far beyond \ALC such as \SROIQ~\cite{HoKS-KR06}.

As an example of a concept that is expressible in \ALC, consider
\[\lnot\Dromedary\sqcap\exists\likes.\Foliage,\]
where \Dromedary and \Foliage are concept names, and \likes is a role name.
Intuitively, this concept describes all domain elements that are not dromedaries
and like foliage.

Besides the logic-based semantics, another advantage of DLs is that most of them
can be seen as \emph{decidable} fragments of first-order predicate logic (FOL),
which are still more expressive than propositional logic.  Hence, important
reasoning problems such as concept satisfiability are decidable for DLs.
Concepts formulated in \ALC, for instance, can be translated into formulas of
the two-variable fragment of FOL\@.  The concept above can be formulated as
\[\lnot\dromedary(x)\land\exists y.\bigl(\likes(x,y)\land\foliage(y)\bigr),\]
where \dromedary and \foliage are unary predicates, and \likes is a binary
predicate.  In this FOL-formula, the free variable $x$ captures all domain
elements that are not dromedaries and like foliage.

There is also a close connection between description logics and modal logics.
In fact, \ALC can be seen as a notational variant of the multi-modal logic~\Kn
(see e.g.~\cite{GKW+-03,MLhandbook-07}).  For instance, the above concept can
be expressed in~\Kn as
\[\lnot\dromedary\land\Diamond_\likes\foliage,\]
where \dromedary and \foliage are propositional variables, and $\Diamond_\likes$
is a modal operator.

However, description logics do not only offer a language for describing
concepts, but allow to state knowledge in so-called \emph{knowledge bases},
which are split into an assertional part (the \emph{ABox}) and a terminological
part (the \emph{TBox}).  The ABox consists of a finite sets of
\emph{ABox-axioms} (or \emph{assertions}) such as concept assertions and role
assertions.  For instance, consider the ABox
\[\bigl\{(\lnot\Dromedary\sqcap\exists\likes.\Foliage)(\leah),\
    \isFatherOf(\hassan,\leah)\bigr\},\]
which states that Leah is not a dromedary and likes foliage, and that Hassan is
Leah's father.
%
The following TBox captures the terminological knowledge:
\[\bigl\{\lnot\Dromedary\sqcap\exists\likes.\Foliage\sqsubseteq\NiceCamel\bigr\}.\]
It states that each domain element which is not a dromedary and likes foliage is
actually a nice camel.

Interesting reasoning problems for such knowledge bases are, for instance,
consistency and entailment, i.e.~the question whether such a knowledge base has
a model and whether it entails certain implicit facts.
%
The formal definition of the notions introduced intuitively above that are
relevant for this thesis can be found in Section~\ref{sec:dls}.

Description logics are successfully employed in many areas such as
natural-language processing, conceptual modelling, and databases.  The most
notable success, however, has been achieved lately by adopting the \emph{Web
Ontology Language (OWL)}, which is based on an expressive DL, as standard
language for the semantic web~\cite{HoPH-JWS03}.  Recently, the second
refinement OWL\,2 has been endorsed by the World Wide Web Consortium (W3C) as
W3C Recommendation.\footnote{%
    See \url{http://www.w3.org/TR/owl2-overview/}.}
%
Additionally, tractable description logics are successfully employed for
defining medical ontologies, see e.g.~\cite{SBS+-AIME07}.

However, DLs are not expressive enough to describe the temporal behaviour of
systems.  Therefore, description logics have been \enquote{temporalised},
i.e.~equipped with temporal operators.  The next section gives a brief overview
on temporalised DLs.


\section{Temporalised Description Logics}\label{sec:temporalised-dls}

In the literature, a plethora of temporalised description logics has been
introduced (surveyed e.g.\ in~\cite{ArFr-AMAI00,ArFr-05,LuWZ-TIME08}).
%
To obtain a temporalised DL, one combines a DL with a \emph{temporal
logic}.\footnote{%
    Other approaches to obtain a temporalised DL, which are out of the scope of
    this thesis, include using fixpoint
    extensions~\cite{FrTo-DL03,FrTo-IJCAI11}, and encoding time in so-called
    concrete domains~\cite{Lut-IJCAI01}.}
%
For a comprehensive introduction to temporal logics, the reader is referred to
e.g.~\cite{Eme-90,GaHR-94,BaKa-08}.
%
In \emph{linear-time} temporal logics, the flow of time is assumed to be linear,
i.e.~each point in time has exactly one successor, whereas in
\emph{branching-time} temporal logics, each point in time may have more than one
successor, i.e.~the flow of time is assumed to be a tree.  A well-investigated
temporal logic with linear flow of time is \emph{linear-time temporal logic
(LTL)}~\cite{Pnu-FOCS77}, and logics with branching flow of time include
\emph{computation-tree logic (CTL)}~\cite{ClEm-LP81} and its extension
\emph{CTL$^*$}~\cite{EmHa-JACM86}.
%
It is a long-standing debate whether linear-time temporal logics or
branching-time temporal logics should be adopted, as both have strengths and
weaknesses in terms of expressiveness and computational complexity of
reasoning~\cite{Var-TACAS01,NaVa-ATVA07}.

In this thesis, we consider only combinations of LTL with description
logics.\footnote{%
    Combinations of a branching-time temporal logic such as CTL or CTL$^*$ with
    a description logic have been investigated e.g.\
    in~\cite{HoWZ-LICS02,BHW+-JLC04,GuJL-ECAI12}.}
%
The first linear-time temporalised DL, called LTL$_\ALC$, was introduced by
Schild~\cite{Sch-EPIA93}.  This temporalised DL is a combination of LTL with the
description logic \ALC.  In LTL$_\ALC$, concepts are built using concept
constructors \emph{and} temporal operators, i.e.~temporal operators are allowed
to occur within concepts.  For instance,
\[\lnot\Dromedary\sqcap\Diamond\exists\likes.\Foliage\]
is a concept formulated in LTL$_\ALC$.  The semantics of LTL$_\ALC$ is
two-dimensional, i.e.~one dimension describes the flow of time and a second
dimension is used for the domains.  Thus, the above concept captures all domain
elements which are not dromedaries (now), and will like foliage at some point in
the future.

An important question is what assumptions are made on the domains.  If we make
the \emph{constant-domain assumption}, the domain elements are global, i.e.~the
same domain elements are available at all points in time.  If the domains are
assumed to be \emph{expanding} (\emph{increasing}), the domain of the next point
in time contains always the current domain.  Similarly, if the domains are
assumed to be \emph{decreasing}, the domain of the next point in time is always
contained in the current domain.  If no restriction on the domains is imposed,
we speak of \emph{varying domains}.
%
However, for the satisfiability problem in LTL$_\ALC$ is is enough to consider
constant domains, since the satisfiability problem with expanding, decreasing
and varying domains can be polynomially reduced to the satisfiability problem
with constant domains~\cite{GKW+-03}.  Note that this reduction works also for
other temporalised DLs that allow temporal operators to occur in front of
concepts such as the ones in~\cite{WoZa-FroCoS98}.

Moreover, it is often desirable to ensure that the interpretation of certain
concept and role names does not change over time.  We call such concept and role
names \emph{rigid}.  For instance, one can argue that the concept name
\Dromedary should be rigid, i.e.~if a domain element is a dromedary, it will
always stay a dromedary in the future, and conversely, if a domain element is
not a dromedary, it will never become a dromedary.
%
\emph{Rigid concept names} can be simulated in LTL$_\ALC$ and similar
temporalised DLs.  To ensure that e.g.\ \Dromedary is rigid, the following two
TBox-axioms are sufficient:
\[\Dromedary\sqsubseteq\Box\Dromedary,\quad%
    \lnot\Dromedary\sqsubseteq\Box\lnot\Dromedary.\]
%
However, \emph{rigid role names} cannot be simulated in LTL$_\ALC$.
Interestingly, already one rigid role name causes the satisfiability problem of
LTL$_\ALC$-concepts w.r.t.\ global TBoxes to be
undecidable~\cite{GKW+-03,LuWZ-TIME08}.  Moreover, one can show by a reduction
of the recurrent tiling problem that the problem is actually \SigmaOneOne-hard,
i.e.~not even recursively enumerable~\cite{LuWZ-TIME08}.
%
The reduction crucially depends on the presence of a global TBox.  One way to
regain decidability is to restrict the TBox to an \emph{acyclic} one.\footnote{%
    For a formal definition of the syntax of acyclic TBoxes, see
    Definition~\ref{def:syntax-tbox}.}
%
However, the problem of deciding satisfiability of LTL$_\ALC$-concepts w.r.t.\
rigid roles and acyclic TBoxes is still hard for non-elementary
time~\cite{GKW+-03}.

If no rigid role names are allowed, the satisfiability problem of
LTL$_\ALC$-concepts w.r.t.\ global TBoxes is
\ExpTime-complete, which was originally stated in~\cite{Sch-EPIA93}.\footnote{%
    As remarked in~\cite{LuWZ-TIME08}, the original proof is incorrect.  For a
    correct proof, see~\cite{LuWZ-TIME08}.}
%
More generally, for any description logic~\Lmc between \ALC and \SHIQ, the
complexity of deciding satisfiability in LTL$_\Lmc$ is the same as the
complexity of deciding satisfiability of \Lmc-concepts~\cite{LuWZ-TIME08}.
Hence, the satisfiability problem of LTL$_\ALC$-concepts (without a global TBox)
is \PSpace-complete, since the satisfiability problem of \ALC-concepts is
\PSpace-complete~\cite{ScSm-AIJ91}.

Moreover, if one additionally allows temporal operators to occur in front of
axioms, i.e.~deals with temporal knowledge bases, the complexity of deciding
whether such a temporal knowledge base has a model, turns out to be
\ExpSpace-complete~\cite{GKW+-03} in LTL$_\ALC$ if no rigid role names are
considered.

Due to the high complexity of reasoning and the undecidability results for the
case where rigid role names are allowed, in~\cite{AKL+-TIME07,AKR+-AAAI10},
light-weight DLs have been extended by allowing temporal operators to
occur in front of concepts.  There, various complexity results for temporal
extensions of members of the \DLLite family are shown.  However,
in~\cite{AKL+-TIME07}, the authors also show that reasoning easily becomes
undecidable (if rigid role names are allowed) already in a small temporal
extension of \EL that is subsumed by LTL$_\EL$.

Another temporalised DL, which is a combination of \ALC and LTL, is
\ALC-LTL~\cite{BaGL-ToCL12}.  In \ALC-LTL, temporal operators are not allowed to
occur in front of concepts, but rather in front of axioms.  More precisely,
\ALC-LTL is \enquote{LTL over \ALC-axioms}, i.e.~it is LTL with \ALC-axioms
instead of propositional variables.  For instance, the following is an
\ALC-LTL-formula:
\[\lnot\Dromedary(\leah)\land\Diamond(\exists\likes.\Foliage)(\leah).\]
It states that Leah is not a dromedary (now), and she will like foliage at some
point in the future.  The complexity of deciding whether an \ALC-LTL-formula is
satisfiable is investigated in detail in~\cite{BaGL-ToCL12}.  For \ALC-LTL, one
needs to distinguish three different settings: (i)~no rigid names are available,
(ii)~only rigid concept names are available, and (iii)~rigid role names are
available.  Since temporal operators are not allowed to occur in front of
concepts, rigid concept names are no longer expressible in the logic.  However,
it is well-known that rigid role names can simulate rigid concept
names~\cite{BaGL-ToCL12}, and thus there are only three cases to consider.
Complexity results for the satisfiability problem in all three settings are
obtained in~\cite{BaGL-ToCL12} for the case of constant domains.  If no rigid
names are available, the satisfiability problem is \ExpTime-complete.  If only
rigid concept names are available, the complexity increases to
\NExpTime-complete, and if rigid role names are available, we have
\TwoExpTime-completeness.  Note that it is not clear whether considering
different kinds of domains like varying, expanding, or decreasing domains has an
impact on the complexity of the satisfiability problem, since
the reduction from varying, expanding, and decreasing domains to the
constant-domain case in~\cite{GKW+-03} does not work if temporal operators are
not allowed to occur in front of concepts.  In~\cite{BaGL-ToCL12}, it is
conjectured, however, that in some settings the complexity of the satisfiability
problem may decrease.

In this thesis, we use temporalised description logics to formulate knowledge
about the temporal behaviour of systems.  The next section gives an overview of
their application in runtime verification.


\section{Runtime Verification}\label{sec:intro-monitor}

Runtime verification~\cite{CoMa-MBTRS04} allows to verify whether an observed
system has certain (wanted or unwanted) properties.  These properties are
usually expressed in a temporal formalism.  This property is then
\enquote{translated} into a so-called \emph{monitor}.  Intuitively, such a
monitor solves the following task.  Having consumed a finite prefix of the
actual behaviour of the system, the monitor indicates whether the property is
satisfied or not.

In the literature, there is a plethora of approaches to constructing such
monitors, see
e.g.~\cite{HaRo-STTT04,RoHa-ASE05,Ros-SACS12,dARo-CAV05,BaLS-JLC10,BaLS-ToSEM11,BGH+-IPDPS04,BaRH-RV07,BaBL-FroCoS09}.
%
Here, we extend the work of~\cite{BaLS-ToSEM11} where the property is specified
in propositional LTL~\cite{Pnu-FOCS77} and a three-valued approach is developed.
To illustrate the idea, consider the following example.  Suppose that the vital
parameters of patient Bob are measured in an emergency ward.  If Bob has a high
heart rate and a low blood pressure, then an alarm should be raised.  This
property can be expressed using the propositional LTL-formula
\[\phi_\text{Bob}:=\Box(\lnot(\highHeartRateBob\land\lowBloodPressureBob)),\]
where \highHeartRateBob and \lowBloodPressureBob are propositional variables
whose validity at each point in time can be checked by evaluating the results of
sensing.  Intuitively, $\phi_\text{Bob}$ expresses that it is always the case
that Bob has not both a high hear rate and a low blood pressure.  If the formula
is violated, then we raise an alarm.  The information about Bob's health status
at each point in time now yields a finite prefix~$u$.  We need to check whether
\emph{all} continuations of this prefix satisfy or violate~$\phi_\text{Bob}$,
i.e.~no matter how the system's behaviour evolves over time, we certainly know
whether the formula is satisfied or not.  Thus, there are three possible answers
that a monitor may give having read such a prefix~$u$:
\begin{itemize}
    \item \enquote{true} if \emph{all} continuations of~$u$
        satisfy~$\phi_\text{Bob}$;
    \item \enquote{false} if \emph{all} continuations of~$u$
        do \emph{not} satisfy~$\phi_\text{Bob}$; and
    \item \enquote{inconclusive} if none of the above holds, i.e.~no definite
        answer can be given.
\end{itemize}
%
In our example, it should be clear that the monitor can never output
\enquote{true} since the \enquote{bad} state of the system could still be
observed in the future.  However, as soon as it outputs \enquote{false}, we can
raise an alarm and call for intervention by the medical staff.

Note that runtime verification is not about answering such a single question
given a prefix~$u$ and a propositional LTL-formula~$\phi$.  In fact, since the
behaviour of the system is observed over time, this prefix~$u$ is continuously
extended by adding new observations.  The monitor should not answer the
questions for the prefixes one after another independently of each other.  On
the contrary, the monitor should successively read the input, and based on this
information it should compute the answer in constant time (if the size of the
propositional LTL-formula is assumed to be constant).  Thus, the answer of the
monitor does \emph{not} depend on the length of the already observed prefix.

This approach presupposes, however, that the relevant propositional variables
can be evaluated at each point in time.  This is not always realistic due to the
following reasons.  Firstly, the states of the system may have a complex
internal structure, and secondly, the assumption that we have complete
information about the system's status at each point in time may be too strict.
%
Temporalised description logics help to overcome these issues.  A first step in
that direction was done in~\cite{BaBL-FroCoS09} where the temporalised
description logic \ALC-LTL~\cite{BaGL-ToCL12} was used for constructing
monitors.  There, \ALC-axioms capture the observations and at each point in
time, the monitor observes the system's status incompletely by reading an ABox.
Unfortunately, in the presence of rigid names, the approach developed
in~\cite{BaBL-FroCoS09} does not work.  In Chapter~\ref{ch:monitor}, we give a
correct monitor construction for the even more expressive temporalised
description logic \SHOQ-LTL\@.  Moreover, the approach developed here has a
better computational complexity, even though it also takes into account
background knowledge in the form of another \SHOQ-LTL-formula that specifies
\emph{temporal} information about the future behaviour of the system.

Considering again the example above, patient Bob's medical status can be
captured in ABoxes, whereas additional information about Bob is available from
the patient record and added by the medical staff.  Such background information
is encoded using concepts defined in a medical ontology like SNOMED~CT.%
\footnote{See~\url{http://www.ihtsdo.org/snomed-ct/}.}
The observed sequence of ABoxes contain a high-level view of the patient's
medical status, which the monitor uses to determine whether a critical situation
specified by a \SHOQ-LTL-formula has arisen.

In the next section, we consider a similar problem, namely temporalised query
entailment.


\section{Temporalised Query Entailment}\label{sec:intro-tcqs}

In a simple setting, one could realise a situation-awareness tool that helps the
user to detect certain situations by using standard database techniques.  For
that, the information obtained from the sensors is stored in a relational
database, and the situations to be recognised are specified by queries in an
appropriate query language, e.g.~conjunctive queries~\cite{AbHV-95}.
%
However, database systems employ the closed-world assumption (CWA),
i.e.~knowledge that is not present in the database is assumed to be false rather
than unknown.  As argued above, this assumption is not always appropriate as the
sensors may provide us with incomplete information about the current state of
the system.  Also, additional global knowledge, i.e.~knowledge that holds true
at each point in time, may be available, which allows for a limited projection
into the future such that maybe more (or less) answers to the query are found.

These requirements are addressed in the research field of ontology-based data
access (OBDA)~\cite{DEF+-DS99,PCD+-JoDS08}.  There, the fact base is an ABox,
which is interpreted with the open-world assumption, and an ontology is used to
encode the background knowledge.
%
In OBDA, one usually assumes that the ABox is obtained from external data sources
(in the case of situation awareness, the raw sensor data) through appropriate
mappings (which in our case realise the preprocessing and aggregation of the
sensor data), but here we abstract from the mapping step and assume that the
result of the preprocessing is explicitly represented in an ABox.

As an example, consider again patient Bob.  Assume that the ABox contains the
following information:
\[\begin{array}{lll}
     \systolicPressure(\bob,\pOne),
    &\HighPressure(\pOne),
    &\\
     \history(\bob,\hOne),
    &\Hypertension(\hOne),
    &\Male(\bob).
\end{array}\]

\noindent
Intuitively, the first line expresses that Bob has high blood pressure, which is
information obtained from sensor data.  The second line expresses that Bob has a
history of hypertension and that he is male, which is information obtained from
the patient record.
%
In addition, we have an ontology that states that patients with high blood
pressure have hypertension, and that patients that currently have hypertension
and also have a history of hypertension are at risk of a heart attack:
\begin{align*}
     \exists\systolicPressure.\HighPressure\
    &\sqsubseteq\ \exists\finding.\Hypertension,\\
     \exists\finding.\Hypertension\sqcap\exists\history.\Hypertension\
    &\sqsubseteq\ \exists\risk.\MyocardialInfarction.
\end{align*}

Assume that the situation we want to recognise for a given patient~$x$ is
whether this patient is a male person who is at risk of a heart attack.  This
situation can be described by the conjunctive query
\[\exists y.\Male(x)\land\risk(x,y)\land\MyocardialInfarction(y),\]
i.e.~\enquote{give me all~$x$ such that $x$ is male and at risk of~$y$, which is
a heart attack}.

Given the information in the ABox and the axioms in the ontology, we can derive
that \bob is a \emph{certain answer} to the query in the ABox w.r.t.\ the
ontology, i.e.~the following conjunctive query (without free variables)
\[\exists y.\Male(\bob)\land\risk(\bob,y)\land\MyocardialInfarction(y)\]
is \emph{entailed} by the ABox and the ontology.
%
Obviously, this answer cannot be derived without the ontology.

The complexity of \emph{query entailment} w.r.t.\ an ontology, i.e.~the
complexity of checking whether a given tuple of individual names is a certain
answer to a query in an ABox w.r.t.\ an ontology, has been investigated in
detail for cases where the ontology is expressed in an appropriate description
and the query is a conjunctive query.
%
One can either consider the \emph{combined complexity}, which is measured in the
size of the whole input (consisting of the query, the ontology, and the ABox),
or the \emph{data complexity}, which is measured in the size of the ABox only
(i.e.~the query and the ontology are assumed to be of constant size).  The
underlying assumption is that the query and the ontology are usually relatively
small, whereas the size of the data may be huge.

In the database setting (where there is no ontology and the CWA is used),
conjunctive-query entailment is \NP-complete w.r.t.\ combined complexity and in
\ACzero w.r.t.\ data complexity~\cite{ChMe-STOC77,AbHV-95}.
For expressive DLs, the complexity of checking certain answers is considerably
higher: for \ALC, the query entailment problem is \ExpTime-complete w.r.t.\
combined complexity and \coNP-complete w.r.t.\ data
complexity~\cite{CaDL-PODS98,Lut-IJCAR08,CDL+-KR06}.  For this reason, the more
light-weight DLs of the \DLLite-family have been developed, for which the
entailment problem is still in \ACzero w.r.t.\ data complexity, and for which
computing certain answers can be reduced to answering conjunctive queries in the
database setting~\cite{CDL+-RW09}.

Unfortunately, OBDA as described until now is not sufficient to achieve
high-level situation awareness.  The reason is that the situations we want to
recognise may depend on states of the system at different time points.
%
To illustrate this, consider again the above example.  There, it was explicitly
stated in the patient records that Bob has a history of hypertension.  It makes
more sense to derive this piece of information automatically from the data
obtained about blood pressure etc.

To achieve this, we propose a temporal query language that extends propositional
LTL by allowing conjunctive queries in place of propositional variables, which
is very similar to some of the temporalised DLs considered above.
%
In our example, we want to query for male patients with a history of
hypertension, which can be expressed by the temporal query
\[\Male(x)\land\Previous\Diamondm\bigl(\exists y.\finding(x,y)\land\Hypertension(y)\bigr).\]
Intuitively, this temporal query asks for all~$x$ such that $x$ is male and at
some point in the (strict) past, $x$ had a finding~$y$ of hypertension.

As for temporalised description logics, it makes perfect sense to consider rigid
concept and role names.  For example, we may want to assume that the concept
\Male is rigid, and thus a patient that is male now also has been male in the
past and will stay male in the future.

Thus, our overall setting for recognising situations will be the following.  In
addition to a global ontology (which describes properties of the system
that hold at every point in time), we have a sequence of ABoxes
$\Amc_0,\Amc_1,\dots,\Amc_n$, which (incompletely) describe the
states of the system at the previous time points $0,1,\dots,n-1$ and the
current time point~$n$.  The situation to be recognised is expressed
by a temporal conjunctive query, as introduced above, which is evaluated w.r.t.\
the current time point~$n$.
%
In Chapter~\ref{ch:tcqs}, we consider the complexity of this approach, i.e.~the
complexity of temporalised query entailment.  There, we adopt \SHQ as
description logic, which is an extension of the description logic \ALC.


\section{Verification in DL-Based Action Formalisms}\label{sec:intro-actions}

In this section, we consider a different scenario.  We assume that we have a
system that executes predefined \emph{actions}.  Moreover, we assume that we
have some knowledge about which actions the system executes.  This is captured
in a so-called \emph{action program}.  We consider non-terminating action
programs, and are interested in the \emph{verification problem}, i.e.~the
problem of deciding whether a certain (temporal) property holds after executing
an action program.

High-level action programming languages such as \textsc{Golog}~\cite{LRL+-JLP97}
and \textsc{Flux}~\cite{Thi-TPLP05} are based on the situation
calculus~\cite{Rei-01} and the fluent calculus~\cite{Thi-05}, respectively.
These calculi encompass full first-order logic, which implies that interesting
reasoning problems in them are undecidable.  To regain decidability, we restrict
our action formalism in two directions: (i)~our action formalism is based on a
decidable description logic, and (ii)~the action program is \enquote{generated}
by an $\omega$-automaton (i.e.~an automaton working on infinite words).

In~\cite{BaLM-ECAI10}, it is shown that the verification problem is decidable
for the DL-based action formalism introduced in~\cite{BLM+-AAAI05}.  The
properties are formulated in a restricted version of the temporalised
description logic \ALCO-LTL\@.  The authors of~\cite{BLM+-AAAI05,BaLM-ECAI10}
consider, however, only the case where domain constraints are encoded with
\emph{acyclic} TBoxes.  If one allows general TBoxes, one has to deal with the
\emph{ramification problem}, i.e.~the question which additional effects the
executing of an action has in order to satisfy the domain constraints.
%
In Chapter~\ref{ch:ramifications}, we introduce a DL-based action formalism that
is extended with so-called \emph{causal relationships} that take care of this
issue.  We prove that important inference problems in the extended action
formalism stay decidable and derive complexity results from the obtained
decision procedures.  Moreover, we continue the work of~\cite{BaLM-ECAI10} by
considering the verification problem in the extended action formalism.


\section{Outline and Contributions of the Thesis}\label{sec:outline}

In the following, we give a broad outline of the present thesis and summarise
the main scientific contributions.

In \emph{Chapter~\ref{ch:preliminaries}}, we formally introduce the basic
notions that are needed for the thesis.  These are foundations of description
logics and of propositional LTL\@.  Moreover, we revisit the relationship
between propositional LTL and $\omega$-automata.

In \emph{Chapter~\ref{ch:shoq-ltl}}, we introduce the temporalised description
logic \SHOQ-LTL, and prove complexity results for the satisfiability problem in
this logic.  We consider three different settings: (i)~neither concept names nor
role names are allowed to be rigid, (ii)~only concept names are allowed to be
rigid, and (iii)~both concept and role names are allowed to be rigid.  We can
show that the complexity is the same as in the less expressive temporalised
description logic \ALC-LTL~\cite{BaGL-ToCL12}, namely \ExpTime-complete in
Setting~(i), \NExpTime-complete in Setting~(ii), and \TwoExpTime-complete in
Setting~(iii).  In order to prove these results, we need to consider also the
consistency problem of Boolean \SHOQ-knowledge bases (w.r.t.\ some side
condition).  We can prove that this problem, which is interesting on its own,
can be decided in exponential time.
%
Some of the (ideas of the proofs of the) results of this chapter are already
published:
\begin{itemize}
    %\item \fullcite{tcq-journal}
    \item \fullcite{BaBL-CADE13}
    \item \fullcite{BaBL-LTCS-13-01}
\end{itemize}

In \emph{Chapter~\ref{ch:monitor}}, we consider runtime verification using the
temporalised description logic \SHOQ-LTL\@.  Before we can construct monitors
for \SHOQ-LTL-formulas, we need to construct $\omega$-automata for
\SHOQ-LTL-formulas.  For this construction, we reuse certain results from
Chapter~\ref{ch:shoq-ltl}.  We are able to show that even in the most complex
case where both rigid concept names and rigid role names are allowed, monitors
of doubly exponential size can be constructed using doubly exponential time.
Moreover, we show that this doubly exponential blow-up in the construction of
the monitor cannot be avoided.  Indeed, as we show, such a blow-up in
unavoidable even for propositional LTL\@.  Finally, we consider the complexity
of the deciding \emph{liveness} and \emph{monitorability}, which are two
important related decision problems, for the three settings above.  Our results
are only tight for the case where both rigid concept and role names are allowed.
There, both problems are \TwoExpTime-complete.  For the other cases, a gap
remains: both problems are \ExpTime-hard and in \TwoExpTime if no rigid names
are allowed, and \coNExpTime-hard and in \TwoExpTime if only rigid concept names
are allowed.  However, the exact complexity of these problems are not even known
for propositional LTL\@.
%
Some of the results of this chapter are already published.  However, we
considered only the less expressive temporalised description logic \ALC-LTL in
the following publications:
\begin{itemize}
    \item \fullcite{BaBL-FroCoS09}
    \item \fullcite{BaLi-LTCS-14-01}
\end{itemize}

In \emph{Chapter~\ref{ch:tcqs}}, we consider temporalised query entailment in
any description logic between \ALC and~\SHQ.  After formally introducing the
temporal query language that we consider, we provide complexity results of the
corresponding entailment problem.  For the three settings above, we have shown
results both for data complexity and combined complexity.  If neither concept
nor role names are allowed to be rigid, temporalised query entailment is
\coNP-complete w.r.t.\ data complexity and \ExpTime-complete w.r.t.\ combined
complexity.  If only concept names may be rigid, the problem is \coNP-complete
w.r.t.\ data complexity and \coNExpTime-complete w.r.t.\ combined complexity.
Finally, if both concept and role names are allowed to be rigid, the problem is
\coNP-hard and in \ExpTime w.r.t.\ data complexity and \TwoExpTime-complete
w.r.t.\ combined complexity.  For showing these results, some results of
Chapter~\ref{ch:shoq-ltl} are used.
%
Most of the results of this chapter are already published:
\begin{itemize}
    %\item \fullcite{tcq-journal}
    \item \fullcite{BaBL-CADE13}
    \item \fullcite{BaBL-LTCS-13-01}
\end{itemize}

Finally, in \emph{Chapter~\ref{ch:ramifications}}, we consider the verification
problem in action formalisms based on description logics between \ALC and
\ALCQIO.  To solve the ramification problem, i.e.~the question how to deal with
indirect effects caused by domain constraints (which arises if we allow general
TBoxes), we extend the DL-based action formalism introduced
in~\cite{BLM+-AAAI05} (which could deal only with acyclic TBoxes) with causal
relationships.  We show that important inference problems such as the
consistency problem and the projection problem are decidable in our new
formalism, and continue the work of~\cite{BaLM-ECAI10} by generalising the
verification problem.  We derive a number of complexity results from the
obtained decision procedures.  Depending on the base DL, the complexity results
range from \PSpace-complete to \coNExpTime-hard and in \PTimeToNExpTime for the
consistency problem, and from \PSpace-complete to \coNExpTime-complete for the
projection problem.  For the verification problem, the complexity ranges from in
\ExpSpace to in \coTwoNExpTime, and it is unknown whether these bounds are
tight.
%
Some of the results of this chapter are already published:
\begin{itemize}
    \item \fullcite{BaLL-LPAR10}
    \item \fullcite{YLL+-DL12}
    \item \fullcite{BaLL-LTCS-10-01}
\end{itemize}
